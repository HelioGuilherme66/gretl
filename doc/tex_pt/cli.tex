\chapter{O interface de linha de comandos (CLI)}
\label{cli}

A aplicação \app{gretl} inclui o programa de linha de comandos \app{gretlcli}.
Em Linux pode ser iniciado a partir de uma janela de terminal (xterm,
rxvt, ou semelhante), ou de uma consola de texto.  No MS Windows pode ser 
executado numa janela de linha de comandos (algumas vezes incorrectamente
chamada de ``janela MS DOS'').
\app{gretlcli} tem o seu próprio ficheiro de ajuda, que pode ser acedido
escrevendo o comando ``help'' no interface CLI. Pode ser executado em modo de
script, enviando os resultados directamente para um ficheiro
(ver também o \GCR).
    
Se \app{gretlcli} tiver sido ligado no momento da compilação à biblioteca de
programas \app{readline} (o que acontece sempre no caso da versão MS Windows;
ver também Apêndice~\ref{app-build}), é possível repetir e editar as linhas
de comandos, e também completar comandos automaticamente.  Você pode usar as
teclas seta-Acima e seta-Abaixo para percorrer os comandos anteriormente
executados.  Numa dada linha de comando, você pode usar as setas para mover o
cursor, em conjunto com as combinações de teclas do editor Emacs.\footnote{Na realidade, as combinações de  teclas referidas abaixo são apenas as definidas por omissão; elas poderão ser personalizadas por si.
  Ver o \href{http://cnswww.cns.cwru.edu/~chet/readline/readline.html}{manual
    do readline}.} As mais comuns são:
%    
\begin{center}
  \begin{tabular}{cl}
    \textit{Combinação} & \multicolumn{1}{c}{\textit{Efeito}}\\
    \verb+Ctrl-a+ & ir para o início da linha\\
    \verb+Ctrl-e+ & ir para o fim da linha\\
    \verb+Ctrl-d+ & apagar o caracter à direita\\
  \end{tabular}
\end{center}
%
onde ``\verb+Ctrl-a+'' significa premir a tecla ``\verb+a+'' ao mesmo tempo
que a tecla ``\verb+Ctrl+'' é premida.  Assim, se você quiser alterar algo no
início de um comando, você \emph{não} precisa de apagar caracter a caracter
na linha toda.  Basta saltar para o início e acrescentar ou apagar caracteres.
Se você escrever as primeiras letras de um comando e pressionar a tecla Tab, o
sistema \app{readline} vai tentar completar o comando por você.  Se houver uma
única possibilidade, o comando é automaticamente completado.  Se houver mais
que uma, pressionando Tab uma segunda vez faz aparecer uma lista.

Provavelmente o modo mais produtivo para análises intensivas com o
\app{gretlcli} é em modo de script (não-interactivo), no qual o
programa lê e processa um ficheiro de script, e envia a saída
para um ficheiro.  Por exemplo
\begin{code}
gretlcli -b ficheirodescript > ficheiroderesultados
\end{code}

O \textsl{ficheirodescript} é tratado como um argumento de programa; apenas o ficheiro de resultados requer redirecionamento (\verb|>|).  Não esquecer a opção \texttt{-b}
(\textit{batch}=lote), de outro modo o programa ficará a aguardar comandos do
utilizador após a execução do script (e se a saída estiver redirecionada, o programa aparentará estar "pendurado").

    
%%% Local Variables: 
%%% mode: latex
%%% TeX-master: "gretl-guide-pt"
%%% End: 

