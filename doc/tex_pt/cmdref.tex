\chapter{Gretl commands}
\label{cmdref}

\section{Introduction}
\label{cmd-intro}

The commands defined below may be executed interactively in the
command-line client program or in the console window of the GUI
program.  They may also be placed in a ``script'' or batch file for
non-interactive execution.
    
The following notational conventions are used below:
    
\begin{itemize}

\item A \texttt{typewriter font} is used for material that you would
  type directly, and also for internal names of variables.

\item Terms in a \textsl{slanted font} are place-holders: you should
  substitute some specific replacement.  For example, you might type
  \texttt{income} in place of the generic \textsl{xvar}.

\item The construction \texttt{[} \textsl{arg} \texttt{]} means that
  the argument \textsl{arg} is optional: you may supply it or not (but
  in any case don't type the brackets).

\item The phrase ``estimation command'' means a command that generates
  estimates for a given model, for example \cmd{ols}, \cmd{ar} or
  \cmd{wls}.

\end{itemize}

In general, each line of a command script should contain one and only
one complete gretl command.  There are, however, two means of
continuing a long command from one line of input to another.  First,
if the last non-space character on a line is a backslash, this is
taken as an indication that the command is continued on the following
line.  In addition, if the comma is a valid character in a given
command (for instance, as a separator between function arguments, or
as punctuation in the command \texttt{printf}) then a trailing comma
also indicates continuation.  To emphasize the point: a backslash may
be inserted ``arbitrarily'' to indicate continuation, but a comma
works in this capacity only if it is syntactically valid as part of
the command.

\section{Commands}
\label{cmd-cmd}

%% auto-generated from XML base, gretl_commands.xml

\subsection{add}
\hypertarget{cmd-add}{}

\begin{tabular}{ll}
Argumento:   & \textsl{lista-de-vari�veis} \\
Op��es:      & \verb|--vcv| (mostrar matriz de covari�ncia) \\
 & \verb|--quiet| (n�o mostrar estimativas para o modelo aumentado) \\
 & \verb|--silent| (n�o mostrar nada) \\
 & \verb|--inst| (acrescentar como instrumento, apenas para TSLS) \\
 & \verb|--both| (acrescentar tanto como regressor como instrumento, apenas para TSLS) \\
Exemplos:    & \texttt{add 5 7 9} \\ 
 & \verb@add xx yy zz --quiet@ \\ 
\end{tabular}

	Tem que ser invocado ap�s um comando de estima��o. As vari�veis na
	\textsl{lista-de-vari�veis} s�o acrescentadas ao modelo anterior
        e o novo modelo � estimado.  � apresentada uma estat�stica de 
        signific�ncia conjunta, juntamente com o seu p-value.  O teste
        estat�stico � o F no caso de estima��o por m�nimos
        quadrados (OLS), ou qui-quadrado assimpt�tico de Wald nos outros
        casos.  Um p-value abaixo de 0,05 significa que os coeficientes s�o
        conjuntamente significantes num n�vel de 5 porcento.

	Se foi fornecida a op��o \verb@--quiet@ os resultados apresentados
	ficam confinados ao teste da signific�ncia conjunta das vari�veis
	acrescentadas, caso contr�rio, tamb�m ser�o mostradas as estimativas
        para o modelo aumentado.  Neste caso, a op��o \verb@--vcv@ faz com
        que a matriz de covari�ncias dos coeficientes tamb�m seja apresentada.
	  Se for usada a op��o \verb@--silent@, n�o ser� mostrado nada; em
        em todo o caso, os resultados do teste podem ser obtidos usando as
        vari�veis especiais \verb@$test@ e \verb@$pvalue@.

	Se o modelo original foi estimado usando o m�todo dos m�nimos quadrados
        de duas fases, ocorre uma situa��o amb�gua: devem as novas vari�veis
        serem acrescentadas como sendo regressores, como instrumentos, ou como
	ambos?  Isto resolve-se do seguinte modo: por omiss�o as novas vari�veis
	s�o acrescentadas como regressores end�genos, mas se foi dada a op��o 
	\verb@ --inst@ elas s�o acrescentadas como instrumentos, ou se a op��o 
	\verb@--both@ est� presente elas s�o acrescentadas como regressores
	ex�genos.

Caminho de Menu:    Janela do modelo, /Testes/Acrescentar vari�veis

\subsection{adf}
\hypertarget{cmd-adf}{}

\begin{tabular}{ll}
Argumentos:  & \textsl{ordem} \textsl{nome-de-vari�vel} \\
Op��es:      & \verb|--nc| (teste sem constante) \\
 & \verb|--c| (apenas com constante) \\
 & \verb|--ct| (com constante e tend�ncia) \\
 & \verb|--ctt| (com constante, tend�ncia e quadrado da tend�ncia) \\
 & \verb|--seasonals| (incluir vari�veis sazonais auxiliares) \\
 & \verb|--verbose| (mostrar resultados da regress�o) \\
 & \verb|--quiet| (n�o mostrar resultados da regress�o) \\
 & \verb|--difference| (usar a primeira diferen�a da vari�vel) \\
 & \verb|--test-down| (ordem 'lag' autom�tica) \\
Exemplos:    & \texttt{adf 0 y} \\ 
 & \verb@adf 2 y --nc --c --ct@ \\ 
 & \verb@adf 12 y --c --test-down@ \\ 
 & Ver tamb�m\texttt{jgm-1996.inp}
\end{tabular}

	Determina estat�sticas para um conjunto de testes de Dickey--Fuller
        sobre a vari�vel especificada, com a hip�tese nula de que a vari�vel
        tem uma raiz unit�ria.  (Mas se a op��o de diferencia��o tiver sido 
        dada, a primeira diferen�a da vari�vel � obtida, e a discuss�o abaixo
        deve ser interpretada como sendo referente � vari�vel transformada.)

Por omiss�o, s�o apresentadas tr�s variantes do teste: uma baseada
        na regress�o contendo uma constante, uma usando uma constante e uma
        tend�ncia linear, e uma usando uma constante e uma tend�ncia quadr�tica.
          Voc� pode controlar as variantes que s�o apresentadas ao especificar
        uma ou mais op��es.

Em todos os casos a vari�vel dependente � a primeira diferen�a
        da vari�vel especificada, y,	e a vari�vel independente
        chave � o primeiro 'lag' de y.  O modelo � construido
        de modo a que o coeficiente do 'lag' de y iguale 1 
        menos a raiz em quest�o.  Por exemplo, o modelo com uma constante pode
        ser escrito como \[(1-L)y_t=\beta_0+(1-\alpha)y_{t-1}+\epsilon_t\]

	Se a ordem de 'lag', k, � maior que 0, ent�o
        k 'lags' da vari�vel dependente s�o incluidos 
        no lado direito das regress�es de teste, sujeitos � seguinte
        qualifica��o.  Se a op��o \verb@--test-down@ foi dada, 
        k � considerada como sendo o 'lag' m�ximo e a
        ordem de 'lag' efectivamente usada � obtida testando para baixo,
        de acordo com o seguinte algoritmo:

\begin{enumerate}
\item Estimar a regress�o de Dickey--Fuller com
	    k 'lags' da vari�vel dependente.

\item O �ltimo 'lag' � significante?  Se sim, executar o teste
            com com a ordem de 'lag', k.  Sen�o, fazer 
	    k = k $-$ 1; se 
	    k for igual a 0, executar o teste com a
            ordem de 'lag' 0, sen�o saltar para o passo 1.

\end{enumerate}

No contexto do passo 2 acima, ``significante''
	quer dizer que para o �ltimo 'lag', a estat�stica-t,
        que segue uma distribui��o normal, tem um \emph{p}-value
        bilateral assimpt�tico menor ou igual a 0,10.

Os \emph{p}-values para os testes de Dickey--Fuller
	baseiam-se em MacKinnon (1996).  O c�digo relevante � inclu�do com a
        generosa permiss�o do autor.

Caminho de Menu:    /Vari�vel/Teste de Dickey-Fuller aumentado

\subsection{append}
\hypertarget{cmd-append}{}

\begin{tabular}{ll}
Argumento:   & \textsl{ficheiro-de-dados} \\
\end{tabular}

	Abre um ficheiro de dados e acrescenta esse conte�do ao conjunto 
        de dados actual, se os novos dados forem compat�veis.  O programa
        tentar� determinar o formato do ficheiro de dados (nativo, texto 
        simples, CVS, Gnumeric, Excel, etc.).

Caminho de Menu:    /Ficheiro/Acrescentar dados

\subsection{ar}
\hypertarget{cmd-ar}{}

\begin{tabular}{ll}
Argumentos:  & \textsl{'lags'} \texttt{;} \textsl{vari�vel-dependente} \textsl{vari�veis-independentes} \\
Op��o:       & \verb|--vcv| (mostrar matriz de covari�ncia) \\
Exemplo:     & \texttt{ar 1 3 4 ; y 0 x1 x2 x3} \\ 
\end{tabular}

	Determina estimativas para os par�metros usando o
        procedimento iteractivo e generalizado de 
        Cochrane--Orcutt (ver a Sec��o 9.5 de Ramanathan, 
        2002). A itera��o termina quando os erros das somas de 
        quadrados sucessivos n�o difiram em mais que 0,005 porcento 
        ou ap�s 20 itera��es.

	\textsl{'lags'} � uma lista de 'lags' nos 
        res�duos, terminada por um ponto-e-v�rgula. No exemplo acima
        o termo do erro � especificado como
	\[u_t = \rho_1u_{t-1} + \rho_3 u_{t-3} + \rho_4 u_{t-4} + e_t\]

Caminho de Menu:    /Modelo/S�rie temporal/Estima��o autoregressiva

\subsection{ar1}
\hypertarget{cmd-ar1}{}

\begin{tabular}{ll}
Argumentos:  & \textsl{depvar} \textsl{indepvars} \\
Op��es:      & \verb|--hilu| (use Hildreth--Lu procedure) \\
 & \verb|--pwe| (use Prais--Winsten estimator) \\
 & \verb|--vcv| (print covariance matrix) \\
 & \verb|--no-corc| (do not fine-tune results with Cochrane-Orcutt) \\
Exemplos:    & \texttt{ar1 1 0 2 4 6 7} \\ 
 & \verb@ar1 y 0 xlist --hilu --no-corc@ \\ 
 & \verb@ar1 y 0 xlist --pwe@ \\ 
\end{tabular}

	Computes feasible GLS estimates for a model in which the error term is
	assumed to follow a first-order autoregressive process.

	The default method is the Cochrane--Orcutt iterative procedure
	(see, for example, Section 9.4 of Ramanathan, 2002). Iteration is
	terminated when successive estimates of the autocorrelation
	coefficient do not differ by more than 0.001 or after 20 iterations.

	If the \verb@--hilu@ option is given, the Hildreth--Lu search
	procedure is used.  The results are then fine-tuned using the
	Cochrane--Orcutt method, unless the \verb@--no-corc@ flag is
	specified.  (The latter option is ignored if \verb@--hilu@
	is not specified.)

	If the \verb@--pwe@ option is given, the Prais--Winsten
	estimator is used.  This involves an an iteration similar to
	Cochrane--Orcutt; the difference is that while
	Cochrane--Orcutt discards the first observation,
	Prais--Winsten makes use of it. See, for example, Chapter 13 of
	Greene's \emph{Econometric Analysis} (2000) for details.

Caminho de Menu:    /Model/Time series/Cochrane-Orcutt

Caminho de Menu:    /Model/Time series/Hildreth-Lu

Caminho de Menu:    /Model/Time series/Prais-Winsten

\subsection{arbond}
\hypertarget{cmd-arbond}{}

\begin{tabular}{ll}
Argumento:   & \textsl{p} \texttt{[} \textsl{q} \texttt{]} \texttt{;} \textsl{vari�vel-dependente} \textsl{vari�veis-independentes} \texttt{[} \texttt{;} \textsl{instrumentos} \texttt{]} \\
Op��es:      & \verb|--vcv| (mostrar matriz de covari�ncia) \\
 & \verb|--two-step| (executa estima��o pelo M�todo Generalizado dos Momentos (GMM) de 2-fases) \\
 & \verb|--time-dummies| (acrescenta vari�veis auxiliares tempo) \\
 & \verb|--asymptotic| (erros padr�o assimpt�ticos) \\
Exemplos:    & \texttt{arbond 2 ; y Dx1 Dx2} \\ 
 & \texttt{arbond 2 5 ; y Dx1 Dx2 ; Dx1} \\ 
 & \texttt{arbond 1 ; y Dx1 Dx2 ; Dx1 GMM(x2,2,3)} \\ 
 & Ver tamb�m\texttt{arbond91.inp}
\end{tabular}

	Executa a estima��o de modelos de painel din�mico (ou seja,
        modelos de painel que contenham um ou mais 'lags' da vari�vel
        dependente) recorrendo ao m�todo M�todo Generalizado dos Momentos 
       (GMM) desenvolvido por Arellano e Bond (1991).  

	O par�metro \textsl{p} representa a ordem da autoregress�o 
        para a vari�vel dependente.  O par�metro opcional \textsl{q}
        indica o m�ximo 'lag' do n�vel da vari�vel dependente a ser usada
        como um instrumento.  Se este argumento for omitido, ou de valor 
        0, todos os 'lags' dispon�veis s�o usados.

	A vari�vel dependente deve ser dada na forma de n�veis; ela ser�
        automaticamente diferenciada (pois este estimador usa 
        diferencia��o para anular os efeitos individuais).  As vari�veis
        independentes n�o s�o automaticamente diferenciadas; se voc� 
        pretende usar diferen�as (o que acontece em geral para vari�veis
        quantitativas, mas n�o ser�, por exemplo, para vari�veis 
        auxiliares temporais), deve primeiro criar essas diferen�as e 
        depois especificar estas como sendo regressoras.

	O �ltimo campo (opcional) do comando serve para especificar 
        instrumentos.  Se n�o for dado nenhum, ent�o � assumido que todas
        as vari�veis independentes s�o estritamente ex�genas.  Se voc�
        especificar alguns instrumentos, ent�o deve incluir na lista 
        quaisquer vari�veis independentes estritamente ex�genas.  Para
        regressores predeterminados, voc� pode usar a fun��o 
        \texttt{GMM} para incluir uma gama de 'lags' especificada no 
        modo bloco-diagonal.  Isto � ilustrado no terceiro exemplo acima.
          O primeiro argumento de \texttt{GMM} � o nome da vari�vel em 
        quest�o, o segundo � o m�nimo 'lag' a ser usado como instrumento,
        e o terceiro � o m�ximo 'lag'.  Se o terceiro argumento for dado
        como 0, todos os 'lags' dispon�veis s�o usados.

	Por omiss�o s�o apresentados os resultados da estima��o 1-fase 
        (com erros padr�o robustos).  Opcionalmente, voc� pode escolher 
        estima��o de 2-fases.  Em ambos os casos s�o efectuados testes de
        autocorrela��o de ordem 1 e 2, assim como o teste de 
        sobre-identifica��o de Sargan e o teste de Wald para a 
        signific�ncia conjunta dos regressores.  Note-se que este modelo 
        de diferencia��o com autocorrela��o de primeira-ordem n�o invalida 
        o modelo, mas que a autocorrela��o de segunda-ordem n�o respeita 
        as assun��es estat�sticas presentes.

	No caso da estima��o de 2-fases, por omiss�o, os erros padr�o s�o
        determinados usando a correc��o de amostra-finita sugerida por 
        Windmeijer (2005).  Os erros padr�o assimpt�ticos associados ao 
        estimador de 2-fases s�o em geral considerados como guias para 
        infer�ncia pouco fi�veis, mas se por alguma raz�o os pretender 
        ver, voc� pode usar a op��o \verb@--asymptotic@ para desligar
        a correc��o de Windmeijer.

	Se for dada a op��o \verb@--time-dummies@, s�o acrescentadas
        vari�veis auxiliares temporais aos regressores especificados.  
        Para evitar colinearidade perfeita com a constante, o n�mero de 
        auxiliares � uma unidade a menos que o n�mero m�ximo de per�odos
        usados na estima��o.  As auxiliares s�o introduzidas por n�veis;
        se voc� deseja usar auxiliares de tempo na forma de 
        primeiras-diferen�as, voc� ter� que definir e acrescentar essas
        vari�veis manualmente.

Caminho de Menu:    /Modelo/Painel/Arellano-Bond

\subsection{arch}
\hypertarget{cmd-arch}{}

\begin{tabular}{ll}
Argumentos:  & \textsl{ordem} \textsl{vari�vel-dependente} \textsl{vari�veis-independentes} \\
Exemplo:     & \texttt{arch 4 y 0 x1 x2 x3} \\ 
\end{tabular}

	Testa o modelo em ARCH (Heterosquedicidade Condicional 
        Autoregressiva) da ordem de 'lag' especificada. Se a 
        estat�stica de teste LM tiver um p-value abaixo de 0,10,
        ent�o a estima��o ARCH tamb�m � executada.  Se a vari�ncia
        predita de qualquer observa��o na regress�o auxiliar n�o for
        positiva, ent�o � usado o correspondente res�duo ao quadrado.
        Segue-se uma estima��o por M�nimos Quadrados com Pesos sobre 
        o modelo original.

	Ver tamb�m \hyperlink{cmd-garch}{garch}.

Caminho de Menu:    Janela do modelo, /Testes/ARCH

\subsection{arima}
\hypertarget{cmd-arima}{}

\begin{tabular}{ll}
Argumentos:  & \textsl{p} \textsl{d} \textsl{q} \texttt{[} \texttt{;} \textsl{P} \textsl{D} \textsl{Q} \texttt{]} \texttt{;} \textsl{vari�vel-dependente} \texttt{[} \textsl{vari�veis-independentes} \texttt{]} \\
Op��es:      & \verb|--verbose| (mostrar detalhes das itera��es) \\
 & \verb|--vcv| (mostrar matriz de covari�ncia) \\
 & \verb|--nc| (n�o incluir uma constante) \\
 & \verb|--conditional| (usar verosimilhan�a m�xima condicional) \\
 & \verb|--x-12-arima| (usar X-12-ARIMA para estima��o) \\
Exemplos:    & \texttt{arima 1 0 2 ; y} \\ 
 & \verb@arima 2 0 2 ; y 0 x1 x2 --verbose@ \\ 
 & \verb@arima 0 1 1 ; 0 1 1 ; y --nc@ \\ 
\end{tabular}

	Se n�o for dada a lista de \textsl{vari�veis-independentes},
        � estimado um modelo ARIMA (M�dia M�vel, Autoregressiva, Integrada)
        univariado.  O valores inteiros \textsl{p}, \textsl{d} e 
        \textsl{q} representam respectivamente, a ordem autoregressiva 
        (AR), a ordem de diferencia��o, e ordem da m�dia m�vel (MA). 
	  Estes valores podem ser fornecidos na forma num�rica, ou como
        nome de vari�veis escalares pr�-existentes.  Por exemplo, um valor
        de 1 em \textsl{d}, significa que a primeira diferen�a da vari�vel
        dependente deve ser obtida antes de estimar os par�metros ARMA.

	Os valores inteiros opcionais,\textsl{P}, \textsl{D} e
	\textsl{Q} representam respectivamente, a sazonalidade AR,
        a ordem para diferencia��o de sazonalidade e a ordem de 
        sazonalidade MA.  Estes s�o apenas aplic�veis se os dados tiverem
        uma frequ�ncia superior a 1 (por exemplo, quadrimestral ou 
        mensal). Mais uma vez, estas ordens podem ser dadas na forma 
        num�rica ou como vari�veis.

	No caso univariado � inclu�do no modelo por omiss�o, um interceptor,  
        mas isto pode ser suprimido com a op��o	\verb@--nc@.  Se forem
        fornecidas \textsl{vari�veis-independentes}, o modelo passa a 
        ser ARMAX; neste caso a constante deve ser explicitamente inclu�da
        se voc� pretender um interceptor (tal como no segundo exemplo acima).

	Existe outra forma alternativa para este comando: se voc� n�o 
        pretende aplicar diferencia��o (seja sazonal ou n�o-sazonal), 
        voc� pode omitir ambos os par�metros \textsl{d} e 
        \textsl{D}, em vez de entrar explicitamente zeros.  Al�m 
        disso, \texttt{arma} � um sin�nimo ou ali�s para 
        \texttt{arima}.  Assim, por exemplo, o comando seguinte � 
        v�lido para especificar o modelo ARMA(2, 1):

\begin{code}
	arma 2 1 ; y
      \end{code}

	O normal � usar a funcionalidade ``nativa'' gretl
        ARMA, com estima��o de M�xima Verosimilhan�a (ML) exacta 
        (usando o filtro de Kalman).  Outras op��es s�o: c�digo nativo,
         ML condicional; X-12-ARIMA, ML exacta; e 
	X-12-ARIMA, ML condicional.  (As �ltimas
        duas op��es est�o dispon�veis apenas se o programa
	X-12-ARIMA estiver instalado.)  Para detalhes
        sobre estas op��es, veja por favor \GUG{}.

O valor AIC retornado em liga��o com os modelos ARIMA �
	calculado conforme a defini��o usada no programa 
	X-12-ARIMA, nomeadamente
	  \[\mbox{AIC}=-2\ell + 2k\] onde 
	$\ell$ � o
	logaritmo da verosimilhan�a e k � o n�mero
        total de par�metros estimados.  Note-se que o programa 
        X-12-ARIMA n�o produz crit�rios de informa��o
        tal como o AIC quando a estima��o � por ML condicional.

	A imagem da ``frequ�ncia'' apresentada
	em liga��o com as ra�zes AR e MA � valor $\lambda$ que resolve
	  \[z=re^{i2\pi\lambda}\] onde z � a raiz em 
        quest�o e r o seu m�dulo.

Caminho de Menu:    /Modelo/S�rie temporal/ARIMA

Acesso alternativo: Menu de contexto da janela principal (selec��o singular)

\subsection{boxplot}
\hypertarget{cmd-boxplot}{}

\begin{tabular}{ll}
Argumento:   & \textsl{lista-de-vari�veis} \\
Op��o:       & \verb|--notches| (mostrar intervalo de 90 porcento para a mediana delimitado por entalhes) \\
\end{tabular}

Estes gr�ficos (criados por Tukey e Chambers) apresentam a
        distribui��o de uma vari�vel.  A caixa central cont�m os 50 
        porcento dos dados centrais, i.e.{} est� limitada pelos primeiro
        e terceiro quartis.  Os ``bigodes'' estendem-se at� 
        aos valores m�nimo e m�ximo.  � desenhada uma linha que corta a
        caixa na mediana.

No caso de caixas com entalhes, os entalhes representam os 
        limites do intervalo de confian�a para a mediana de cerca de 90
        porcento.  Isto � obtido usando o m�todo 'bootstrap'.

A seguir a cada vari�vel indicada no 
        comando caixa-com-bigodes, pode-se acrescentar uma express�o
        Booleana para restringir a vari�vel em quest�o.  Tem que se 
        inserir um espa�o entre o nome da vari�vel ou n�mero, e a 
        express�o.  Suponha que voc� disp�e de valores de sal�rios
        (\texttt{salary}) para homens e mulheres, e que tem a vari�vel
        auxiliar \texttt{GENDER} com valor 1 para homens e 0 para mulheres.  
        Nesse caso voc� podia ter gr�ficos caixa-com-bigodes 
        comparativos com a seguinte \textsl{lista-de-vari�veis}:

\begin{code}
	salary (GENDER=1) salary (GENDER=0)
      \end{code}
Alguns detalhes das caixas-com-bigodes do gretl podem ser
        controlados por interm�dio de um ficheiro (de texto simples)
        com o nome \texttt{.boxplotrc}.  Para mais detalhes
        sobre isto veja \GUG{}.

Caminho de Menu:    /Ver/Gr�fico das vari�veis/Caixa com bigodes

\subsection{break}
\hypertarget{cmd-break}{}
Sai de um ciclo.  Este comando pode apenas ser usado
        dentro de um ciclo; ele termina a execu��o de comandos e
        sai de dentro do ciclo (o mais interior).  
        Ver tamb�m \hyperlink{cmd-loop}{loop}.

\subsection{chow}
\hypertarget{cmd-chow}{}

\begin{tabular}{ll}
Argumento:   & \textsl{obs} \\
Exemplos:    & \texttt{chow 25} \\ 
 & \texttt{chow 1988:1} \\ 
\end{tabular}

	Tem que se seguir a uma regress�o de M�nimos Quadrados (OLS).
        Cria uma vari�vel auxiliar que � igual a 1 a partir do ponto
        especificado por \textsl{obs} at� ao final da amostra,
        caso contr�rio � 0, e cria tamb�m termos de interac��o entre
        esta vari�vel auxiliar e as vari�veis independentes originais.  
        � executada uma regress�o aumentada que inclui estes termos e
        � calculada a estat�stica F, considerando 
        a regress�o aumentada como n�o restringida e a original como
        restringida.  Esta estat�stica � apropriada para testar a
        hip�tese nula de n�o exist�ncia de quebra estrutural no ponto
        de separa��o dado.

Caminho de Menu:    Janela do modelo, /Testes/Teste de Chow

\subsection{coeffsum}
\hypertarget{cmd-coeffsum}{}

\begin{tabular}{ll}
Argumento:   & \textsl{lista-de-vari�veis} \\
Exemplo:     & \verb@coeffsum xt xt_1 xr_2@ \\ 
 & \texttt{restrict.inp}
\end{tabular}

	Tem que se seguir a uma regress�o.  Calcula a soma dos 
        coeficientes nas vari�veis indicadas na \textsl{lista-de-vari�veis}.
        Apresenta esta soma juntamente com o seu erro padr�o e o p-value
        para a hip�tese nula de que a soma � zero.

Note-se a diferen�a entre este teste e \hyperlink{cmd-omit}{omit}, que testa a hip�tese nula de que os coeficientes
        num conjunto especificado de vari�veis independentes s�o
        \emph{todos} iguais a zero.

Caminho de Menu:    Janela do modelo, /Testes/Soma de coeficientes

\subsection{coint}
\hypertarget{cmd-coint}{}

\begin{tabular}{ll}
Argumentos:  & \textsl{ordem} \textsl{vari�vel-dependente} \textsl{vari�veis-independentes} \\
Op��es:      & \verb|--nc| (n�o incluir uma constante) \\
 & \verb|--ct| (incluir constante e tend�ncia) \\
 & \verb|--ctt| (incluir constante e tend�ncia quadr�tica) \\
 & \verb|--skip-df| (n�o efectuar testes DF nas vari�veis individuais) \\
Exemplos:    & \texttt{coint 4 y x1 x2} \\ 
 & \verb@coint 0 y x1 x2 --ct --skip-df@ \\ 
\end{tabular}

	O teste de cointegra��o Engle--Granger.  O procedimento 
        por omiss�o �: (1) efectuar testes de Dickey--Fuller (DF) 
        segundo a hip�tese nula de que cada vari�vel listada tem uma raiz 
        unit�ria; (2) estima a regress�o de cointegra��o; e (3)executar um 
        teste DF sobre os res�duos da regress�o de cointegra��o. Se for 
        dada a op��o \verb@--skip-df@, o passo (1) � omitido.

	Se a ordem de 'lag' especificada � positiva, todos os testes
	Dickey--Fuller usam essa ordem.  Se a ordem for antecedida
        de um sinal menos, ela � encarada como sendo o m�ximo 'lag' e a 
        ordem efectivamente usada em cada caso � obtida testando para
        baixo: ver o comando \hyperlink{cmd-adf}{adf} para detalhes.

	Por omiss�o, a regress�o de cointegra��o cont�m uma constante.  Se
        voc� deseja suprimir a constante, acrescente a op��o \verb@--nc@.
	Se voc� deseja aumentar a lista de termos determin�sticos na regress�o
        de cointegra��o com uma tend�ncia linear ou quadr�tica, use a op��o
        \verb@--ct@ ou \verb@--ctt@.  Estas op��es s�o mutualmente
        exclusivas.

Os \emph{P-}values para este teste s�o baseados em
	MacKinnon (1996).  O c�digo relevante � inclu�do com a
        generosa permiss�o do autor.

Caminho de Menu:    /Modelo/S�rie temporal/Testes de Cointegra��o/Engle-Granger

\subsection{coint2}
\hypertarget{cmd-coint2}{}

\begin{tabular}{ll}
Argumentos:  & \textsl{ordem} \textsl{vari�vel-dependente} \textsl{vari�veis-independentes} \\
Op��es:      & \verb|--nc| (sem constante) \\
 & \verb|--rc| (constante restringida) \\
 & \verb|--crt| (constante e tend�ncia restringida) \\
 & \verb|--ct| (constante e tend�ncia n�o restringida) \\
 & \verb|--seasonals| (incluir auxiliares sazonais centradas) \\
 & \verb|--quiet| (apenas mostrar os testes) \\
 & \verb|--verbose| (mostrar detalhes das regress�es auxiliares) \\
Exemplos:    & \texttt{coint2 2 y x} \\ 
 & \verb@coint2 4 y x1 x2 --verbose@ \\ 
 & \verb@coint2 3 y x1 x2 --rc@ \\ 
\end{tabular}

	Executa o teste de Johansen para a cointegra��o entre as vari�veis
        listadas para a dada ordem de 'lag'.  Os valores cr�ticos s�o
        determinados usando a aproxima��o gamma de J. Doornik (Doornik, 1998).
        Para detalhes sobre este teste ver o Cap�tulo 20 do livro de 
        Hamilton, \emph{Time Series Analysis} (1994).

	A inclus�o de termos determin�sticos no modelo � controlada por
       interm�dio das op��es.  Por omiss�o, se n�o tiver sido indicada nenhuma
       op��o, ser� inclu�da uma ``constante n�o restringida'', o
       que permite a presen�a de um interceptor n�o-nulo nas rela��es
       cointegrantes assim como uma tend�ncia nos n�veis das vari�veis
       end�genas.  Na literatura derivada do trabalho de Johansen (ver por
       exemplo o livro dele de 1995) isto � frequentemente referido como sendo
       o ``caso 3''.  As primeiras quatro op��es apresentadas
       acima, que s�o mutualmente exclusivas, produzem respectivamente os
       casos 1, 2, 4, e 5.  O significado destes casos e os crit�rios para
       seleccionar um caso est�o explicados no 
       \GUG{}.

	A op��o \verb@--seasonals@, que pode ser combinada com qualquer
        outra op��o, especifica a inclus�o de um conjunto de vari�veis
        auxiliares sazonais.  Esta op��o apenas est� dispon�vel para dados
        trimestrais ou mensais.

	A seguinte tabela serve como uma guia � interpreta��o dos resultados
        apresentados pelo teste, num caso de 3-vari�veis.  \texttt{H0}
        significa a hip�tese nula, \texttt{H1} a hip�tese alternativa, e
        \texttt{c} o n�mero de rela��es cointegrantes.

\begin{code}
                 Ordem    Teste Tra�o        Teste Lmax
                          H0     H1          H0     H1
                 ---------------------------------------
                  0      c = 0  c = 3       c = 0  c = 1
                  1      c = 1  c = 3       c = 1  c = 2
                  2      c = 2  c = 3       c = 2  c = 3
                 ---------------------------------------
\end{code}

	Ver tamb�m o comando \hyperlink{cmd-vecm}{vecm}.

Caminho de Menu:    /Modelo/S�rie temporal/Testes de Cointegra��o/Johansen

\subsection{corr}
\hypertarget{cmd-corr}{}

\begin{tabular}{ll}
Argumento:   & \texttt{[} \textsl{lista-de-vari�veis} \texttt{]} \\
Exemplo:     & \texttt{corr y x1 x2 x3} \\ 
\end{tabular}

	Apresenta os coeficientes de correla��o emparelhados das vari�veis em
        \textsl{lista-de-vari�veis}, ou de todas as vari�veis no conjunto
        de dados se n�o for dada a \textsl{lista-de-vari�veis}.

Caminho de Menu:    /Ver/Matriz de correla��o

Acesso alternativo: Menu de contexto da janela principal (selec��o m�ltipla)

\subsection{corrgm}
\hypertarget{cmd-corrgm}{}

\begin{tabular}{ll}
Argumentos:  & \textsl{vari�vel} \texttt{[} \textsl{maxlag} \texttt{]} \\
Exemplo:     & \texttt{corrgm x 12} \\ 
\end{tabular}

	Apresenta os valores da fun��o de autocorrela��o para a
	\textsl{vari�vel}, que pode ser especificada por nome ou por
        n�mero.  Os valores s�o definidos pela equa��o
	$\hat{\rho}(u_t, u_{t-s})$ onde 
	u\ensuremath{_{t}} � a t--�sima
	observa��o da vari�vel u e s
        � o n�mero de "lags".

	Tamb�m s�o apresentadas as autocorrela��es parciais (obtidas segundo
        o algoritmo de Durbin--Levinson): estas constituem a rede dos
        efeitos dos "lags" intervenientes.  O comando tamb�m produz o gr�fico
        correlograma e apresenta a estat�stica de teste 
        Q de Box--Pierce, para a hip�tese nula de que
        a s�rie temporal � ``ru�do branco'': ter� uma distribui��o
        qui-quadrado assimpt�tico com os graus de liberdade iguais ao n�mero de
        "lags" usados.

	Se o valor \textsl{maxlag} for especificado o comprimento do
        correlograma fica limitado a esse m�ximo n�mero de "lags", sen�o o 
        comprimento � determinado autom�ticamente, como uma fun��o da
        frequ�ncia dos dados e do n�mero de observa��es.

Caminho de Menu:    /Vari�vel/Correlograma

Acesso alternativo: Menu de contexto da janela principal (selec��o singular)

\subsection{criteria}
\hypertarget{cmd-criteria}{}

\begin{tabular}{ll}
Argumentos:  & \textsl{ess} \textsl{T} \textsl{k} \\
Exemplo:     & \texttt{criteria 23.45 45 8} \\ 
\end{tabular}

	Determina o Crit�rio de Informa��o de Akaike (AIC) e o Crit�rio
        de Informa��o Bayesiano de Schwarz (BIC), dados \textsl{ess}
	(erro da soma de quadrados), o n�mero de observa��es
	(T), e o n�mero de coeficientes 
	(k). T,
	k, e \textsl{ess} podem ser valores 
        num�ricos ou nomes de vari�veis pr�viamente definidas.

	O AIC � obtido segundo a formula��o original de Akaike (1974),
	nomeadamente
	\[{\rm AIC} = -2 \ell + 2k\]
	onde $\ell$ designa a verosimilhan�a logaritmica maximizada.
	O BIC � calculado por 
	\[{\rm BIC} = -2 \ell + k \log T\]
	Por favor consulte \GUG{} para mais
        pormenores.

\subsection{cusum}
\hypertarget{cmd-cusum}{}

\begin{tabular}{ll}
Op��o:       & \verb|--squares| (executa o teste CUSUMSQ) \\
\end{tabular}

	Tem que se seguir � estima��o de um modelo por via de OLS.  Executa 
        o teste CUSUM ---ou se for dada a op��o \verb@--squares@,
        o teste CUSUMSQ ---para a estabilidade dos par�metros.  � obtida
        uma s�rie temporal de erros de predi��o um passo-�-frente, pela 
        execu��o de s�ries de regress�es: a primeira regress�o usa as 
        primeiras k observa��es e � usada para gerar a
        predi��o da vari�vel dependente na observa��o
        k + 1; a segunda usa a primeira predi��o para a
        observa��o k + 2, e por a� a diante 
	(onde k � o n�mero de par�metros no modelo original).

	A soma acumulada dos erros de predi��o escalados, ou os quadrados desses
        erros, � mostrada e apresentada em gr�fico.  A hip�tese nula para a
        estabilidade dos par�metros � rejeitada ao n�vel de cinco porcento, 
        se a soma acumulada se desviar do intervalo de confian�a de 95 porcento.

	No caso do teste  CUSUM, � tamb�m apresentada a estat�stica de teste
        t de Harvey--Collier, para a hip�tese nula
        da estabilidade dos par�metros.  Ver o livro 
        \emph{Econometric Analysis} de Greene para mais detalhes.  Para
        o teste CUSUMSQ, o intervalo de confian�a a 95 porcento � calculado de
        acordo com o algoritmo apresentado por Edgerton e Wells (1994).

Caminho de Menu:    Janela do modelo, /Testes/Teste CUSUM(SQ)

\subsection{data}
\hypertarget{cmd-data}{}

\begin{tabular}{ll}
Argumento:   & \textsl{lista-de-vari�veis} \\
\end{tabular}

	Reads the variables in \textsl{lista-de-vari�veis} from a database
	(gretl, RATS 4.0 or PcGive), which must have been opened
	previously using the \hyperlink{cmd-open}{open} command.  The
	data frequency and sample range may be established via the
	\hyperlink{cmd-setobs}{setobs} and \hyperlink{cmd-smpl}{smpl} commands prior
	to using this command. Here is a full example:

\begin{code}
	open macrodat.rat
	setobs 4 1959:1
	smpl ; 1999:4
	data GDP_JP GDP_UK
\end{code}

	The commands above open a database named
	\texttt{macrodat.rat}, establish a quarterly data set
	starting in the first quarter of 1959 and ending in the fourth
	quarter of 1999, and then import the series named
	\verb@GDP_JP@ and \verb@GDP_UK@.

	If \texttt{setobs} and \texttt{smpl} are not specified in this
	way, the data frequency and sample range are set using the first
	variable read from the database.

	If the series to be read are of higher frequency than the working
	data set, you may specify a compaction method as below:

\begin{code}
	data (compact=average) LHUR PUNEW
      \end{code}

	The four available compaction methods are ``average''
	(takes the mean of the high frequency observations),
	``last'' (uses the last observation),
	``first'' and ``sum''.  If no method is
	specified, the default is to use the average.

Caminho de Menu:    /Ficheiro/Databases

\subsection{dataset}
\hypertarget{cmd-dataset}{}

\begin{tabular}{ll}
Argumentos:  & \textsl{keyword} \textsl{parameters} \\
Exemplos:    & \texttt{dataset addobs 24} \\ 
 & \texttt{dataset compact 1} \\ 
 & \texttt{dataset compact 4 last} \\ 
 & \texttt{dataset expand 12} \\ 
 & \texttt{dataset transpose} \\ 
 & \texttt{dataset sortby x1} \\ 
\end{tabular}

	Performs various operations on the data set as a whole, depending
	on the given \textsl{keyword}, which must be
	\texttt{addobs}, \texttt{compact}, \texttt{expand},
	\texttt{transpose}, \texttt{sortby} or \texttt{dsortby}.
	Note: these actions are not available when the dataset is
	currently subsampled by selection of cases on some Boolean
	criterion.

	\texttt{addobs}: Must be followed by a positive integer.  Adds
	the specified number of extra observations to the end of the
	working dataset.  This is primarily intended for forecasting
	purposes.  The values of most variables over the additional range
	will be set to missing, but certain deterministic variables are
	recognized and extended, namely, a simple linear trend and
	periodic dummy variables. 

	\texttt{compact}: Must be followed by a positive integer
	representing a new data frequency, which should be lower than the
	current frequency (for example, a value of 4 when the current
	frequency is 12 indicates compaction from monthly to quarterly).
	This command is available for time series data only; it compacts
	all the series in the data set to the new frequency.  A second
	parameter may be given, namely one of \texttt{sum},
	\texttt{first} or \texttt{last}, to specify, respectively,
	compaction using the sum of the higher-frequency values,
	start-of-period values or end-of-period values.  The default
	is to compact by averaging.

	\texttt{expand}: Must be followed by a positive integer
	representing a new data frequency, which should be higher than the
	current frequency.  This command is only available for annual or
	quarterly time series data.  Annual data can be expanded to
	quarterly or monthly; quarterly data can be expanded to monthly.
	All the series in the data set are padded out to the new
	frequency by repeating the existing values.

	\texttt{transpose}: No additional parameter required.
	Transposes the current data set.  That is, each observation (row)
	in the current data set will be treated as a variable (column),
	and each variable as an observation.  This command may be useful
	if data have been read from some external source in which the rows
	of the data table represent variables.

	\texttt{sortby}: One variable name is required; this variable is
	used as a sort key.  The observations on all variables in the
	dataset are re-ordered by increasing value of the key variable.
	This command is available only for undated data.

	\texttt{dsortby}: Works as \texttt{sortby} except that
	the re-ordering is by decreasing value of the key variable.

Caminho de Menu:    /Dados/Acrescentar observa��es

Caminho de Menu:    /Dados/Compactar datos

Caminho de Menu:    /Dados/Expandir datos

Caminho de Menu:    /Dados/Transp�r

Caminho de Menu:    /Dados/Ordenar

\subsection{delete}
\hypertarget{cmd-delete}{}

\begin{tabular}{ll}
Argumento:   & \texttt{[} \textsl{lista-de-vari�veis} \texttt{]} \\
\end{tabular}

Removes the listed variables (given by name or number)
	from the dataset. \emph{Use with caution}: no
	confirmation is asked, and any variables with higher ID
	numbers will be re-numbered.

If no \textsl{lista-de-vari�veis} is given with this
        command, it deletes the last (highest numbered) variable from the
	dataset.

Caminho de Menu:    Main window pop-up (single selection)

\subsection{diff}
\hypertarget{cmd-diff}{}

\begin{tabular}{ll}
Argumento:   & \textsl{lista-de-vari�veis} \\
\end{tabular}

	The first difference of each variable in \textsl{lista-de-vari�veis}
	is obtained and the result stored in a new variable with the
	prefix \verb@d_@.  Thus diff x y creates the
	new variables

\begin{code}
	d_x = x(t) - x(t-1)
	d_y = y(t) - y(t-1)
\end{code}
Caminho de Menu:    /Add/First differences of selected variables

\subsection{difftest}
\hypertarget{cmd-difftest}{}

\begin{tabular}{ll}
Argumentos:  & \textsl{var1} \textsl{var2} \\
Op��es:      & \verb|--sign| (Sign test, the default) \\
 & \verb|--rank-sum| (Wilcoxon rank-sum test) \\
 & \verb|--signed-rank| (Wilcoxon signed-rank test) \\
 & \verb|--verbose| (print extra output) \\
\end{tabular}

	Carries out a nonparametric test for a difference between two
	populations or groups, the specific test depending on the option
	selected.

	With the \verb@--sign@ option, the Sign test is performed.
	This test is based on the fact that if two samples,
	x and y, are drawn randomly
	from the same distribution, the probability that
	x\ensuremath{_{i}} >
	y\ensuremath{_{i}}, for each observation
	i, should equal 0.5.  The test statistic is
	w, the number of observations for which
	x\ensuremath{_{i}} >
	y\ensuremath{_{i}}. Under the null hypothesis this
	follows the Binomial distribution with parameters
	(n, 0.5), where n is the
	number of observations.

	With the \verb@--rank-sum@ option, the Wilcoxon rank-sum test
	is performed.  This test proceeds by ranking the observations from
	both samples jointly, from smallest to largest, then finding the
	sum of the ranks of the observations from one of the samples.  The
	two samples do not have to be of the same size, and if they differ
	the smaller sample is used in calculating the rank-sum.  Under the
	null hypothesis that the samples are drawn from populations with
	the same median, the probability distribution of the rank-sum can
	be computed for any given sample sizes; and for reasonably large
	samples a close Normal approximation exists.

	With the \verb@--signed-rank@ option, the Wilcoxon signed-rank
	test is performed.  This is designed for matched data pairs such
	as, for example, the values of a variable for a sample of
	individuals before and after some treatment.  The test proceeds by
	finding the differences between the paired observations,
	x\ensuremath{_{i}} $-$
	y\ensuremath{_{i}}, ranking these differences by
	absolute value, then assigning to each pair a signed rank, the
	sign agreeing with the sign of the difference.  One then
	calculates W\ensuremath{_{+}}, the sum of the
	positive signed ranks.  As with the rank-sum test, this statistic
	has a well-defined distribution under the null that the median
	difference is zero, which converges to the Normal for samples
	of reasonable size.

	For the Wilcoxon tests, if the \verb@--verbose@ option is
	given then the ranking is printed.  (This option has no effect if
	the Sign test is selected.)

\subsection{discrete}
\hypertarget{cmd-discrete}{}

\begin{tabular}{ll}
Argumento:   & \textsl{lista-de-vari�veis} \\
Op��o:       & \verb|--reverse| (mark variables as continuous) \\
\end{tabular}

	Marks each variable in \textsl{lista-de-vari�veis} as being discrete. By
	default all variables are treated as continuous; marking a
	variable as discrete affects the way the variable is handled in
	frequency plots, and also allows you to select the variable for
	the command \hyperlink{cmd-dummify}{dummify}.

	If the \verb@--reverse@ flag is given, the operation is reversed;
	that is, the variables in \textsl{lista-de-vari�veis} are marked as
	being continuous.

Caminho de Menu:    /Vari�vel/Edit attributes

\subsection{dummify}
\hypertarget{cmd-dummify}{}

\begin{tabular}{ll}
Argumento:   & \textsl{lista-de-vari�veis} \\
Op��es:      & \verb|--drop-first| (omit lowest value from encoding) \\
 & \verb|--drop-last| (omit highest value from encoding) \\
\end{tabular}

	For any suitable variables in \textsl{lista-de-vari�veis}, creates a set
	of dummy variables coding for the distinct values of that
	variable.   Suitable variables are those that have been explicitly
	marked as discrete, or those that take on a fairly small number of
	values all of which are ``fairly round'' (multiples of
	0.25).

	By default a dummy variable is added for each distinct value of
	the variable in question.  For example if a discrete variable
	\texttt{x} has 5 distinct values, 5 dummy variables will be
	added to the data set, with names \verb@Dx_1@, \verb@Dx_2@
	and so on.  The first dummy variable will have value 1 for
	observations where \texttt{x} takes on its smallest value, 0
	otherwise; the next dummy will have value 1 when \texttt{x}
	takes on its second-smallest value, and so on.  If one of the
	option flags \verb@--drop-first@ or \verb@--drop-last@
	is added, then either the lowest or the highest value of each
	variable is omitted from the encoding (which may be useful for
	avoiding the ``dummy variable trap'').

	This command can also be embedded in the context of a regression
	specification.  For example, the following line specifies a model
	where \texttt{y} is regressed on the set of dummy variables
	coding for \texttt{x}.  (Option flags cannot be passed to
	dummify in this context.)

\begin{code}
	ols y dummify(x)
      \end{code}

\subsection{elif}
\hypertarget{cmd-elif}{}
See \hyperlink{cmd-if}{if}.

\subsection{else}
\hypertarget{cmd-else}{}
See \hyperlink{cmd-if}{if}.

\subsection{end}
\hypertarget{cmd-end}{}

	Ends a block of commands of some sort.  For example, end
	  system terminates an equation \hyperlink{cmd-system}{system}.

\subsection{endif}
\hypertarget{cmd-endif}{}
See \hyperlink{cmd-if}{if}.

\subsection{endloop}
\hypertarget{cmd-endloop}{}

	Marks the end of a command loop.  See \hyperlink{cmd-loop}{loop}.

\subsection{eqnprint}
\hypertarget{cmd-eqnprint}{}

\begin{tabular}{ll}
Argumento:   & \texttt{[} \textsl{-f filename} \texttt{]} \\
Op��o:       & \verb|--complete| (Create a complete document) \\
\end{tabular}

	Must follow the estimation of a model.  Prints the estimated
	model in the form of a {\LaTeX} equation.  If a filename is
	specified using the \texttt{-f} flag output goes to that
	file, otherwise it goes to a file with a name of the form
	\verb@equation_N.tex@, where \texttt{N} is the
	number of models estimated to date in the current session.
	See also \hyperlink{cmd-tabprint}{tabprint}.

	If the \verb@--complete@ flag is given, the {\LaTeX} file is
	a complete document, ready for processing; otherwise it must
	be included in a document.

Caminho de Menu:    Janela do modelo, /LaTeX

\subsection{equation}
\hypertarget{cmd-equation}{}

\begin{tabular}{ll}
Argumentos:  & \textsl{vari�vel-dependente} \textsl{vari�veis-independentes} \\
Exemplo:     & \texttt{equation y x1 x2 x3 const} \\ 
\end{tabular}

	Specifies an equation within a system of equations (see 
	\hyperlink{cmd-system}{system}).  The syntax for specifying an
	equation within an SUR system is the same as that for, e.g.{}, 
	\hyperlink{cmd-ols}{ols}.  For an equation within a Three-Stage Least
	Squares system you may either (a) give an OLS-type equation
	specification and provide a common list of instruments using the
	instr keyword (again, see \hyperlink{cmd-system}{system}),
	or (b) use the same equation syntax as for \hyperlink{cmd-tsls}{tsls}.

\subsection{estimate}
\hypertarget{cmd-estimate}{}

\begin{tabular}{ll}
Argumentos:  & \textsl{systemname} \textsl{estimator} \\
Op��es:      & \verb|--iterate| (iterate to convergence) \\
 & \verb|--no-df-corr| (no degrees of freedom correction) \\
 & \verb|--geomean| (see below) \\
Exemplos:    & \texttt{estimate "Klein Model 1" method=fiml} \\ 
 & \texttt{estimate Sys1 method=sur} \\ 
 & \verb@estimate Sys1 method=sur --iterate@ \\ 
\end{tabular}

	Calls for estimation of a system of equations, which must have
	been previously defined using the \hyperlink{cmd-system}{system}
	command.  The name of the system should be given first,
	surrounded by double quotes if the name contains spaces.  The
	estimator, which must be one of ols,
	tsls, sur, 3sls,
	fiml or liml, is preceded by the string
	\texttt{method=}.

If the system in question has had a set of restrictions
	applied (see the \hyperlink{cmd-restrict}{restrict} command),
	estimation will be subject to the specified restrictions.

	If the estimation method is sur or 3sls
	and the \verb@--iterate@ flag is given, the estimator will
	be iterated.  In the case of SUR, if the procedure converges
	the results are maximum likelihood estimates.  Iteration of
	three-stage least squares, however, does not in general
	converge on the full-information maximum likelihood results.
	The \verb@--iterate@ flag is ignored for other methods of
	estimation.  

If the equation-by-equation estimators ols or
	tsls are chosen, the default is to apply a degrees
	of freedom correction when calculating standard errors. This
	can be suppressed using the \verb@--no-df-corr@ flag. This
	flag has no effect with the other estimators; no degrees of
	freedom correction is applied in any case.

By default, the formula used in calculating the
	elements of the cross-equation covariance matrix is
	\[\hat{\sigma}_{i,j}=\frac{\hat{u}_i' \hat{u}_j}{T}\]
	If the \verb@--geomean@ flag is
	given, a degrees of freedom correction is applied: the
	formula is
	\[\hat{\sigma}_{i,j}=\frac{\hat{u}_i' \hat{u}_j}{\sqrt{(T-k_i)(T-k_j)}}\]
	where the ks denote the number of
	independent parameters in each equation.

\subsection{fcast}
\hypertarget{cmd-fcast}{}

\begin{tabular}{ll}
Argumentos:  & \texttt{[} \textsl{startobs endobs} \texttt{]} \textsl{fitvar} \\
Op��es:      & \verb|--dynamic| (create dynamic forecast) \\
 & \verb|--static| (create static forecast) \\
Exemplos:    & \texttt{fcast 1997:1 2001:4 f1} \\ 
 & \texttt{fcast fit2} \\ 
\end{tabular}

	Must follow an estimation command.  Forecasts are generated for the
	specified range (or the largest possible range if no
	\textsl{startobs} and \textsl{endobs} are given) and the values
	saved as \textsl{fitvar}, which can be printed, graphed, or plotted.
	The right-hand side variables are those in the original model.  There is
	no provision to substitute other variables.  If an autoregressive error
	process is specified the forecast incorporates the predictable fraction
	of the error process.

	The choice between a static and a dynamic forecast applies
	only in the case of dynamic models, with an autoregressive
	error process or including one or more lagged values of the
	dependent variable as regressors.  See \hyperlink{cmd-fcasterr}{fcasterr} for more details.

Caminho de Menu:    Janela do modelo, /Analysis/Forecasts

\subsection{foreign}
\hypertarget{cmd-foreign}{}

\begin{tabular}{ll}
Argumento:   & \texttt{[} \textsl{language-spec} \texttt{]} \\
Op��es:      & \verb|--send-data| (pre-load the current dataset) \\
 & \verb|--quiet| (suppress output from foreign program) \\
\end{tabular}

	This command opens a special mode in which commands to be executed by
	another program are accepted.  You exit this mode with \texttt{end	  foreign}; at this point the stacked commands are executed.

	At present the only ``foreign'' program supported in this
	way is GNU R, and the only acceptable \textsl{language-spec} is
	\texttt{language=R} (which is implicit, and may be omitted).  In
	future other programs may be supported.

	See \GUG{} for details and examples.

\subsection{freq}
\hypertarget{cmd-freq}{}

\begin{tabular}{ll}
Argumento:   & \textsl{var} \\
Op��es:      & \verb|--quiet| (suppress printing of histogram) \\
 & \verb|--gamma| (test for gamma distribution) \\
\end{tabular}

	With no options given, displays the frequency distribution for
	\textsl{var} (given by name or number) and shows the
	results of the Doornik--Hansen chi-square test for
	normality.

	If the \verb@--quiet@ option is given, the histogram is
	not shown.  If the \verb@--gamma@ option is given, the
	test for normality is replaced by Locke's nonparametric test
	for the null hypothesis that the variable follows the gamma
	distribution; see Locke (1976), Shapiro and Chen (2001).

	In interactive mode a graph of the distribution is displayed.

Caminho de Menu:    /Vari�vel/Frequency distribution

\subsection{function}
\hypertarget{cmd-function}{}

\begin{tabular}{ll}
Argumento:   & \textsl{fnname} \\
\end{tabular}

	Opens a block of statements in which a function is defined.  This
	block must be closed with \texttt{end function}.  Please see
	\GUG{} for details.

\subsection{garch}
\hypertarget{cmd-garch}{}

\begin{tabular}{ll}
Argumentos:  & \textsl{p} \textsl{q} \texttt{;} \textsl{vari�vel-dependente} \texttt{[} \textsl{vari�veis-independentes} \texttt{]} \\
Op��es:      & \verb|--robust| (robust standard errors) \\
 & \verb|--verbose| (mostrar detalhes das itera��es) \\
 & \verb|--vcv| (mostrar matriz de covari�ncia) \\
 & \verb|--arma-init| (initial variance parameters from ARMA) \\
Exemplos:    & \texttt{garch 1 1 ; y} \\ 
 & \verb@garch 1 1 ; y 0 x1 x2 --robust@ \\ 
\end{tabular}

	Estimates a GARCH model (GARCH = Generalized Autoregressive
	Conditional Heteroskedasticity), either a univariate model or,
	if \textsl{vari�veis-independentes} are specified, including the given
	exogenous variables.  The integer values \textsl{p} and
	\textsl{q} (which may be given in numerical form or as the
	names of pre-existing scalar variables) represent the lag
	orders in the conditional variance equation:
	\[h_t = \alpha_0 + \sum_{i=1}^q \alpha_i \varepsilon^2_{t-i} + \sum_{j=1}^p \beta_i h_{t-j}\]

The gretl GARCH algorithm is basically that of Fiorentini,
	Calzolari and Panattoni (1996), used by kind permission of
	Professor Fiorentini.

Several variant estimates of the coefficient
	covariance matrix are available with this command.  By
	default, the Hessian is used unless the \verb@--robust@
	option is given, in which case the QML (White) covariance
	matrix is used.  Other possibilities (e.g.{} the information
	matrix, or the Bollerslev--Wooldridge estimator) can be
	specified using the \hyperlink{cmd-set}{set} command.

	By default, the estimates of the variance parameters are
	initialized using the unconditional error variance from
	initial OLS estimation for the constant, and small positive
	values for the coefficients on the past values of the squared
	error and the error variance.  The flag \verb@--arma-init@
	calls for the starting values of these parameters to be
	set using an initial ARMA model, exploiting the relationship
	between GARCH and ARMA set out in Chapter 21 of Hamilton's
	\emph{Time Series Analysis}.  In some cases this may 
	improve the chances of convergence.

	The GARCH residuals and estimated conditional variance can be
	retrieved as \verb@$uhat@ and \verb@$h@ respectively.  For
	example, to get the conditional variance:  

\begin{code}
	genr ht = $h
      \end{code}
Caminho de Menu:    /Modelo/S�rie temporal/GARCH

\subsection{genr}
\hypertarget{cmd-genr}{}

\begin{tabular}{ll}
Argumentos:  & \textsl{newvar} \textsl{= formula} \\
\end{tabular}

	Creates new variables, usually through transformations of existing
	variables. See also \hyperlink{cmd-diff}{diff}, \hyperlink{cmd-logs}{logs},
	\hyperlink{cmd-lags}{lags}, \hyperlink{cmd-ldiff}{ldiff}, \hyperlink{cmd-multiply}{multiply}, \hyperlink{cmd-sdiff}{sdiff} and \hyperlink{cmd-square}{square}
	for shortcuts. In the context of a \texttt{genr} formula, existing
	variables must be referenced by name, not ID number.  The formula should
	be a well-formed combination of variable names, constants, operators and
	functions (described below).  Note that further details on some aspects
	of this command can be found in \GUG{}.

	This command may yield either a series or a scalar result.  For example,
	the formula \texttt{x2 = x * 2} naturally yields a series if the
	variable \texttt{x} is a series and a scalar if \texttt{x} is a
	scalar.  The formulae \texttt{x = 0} and \texttt{mx = mean(x)}
	naturally return scalars. Under some circumstances you may want to have
	a scalar result expanded into a series or vector.  You can do this by
	using \texttt{series} as an ``alias'' for the
	\texttt{genr} command.  For example, \texttt{series x = 0} produces
	a series all of whose values are set to 0.  You can also use
	\texttt{scalar} as an alias for \texttt{genr}.  It is not possible
	to coerce a vector result into a scalar, but use of this keyword 
	indicates that the result \emph{should be} a scalar: if it
	is not, an error occurs.

	When a formula yields a series or vector result, the range over which
	the result is written to the target variable depends on the current
	sample setting.  It is possible, therefore, to define a series piecewise
	using the \texttt{smpl} command in conjunction with \texttt{genr}.

	Supported \emph{arithmetical operators} are, in
	order of precedence: \verb@^@ (exponentiation);
	\texttt{*}, \texttt{/} and \verb@%@ (modulus or
	remainder); \texttt{+} and \texttt{-}. 

	The available \emph{Boolean operators} are (again,
	in order of precedence): \texttt{!} (negation),
	\verb@&&@ (logical AND), \texttt{||} (logical OR),
	\texttt{>}, \texttt{<}, \texttt{=}, \texttt{>=}
	(greater than or equal), \texttt{<=} (less than or equal) and
	\texttt{!=} (not equal).  The Boolean operators can be used in
	constructing dummy variables: for instance \texttt{(x > 10)}
	returns 1 if \texttt{x} > 10, 0 otherwise.

Built-in constants are \texttt{pi} and \texttt{NA}.  The latter
	is the missing value code: you can initialize a variable to the missing
	value with \texttt{scalar x = NA}.

Supported \emph{functions} fall into these
	groups:

\begin{itemize}
\item Standard mathematical functions: abs,
	    exp, int (integer part),
	    ln (natural logarithm: log is a
	    synonym), log10 (log to the base 10),
	    log2 (log to the base 2), sqrt.  All
	    of these take a single argument, which may be either a series
	    or a scalar.

\item Trigonometric functions: cos,
	    sin, tan and atan (arc
	    tangent). These functions also take a single series or scalar
	    argument.

\item 
	    Standard statistical functions taking a single argument and yielding
	    a scalar result: max (maximum value in a series),
	    min (minimum value in series), mean
	    (arithmetic mean), median, var (variance),
	    sd (standard deviation), sst (sum of
	    squared deviations from the mean), sum,
	    gini (Gini coefficient).

\item 
	    Statistical functions taking one series as argument and
	    yielding a series or vector result: sort (sort a
	    series in ascending order of magnitude), dsort
	    (sort in descending order), cum (cumulate, or
	    running sum).  For panel data only, the functions
	    pmean and psd take a single series
	    as argument and return series containing, respectively, the
	    means and standard deviations of the variable for each of the
	    cross-sectional units.  See \GUG{} for
	    details.

\item 
	    Statistical functions taking two series as arguments and yielding a
	    scalar result: cov (covariance), corr
	    (correlation coefficient).

\item Special statistical functions: pvalue,
	    cdf and critical (see below),
	    cnorm (standard normal CDF), dnorm
	    (standard normal PDF), qnorm (standard normal
	    quantiles, or inverse of standard normal PDF),
	    resample (resample a series with replacement, for
	    bootstrap purposes), hpfilt
	    (Hodrick--Prescott filter: this function returns the
	    ``cycle'' component of the series),
	    bkfilt (Baxter--King bandpass filter).

\item 
	    Time-series functions: diff (first difference),
	    ldiff (log-difference, or first difference of natural
	    logs), sdiff (seasonal difference),
	    fracdiff (fractional difference).  To generate lags of
	    a variable \texttt{x}, use the syntax x(-N), where
	    \texttt{N} represents the desired lag length; to generate leads,
	    use x(+N).

\item 
	    Dataset functions yielding a series: misszero (replaces
	    the missing observation code in a given series with zeros);
	    zeromiss (the inverse operation to
	    misszero); missing (at each observation, 1
	    if the argument has a missing value, 0 otherwise); ok
	    (the opposite of missing).

\item 
	    Dataset functions yielding a scalar: nobs (gives the
	    number of valid observations in a data series),
	    firstobs (gives the 1-based observation number of the
	    first non-missing value in a series), lastobs
	    (observation number of the last non-missing observation in a
	    series).

\item 
	    Pseudo-random numbers: uniform,
	    normal, chisq, student,
	    binomial and poisson. The first two
	    functions do not take an argument and should be written with
	    empty parentheses: \texttt{uniform()}, \texttt{normal()}.
	    They create pseudo-random series drawn from the uniform
	    (0--1) and standard normal distributions respectively.
	    The next two take a single argument, namely the (integer)
	    degrees of freedom; they generate drawings from the Chi-square
	    and Student's t distributions respectively.  The
	    binomial function takes two parameters: an
	    integer number of trials and a ``success''
	    probability.  The poisson function takes a single
	    argument, the mean, which may be either a scalar or a series.
	    Uniform series, which form the basis for the more complex
	    distributions, are generated using the Mersenne
	    Twister;See Matsumoto and Nishimura (1998).
		The implementation is provided by glib,
		if available, or by the C code written by Nishimura and
		Matsumoto.

 for normal series the method
	    of Box and Muller (1958) is used, taking input from the
	    Mersenne Twister. Ver tamb�m o comando \hyperlink{cmd-set}{set} command, \texttt{seed} option.

\end{itemize}

	With a few exceptions as noted, all the above functions take as
	their single argument either the name of a variable or an
	expression that evaluates to a variable (e.g.{}
	ln((x1+x2)/2)).

	The pvalue, critical and
	cdf functions take, as their first parameter, a
	one-character code for the distribution of interest:

\begin{itemize}
\item z or N: Normal;

\item t: Student's \emph{t};

\item X: chi-square;

\item F: Snedecor;

\item G: Gamma (only pvalue and cdf);

\item B: Binomial (only pvalue and cdf);

\item D: Bivariate normal (only cdf).

\end{itemize}

	The pvalue function takes the same arguments as the
	\hyperlink{cmd-pvalue}{pvalue} command, but in this context commas should
	be placed between the arguments. It returns a one-tailed,
	right-hand p-value, that is, the integral of the given density
	function from the specified value to plus infinity. 

        The cdf function also takes the same arguments as
	the \hyperlink{cmd-pvalue}{pvalue} command, but it returns the
	complementary value, that is, the integral of the relevant
	density function from its lower limit (either minus infinity
	or zero) to the specified value.  For the bivariate normal
	distribution, three arguments must be given:
	\emph{x}, \emph{y} and the
	correlation coefficient \emph{rho}, in this order.
	For additional discussion of these functions see
	\GUG{}.

	The critical function returns the critical value
	for a specified probability distribution and a specified
	proportion in the right-hand tail (with specified degrees of
	freedom where applicable). 
	The last parameter is the right-hand tail
	proportion.  If the first parameter is t or
	X, a second parameter must give the degrees of
	freedom.  For the \emph{F} distribution, the
	second and third parameters must give the numerator and
	denominator degrees of freedom.  

	Here are some examples of use of the pvalue and
	critical functions (spaces between the arguments
	are optional):

\begin{code}
	genr p1 = pvalue(z, 2.2)
	genr p2 = pvalue(X, 3, 5.67)
	genr p3 = cdf(D, -1, 1, 0.25)
	genr c1 = critical(t, 20, 0.025)
	genr c2 = critical(F, 4, 48, 0.05)
      \end{code}

	The functions relating to the standard normal density work thus,
	for a given argument x: cnorm
	returns the area under the standard normal density function
	integrated from minus infinity to x;
	dnorm returns the standard normal density evaluated
	at x; and qnorm returns the
	z such that the area under the standard normal
	density, integrated from minus infinity to z,
	equals x.

	The fracdiff function takes two arguments: the
	name of the series and a fraction, in the range $-$1 to 1.

Besides the operators and functions noted above there are
	some special uses of genr:

\begin{itemize}
\item 
	    genr time creates a time trend variable (1,2,3,\dots{})
	    called time. genr index does the same thing
	    except that the variable is called \texttt{index}.

\item 
	    genr dummy creates dummy variables up to the periodicity
	    of the data.  For example, in the case of quarterly data
	    (periodicity 4), the program creates \verb@dummy_1@ = 1 for
	    first quarter and 0 in other quarters, \verb@dummy_2@ = 1 for
	    the second quarter and 0 in other quarters, and so on.

\item 
	    genr unitdum and genr timedum create 
	    sets of special dummy variables for use with panel data.
	    The first codes for the cross-sectional units and the second
	    for the time period of the observations.

\end{itemize}

	Various internal variables defined in the course of running a
	regression can be retrieved using genr, as follows:

\begin{itemize}
\item 
	   \verb@$ess@: error sum of squares

\item 
	   \verb@$rsq@: unadjusted \emph{R}-squared

\item 
	   \verb@$T@: number of observations used

\item 
	   \verb@$df@: degrees of freedom

\item 
	   \verb@$ncoeff@: total number of estimated coefficients

\item 
	    \verb@$trsq@: \emph{TR}-squared (sample size times
	    \emph{R}-squared)

\item 
	   \verb@$sigma@: standard error of residuals

\item 
	   \verb@$aic@: Akaike Information Criterion

\item 
	   \verb@$bic@: Schwarz's Bayesian Information Criterion

\item 
	   \verb@$hqc@: Hannan--Quinn Criterion

\item 
	   \verb@$lnl@: log-likelihood (where applicable)

\item 
	    \verb@$coeff@(\textsl{var}): estimated coefficient for
	    variable \textsl{var}

\item 
	    \verb@$stderr@(\textsl{var}): estimated standard error for
	    variable \textsl{var}

\item 
	    \verb@$rho@(\textsl{i}): \textsl{i}th order
	    autoregressive coefficient for residuals

\item 
	    \verb@$vcv@(\textsl{x1},\textsl{x2}): estimated
	    covariance between coefficients for the variables \textsl{x1}
	    and \textsl{x2} 

\end{itemize}

	\emph{Note}: In the command-line program, genr
	commands that retrieve model-related data always reference the model
	that was estimated most recently. This is also true in the GUI program,
	if one uses genr in the ``gretl console'' or
	enters a formula using the ``Define new variable'' option
	under the Variable menu in the main window.  With the GUI, however, you
	have the option of retrieving data from any model currently displayed in
	a window (whether or not it's the most recent model).  You do this under
	the ``Model data'' menu in the model's window.

The internal series \verb@$uhat@ and \verb@$yhat@ hold,
	respectively, the residuals and fitted values from the last regression.
	For GARCH models, the conditional variance is held in the internal variable 
	\verb@$h@.

	Three ``internal'' dataset variables are available:
	\verb@$nobs@ holds the number of observations in the current sample
	range (which may or may not equal the value of \verb@$T@, the number
	of observations used in estimating the last model); \verb@$nvars@
	holds the number of variables in the dataset (including the constant);
	and \verb@$pd@ holds the frequency or periodicity of the data (e.g.{}
	4 for quarterly data).

	Two special internal scalars, \verb@$test@ and \verb@$pvalue@,
	hold respectively the value and the p-value of the test statistic that
	was generated by the last explicit hypothesis-testing command, if any
	(e.g.{} \texttt{chow}).  Please see 
	\GUG{} for details on this.

	The variable \texttt{t} serves as an index of the observations. For
	instance \texttt{genr dum = (t=15)} will generate a dummy variable
	that has value 1 for observation 15, 0 otherwise.  The variable
	\texttt{obs} is similar but more flexible: you can use this to pick
	out particular observations by date or name.  For example, \texttt{genr d =	  (obs>1986:4)} or \texttt{genr d = (obs="CA")}. The last form
	presumes that the observations are labeled; the label must be put in
	double quotes.

	Scalar values can be pulled from a series in the context of a
	\texttt{genr} formula, using the syntax
	\textsl{varname}\texttt{[}\textsl{obs}\texttt{]}. The
	\textsl{obs} value can be given by number or date. Examples:
	\texttt{x[5]}, \texttt{CPI[1996:01]}.  For daily data, the form
	\textsl{YYYY/MM/DD} should be used, e.g.{} \texttt{ibm[1970/01/23]}.

	An individual observation in a series can be modified via
	\texttt{genr}.  To do this, a valid observation number or date, in
	square brackets, must be appended to the name of the variable on the
	left-hand side of the formula.  For example, \texttt{genr x[3] = 30}
	or \texttt{genr x[1950:04] = 303.7}.

	Here are a couple of tips on dummy variables:

\begin{itemize}
\item 
	    Suppose \texttt{x} is coded with values 1, 2, or 3 and you want
	    three dummy variables, \texttt{d1} = 1 if \texttt{x} = 1, 0
	    otherwise, \texttt{d2} = 1 if \texttt{x} = 2, and so on.  To
	    create these, use the commands:

\begin{code}
	    genr d1 = (x=1)
	    genr d2 = (x=2)
	    genr d3 = (x=3)
\end{code}

\item To create \texttt{z} = \texttt{max(x,y)}
	    do

\begin{code}
	    genr d = x>y
	    genr z = (x*d)+(y*(1-d))
\end{code}

\end{itemize}

\begin{table}[htbp]
\caption{Examples of use of genr command}
\centering
{\small
\begin{tabular}{lp{300pt}}
\textit{Formula} & \textit{Comment} \\[4pt]
\verb@y = x1^3@ & \texttt{x1} cubed\\
\texttt{y = ln((x1+x2)/x3)} & \\
\texttt{z = x>y} & \texttt{z(t)} = 1 if \texttt{x(t) > y(t)},
	    otherwise 0\\
\texttt{y = x(-2)} & \texttt{x} lagged 2 periods\\
\texttt{y = x(+2)} & \texttt{x} led 2 periods\\
\texttt{y = diff(x)} & \texttt{y(t) = x(t) - x(t-1)}\\
\texttt{y = ldiff(x)} & \texttt{y(t) = log x(t) - log x(t-1)}, the
	    instantaneous rate of growth of \texttt{x}\\
\texttt{y = sort(x)} & sorts \texttt{x} in increasing order and stores in
	    \texttt{y}\\
\texttt{y = dsort(x)} & sort \texttt{x} in decreasing order\\
\texttt{y = int(x)} & truncate \texttt{x} and store its integer value as
	    \texttt{y}\\
\texttt{y = abs(x)} & store the absolute values of \texttt{x}\\
\texttt{y = sum(x)} & sum \texttt{x} values excluding missing $-$999
	    entries\\
\texttt{y = cum(x)} & cumulation: 
		$y_t = \sum_{\tau=1}^t x_{\tau}$
	  \\
\verb@aa = $ess@ & set \texttt{aa} equal to the Error Sum of Squares
	    from last regression\\
\verb@x = $coeff(sqft)@ & grab the estimated coefficient on the variable
	    \texttt{sqft} from the last regression\\
\verb@rho4 = $rho(4)@ & grab the 4th-order autoregressive coefficient from the
	    last model (presumes an \texttt{ar} model)\\
\verb@cvx1x2 = $vcv(x1, x2)@ & grab the estimated coefficient covariance of vars
	    \texttt{x1} and \texttt{x2} from the last model\\
\texttt{foo = uniform()} & uniform pseudo-random variable in range
	    0--1\\
\texttt{bar = 3 * normal()} & normal pseudo-random variable, $\mu$ = 0, $\sigma$ =
	    3\\
\texttt{samp = ok(x)} & = 1 for observations where \texttt{x} is not
	    missing.\end{tabular}
}
\end{table}
Caminho de Menu:    /Vari�vel/Define new variable

Acesso alternativo: Menu de contexto da janela principal

\subsection{gmm}
\hypertarget{cmd-gmm}{}

\begin{tabular}{ll}
Op��es:      & \verb|--two-step| (two step estimation) \\
 & \verb|--iterate| (iterated GMM) \\
 & \verb|--vcv| (print covariance matrix) \\
 & \verb|--verbose| (mostrar detalhes das itera��es) \\
\end{tabular}

	Performs Generalized Method of Moments (GMM) estimation using the
	BFGS (Broyden, Fletcher, Goldfarb, Shanno) algorithm. You must
	specify one or more commands for updating the relevant quantities
	(tyically, GMM residuals), one or more sets of orthogonality
	conditions, an initial matrix of weights, and a listing of the
	parameters to be estimated, all enclosed between the tags
	\texttt{gmm} and \texttt{end gmm}.

	Please see \GUG{} for details on this command.
	Here we just illustrate with a simple example.

\begin{code}
	gmm e = y - X*b
	  orthog e ; W
	  weights V
	  params b
	end gmm
      \end{code}

	In the example above we assume that \texttt{y} and \texttt{X}
	are data matrices, \texttt{b} is an appropriately sized vector
	of parameter values, \texttt{W} is a matrix of instruments, and
	\texttt{V} is a suitable matrix of weights.  The statement

\begin{code}
	orthog e ; W
      \end{code}

	indicates that the residual vector \texttt{e} is in principle
	orthogonal to each of the instruments composing the columns of
	\texttt{W}.

Caminho de Menu:    /Model/GMM

\subsection{gnuplot}
\hypertarget{cmd-gnuplot}{}

\begin{tabular}{ll}
Argumentos:  & \textsl{yvars} \textsl{xvar} \texttt{[} \textsl{dumvar} \texttt{]} \\
Op��es:      & \verb|--with-lines| (use lines, not points) \\
 & \verb|--with-impulses| (use vertical lines) \\
 & \verb|--suppress-fit| (don't show fitted line) \\
 & \verb|--linear-fit| (show least squares fit) \\
 & \verb|--inverse-fit| (show inverse fit) \\
 & \verb|--quadratic-fit| (show quadratic fit) \\
 & \verb|--loess-fit| (show loess fit) \\
 & \verb|--dummy| (see below) \\
Exemplos:    & \texttt{gnuplot y1 y2 x} \\ 
 & \verb@gnuplot x time --with-lines@ \\ 
 & \verb@gnuplot wages educ gender --dummy@ \\ 
\end{tabular}

Without the \verb@--dummy@ option, the
	\textsl{yvars} are graphed against \textsl{xvar}. With
	\verb@--dummy@, \textsl{yvar} is graphed against
	\textsl{xvar} with the points shown in different colors
	depending on whether the value of \textsl{dumvar} is 1 or
	0.

	The \texttt{time} variable behaves specially: if it does not
	already exist then it will be generated automatically.

	In interactive mode the result is displayed immediately. In
	batch mode a gnuplot command file is written, with a name on
	the pattern \texttt{gpttmpN.plt}, starting with N
	= \texttt{01}. The actual plots may be generated later using
	gnuplot (under MS Windows,
	wgnuplot).

	The various ``fit'' options are applicable only in
	the case of a bivariate scatterplot.  The default behavior
	is to show the OLS fitted line if and only if the slope
	coefficient is significant at the 10 percent level. If the
	\texttt{suppress} option is given, no fitted line is shown.
	If the \texttt{linear} option is given, the OLS line is
	shown regardless of whether or not it is significant.  The other
	options---\texttt{inverse}, \texttt{quadratic} and
	loess---produce respectively an inverse fit
	(regression of y on 1/x), a
	quadratic fit, or a loess fit.  Loess (also sometimes called
	``lowess'') is a robust locally weighted regression.  

A further option to this command is available: following
	the specification of the variables to be plotted and the
	option flag (if any), you may add literal gnuplot commands to
	control the appearance of the plot (for example, setting the
	plot title and/or the axis ranges).  These commands should be
	enclosed in braces, and each gnuplot command must be
	terminated with a semi-colon.  A backslash may be used to
	continue a set of gnuplot commands over more than one line.
	Here is an example of the syntax:

	\verb@{ set title 'My Title'; set yrange [0:1000]; }@

Caminho de Menu:    /View/Graph specified vars

Acesso alternativo: Menu de contexto da janela principal, graph button on toolbar

\subsection{graph}
\hypertarget{cmd-graph}{}

\begin{tabular}{ll}
Argumentos:  & \textsl{yvars} \textsl{xvar} \\
Op��o:       & \verb|--tall| (use 40 rows) \\
\end{tabular}

	ASCII graphics.  The \textsl{yvars} (which may be given by
	name or number) are  graphed against \textsl{xvar} using
	ASCII symbols. The \verb@--tall@ flag will produce a graph
	with 40 rows and 60 columns. Without it, the graph will be 20
	by 60 (for screen output).  See also \hyperlink{cmd-gnuplot}{gnuplot}.

\subsection{hausman}
\hypertarget{cmd-hausman}{}

	This test is available only after estimating an OLS model
	using panel data (see also setobs).  It tests the
	simple pooled model against the principal alternatives, the fixed
	effects and random effects models.

	The fixed effects model allows the intercept of the regression to
	vary across the cross-sectional units.  An
	F-test is reported for the null hypotheses that
	the intercepts do not differ. The random effects model decomposes
	the residual variance into two parts, one part specific to the
	cross-sectional unit and the other specific to the particular
	observation.  (This estimator can be computed only if the number
	of cross-sectional units in the data set exceeds the number of
	parameters to be estimated.) The Breusch--Pagan LM statistic
	tests the null hypothesis that the pooled OLS estimator is
	adequate against the random effects alternative.

	The pooled OLS model may be rejected against both of the
	alternatives, fixed effects and random effects. Provided the
	unit- or group-specific error is uncorrelated with the
	independent variables, the random effects estimator is more
	efficient than the fixed effects estimator; otherwise the
	random effects estimator is inconsistent and the fixed effects
	estimator is to be preferred. The null hypothesis for the
	Hausman test is that the group-specific error is not so
	correlated (and therefore the random effects model is
	preferable).  A low p-value for this test counts against the
	random effects model and in favor of fixed effects.

Caminho de Menu:    Janela do modelo, /Testes/Panel diagnostics

\subsection{hccm}
\hypertarget{cmd-hccm}{}

\begin{tabular}{ll}
Argumentos:  & \textsl{vari�vel-dependente} \textsl{vari�veis-independentes} \\
Op��o:       & \verb|--vcv| (mostrar matriz de covari�ncia) \\
\end{tabular}

	Heteroskedasticity-Consistent Covariance Matrix: this command
	runs a regression where the coefficients are estimated via the
	standard OLS procedure, but the standard errors of the
	coefficient estimates are computed in a manner that is robust
	in the face of heteroskedasticity, namely using the
	MacKinnon--White ``jackknife'' procedure.

\subsection{heckit}
\hypertarget{cmd-heckit}{}

\begin{tabular}{ll}
Argumentos:  & \textsl{vari�vel-dependente} \textsl{vari�veis-independentes} \texttt{;} \textsl{selection equation} \\
Op��es:      & \verb|--two-step| (perform two-step estimation) \\
 & \verb|--vcv| (print covariance matrix) \\
 & \verb|--verbose| (print extra output) \\
Exemplo:     & \texttt{heckit y 0 x1 x2 ; ys 0 x3 x4} \\ 
\end{tabular}

	Heckman-type selection model.  In the specification, the list
	before the semicolon represents the outcome equation, and the
	second list represents the selection equation.  The dependent
	variable in the selection equation (\texttt{ys} in the
	example above) must be a binary variable.  

	By default, the parameters are estimated by maximum
	likelihood. The covariance matrix of the parameters is
	computed using the negative inverse of the Hessian. If
	two-step estimation is desired, use the \verb@--two-step@
	option. In this case, the covariance matrix of the parameters
	of the outcome equation is appropriately adjusted as per
	Heckman (1979).

      FIXME this entry needs to be completed.

Caminho de Menu:    /Model/Nonlinear models/Heckit

\subsection{help}
\hypertarget{cmd-help}{}

	Gives a list of available commands. help
	\textsl{command} describes \textsl{command} (e.g.{}
	help smpl).  You can type man instead of
	help if you like. 

Caminho de Menu:    /Help

\subsection{hsk}
\hypertarget{cmd-hsk}{}

\begin{tabular}{ll}
Argumentos:  & \textsl{vari�vel-dependente} \textsl{vari�veis-independentes} \\
Op��o:       & \verb|--vcv| (mostrar matriz de covari�ncia) \\
\end{tabular}

	An OLS regression is run and the residuals are saved.  The
	logs of the squares of these residuals then become the
	dependent variable in an auxiliary regression, on the
	right-hand side of which are the original independent
	variables plus their squares.  The fitted values from the
	auxiliary regression are then used to construct a weight
	series, and the original model is re-estimated using weighted
	least squares.  This final result is reported.

The weight series is formed as
	$1/\sqrt{e^{y^*}}$, where y*
	denotes the fitted values from the auxiliary
	regression.

Caminho de Menu:    /Model/Other linear models/Heteroskedasticity corrected

\subsection{hurst}
\hypertarget{cmd-hurst}{}

\begin{tabular}{ll}
Argumento:   & \textsl{nome-de-vari�vel} \\
\end{tabular}

	Calculates the Hurst exponent (a measure of persistence or
	long memory) for a time-series variable having at least 128
	observations.

	The Hurst exponent is discussed by Mandelbrot.  In theoretical
	terms it is the exponent, H, in the
	relationship 
	\[\mathrm{RS}(x) = an^H\]where RS is the ``rescaled
	  range'' of the variable x in
	samples of size n and a
	is a constant. The rescaled range is the range (maximum minus
	minimum) of the cumulated value or partial sum of
	x over the sample period (after subtraction
	of the sample mean), divided by the sample standard deviation.

	As a reference point, if x is white noise
	(zero mean, zero persistence) then the range of its cumulated
	``wandering'' (which forms a random walk), scaled
	by the standard deviation, grows as the square root of the
	sample size, giving an expected Hurst exponent of 0.5.  Values
	of the exponent significantly in excess of 0.5 indicate
	persistence, and values less than 0.5 indicate
	anti-persistence (negative autocorrelation).  In principle the
	exponent is bounded by 0 and 1, although in finite samples it
	is possible to get an estimated exponent greater than 1.  

	In gretl, the exponent is estimated using binary sub-sampling:
	we start with the entire data range, then the two halves of
	the range, then the four quarters, and so on.  For sample
	sizes smaller than the data range, the RS value is the mean
	across the available samples.  The exponent is then estimated
	as the slope coefficient in a regression of the log of RS on
	the log of sample size.

Caminho de Menu:    /Vari�vel/Hurst exponent

\subsection{if}
\hypertarget{cmd-if}{}
Flow control for command execution.  Three sorts of
	construction are supported, as follows.

\begin{raggedright}
	\verb@# simple form@\\
	\texttt{if} \textsl{condition}\\
	\texttt{ }\texttt{ } \textsl{commands}\\
	\texttt{endif}
      \end{raggedright}

\begin{raggedright}
	\verb@# two branches@\\
	\texttt{if} \textsl{condition}\\
	\texttt{ }\texttt{ } \textsl{commands1}\\
	\texttt{else}\\
	\texttt{ }\texttt{ } \textsl{commands2}\\
	\texttt{endif}
      \end{raggedright}

\begin{raggedright}
	\verb@# three or more branches@\\
	\texttt{if} \textsl{condition1}\\
	\texttt{ }\texttt{ } \textsl{commands1}\\
	\texttt{elif} \textsl{condition2}\\
	\texttt{ }\texttt{ } \textsl{commands2}\\
	\texttt{else}\\
	\texttt{ }\texttt{ } \textsl{commands3}\\
	\texttt{endif}
      \end{raggedright}

	\textsl{condition} must be a Boolean expression, for
	the syntax of which see \hyperlink{cmd-genr}{genr}.  More than one
	elif block may be included.  In addition, \texttt{if}
	\dots{} \texttt{endif} blocks may be nested.

\subsection{include}
\hypertarget{cmd-include}{}

\begin{tabular}{ll}
Argumento:   & \textsl{inputfile} \\
Exemplos:    & \texttt{include myfile.inp} \\ 
 & \texttt{include sols.gfn} \\ 
\end{tabular}

	Intended for use in a command script, primarily for including definitions of
	functions.  Executes the commands in \textsl{inputfile} then returns
	control to the main script. To include a packaged function, be
	sure to include the filename extension.

	See also \hyperlink{cmd-run}{run}.

\subsection{info}
\hypertarget{cmd-info}{}

	Prints out any supplementary information stored with the
	current datafile.

Caminho de Menu:    /Data/Read info

Acesso alternativo: Data browser windows

\subsection{kpss}
\hypertarget{cmd-kpss}{}

\begin{tabular}{ll}
Argumentos:  & \textsl{order} \textsl{varlist} \\
Op��es:      & \verb|--trend| (include a trend) \\
 & \verb|--verbose| (print regression results) \\
 & \verb|--quiet| (suppress printing of results) \\
 & \verb|--difference| (use first difference of variable) \\
Exemplos:    & \texttt{kpss 8 y} \\ 
 & \verb@kpss 4 x1 --trend@ \\ 
\end{tabular}

	Computes the KPSS test (Kwiatkowski, Phillips, Schmidt and Shin,
	1992) for stationarity, for each of the specified variables (or
	their first difference, if the \verb@--difference@ option is
	selected). The null hypothesis is that the variable in question is
	stationary, either around a level or, if the \verb@--trend@
	option is given, around a deterministic linear trend.  

	The order argument determines the size of the window used for
	Bartlett smoothing.  If the \verb@--verbose@ option is
	chosen the results of the auxiliary regression are printed,
	along with the estimated variance of the random walk component
	of the variable.

Caminho de Menu:    /Variable/KPSS test

\subsection{labels}
\hypertarget{cmd-labels}{}

	Prints out the informative labels for any variables that have
	been generated using genr, and any labels added to
	the data set via the GUI.

\subsection{lad}
\hypertarget{cmd-lad}{}

\begin{tabular}{ll}
Argumentos:  & \textsl{depvar} \textsl{indepvars} \\
Op��o:       & \verb|--vcv| (print covariance matrix) \\
\end{tabular}

	Calculates a regression that minimizes the sum of the absolute
	deviations of the observed from the fitted values of the
	dependent variable.  Coefficient estimates are derived using
	the Barrodale--Roberts simplex algorithm; a warning is
	printed if the solution is not unique.

	Standard errors are derived using the bootstrap procedure with
	500 drawings. The covariance matrix for the parameter
	estimates, printed when the \verb@--vcv@ flag is given, is
	based on the same bootstrap.

Caminho de Menu:    /Model/Robust estimation/Least Absolute Deviation

\subsection{lags}
\hypertarget{cmd-lags}{}

\begin{tabular}{ll}
Variantes:   & \texttt{lags} \textsl{varlist}\\
 & \texttt{lags} \textsl{order} \texttt{;} \textsl{varlist}\\
Exemplos:    & \texttt{lags x y} \\ 
 & \texttt{lags 12 ; x y} \\ 
\end{tabular}

	Creates new variables which are lagged values of each of the variables in
	\textsl{varlist}.  By default the number of lagged variables equals the
	periodicity of the data. For example, if the periodicity is 4 (quarterly),
	the command lags x creates

\begin{code}
	x_1 = x(t-1)
	x_2 = x(t-2)
	x_3 = x(t-3)
	x_4 = x(t-4)
      \end{code}

	The number of lags created can be controlled by the optional
	first parameter.

Caminho de Menu:    /Add/Lags of selected variables

\subsection{ldiff}
\hypertarget{cmd-ldiff}{}

\begin{tabular}{ll}
Argumento:   & \textsl{varlist} \\
\end{tabular}

	The first difference of the natural log of each variable in
	\textsl{varlist} is obtained and the result stored in a
	new variable with the prefix \verb@ld_@.  Thus ldiff
	  x y creates the new variables

\begin{code}
	ld_x = log(x) - log(x(-1))
	ld_y = log(y) - log(y(-1))
      \end{code}
Caminho de Menu:    /Add/Log differences of selected variables

\subsection{leverage}
\hypertarget{cmd-leverage}{}

\begin{tabular}{ll}
Op��o:       & \verb|--save| (save variables) \\
\end{tabular}

	Must immediately follow an ols command. Calculates
	the leverage (\ensuremath{h}, which must lie in the
	range 0 to 1) for each data point in the sample on which the
	previous model was estimated.  Displays the residual
	(\ensuremath{u}) for each observation along with its
	leverage and a measure of its influence on the estimates, 
	  $uh/(1 - h)$. ``Leverage points'' for
	which the value of \ensuremath{h} exceeds
	2\ensuremath{k}/\ensuremath{n} (where
	\ensuremath{k} is the number of parameters being
	estimated and \ensuremath{n} is the sample size) are
	flagged with an asterisk.  For details on the concepts of
	leverage and influence see Davidson and MacKinnon (1993,
	Chapter 2).

	DFFITS values are also shown: these are ``studentized
	  residuals'' (predicted residuals divided by their
	standard errors) multiplied by 
	  $\sqrt{h/(1 - h)}$. For a discussion of studentized residuals
	and DFFITS see G. S. Maddala, \emph{Introduction to
	  Econometrics}, chapter 12; also Belsley, Kuh and
	Welsch (1980).

	Briefly, a ``predicted residual'' is the difference
	between the observed value of the dependent variable at
	observation \ensuremath{t}, and the fitted value for
	observation \ensuremath{t} obtained from a regression in
	which that observation is omitted (or a dummy variable with
	value 1 for observation \ensuremath{t} alone has been
	added); the studentized residual is obtained by dividing the
	predicted residual by its standard error.

If the \verb@--save@ flag is given with
	this command, then the leverage, influence and DFFITS values
	are added to the current data set.

Caminho de Menu:    Model window, /Tests/Influential observations

\subsection{lmtest}
\hypertarget{cmd-lmtest}{}

\begin{tabular}{ll}
Argumento:   & \texttt{[} \textsl{order} \texttt{]} \\
Op��es:      & \verb|--logs| (non-linearity, logs) \\
 & \verb|--autocorr| (serial correlation) \\
 & \verb|--arch| (ARCH) \\
 & \verb|--squares| (non-linearity, squares) \\
 & \verb|--white| (heteroskedasticity, White's test) \\
 & \verb|--panel| (heteroskedasticity, groupwise) \\
 & \verb|--quiet| (don't print auxiliary regression) \\
\end{tabular}

	Must immediately follow an ols command. Depending on
	the options given, this command carries out some combination of
	the following: Lagrange Multiplier tests for nonlinearity (logs
	and squares); White's test for heteroskedasticity; the LMF test
	for serial correlation (see Kiviet, 1986); and a test for ARCH
	(Autoregressive Conditional Heteroskedasticity, see also the
	arch command).

	The optional \texttt{order} argument is relevant only in case
	the \verb@--autocorr@ or \verb@--arch@ options are
	selected.  The default is to run these tests using a lag order
	equal to the periodicity of the data, but this can be adjusted by
	supplying a specific lag order.

	The \verb@--panel@ option is available only when the model
	is estimated on panel data: in this case a test for groupwise
	heteroskedasticity is performed (that is, for a differing
	error variance across the cross-sectional units).

	By default, the program prints the auxiliary regression on which
	the test statistic is based.  This may be suppressed by using the
	\verb@--quiet@ flag.

Caminho de Menu:    Janela do modelo, /Tests

\subsection{logistic}
\hypertarget{cmd-logistic}{}

\begin{tabular}{ll}
Argumentos:  & \textsl{vari�vel-dependente} \textsl{vari�veis-independentes} \texttt{[} \texttt{ymax=}\textsl{value} \texttt{]} \\
Op��o:       & \verb|--vcv| (mostrar matriz de covari�ncia) \\
Exemplos:    & \texttt{logistic y const x} \\ 
 & \texttt{logistic y const x ymax=50} \\ 
\end{tabular}

	Logistic regression: carries out an OLS regression using the
	logistic transformation of the dependent variable,
	\[\log\left(\frac{y}{y^*-y}\right)\]

The dependent variable must be strictly
	positive.  If it is a decimal fraction, between 0 and 1, the
	default is to use a y* value
	(the asymptotic maximum of the dependent variable) of 1. If
	the dependent variable is a percentage, between 0 and 100, the
	default y* is 100.

	If you wish to set a different maximum, use the optional
	\texttt{ymax=}\textsl{value} syntax following the list of
	regressors.  The supplied value must be greater than all of the observed
	values of the dependent variable.

The fitted values and residuals from the regression are
	automatically transformed using 	  
	\[y=\frac{y^*}{1+e^{-x}}\] where x represents
	either a fitted value or a residual from the OLS regression
	using the transformed dependent variable.  The reported values
	are therefore comparable with the original dependent
	variable.

	Note that if the dependent variable is binary, you should
	use the \hyperlink{cmd-logit}{logit} command instead.

Caminho de Menu:    /Model/Nonlinear models/Logistic

\subsection{logit}
\hypertarget{cmd-logit}{}

\begin{tabular}{ll}
Argumentos:  & \textsl{vari�vel-dependente} \textsl{vari�veis-independentes} \\
Op��es:      & \verb|--robust| (robust standard errors) \\
 & \verb|--vcv| (mostrar matriz de covari�ncia) \\
 & \verb|--verbose| (mostrar detalhes das itera��es) \\
\end{tabular}

	If the dependent variable is a binary variable (all values are
	0 or 1) maximum likelihood estimates of the coefficients on
	\textsl{vari�veis-independentes} are obtained via the ``binary
	  response model regression'' (BRMR) method outlined by
	Davidson and MacKinnon (2004). As the model is nonlinear the
	slopes depend on the values of the independent variables: the
	reported slopes are evaluated at the means of those variables.
	The chi-square statistic tests the null hypothesis that all
	coefficients are zero apart from the constant.

	By default, standard errors are computed using the negative
	inverse of the Hessian.  If the \verb@--robust@ flag is given,
	then QML or Huber--White standard errors are calculated
	instead. In this case the estimated covariance matrix is a
	``sandwich'' of the inverse of the estimated Hessian
	and the outer product of the gradient. See Davidson and MacKinnon
	(2004, Chapter 10) for details.

	If the dependent variable is not binary, but is discrete and has a
	minimum value of 0, then Ordered Logit estimates are obtained.  In
	this case robust standard errors are not yet available.  (If the
	variable selected as dependent is not discrete, or does not have a
	minimum of zero, an error is flagged.)

	If you want to use logit for analysis of proportions (where
	the dependent variable is the proportion of cases having a
	certain characteristic, at each observation, rather than a 1
	or 0 variable indicating whether the characteristic is present
	or not) you should not use the logit command, but
	rather construct the logit variable, as in

\begin{code}
	genr lgt_p = log(p/(1 - p))
      \end{code}
and use this as the dependent variable in an OLS regression.  
	See Ramanathan (2002, Chapter 12).

Caminho de Menu:    /Model/Nonlinear models/Logit

\subsection{logs}
\hypertarget{cmd-logs}{}

\begin{tabular}{ll}
Argumento:   & \textsl{lista-de-vari�veis} \\
\end{tabular}

	The natural log of each of the variables in \textsl{lista-de-vari�veis}
	is obtained and the result stored in a new variable with the
	prefix \verb@l_@ (``el'' underscore).  For example,
	logs x y creates the new variables \verb@l_x@ =
	ln(\texttt{x}) and \verb@l_y@ = ln(\texttt{y}).

Caminho de Menu:    /Add/Logs of selected variables

\subsection{loop}
\hypertarget{cmd-loop}{}

\begin{tabular}{ll}
Argumento:   & \textsl{control} \\
Op��es:      & \verb|--progressive| (enable special forms of certain commands) \\
 & \verb|--verbose| (report details of genr commands) \\
 & \verb|--quiet| (do not report number of iterations performed) \\
Exemplos:    & \texttt{loop 1000} \\ 
 & \verb@loop 1000 --progressive@ \\ 
 & \texttt{loop while essdiff > .00001} \\ 
 & \texttt{loop for i=1991..2000} \\ 
 & \texttt{loop for (r=-.99; r<=.99; r+=.01)} \\ 
\end{tabular}

The parameter \textsl{control} must take
	one of four forms, as shown in the examples: an integer number
	of times to repeat the commands within the loop;
	``\texttt{while}'' plus a numerical condition;
	``\texttt{for}'' plus a range of values for the
	internal index variable \texttt{i}; or
	``\texttt{for}'' plus three expressions in
	parentheses, separated by semicolons.  In the last form the
	left-hand expression initializes a variable, the middle
	expression sets a condition for iteration to continue, and the
	right-hand expression sets an increment or decrement to be
	applied at the start of the second and subsequent iterations.
	(This is a restricted form of the \texttt{for} statement in
	the C programming language.)

This command opens a special mode in which the program
	accepts commands to be executed repeatedly.  You exit the mode
	of entering loop commands with endloop: at this
	point the stacked commands are executed.

See \GUG{} for further details and
	examples.  The effect of the \verb@--progressive@ option
	(which is designed for use in Monte Carlo simulations) is
	explained there. Not all gretl commands may be used within
	a loop; the commands available in this context are also
	set out there.

\subsection{mahal}
\hypertarget{cmd-mahal}{}

\begin{tabular}{ll}
Argumento:   & \textsl{lista-de-vari�veis} \\
Op��es:      & \verb|--save| (add distances to the dataset) \\
 & \verb|--vcv| (mostrar matriz de covari�ncia) \\
\end{tabular}

	The Mahalanobis distance is the distance between two points in
	an k-dimensional space, scaled by the
	statistical variation in each dimension of the space.  For
	example, if p and q are
	two observations on a set of k variables
	with covariance matrix C, then the
	Mahalanobis distance between the observations is given by
	\[\sqrt{(p-q)^{\prime}C^{-1}(p-q)}\]where 
	$(p-q)$ is a
	k-vector. This reduces to Euclidean
	distance if the covariance matrix is the identity
	matrix.

The space for which distances are computed is defined by
	the selected variables.  For each observation in the current
	sample range, the distance is computed between the observation
	and the centroid of the selected variables.  This distance is
	the multidimensional counterpart of a standard
	z-score, and can be used to judge whether a
	given observation ``belongs'' with a group of other
	observations.

If the \verb@--vcv@ option is given, the
	covariance matrix and its inverse are printed.  If the
	\verb@--save@ option is given, the distances are saved to
	the dataset under the name \texttt{mdist} (or
	\texttt{mdist1}, \texttt{mdist2} and so on if there is
	already a variable of that name).

Caminho de Menu:    /View/Mahalanobis distances

\subsection{meantest}
\hypertarget{cmd-meantest}{}

\begin{tabular}{ll}
Argumentos:  & \textsl{var1} \textsl{var2} \\
Op��o:       & \verb|--unequal-vars| (assume variances are unequal) \\
\end{tabular}

	Calculates the t statistic for the null
	hypothesis that the population means are equal for the
	variables \textsl{var1} and \textsl{var2}, and shows
	its p-value.

	By default the test statistic is calculated on the assumption
	that the variances are equal for the two variables; with the
	\verb@--unequal-vars@ option the variances are assumed to
	be different.  This will make a difference to the test
	statistic only if there are different numbers of non-missing
	observations for the two variables.

Caminho de Menu:    /Model/Bivariate tests/Difference of means

\subsection{mle}
\hypertarget{cmd-mle}{}

\begin{tabular}{ll}
Argumentos:  & \textsl{log-likelihood function} \textsl{derivatives} \\
Op��es:      & \verb|--vcv| (mostrar matriz de covari�ncia) \\
 & \verb|--hessian| (base covariance matrix on the Hessian) \\
 & \verb|--robust| (QML covariance matrix) \\
 & \verb|--verbose| (mostrar detalhes das itera��es) \\
Exemplo:     & \texttt{weibull.inp}
\end{tabular}

	Performs Maximum Likelihood (ML) estimation using the BFGS
	(Broyden, Fletcher, Goldfarb, Shanno) algorithm. The user must
	specify the log-likelihood function.  The parameters of this
	function must be declared and given starting values (using the
	genr command) prior to estimation.  Optionally, the
	user may specify the derivatives of the log-likelihood
	function with respect to each of the parameters; if analytical
	derivatives are not supplied, a numerical approximation is
	computed.

Simple example: Suppose we have a series \texttt{X} with values 0
	or 1 and we wish to obtain the maximum likelihood estimate of the
	probability, \texttt{p}, that \texttt{X} = 1.  (In this simple case
	we can guess in advance that the ML estimate of \texttt{p} will simply
	equal the proportion of Xs equal to 1 in the sample.)

The parameter \texttt{p} must first be added to the dataset and
	given an initial value.  This can be done using the genr
	command.  For example, \texttt{genr p = 0.5}.

We then construct the MLE command block:

\begin{code}
	mle loglik = X*log(p) + (1-X)*log(1-p)
	  deriv p = X/p - (1-X)/(1-p)
	end mle
\end{code}

	The first line above specifies the log-likelihood function. It
	starts with the keyword \texttt{mle}, then a dependent
	variable is specified and an expression for the log-likelihood
	is given (using the same syntax as in the genr
	command).  The next line (which is optional) starts with the
	keyword \texttt{deriv} and supplies the derivative of
	the log-likelihood function with respect to the parameter
	\texttt{p}. If no derivatives are given, you should include
	a statement using the keyword \texttt{params} which
	identifies the free parameters: these are listed on one line,
	separated by spaces.  For example, the above could be changed
	to:

\begin{code}
	mle loglik = X*log(p) + (1-X)*log(1-p)
	  params p
	end mle
\end{code}

	in which case numerical derivatives would be used.

	Note that any option flags should be appended to the ending line
	of the MLE block.

	By default, estimated standard errors are based on the Outer
	Product of the Gradient.  If the \verb@--hessian@ option is
	given, they are instead based on the negative inverse of the
	Hessian (which is approximated numerically).  If the
	\verb@--robust@ option is given, a QML estimator is used
	(namely, a sandwich of the negative inverse of the Hessian and the
	covariance matrix of the gradient). 

Caminho de Menu:    /Model/Maximum likelihood

\subsection{modeltab}
\hypertarget{cmd-modeltab}{}

\begin{tabular}{ll}
Argumentos:  & \textsl{add}  ou \textsl{show}  ou \textsl{free} \\
\end{tabular}

Manipulates the gretl ``model table''. See 
	\GUG{} for details. The sub-commands have
	the following effects: add adds the last model
	estimated to the model table, if possible; show
	displays the model table in a window; and free
	clears the table.

Caminho de Menu:    Session window, Model table icon

\subsection{mpols}
\hypertarget{cmd-mpols}{}

\begin{tabular}{ll}
Argumentos:  & \textsl{vari�vel-dependente} \textsl{vari�veis-independentes} \\
Op��es:      & \verb|--vcv| (mostrar matriz de covari�ncia) \\
 & \verb|--simple-print| (do not print auxiliary statistics) \\
 & \verb|--quiet| (n�o mostrar resultados da regress�o) \\
\end{tabular}

	Computes OLS estimates for the specified model using multiple
	precision floating-point arithmetic.  This command is available
	only if gretl is compiled with support for the
	Gnu Multiple Precision (GMP) library.  By default 256 bits of
	precision are used for the calculations, but this can be increased
	via the environment variable \verb@GRETL_MP_BITS@.  For
	example, when using the bash shell one could issue the following
	command, before starting gretl, to set a precision of 1024 bits.

\begin{code}
	export GRETL_MP_BITS=1024
      \end{code}

	A rather arcane option is available for this command (primarily
	for testing purposes): if the \textsl{vari�veis-independentes} list is
	followed by a semicolon and a further list of numbers, those
	numbers are taken as powers of \textsl{x} to be added to the
	regression, where \textsl{x} is the last variable in
	\textsl{vari�veis-independentes}.  These additional terms are computed and
	stored in multiple precision.  In the following example
	\texttt{y} is regressed on \texttt{x} and the second, third
	and fourth powers of \texttt{x}:

\begin{code}
	mpols y 0 x ; 2 3 4
      \end{code}
Caminho de Menu:    /Model/Other linear models/High precision OLS

\subsection{nls}
\hypertarget{cmd-nls}{}

\begin{tabular}{ll}
Argumentos:  & \textsl{function} \texttt{[} \textsl{derivatives} \texttt{]} \\
Op��es:      & \verb|--robust| (robust standard errors) \\
 & \verb|--vcv| (mostrar matriz de covari�ncia) \\
 & \verb|--verbose| (mostrar detalhes das itera��es) \\
Exemplo:     & \verb@wg_nls.inp@
\end{tabular}

	Performs Nonlinear Least Squares (NLS) estimation using a modified
	version of the Levenberg--Marquandt algorithm.  You must
	supply a function specification.  The parameters of this function
	must be declared and given starting values (using the
	genr command) prior to estimation.  Optionally, you may
	specify the derivatives of the regression function with respect to
	each of the parameters.  If you do not supply derivatives you
	should instead give a list of the parameters to be estimated
	(separated by spaces or commas), preceded by the keyword
	\texttt{params}.  In the latter case a numerical approximation
	to the Jacobian is computed.

	It is easiest to show what is required by example.  The
	following is a complete script to estimate the nonlinear
	consumption function set out in William Greene's
	\emph{Econometric Analysis} (Chapter 11 of the 4th
	edition, or Chapter 9 of the 5th).  The numbers to the left of
	the lines are for reference and are not part of the commands.
	Note that any option flags, such as \verb@--vcv@ for
	printing the covariance matrix of the parameter estimates,
	should be appended to the final command, \texttt{end nls}.

\begin{code}
	1   open greene11_3.gdt
	2   ols C 0 Y
	3   genr a = $coeff(0)
	4   genr b = $coeff(Y)
	5   genr g = 1.0
	6   nls C = a + b * Y^g
	7    deriv a = 1
	8    deriv b = Y^g
	9    deriv g = b * Y^g * log(Y)
	10  end nls --vcv
\end{code}

	It is often convenient to initialize the parameters by
	reference to a related linear model; that is accomplished here
	on lines 2 to 5.  The parameters alpha, beta and gamma could
	be set to any initial values (not necessarily based on a model
	estimated with OLS), although convergence of the NLS procedure
	is not guaranteed for an arbitrary starting point.

	The actual NLS commands occupy lines 6 to 10. On line 6 the
	nls command is given: a dependent variable is
	specified, followed by an equals sign, followed by a function
	specification.  The syntax for the expression on the right is
	the same as that for the genr command.  The next
	three lines specify the derivatives of the regression function
	with respect to each of the parameters in turn.  Each line
	begins with the keyword deriv, gives the name of a
	parameter, an equals sign, and an expression whereby the
	derivative can be calculated (again, the syntax here is the
	same as for genr). As an alternative to supplying
	numerical derivatives, you could substitute the following for
	lines 7 to 9:

\begin{code}
	params a b g
      \end{code}

	Line 10, end nls, completes the command and calls for
	estimation.

For further details on NLS estimation please see
	\GUG{}.

Caminho de Menu:    /Model/Nonlinear models/Nonlinear Least Squares

\subsection{normtest}
\hypertarget{cmd-normtest}{}

\begin{tabular}{ll}
Argumento:   & \textsl{series} \\
Op��es:      & \verb|--dhansen| (Doornik--Hansen test) \\
 & \verb|--swilk| (Shapiro--Wilk test) \\
 & \verb|--lillie| (Lilliefors test) \\
 & \verb|--jbera| (Jarque--Bera test) \\
 & \verb|--all| (do all tests) \\
 & \verb|--quiet| (suppress printed output) \\
\end{tabular}

	Carries out a test for normality for the given
	\textsl{series}.  The specific test is controlled by the
	option flags (but if no flag is given, the Doornik--Hansen
	test is performed).  Note: the Doornik--Hansen and
	Shapiro--Wilk tests are recommended over the others, on
	account of their superior small-sample properties.

	The test statistic and its p-value may be retrieved
	using the accessors \verb@$test@ and \verb@$pvalue@.
	Please note that if the \verb@--all@ option is given,
	the result recorded is that from the Doornik--Hansen
	test.

Caminho de Menu:    /Variable/Normality test

\subsection{nulldata}
\hypertarget{cmd-nulldata}{}

\begin{tabular}{ll}
Argumento:   & \textsl{series-length} \\
Exemplo:     & \texttt{nulldata 500} \\ 
\end{tabular}

	Establishes a ``blank'' data set, containing only a
	constant and an index variable, with periodicity 1 and the
	specified number of observations. This may be used for
	simulation purposes: some of the genr commands
	(e.g.{} genr uniform(), genr normal())
	will generate dummy data from scratch to fill out the data
	set. This command may be useful in conjunction with
	loop.  See also the ``seed'' option to
	the \hyperlink{cmd-set}{set} command.

Caminho de Menu:    /Ficheiro/New data set

\subsection{ols}
\hypertarget{cmd-ols}{}

\begin{tabular}{ll}
Argumentos:  & \textsl{vari�vel-dependente} \textsl{vari�veis-independentes} \\
Op��es:      & \verb|--vcv| (mostrar matriz de covari�ncia) \\
 & \verb|--robust| (robust standard errors) \\
 & \verb|--simple-print| (do not print auxiliary statistics) \\
 & \verb|--quiet| (n�o mostrar resultados da regress�o) \\
 & \verb|--no-df-corr| (suppress degrees of freedom correction) \\
 & \verb|--print-final| (see below) \\
Exemplos:    & \texttt{ols 1 0 2 4 6 7} \\ 
 & \verb@ols y 0 x1 x2 x3 --vcv@ \\ 
 & \verb@ols y 0 x1 x2 x3 --quiet@ \\ 
\end{tabular}

        Computes ordinary least squares (OLS) estimates with
	\textsl{depvar} as the dependent variable and
	\textsl{vari�veis-independentes} as the list of independent variables.
	Variables may be specified by name or number; use the number
	zero for a constant term. 

Besides coefficient estimates and standard errors, the
	program also prints p-values for t
	(two-tailed) and F-statistics.  A p-value
	below 0.01 indicates statistical significance at the 1 percent
	level and is marked with \texttt{***}. \texttt{**}
	indicates significance between 1 and 5 percent and
	\texttt{*} indicates significance between the 5 and 10
	percent levels. Model selection statistics (the Akaike
	Information Criterion or AIC and Schwarz's Bayesian Information
	Criterion) are also printed.  The formula used for the AIC is
	that given by Akaike (1974), namely minus two times the
	maximized log-likelihood plus two times the number of
	parameters estimated.

If the option \verb@--no-df-corr@ is
	given, the usual degrees of freedom correction is not applied
	when calculating the estimated error variance (and hence also
	the standard errors of the parameter estimates).

	The option \verb@--print-final@ is applicable only in the
	context of a \hyperlink{cmd-loop}{loop}.  It arranges for the
	regression to be run silently on all but the final iteration
	of the loop. See \GUG{} for details.

Various internal variables may be retrieved
	using the \hyperlink{cmd-genr}{genr} command, provided
	genr is invoked immediately after this command.

The specific formula used for generating
	robust standard errors (when the \verb@--robust@ option is
	given) can be adjusted via the \hyperlink{cmd-set}{set} command.

Caminho de Menu:    /Model/Ordinary Least Squares

Acesso alternativo: Beta-hat button on toolbar

\subsection{omit}
\hypertarget{cmd-omit}{}

\begin{tabular}{ll}
Argumento:   & \textsl{lista-de-vari�veis} \\
Op��es:      & \verb|--wald| (do a Wald test rather than an F-test) \\
 & \verb|--vcv| (mostrar matriz de covari�ncia) \\
 & \verb|--quiet| (don't print estimates for reduced model) \\
 & \verb|--silent| (don't print anything) \\
 & \verb|--inst| (omit as instrument, TSLS only) \\
 & \verb|--both| (omit as both regressor and instrument, TSLS only) \\
Exemplos:    & \texttt{omit 5 7 9} \\ 
 & \verb@omit seasonals --quiet@ \\ 
\end{tabular}

	This command must follow an estimation command.  It calculates
	a test statistic for the joint significance of the variables
	in \textsl{lista-de-vari�veis}, which should be a subset of the
	independent variables in the model previously estimated.

	If the original model was estimated by OLS, the test statistic is
	by default an F-value.  This is based on the
	sums of squared residuals for the restricted and unrestricted
	models, unless the original model was estimated with robust
	standard errors.  In the latter case F is
	computed from the robust estimate of the covariance matrix for the
	original model.  (It is the F-form of a Wald
	test).

	For estimators other than OLS, or if the \verb@--wald@ option
	is given, the statistic is an asymptotic Wald chi-square value
	based on the covariance matrix of the original model.

	By default, the restricted model is estimated, the estimates
	are printed, and the restricted model replaces the original as
	the ``current model'' for the purposes of, for
	example, retrieving the residuals as \verb@$uhat@ (or
	doing further tests such as add or
	omit).  

	If the \verb@--wald@ option is selected, the restricted
	model is not estimated (and so the current model is not
	replaced). The \verb@--quiet@ option suppresses the
	printout of the restricted model (if applicable): only the
	result of the test is printed.  If the restricted model is
	both estimated and printed, the \verb@--vcv@ option has the
	effect of printing the covariance matrix for the coefficients
	in the restricted model, otherwise this option is ignored.

	If the \verb@--silent@ option is given, nothing is printed;
	nonetheless, the results of the test can be retrieved using the
	special variables \verb@$test@ and \verb@$pvalue@.

	If the original model was estimated using two-stage least
	squares, an ambiguity arises: should the selected variables be
	omitted as regressors, as instruments, or as both?  This is
	resolved as follows: by default the variables are dropped from
	the list of regressors, but if the \verb@--inst@ flag is
	given they are dropped as instruments, or if the
	\verb@--both@ flag is given they are dropped from the
	model altogether.  These two options are incompatible with the
	\verb@--wald@ option; if one or more instruments are
	omitted the model must be re-estimated.

Caminho de Menu:    Janela do modelo, /Testes/Omit variables

\subsection{open}
\hypertarget{cmd-open}{}

\begin{tabular}{ll}
Argumento:   & \textsl{ficheiro-de-dados} \\
Op��o:       & \verb|--www| (use a database on the gretl server) \\
Exemplos:    & \texttt{open data4-1} \\ 
 & \texttt{open voter.dta} \\ 
 & \verb@open fedbog --www@ \\ 
\end{tabular}

	Opens a data file.  If a data file is already open, it is replaced
	by the newly opened one.  If a full path is not given, the program
	will search some relevant paths to try to find the file.  If no
	filename suffix is given (as in the first example above), gretl
	assumes a native datafile with suffix \texttt{.gdt}.  Based on
	the name of the file and various heuristics, gretl will try to
	detect the format of the data file (native, plain text, CSV, MS
	Excel, Stata, etc.).

	This command can also be used to open a database (gretl or RATS
	4.0) for reading.  In that case it should be followed by the
	\hyperlink{cmd-data}{data} command to extract particular series from
	the database.  If the \texttt{www} option is given, the program
	will try to access a database of the given name on the gretl
	server --- for instance the Federal Reserve interest rates
	database in the third example above.

Caminho de Menu:    /Ficheiro/Open data

Acesso alternativo: Drag a data file into gretl (MS Windows or Gnome)

\subsection{orthdev}
\hypertarget{cmd-orthdev}{}

\begin{tabular}{ll}
Argumento:   & \textsl{varlist} \\
\end{tabular}

	Applicable with panel data only. A series of forward orthogonal
	deviations is obtained for each variable in \textsl{varlist} and
	stored in a new variable with the prefix \verb@o_@. Thus
	orthdev x y creates the new variables \verb@o_x@ and
	\verb@o_y@.

	The values are stored one step ahead of their true temporal location
	(that is, \verb@o_x@ at observation \ensuremath{t} holds the
	deviation that, strictly speaking, belongs at \ensuremath{t} $-$
	1).  This is for compatibility with first differences: one loses the
	first observation in each time series, not the last.

\subsection{outfile}
\hypertarget{cmd-outfile}{}

\begin{tabular}{ll}
Argumentos:  & \textsl{filename} \textsl{option} \\
Op��es:      & \verb|--append| (append to file) \\
 & \verb|--close| (close file) \\
 & \verb|--write| (overwrite file) \\
Exemplos:    & \verb@outfile --write regress.txt@ \\ 
 & \verb@outfile --close@ \\ 
\end{tabular}

Diverts output to \textsl{filename}, until further
	notice.  Use the flag \verb@--append@ to append output to
	an existing file or \verb@--write@ to start a new file
	(or overwrite an existing one).  Only one file can be opened
	in this way at any given time.

The \verb@--close@ flag is used to close an output
	file that was previously opened as above.  Output will then
	revert to the default stream.

In the first example command above, the file
	\texttt{regress.txt} is opened for writing, and in
	the second it is closed.  This would make sense as a sequence
	only if some commands were issued before the
	\verb@--close@.  For example if an ols command
	intervened, its output would go to
	\texttt{regress.txt} rather than the
	screen.

\subsection{panel}
\hypertarget{cmd-panel}{}

\begin{tabular}{ll}
Op��es:      & \verb|--vcv| (mostrar matriz de covari�ncia) \\
 & \verb|--random-effects| (random rather than fixed effects) \\
 & \verb|--between| (estimate the between-groups model) \\
 & \verb|--time-dummies| (include time dummy variables) \\
 & \verb|--unit-weights| (weighted least squares) \\
 & \verb|--iterate| (iterative estimation) \\
 & \verb|--quiet| (less verbose output) \\
 & \verb|--verbose| (more verbose output) \\
\end{tabular}

	Estimates a panel model, by default using the fixed effects
	estimator.  Depending on the number of cross-sectional units and
	the number of independent variables, this is implemented either by
	adding a set of dummy variables representing the units, or by
	subtracting the group or unit means from the original data.

	If the \verb@--random-effects@ flag is given, random effects
	estimates are computed, using the method of Swamy and Arora.

	Alternatively, if the \verb@--unit-weights@ flag is given, the
	model is estimated via weighted least squares, with the weights
	based on the residual variance for the respective cross-sectional
	units in the sample.  In this case (only) the \verb@--iterate@
	flag may be added to produce iterative estimates: if the
	iteration converges, the resulting estimates are Maximum
	Likelihood.

	As a further alternative, if the \verb@--between@ flag is
	given, the between-groups model is estimated (that is, an OLS
	regression using the group means).

	For more details on panel estimation, please see \GUG{}.

Caminho de Menu:    /Model/Panel

\subsection{pca}
\hypertarget{cmd-pca}{}

\begin{tabular}{ll}
Argumento:   & \textsl{lista-de-vari�veis} \\
Op��es:      & \verb|--save| (Save major components) \\
 & \verb|--save-all| (Save all components) \\
\end{tabular}

Principal Components Analysis.  Prints the eigenvalues of
	the correlation matrix for the variables in
	\textsl{lista-de-vari�veis} along with the proportion of the joint
	variance accounted for by each component.  Also prints the
	corresponding eigenvectors (or ``component
	  loadings'').

If the \verb@--save@ flag is given, components with
	eigenvalues greater than 1.0 are saved to the dataset as
	variables, with names \texttt{PC1}, \texttt{PC2} and so
	on.  These artificial variables are formed as the sum of
	(component loading) times (standardized \texttt{Xi}), where
	\texttt{Xi} denotes the ith variable in
	\textsl{lista-de-vari�veis}.

If the \verb@--save-all@ flag is given, all of the
	components are saved as described above.

Caminho de Menu:    /View/Principal components

Acesso alternativo: Main window pop-up (multiple selection)

\subsection{pergm}
\hypertarget{cmd-pergm}{}

\begin{tabular}{ll}
Argumentos:  & \textsl{nome-de-vari�vel} \texttt{[} \textsl{bandwidth} \texttt{]} \\
Op��es:      & \verb|--bartlett| (use Bartlett lag window) \\
 & \verb|--log| (use log scale) \\
\end{tabular}

	Computes and displays (and if not in batch mode, graphs) the
	spectrum of the specified variable.  By default the sample
	periodogram is given; with the \verb@--bartlett@ flag a
	Bartlett lag window is used in estimating the spectrum (see, for
	example, Greene's \emph{Econometric Analysis} for a
	discussion of this).  The default width of the Bartlett window is
	twice the square root of the sample size but this can be set
	manually using the \textsl{bandwidth} parameter, up to a
	maximum of half the sample size.  If the \verb@--log@ option
	is given the spectrum is represented on a logarithmic scale.

	When the sample periodogram is printed, two tests for fractional
	integration of the series (``long memory'') are given,
	namely the Geweke and Porter-Hudak (GPH) test and the Local
	Whittle Estimator.  The null hypothesis in both cases is that the
	integration order is zero.  By default the order for these tests
	is the lesser of T/2 and
	T0.6.  Again, this value can be
	adjusted using the bandwidth parameter.

Caminho de Menu:    /Vari�vel/Spectrum

Acesso alternativo: Menu de contexto da janela principal (selec��o singular)

\subsection{poisson}
\hypertarget{cmd-poisson}{}

\begin{tabular}{ll}
Argumentos:  & \textsl{vari�vel-dependente} \textsl{vari�veis-independentes} \texttt{[} \texttt{;} \textsl{offset} \texttt{]} \\
Op��es:      & \verb|--vcv| (mostrar matriz de covari�ncia) \\
 & \verb|--verbose| (mostrar detalhes das itera��es) \\
Exemplos:    & \texttt{poisson y 0 x1 x2} \\ 
 & \texttt{poisson y 0 x1 x2 ; S} \\ 
\end{tabular}

Estimates a poisson regression.  The dependent variable is
	taken to represent the occurrence of events of some sort, and
	must take on only non-negative integer values.

If a discrete random variable Y follows
      the Poisson distribution, then
        \[\mathrm{Pr}(Y = y) = \frac{e^{-v} v^y}{y!}\]
	for y = 0, 1,
      2,\dots{}.  The mean and variance of the distribution are both
      equal to v.  In the Poisson regression model,
      the parameter v is represented as a function
      of one or more independent variables.  The most common version
      (and the only one supported by gretl) has
        \[v = \mathrm{exp}(\beta_0+\beta_1 x_1+\beta_2 x_2 + \cdots)\]
	or in other words the log of
      v is a linear function of the independent
      variables.

Optionally, you may add an ``offset'' variable
	to the specification.  This is a scale variable, the log of
	which is added to the linear regression function (implicitly,
	with a coefficient of 1.0).  This makes sense if you expect
	the number of occurrences of the event in question to be
	proportional, other things equal, to some known factor.  For
	example, the number of traffic accidents might be supposed to
	be proportional to traffic volume, other things equal, and in
	that case traffic volume could be specified as an
	``offset'' in a Poisson model of the accident rate.
	The offset variable must be strictly positive.  

Caminho de Menu:    /Model/Nonlinear models/Poisson

\subsection{plot}
\hypertarget{cmd-plot}{}

\begin{tabular}{ll}
Argumento:   & \textsl{lista-de-vari�veis} \\
Op��o:       & \verb|--one-scale| (force a single scale) \\
\end{tabular}

	Plots the values for specified variables, for the range of
	observations currently in effect, using ASCII symbols.  Each
	line stands for an observation and the values are plotted
	horizontally.  By default the variables are scaled
	appropriately.  See also \hyperlink{cmd-gnuplot}{gnuplot}.

\subsection{print}
\hypertarget{cmd-print}{}

\begin{tabular}{ll}
Argumentos:  & \textsl{lista-de-vari�veis}  ou \textsl{string-literal} \\
Op��es:      & \verb|--byobs| (by observations) \\
 & \verb|--long| (use 10 significant digits or more) \\
 & \verb|--no-dates| (use simple observation numbers) \\
Exemplos:    & \verb@print x1 x2 --byobs@ \\ 
 & \texttt{print "This is a string"} \\ 
\end{tabular}

	If \textsl{lista-de-vari�veis} is given, prints the values of the
	specified variables; if no list is given, prints the values of
	all variables in the current data file. If the
	\verb@--byobs@ flag is given the data are printed by
	observation, otherwise they are printed by variable.

	If the \verb@--long@ flag is given the data are printed, by
	variable, to greater than usual precision.  The default in this
	case is to show 10 significant digits but you can adjust that
	figure using the \hyperlink{cmd-set}{set} command.

	If the \verb@--byobs@ flag is given and the data are
	printed by observation, the default is to show the date (with
	time-series data) or the observation marker string (if any) at
	the start of each line.  The \verb@--no-dates@ option
	suppresses the printing of dates or markers; a simple
	observation number is shown instead.

	If the argument to print is a literal string (which
	must start with a double-quote, \texttt{"}), the string is
	printed as is.  See also \hyperlink{cmd-printf}{printf}.

	Note: a special ``hack'' is available with this
	command, in conjunction with the \verb@--byobs@ flag, which
	can be useful when working with missing values in a data set.  If
	you give a list of variables followed by a semi-colon, followed by
	one final variable, then the final variable is not printed but is
	used to screen the observations to print.  Any observations for
	which the screening variable has value 0 are not printed.  As an
	example of use, suppose you have a daily time series
	\texttt{x}, and you want a list of the dates for which
	\texttt{x} is missing.  You can do

\begin{code}
	genr filt = missing(x)
	print x ; filt --byobs
\end{code}
Caminho de Menu:    /Data/Display values

\subsection{printf}
\hypertarget{cmd-printf}{}

\begin{tabular}{ll}
Argumentos:  & \textsl{format} \textsl{args} \\
\end{tabular}

	Prints scalar values and/or strings under the control of a
	format string (providing a small subset of the
	\texttt{printf()} statement in the C programming language).
	Recognized numeric formats are \verb@%e@, \verb@%E@,
	\verb@%f@, \verb@%g@, \verb@%G@ and \verb@%d@,
	in each case with the various modifiers available in C.
	Examples: the format \verb@%.10g@ prints a value to 10
	significant figures; \verb@%12.6f@ prints a value to 6
	decimal places, with a width of 12 characters.  The format
	\verb@%s@ should be used for strings.

The format string itself must be enclosed in double
	quotes.  The values to be printed must follow the format
	string, separated by commas.  These values should take the
	form of either (a) the names of variables in the dataset, (b)
	expressions that are valid for the genr command, or
	(c) the special functions \texttt{varname()} or
	\texttt{date()}.  The following example prints the values of
	two variables plus that of a calculated expression:

\begin{code}
	ols 1 0 2 3
	genr b = $coeff(2)
	genr se_b = $stderr(2)
	printf "b = %.8g, standard error %.8g, t = %.4f\n", b, se_b, b/se_b
\end{code}

	The next lines illustrate the use of the varname and date
	functions, which respectively print the name of a variable,
	given its ID number, and a date string, given a 1-based
	observation number.

\begin{code}
	printf "The name of variable %d is %s\n", i, varname(i)
	printf "The date of observation %d is %s\n", j, date(j)
\end{code}

	The maximum length of a format string is 127 characters.  The
	escape sequences \verb@\n@ (newline), \verb@\t@ (tab),
	\verb@\v@ (vertical tab) and \verb@\\@ (literal
	backslash) are recognized.  To print a literal percent sign,
	use \verb@%%@.

\subsection{probit}
\hypertarget{cmd-probit}{}

\begin{tabular}{ll}
Argumentos:  & \textsl{vari�vel-dependente} \textsl{vari�veis-independentes} \\
Op��es:      & \verb|--robust| (robust standard errors) \\
 & \verb|--vcv| (mostrar matriz de covari�ncia) \\
 & \verb|--verbose| (mostrar detalhes das itera��es) \\
\end{tabular}

	If the dependent variable is a binary variable (all values are
	0 or 1) maximum likelihood estimates of the coefficients on
	\textsl{vari�veis-independentes} are obtained via the ``binary
	  response model regression'' (BRMR) method outlined by
	Davidson and MacKinnon (2004). As the model is nonlinear the
	slopes depend on the values of the independent variables: the
	reported slopes are evaluated at the means of those variables.
	The chi-square statistic tests the null hypothesis that all
	coefficients are zero apart from the constant.

	By default, standard errors are computed using the negative
	inverse of the Hessian.  If the \verb@--robust@ flag is given,
	then QML or Huber--White standard errors are calculated
	instead. In this case the estimated covariance matrix is a
	``sandwich'' of the inverse of the estimated Hessian
	and the outer product of the gradient. See Davidson and MacKinnon
	(2004, Chapter 10) for details.

	If the dependent variable is not binary, but is discrete and
	has a minimum value of 0, then Ordered Probit estimates are
	obtained.  In this case robust standard errors are not yet
	available.  (If the variable selected as dependent is not
	discrete, or does not have a minimum of zero, an error is
	flagged.)

	Probit for analysis of proportions is not implemented in
	gretl at this point.

Caminho de Menu:    /Model/Nonlinear models/Probit

\subsection{pvalue}
\hypertarget{cmd-pvalue}{}

\begin{tabular}{ll}
Argumentos:  & \textsl{dist} \texttt{[} \textsl{params} \texttt{]} \textsl{xval} \\
Exemplos:    & \texttt{pvalue z zscore} \\ 
 & \texttt{pvalue t 25 3.0} \\ 
 & \texttt{pvalue X 3 5.6} \\ 
 & \texttt{pvalue F 4 58 fval} \\ 
 & \texttt{pvalue G xbar varx x} \\ 
 & \texttt{pvalue B bprob 10 6} \\ 
\end{tabular}

	Computes the area to the right of \textsl{xval} in the
	specified distribution (\texttt{z} for Gaussian,
	\texttt{t} for Student's t, \texttt{X}
	for chi-square, \texttt{F} for F,
	\texttt{G} for gamma, or \texttt{B} for binomial).  

	For the t and chi-square distributions the
	degrees of freedom must be given; for F
	numerator and denominator degrees of freedom are required; for
	gamma the mean and variance are needed; and for the binomial
	distribution the ``success'' probability and the
	number of trials must be given.  In each case, these extra
	values are provided before the \textsl{xval}.

	As shown in the examples above, the numerical parameters
	may be given in numeric form or as the names of variables.

Caminho de Menu:    /Tools/P-value finder

\subsection{qlrtest}
\hypertarget{cmd-qlrtest}{}

	For a model estimated on time-series data via OLS, performs the
	Quandt likelihood ratio (QLR) test for a structural break at an
	unknown point in time, with 15 percent trimming at the beginning
	and end of the sample period.

	For each potential break point within the central 70 percent of
	the observations, a Chow test is performed (see \hyperlink{cmd-chow}{chow}). The QLR test statistic is the maximum of the
	F values from these tests.  It follows a
	non-standard distribution, the critical values of which are taken
	from Stock and Watson's \emph{Introduction to Econometrics}
	(2003). If the QLR statistic exceeds the critical value at the
	chosen level of significance, one can infer that the parameters of
	the model are not constant.  This statistic can be used to detect
	forms of instability other than a single discrete break (such as
	multiple breaks or a slow drifting of the parameters). 

Caminho de Menu:    Janela do modelo, /Testes/Teste QLR test

\subsection{quantreg}
\hypertarget{cmd-quantreg}{}

\begin{tabular}{ll}
Argumentos:  & \textsl{tau} \textsl{depvar} \textsl{indepvars} \\
Op��es:      & \verb|--robust| (robust standard errors) \\
 & \verb|--intervals|\texttt{[}=\textsl{level}\texttt{]} (compute confidence intervals) \\
 & \verb|--vcv| (print covariance matrix) \\
Exemplos:    & \texttt{quantreg 0.25 y 0 xlist} \\ 
 & \verb@quantreg 0.5 y 0 xlist --intervals@ \\ 
 & \verb@quantreg 0.5 y 0 xlist --intervals=.95@ \\ 
 & \verb@quantreg tauvec y 0 xlist --robust@ \\ 
 & Ver tamb�m\verb@mrw_qr.inp@
\end{tabular}

	Quantile regression.  The first argument, \textsl{tau}, is the
	conditional quantile for which estimates are wanted.  It may be given
	either as a numerical value or as the name of a pre-defined scalar
	variable; the value must be in the range 0.01 to 0.99. (Alternatively,
	a vector of values may be given for \textsl{tau}; see below for
	details.) The second and subsequent arguments compose a regression
	list on the same pattern as \hyperlink{cmd-ols}{ols}.

	Without the \verb@--intervals@ option, standard errors are printed
	for the quantile estimates.  By default, these are computed according
	to the asymptotic formula given by Koenker and Bassett (1978), but if
	the \verb@--robust@ option is given, standard errors that are
	robust with respect to heteroskedasticity are calculated using the
	method of Koenker and Zhao (1994).

	When the \verb@--intervals@ option is chosen, confidence intervals
	are given for the parameter estimates instead of standard errors.
	These intervals are computed using the rank inversion method, and in
	general they are asymmetrical about the point estimates.  The specifics
	of the calculation are inflected by the \verb@--robust@ option:
	without this, the intervals are computed on the assumption of IID
	errors (Koenker, 1994); with it, they use the robust estimator
	developed by Koenker and Machado (1999).

	By default, 90 percent confidence intervals are produced.  You can
	change this by appending a confidence level (expressed as a decimal
	fraction) to the intervals option, as in \verb@--intervals=0.95@.

	Vector-valued \textsl{tau}:  instead of supplying a scalar, you
	may give the name of a pre-defined matrix.  In this case estimates are
	computed for all the given \textsl{tau} values and the results are
	printed in a special format, showing the sequence of quantile
	estimates for each regressor in turn.

Caminho de Menu:    /Model/Robust estimation/Quantile regression

\subsection{quit}
\hypertarget{cmd-quit}{}

	Exits from the program, giving you the option of saving the
	output from the session on the way out.  

Caminho de Menu:    /Ficheiro/Exit

\subsection{rename}
\hypertarget{cmd-rename}{}

\begin{tabular}{ll}
Variantes:   & \texttt{rename} \textsl{varnumber} \textsl{newname}\\
 & \texttt{rename} \textsl{varname} \textsl{newname}\\
\end{tabular}

Changes the name of the variable with identification number
	\textsl{varnumber} or current name \textsl{varname} to
	\textsl{newname}.  The new name must be of 15 characters maximum,
	must start with a letter, and must be composed of only letters, digits,
	and the underscore character.

Caminho de Menu:    /Vari�vel/Edit attributes

Acesso alternativo: Menu de contexto da janela principal (selec��o singular)

\subsection{reset}
\hypertarget{cmd-reset}{}

	Must follow the estimation of a model via OLS. Carries out
	Ramsey's RESET test for model specification (non-linearity) by
	adding the square and the cube of the fitted values to the
	regression and calculating the F statistic
	for the null hypothesis that the parameters on the two added
	terms are zero.

Caminho de Menu:    Janela do modelo, /Testes/Ramsey's RESET

\subsection{restrict}
\hypertarget{cmd-restrict}{}

	Imposes a set of linear restrictions on either (a) the model
	last estimated or (b) a system of equations previously defined
	and named.  The syntax and effects of the command differ
	slightly in the two cases.

	In both cases the set of restrictions should be started with
	the keyword ``restrict'' and terminated with
	``end restrict''.  In the single equation case the
	restrictions are implicitly to be applied to the last model,
	and they are evaluated as soon as the \texttt{restrict}
	command is terminated.  In the system case the initial
	``restrict'' must be followed by the name of a
	previously defined system of equations (see \hyperlink{cmd-system}{system}).  The restrictions are evaluated when
	the system is next estimated, using the \hyperlink{cmd-estimate}{estimate} command.

	Each restriction in the set should be expressed as an equation,
	with a linear combination of parameters on the left and a numeric
	value to the right of the equals sign. In the single-equation
	case, parameters may be referenced in the form
	\texttt{b[}\textsl{i}\texttt{]}, where \textsl{i}
	represents the position in the list of regressors (starting at 1),
	or \texttt{b[}\textsl{varname}\texttt{]}, where
	\textsl{varname} is the name of the regressor in question. In
	the system case, parameters are referenced using \texttt{b} plus
	two numbers in square brackets. The leading number represents the
	position of the equation within the system and the second number
	indicates position in the list of regressors.  For example
	\texttt{b[2,1]} denotes the first parameter in the second
	equation, and \texttt{b[3,2]} the second parameter in the third
	equation.

The \texttt{b} terms in the equation representing a
	restriction equation may be prefixed with a numeric
	multiplier, using \texttt{*} to represent multiplication,
	for example \texttt{3.5*b[4]}.

Here is an example of a set of restrictions for a
	previously estimated model:

\begin{code}
	restrict
	 b[1] = 0
	 b[2] - b[3] = 0
	 b[4] + 2*b[5] = 1
	end restrict
\end{code}
And here is an example of a set of restrictions to be
	applied to a named system.  (If the name of the system does
	not contain spaces, the surrounding quotes are not required.)

\begin{code}
	restrict "System 1"
	 b[1,1] = 0
	 b[1,2] - b[2,2] = 0
	 b[3,4] + 2*b[3,5] = 1
	end restrict
\end{code}

	In the single-equation case the restrictions are evaluated via
	a Wald F-test, using the coefficient covariance
	matrix of the model in question.  By default, the restricted
	coefficient estimates are printed; if you just want the test
	statistic, you can append the \verb@--quiet@ option flag
	to the initial \texttt{restrict} command.

	In the system case, the test statistic depends on the estimator
	chosen: a Likelihood Ratio test if the system is estimated using a
	Maximum Likelihood method, or an asymptotic
	F-test otherwise.

Caminho de Menu:    Janela do modelo, /Testes/Linear restrictions

\subsection{rhodiff}
\hypertarget{cmd-rhodiff}{}

\begin{tabular}{ll}
Argumentos:  & \textsl{rholist} \texttt{;} \textsl{lista-de-vari�veis} \\
Exemplos:    & \texttt{rhodiff .65 ; 2 3 4} \\ 
 & \texttt{rhodiff r1 r2 ; x1 x2 x3} \\ 
\end{tabular}

	Creates rho-differenced counterparts of the variables (given
	by number or by name) in \textsl{lista-de-vari�veis} and adds them to
	the data set, using the suffix \verb@#@ for the new
	variables. Given variable \texttt{v1} in
	\textsl{lista-de-vari�veis}, and entries \texttt{r1} and
	\texttt{r2} in \textsl{rholist}, the new variable

\begin{code}
	v1# = v1 - r1*v1(-1) - r2*v1(-2)
      \end{code}

	is created. The \textsl{rholist} entries can be given as numerical
	values or as the names of variables previously defined.

\subsection{rmplot}
\hypertarget{cmd-rmplot}{}

\begin{tabular}{ll}
Argumento:   & \textsl{nome-de-vari�vel} \\
\end{tabular}

	Range--mean plot: this command creates a simple graph to
	help in deciding whether a time series,
	y(t), has constant variance or not.  We
	take the full sample t=1,...,T and divide it into small
	subsamples of arbitrary size k. The first
	subsample is formed by
	y(1),...,y(k), the
	second is y(k+1), ...,
	y(2k), and so on.  For each subsample we
	calculate the sample mean and range (= maximum minus minimum),
	and we construct a graph with the means on the horizontal axis
	and the ranges on the vertical. So each subsample is
	represented by a point in this plane.  If the variance of the
	series is constant we would expect the subsample range to be
	independent of the subsample mean; if we see the points
	approximate an upward-sloping line this suggests the variance
	of the series is increasing in its mean; and if the points
	approximate a downward sloping line this suggests the variance
	is decreasing in the mean.

Besides the graph, gretl displays the means and ranges for
	each subsample, along with the slope coefficient for an OLS
	regression of the range on the mean and the p-value for the
	null hypothesis that this slope is zero.  If the slope
	coefficient is significant at the 10 percent significance
	level then the fitted line from the regression of range on
	mean is shown on the graph.

Caminho de Menu:    /Vari�vel/Range-mean graph

\subsection{run}
\hypertarget{cmd-run}{}

\begin{tabular}{ll}
Argumento:   & \textsl{inputfile} \\
\end{tabular}

	Execute the commands in \textsl{inputfile} then return
	control to the interactive prompt.  This command is intended
	for use with the command-line program gretlcli, or at the
	``gretl console'' in the GUI program.

	See also \hyperlink{cmd-include}{include}.

Caminho de Menu:    Run icon in script window

\subsection{runs}
\hypertarget{cmd-runs}{}

\begin{tabular}{ll}
Argumento:   & \textsl{nome-de-vari�vel} \\
\end{tabular}

	Carries out the nonparametric ``runs'' test for
	randomness of the specified variable.  If you want to test for
	randomness of deviations from the median, for a variable named
	\texttt{x1} with a non-zero median, you can do the
	following:

\begin{code}
	genr signx1 = x1 - median(x1)
	runs signx1
\end{code}
Caminho de Menu:    /Vari�vel/Runs test

\subsection{scatters}
\hypertarget{cmd-scatters}{}

\begin{tabular}{ll}
Argumentos:  & \textsl{yvar} \texttt{;} \textsl{xvarlist}  ou \textsl{yvarlist ; xvar} \\
Op��o:       & \verb|--with-lines| (create line graphs) \\
Exemplos:    & \texttt{scatters 1 ; 2 3 4 5} \\ 
 & \texttt{scatters 1 2 3 4 5 6 ; 7} \\ 
 & \verb@scatters y1 y2 y3 ; x --with-lines@ \\ 
\end{tabular}

	Generates pairwise graphs of \textsl{yvar} against all the
	variables in \textsl{xvarlist}, or of all the variables in
	\textsl{yvarlist} against \textsl{xvar}.  The first
	example above puts variable 1 on the y-axis
	and draws four graphs, the first having variable 2 on the
	x-axis, the second variable 3 on the
	x-axis, and so on.  The second example
	plots each of variables 1 through 6 against variable 7 on the
	x-axis. Scanning a set of such plots can be
	a useful step in exploratory data analysis.  The maximum
	number of plots is six; any extra variable in the list will be
	ignored.

	By default the graphs are scatterplots, but if you give the
	\verb@--with-lines@ flag they will be line graphs.

Caminho de Menu:    /View/Multiple graphs

\subsection{sdiff}
\hypertarget{cmd-sdiff}{}

\begin{tabular}{ll}
Argumento:   & \textsl{lista-de-vari�veis} \\
\end{tabular}

	The seasonal difference of each variable in \textsl{lista-de-vari�veis} is
	obtained and the result stored in a new variable with the prefix
	\verb@sd_@.  This command is available only for seasonal time
	series.  

Caminho de Menu:    /Add/Seasonal differences of selected variables

\subsection{set}
\hypertarget{cmd-set}{}

\begin{tabular}{ll}
Argumentos:  & \textsl{vari�vel} \textsl{value} \\
Exemplos:    & \texttt{set qr on} \\ 
 & \verb@set csv_delim tab@ \\ 
 & \texttt{set horizon 10} \\ 
\end{tabular}

	Set the values of various program parameters.  The given value remains
	in force for the duration of the gretl session unless it is changed by a
	further call to set.  The parameters that can be set in this
	way are enumerated below. Note that the settings of \verb@hac_lag@
	and \verb@hc_version@ are used when the \verb@--robust@ option
	is given to the ols command. 

	If the set command is given without any parameters, the
	current settings for all the relevant variables are printed.

\begin{itemize}
\item \texttt{echo}: \texttt{off} or \texttt{on} (the
	    default). Suppress or resume the echoing of commands in gretl's
	    output.

\item \texttt{messages}: \texttt{off} or \texttt{on} (the
	  default). Suppress or resume the printing of non-error messages
	  associated with various commands, for example when a new variable is
	  generated or when the sample range is changed.

\item 
	  \verb@nls_toler@: a floating-point value (the default is the
	  machine precision to the power 3/4).  Sets the tolerance used in
	  judging whether or not convergence has occurred in nonlinear least
	  squares estimation using the \hyperlink{cmd-nls}{nls} command.

\item \texttt{qr}: \texttt{on} or \texttt{off} (the
	    default). Use QR rather than Cholesky decomposition in calculating
	    OLS estimates.

\item \texttt{seed}: an unsigned integer.  Sets the seed for
	    the pseudo-random number generator.  By default this is set from the
	    system time; if you want to generate repeatable sequences of random
	    numbers you must set the seed manually.

\item \verb@hac_lag@: \texttt{nw1} (the default) or
	    \texttt{nw2}, or an integer.  Sets the maximum lag value,
	    p, used when calculating HAC (Heteroskedasticity
	    and Autocorrelation Consistent) standard errors using the Newey-West
	    approach, for time series data. \texttt{nw1} and \texttt{nw2}
	    represent two variant automatic calculations based on the sample
	    size, T: for nw1, 
	  $p = 0.75 \times T^{1/3}$, and for nw2, 
	  $p = 4 \times (T/100)^{2/9}$.	  

\item 
	    \verb@hc_version@: 0 (the default), 1, 2 or 3. Sets the
	    variant used when calculating Heteroskedasticity Consistent standard
	    errors with cross-sectional data.  The options correspond to the
	    HC0, HC1, HC2 and HC3 discussed by Davidson and MacKinnon in
	    \emph{Econometric Theory and Methods}, chapter 5.  HC0
	    produces what are usually called ``White's standard
	      errors''.

\item \verb@force_hc@: \texttt{off} (the default) or
	    \texttt{on}.  By default, with time-series data and when the
	    \verb@--robust@ option is given with \texttt{ols}, the HAC
	    estimator is used.  If you set \verb@force_hc@ to
	    ``on'', this forces calculation of the regular
	    Heteroskedasticity Consistent Covariance Matrix (which does not take
	    autocorrelation into account).

\item \verb@garch_vcv@: \texttt{unset},
	    \texttt{hessian}, \texttt{im} (information matrix) ,
	    \texttt{op} (outer product matrix), \texttt{qml} (QML
	    estimator), \texttt{bw} (Bollerslev--Wooldridge). Specifies
	    the variant that will be used for estimating the coefficient
	    covariance matrix, for GARCH models.  If \texttt{unset} is given
	    (the default) then the Hessian is used unless the
	    ``robust'' option is given for the garch command, in
	    which case QML is used.

\item \verb@hp_lambda@: \texttt{auto} (the default), or a
	    numerical value.  Sets the smoothing parameter for the
	    Hodrick--Prescott filter (see the \texttt{hpfilt} function
	    under the \texttt{genr} command).  The default is to use 100 times
	    the square of the periodicity, which gives 100 for annual data, 1600
	    for quarterly data, and so on.

\item \verb@bkbp_limits@: two integers, the second greater
	    than the first (the defaults are 8 and 32).  Sets the frequency
	    bounds for the Baxter--King bandpass filter (see the
	    \texttt{bkfilt} function under the \texttt{genr} command).

\item \verb@bkbp_k@: one integer (the default is 8).  Sets
	    the approximation order for the Baxter--King bandpass filter.

\item \texttt{horizon}: one integer (the default is based on
	    the frequency of the data).  Sets the horizon for impulse responses
	    and forecast variance decompositions in the context of vector
	    autoregressions.

\item \verb@csv_delim@: either \texttt{comma} (the
	    default), \texttt{space} or \texttt{tab}.  Sets the column
	    delimiter used when saving data to file in CSV format.

\item \verb@bhhh_maxiter@: one integer, maximum number of iterations
	    for gretl's internal BHHH routine, which is used in the
	    arma command for conditional ML estimation. If convergence 
	    is not achieved after \verb@bhhh_maxiter@, the program returns an
	    error. The default is set at 500.

\item \verb@bhhh_toler@: one floating point value, or the string
	    \texttt{default}.  This is used in gretl's internal BHHH routine
	    to check if convergence has occurred. The algorithm stops iterating
	    as soon as the increment in the log-likelihood between iterations is
	    smaller than \verb@bhhh_toler@.  The default value is
	    1.0E$-$06; this value may be re-established by typing
	    \texttt{default} in place of a numeric value.

\item \texttt{longdigits}: one positive integer value, less
	    than or equal to 20.  Determines the number of digits of
	    precision used when printing the values of variables using the
	    \verb@--long@ option (see \hyperlink{cmd-print}{print}).

\item \texttt{initvals}: a pre-specified matrix. Allows manual
	    setting of the initial parameter estimates for ARMA
	    estimation. For details see \GUG{}.

\end{itemize}

\subsection{setinfo}
\hypertarget{cmd-setinfo}{}

\begin{tabular}{ll}
Argumentos:  & \textsl{nome-de-vari�vel} \texttt{-d }\textsl{description} \texttt{-n }\textsl{displayname} \\
Op��es:      & \verb|--discrete| (mark variable as discrete) \\
 & \verb|--continuous| (mark variable as continuous) \\
Exemplos:    & \texttt{setinfo x1 -d "Description of x1" -n "Graph name"} \\ 
 & \verb@setinfo z --discrete@ \\ 
\end{tabular}

	Sets up to three attributes of the named variable, as follows.

	If the \texttt{-d} flag is given followed by a string in double
	quotes, that string is used to set the variable's descriptive
	label. This label is shown in response to the \hyperlink{cmd-labels}{labels} command, and is also shown in the main window of
	the GUI program.

	If the \texttt{-n} flag is given followed by a quoted string,
	that string is used to set the variable's ``display
	  name'', which is then used in place of the variable's name
	in graphs.

	If one or other of the \verb@--discrete@ or
	\verb@--continuous@ option flags is given, the variable's
	numerical character is set accordingly.  The default is to treat
	all variables as continuous; setting a variable as discrete
	affects the way the variable is handled in frequency plots.

Caminho de Menu:    /Vari�vel/Edit attributes

Acesso alternativo: Menu de contexto da janela principal

\subsection{setobs}
\hypertarget{cmd-setobs}{}

\begin{tabular}{ll}
Variantes:   & setobs \textsl{periodicity} \textsl{startobs}\\
 & setobs \textsl{unitvar} \textsl{timevar}\\
Op��es:      & \verb|--cross-section| (interpret as cross section) \\
 & \verb|--time-series| (interpret as time series) \\
 & \verb|--stacked-cross-section| (interpret as panel data) \\
 & \verb|--stacked-time-series| (interpret as panel data) \\
 & \verb|--panel-vars| (use index variables (see below)) \\
Exemplos:    & \verb@setobs 4 1990:1 --time-series@ \\ 
 & \texttt{setobs 12 1978:03} \\ 
 & \verb@setobs 1 1 --cross-section@ \\ 
 & \verb@setobs 20 1:1 --stacked-time-series@ \\ 
 & \verb@setobs unit year --panel-vars@ \\ 
\end{tabular}

	Force the program to interpret the current data set as having
	a specified structure.  

	In the first form of the command the \textsl{periodicity},
	which must be an integer, represents frequency in the case of
	time-series data (1 = annual; 4 = quarterly; 12 = monthly; 52 =
	weekly; 5, 6, or 7 = daily; 24 = hourly).  In the case of panel
	data the periodicity means the number of lines per data block:
	this corresponds to the number of cross-sectional units in the
	case of stacked cross-sections, or the number of time periods in
	the case of stacked time series.  In the case of simple
	cross-sectional data the periodicity should be set to 1.

	The starting observation represents the starting date in the
	case of time series data.  Years may be given with two or four
	digits; subperiods (for example, quarters or months) should be
	separated from the year with a colon.  In the case of panel
	data the starting observation should be given as 1:1; and in
	the case of cross-sectional data, as 1.  Starting observations
	for daily or weekly data should be given in the form YY/MM/DD
	or YYYY/MM/DD (or simply as 1 for undated data).  

	The second form of the command (which requires the
	\verb@--panel-vars@ flag) may be used to impose a panel
	interpretation when the data set contains variables that uniquely
	identify the cross-sectional units and the time periods.  The data
	set will be sorted as stacked time series, by ascending values of
	the units variable, \textsl{unitvar}.

	If no explicit option flag is given to indicate the structure
	of the data the program will attempt to guess the structure
	from the information given.

Caminho de Menu:    /Data/Dataset structure

\subsection{setmiss}
\hypertarget{cmd-setmiss}{}

\begin{tabular}{ll}
Argumentos:  & \textsl{value} \texttt{[} \textsl{lista-de-vari�veis} \texttt{]} \\
Exemplos:    & \texttt{setmiss -1} \\ 
 & \texttt{setmiss 100 x2} \\ 
\end{tabular}

	Get the program to interpret some specific numerical data
	value (the first parameter to the command) as a code for
	``missing'', in the case of imported data.  If this
	value is the only parameter, as in the first example above,
	the interpretation will be applied to all series in the data
	set.  If \textsl{value} is followed by a list
	of variables, by name or number, the interpretation is
	confined to the specified variable(s). Thus in the second
	example the data value 100 is interpreted as a code for
	``missing'', but only for the variable
	\texttt{x2}.

Caminho de Menu:    /Sample/Set missing value code

\subsection{shell}
\hypertarget{cmd-shell}{}

\begin{tabular}{ll}
Argumento:   & \textsl{shellcommand} \\
Exemplos:    & \texttt{! ls -al} \\ 
 & \texttt{! notepad} \\ 
 & \texttt{launch notepad} \\ 
\end{tabular}

	A !, or the keyword launch, at the beginning
	of a command line is interpreted as an escape to the user's shell.
	Thus arbitrary shell commands can be executed from within
	gretl.  When ! is used, the external
	command is executed synchronously.  That is,
	gretl waits for it to complete before
	proceeding.  If you want to start another program from within
	gretl and not wait for its completion
	(asynchronous operation), use launch instead.

	For reasons of security this facility is not enabled by default.
	To activate it, check the box titled ``Allow shell
	commands'' under the File, Preferences menu in the GUI
	program.  This also makes shell commands available in the
	command-line program (and is the only way to do so).  

\subsection{smpl}
\hypertarget{cmd-smpl}{}

\begin{tabular}{ll}
Variantes:   & \texttt{smpl} \textsl{startobs endobs}\\
 & \texttt{smpl} \textsl{+i -j}\\
 & \texttt{smpl} \textsl{dumvar} \verb@--dummy@\\
 & \texttt{smpl} \textsl{condition} \verb@--restrict@\\
 & \texttt{smpl} \verb@--no-missing [ @\textsl{lista-de-vari�veis} \texttt{]}\\
 & \texttt{smpl} \textsl{n} \verb@--random@\\
 & \texttt{smpl full}\\
Exemplos:    & \texttt{smpl 3 10} \\ 
 & \texttt{smpl 1960:2 1982:4} \\ 
 & \texttt{smpl +1 -1} \\ 
 & \verb@smpl x > 3000 --restrict@ \\ 
 & \verb@smpl y > 3000 --restrict --replace@ \\ 
 & \verb@smpl 100 --random@ \\ 
\end{tabular}

	Resets the sample range.  The new range can be defined in
	several ways.  In the first alternate form (and the first two
	examples) above, \textsl{startobs} and \textsl{endobs}
	must be consistent with the periodicity of the data.  Either
	one may be replaced by a semicolon to leave the value
	unchanged.  In the second form, the integers \textsl{i}
	and \textsl{j} (which may be positive or negative, and
	should be signed) are taken as offsets relative to the
	existing sample range. In the third form \textsl{dummyvar}
	must be an indicator variable with values 0 or 1 at each
	observation; the sample will be restricted to observations
	where the value is 1. The fourth form, using
	\verb@--restrict@, restricts the sample to observations
	that satisfy the given Boolean condition (which is specified
	according to the syntax of the \hyperlink{cmd-genr}{genr}
	command).

With the \verb@--no-missing@ form, if
	\textsl{lista-de-vari�veis} is specified observations are selected on
	condition that all variables in \textsl{lista-de-vari�veis} have
	valid values at that observation; otherwise, if no
	\textsl{lista-de-vari�veis} is given, observations are selected on
	condition that \emph{all} variables have valid
	(non-missing) values.

With the \verb@--random@ flag, the specified number of
	cases are selected from the full dataset at random.  If you
	wish to be able to replicate this selection you should 
	set the seed for the random number generator first (see the
	\hyperlink{cmd-set}{set} command).

The final form, \texttt{smpl full}, restores the full
	data range.

Note that sample restrictions are, by default, cumulative:
	the baseline for any \texttt{smpl} command is the current
	sample. If you wish the command to act so as to replace any
	existing restriction you can add the option flag
	\verb@--replace@ to the end of the command.

The internal variable \texttt{obs} may be used with the
	\verb@--restrict@ form of \texttt{smpl} to exclude
	particular observations from the sample.  For example

\begin{code}
	smpl obs!=4 --restrict
      \end{code}
will drop just the fourth observation. If the data points
	are identified by labels,

\begin{code}
	smpl obs!="USA" --restrict
      \end{code}
will drop the observation with label ``USA''.

	One point should be noted about the \verb@--dummy@,
	\verb@--restrict@ and \verb@--no-missing@ forms of
	\texttt{smpl}: Any ``structural'' information in
	the data file (regarding the time series or panel nature of
	the data) is lost when this command is issued.  You may
	reimpose structure with the \hyperlink{cmd-setobs}{setobs} command.

Please see \GUG{} for further details.

Caminho de Menu:    /Sample

\subsection{spearman}
\hypertarget{cmd-spearman}{}

\begin{tabular}{ll}
Argumentos:  & \textsl{x} \textsl{y} \\
Op��o:       & \verb|--verbose| (print ranked data) \\
\end{tabular}

	Prints Spearman's rank correlation coefficient for the two
	variables x and y. The
	variables do not have to be ranked manually in advance; the
	function takes care of this.

	The automatic ranking is from largest to smallest (i.e.{} the
	largest data value gets rank 1).  If you need to invert this
	ranking, create a new variable which is the negative of the
	original first.  For example:

\begin{code}
	genr altx = -x
	spearman altx y
\end{code}
Caminho de Menu:    /Model/Robust estimation/Rank correlation

\subsection{sprintf}
\hypertarget{cmd-sprintf}{}

\begin{tabular}{ll}
Argumentos:  & \textsl{stringvar} \textsl{format} \textsl{args} \\
\end{tabular}

	This command works exactly like the \hyperlink{cmd-printf}{printf}
	command, printing the given arguments under the control of the
	format string, except that the result is written into the named
	string, \textsl{stringvar}.

\subsection{square}
\hypertarget{cmd-square}{}

\begin{tabular}{ll}
Argumento:   & \textsl{lista-de-vari�veis} \\
Op��o:       & \verb|--cross| (generate cross-products as well as squares) \\
\end{tabular}

	Generates new variables which are squares of the variables in
	\textsl{lista-de-vari�veis} (plus cross-products if the
	\verb@--cross@ option is given).  For example, square
	  x y will generate \verb@sq_x@ = \texttt{x}
	squared, \verb@sq_y@ = \texttt{y} squared and
	(optionally) \verb@x_y@ = \texttt{x} times \texttt{y}.
	If a particular variable is a dummy variable it is not squared
	because we will get the same variable.  

Caminho de Menu:    /Add/Squares of selected variables

\subsection{sscanf}
\hypertarget{cmd-sscanf}{}

\begin{tabular}{ll}
Argumentos:  & \textsl{source} \texttt{, }\textsl{format} \texttt{, }\textsl{args} \\
\end{tabular}

	Scans the string \textsl{source} under the control of
	\textsl{format}, assigning zero or more values to the
	given \textsl{args}. This is a simplifed version of the
	\texttt{sscanf} function in the C programming language.

	\textsl{source} may be either a literal string, enclosed in double
	quotes, or the name of a predefined string variable.
	\textsl{format} is defined similarly to the format string in
	\hyperlink{cmd-printf}{printf} (more on this below). \textsl{args} should
	be a comma-separated list containing the names of pre-defined
	variables: these are the targets of conversion from
	\textsl{source}.  (For those used to C: one can prefix the names
	of numerical variables with \verb@&@ but this is not
	required.)

	Literal	text in \textsl{format} is matched against
	\textsl{source}.  Conversion specifiers start with
	\verb@%@, and recognized conversions include \verb@%f@,
	\verb@%g@ or \verb@%lf@ for floating-point numbers;
	\verb@%d@ for integers; \verb@%s@ for strings; and
	\verb@%m@ for matrices.  You may insert a positive integer
	after the percent sign: this sets the maximum number of characters
	to read for the given conversion (or the maximum number of rows in
	the case of matrix conversion). Alternatively, you can insert a
	literal \texttt{*} after the percent to suppress the conversion
	(thereby skipping any characters that would otherwise have been
	converted for the given type).  For example, \verb@%3d@
	converts the next 3 characters in \textsl{source} to an
	integer, if possible; \verb@%*g@ skips as many characters in
	\textsl{source} as could be converted to a single
	floating-point number.

	Matrix conversion works thus: the scanner reads a line of input
	and counts the (space- or tab-separated) number of numeric fields.
	This defines the number of columns in the matrix.  By default,
	reading then proceeds for as many lines (rows) as contain the same
	number of numeric columns, but the maximum number of rows to read
	can be limited as described above.

	In addition to \verb@%s@ conversion for strings, a simplified
	version of the C format
	\verb@%@\textsl{N}\texttt{[}\textsl{chars}\texttt{]} 
	is available.  In this format \textsl{N} is the maximum number
	of characters to read and \textsl{chars} is a set of
	acceptable characters, enclosed in square brackets: reading stops
	if \textsl{N} is reached or if a character not in
	\textsl{chars} is encountered. The function of
	\textsl{chars} can be reversed by giving a circumflex,
	\verb@^@, as the first character; in that case reading stops
	if a character in the given set is found.  (Unlike C, the hyphen
	does not play a special role in the \textsl{chars} set.)

	If the source string does not (fully) match the format, the number
	of conversions may fall short of the number of arguments given.
	This is not in itself an error so far as gretl is concerned.
	However, you may wish to check the number of conversions
	performed; this is given by the built-in scalar variable
	\verb@$nscan@, whose value is refreshed every time
	\texttt{sscanf} is called.  The number of conversions is also
	printed, in interactive use.

	Some examples follow:

\begin{code}
	scalar x
	scalar y
	sscanf "123456", "%3d%3d", x, y

	sprintf S "1 2 3 4\n5 6 7 8"
	S
	matrix m
	sscanf S, "%m", m
	print m
      \end{code}

\subsection{store}
\hypertarget{cmd-store}{}

\begin{tabular}{ll}
Argumentos:  & \textsl{ficheiro-de-dados} \texttt{[} \textsl{lista-de-vari�veis} \texttt{]} \\
Op��es:      & \verb|--csv| (use CSV format) \\
 & \verb|--omit-obs| (see below, on CSV format) \\
 & \verb|--gnu-octave| (use GNU Octave format) \\
 & \verb|--gnu-R| (use GNU R format) \\
 & \verb|--traditional| (use traditional ESL format) \\
 & \verb|--gzipped| (apply gzip compression) \\
 & \verb|--jmulti| (use JMulti ASCII format) \\
 & \verb|--dat| (use PcGive ASCII format) \\
 & \verb|--database| (use gretl database format) \\
 & \verb|--overwrite| (see below, on database format) \\
\end{tabular}

	Saves either the entire dataset or, if a \textsl{lista-de-vari�veis}
	is supplied, a specified subset of the variables in the
	current dataset, to the file given by
	\textsl{datafile}.

	By default the data are saved in ``native'' gretl
	format, but the option flags permit saving in several
	alternative formats.  CSV (Comma-Separated Values) data may be
	read into spreadsheet programs, and can also be manipulated
	using a text editor.  The formats of
	Octave, R and
	PcGive are designed for use with the
	respective programs.  Gzip compression may be useful for large
	datasets.  See \GUG{} for details on the
	various formats.

	The option flag \verb@--omit-obs@ is applicable only when saving
	data in CSV format.  By default, if the data are time series or panel or
	if the dataset includes specific observation markers, the CSV file
	includes a first column identifying the observations (e.g.{} by date).  If
	the \verb@--omit-obs@ flag is given this column is omitted; only the
	actual data are printed.

	Note that any scalar variables will not be saved
	automatically: if you wish to save scalars you must explicitly
	list them in \textsl{lista-de-vari�veis}.

	The option of saving in gretl database format is intended to help with the
	construction of large sets of series, possibly having mixed frequencies and
	ranges of observations.  At present this option is available only for
	annual, quarterly or monthly time-series data. If you save to a file that
	already exists, the default action is to append the newly saved series to
	the existing content of the database.  In this context it is an error if one
	or more of the variables to be saved has the same name as a variable that is
	already present in the database. The \verb@--overwrite@ flag has the
	effect that, if there are variable names in common, the newly saved variable
	replaces the variable of the same name in the original dataset.

Caminho de Menu:    /Ficheiro/Save data; /File/Export data

\subsection{summary}
\hypertarget{cmd-summary}{}

\begin{tabular}{ll}
Argumento:   & \texttt{[} \textsl{lista-de-vari�veis} \texttt{]} \\
\end{tabular}

	Print summary statistics for the variables in
	\textsl{lista-de-vari�veis}, or for all the variables in the data set
	if \textsl{lista-de-vari�veis} is omitted. Output consists of the
	mean, standard deviation (sd), coefficient of variation (=
	sd/mean), median, minimum, maximum, skewness coefficient, and
	excess kurtosis.

Caminho de Menu:    /View/Summary statistics

Acesso alternativo: Menu de contexto da janela principal

\subsection{system}
\hypertarget{cmd-system}{}

\begin{tabular}{ll}
Variantes:   & \texttt{system method=}\textsl{estimator}\\
 & \texttt{system name=}\textsl{sysname}\\
Argumento:   & \textsl{savevars} \\
Exemplos:    & \texttt{system name="Klein Model 1"} \\ 
 & \texttt{system method=sur} \\ 
 & \texttt{system method=sur save=resids} \\ 
 & \texttt{system method=3sls save=resids,fitted} \\ 
 & Ver tamb�m\texttt{klein.inp}, \texttt{kmenta.inp}
\end{tabular}

	Starts a system of equations.  Either of two forms of the
	command may be given, depending on whether you wish to save
	the system for estimation in more than one way or just
	estimate the system once.

	To save the system you should give it a name, as in the first
	example (if the name contains spaces it must be surrounded by
	double quotes).  In this case you estimate the system using
	the \hyperlink{cmd-estimate}{estimate} command.  With a saved system of
	equations, you are able to impose restrictions (including
	cross-equation restrictions) using the \hyperlink{cmd-restrict}{restrict} command.

	Alternatively you can specify an estimator for the system
	using \texttt{method=} followed by a string identifying one
	of the supported estimators: ols (Ordinary Least
	Squares), tsls (Two-Stage Least Squares)
	sur (Seemingly Unrelated Regressions),
	3sls (Three-Stage Least Squares), fiml
	(Full Information Maximum Likelihood) or liml
	(Limited Information Maximum Likelihood).  In this case the
	system is estimated once its definition is complete.  

An equation system is terminated by the line
	end system.  Within the system four sorts of
	statement may be given, as follows.

\begin{itemize}
\item \hyperlink{cmd-equation}{equation}: specify an equation
	    within the system.  At least two such statements must be
	    provided.

\item instr: for a system to be estimated via
	    Three-Stage Least Squares, a list of instruments (by
	    variable name or number). Alternatively, you can put this
	    information into the equation line using the
	    same syntax as in the \hyperlink{cmd-tsls}{tsls}
	    command.

\item endog: for a system of simultaneous
	    equations, a list of endogenous variables.  This is
	    primarily intended for use with FIML estimation, but with
	    Three-Stage Least Squares this approach may be used
	    instead of giving an instr list; then all the
	    variables not identified as endogenous will be used as
	    instruments.

\item identity: for use with FIML, an identity
	    linking two or more of the variables in the system.  This
	    sort of statement is ignored when an estimator other than
	    FIML is used.

\end{itemize}

	In the optional save= field of the command you can
	specify whether to save the residuals (resids)
	and/or the fitted values (fitted).

	For examples of the specification and estimation of systems of
	equations, please see the scripts
	\texttt{klein.inp},
	\texttt{kmenta.inp} and
	\verb@greene14_2.inp@ (supplied with the gretl
	distribution).

Caminho de Menu:    /Model/Simultaneous equations

\subsection{tabprint}
\hypertarget{cmd-tabprint}{}

\begin{tabular}{ll}
Argumento:   & \texttt{[} \textsl{-f filename} \texttt{]} \\
Op��es:      & \verb|--complete| (Create a complete document) \\
 & \verb@--format="f1|f2|f3|f4"@ (Specify a custom format) \\
\end{tabular}

	Must follow the estimation of a model.  Prints the estimated
	model in the form of a {\LaTeX} table.  If a filename is
	specified using the \texttt{-f} flag output goes to that
	file, otherwise it goes to a file with a name of the form
	\verb@model_N.tex@, where \texttt{N} is the
	number of models estimated to date in the current session. See
	also \hyperlink{cmd-eqnprint}{eqnprint}.

	If the \verb@--complete@ flag is given the {\LaTeX} file is
	a complete document, ready for processing; otherwise it must
	be included in a document.

	If you wish alter the appearance of the tabular output, you can
	specify a custom row format using the \verb@--format@ flag.
	The format string must be enclosed in double quotes and must be
	tied to the flag with an equals sign.  The pattern for the format
	string is as follows.  There are four fields, representing the
	coefficient, standard error, t-ratio and
	p-value respectively.  These fields should be separated by
	vertical bars; they may contain a \texttt{printf}-type
	specification for the formatting of the numeric value in question,
	or may be left blank to suppress the printing of that column
	(subject to the constraint that you can't leave all the columns
	blank).  Here are a few examples:

\begin{code}
	--format="%.4f|%.4f|%.4f|%.4f"
	--format="%.4f|%.4f|%.3f|"
	--format="%.5f|%.4f||%.4f"
	--format="%.8g|%.8g||%.4f"
      \end{code}

	The first of these specifications prints the values in all columns
	using 4 decimal places.  The second suppresses the p-value and
	prints the t-ratio to 3 places.  The third
	omits the t-ratio.  The last one again omits
	the t, and prints both coefficient and standard
	error to 8 significant figures.

	Once you set a custom format in this way, it is remembered and
	used for the duration of the gretl session.  To revert to
	the default format you can use the special variant
	\verb@--format=default@.

Caminho de Menu:    Janela do modelo, /LaTeX

\subsection{testuhat}
\hypertarget{cmd-testuhat}{}

	Must follow a model estimation command.  Gives the frequency
	distribution for the residual from the model along with a
	chi-square test for normality, based on the procedure
	suggested by Doornik and Hansen (1984).

Caminho de Menu:    Janela do modelo, /Testes/Normality of residual

\subsection{tobit}
\hypertarget{cmd-tobit}{}

\begin{tabular}{ll}
Argumentos:  & \textsl{vari�vel-dependente} \textsl{vari�veis-independentes} \\
Op��es:      & \verb|--vcv| (mostrar matriz de covari�ncia) \\
 & \verb|--verbose| (mostrar detalhes das itera��es) \\
\end{tabular}

Estimates a Tobit model.  This model may be appropriate
	when the dependent variable is ``censored''.  For
	example, positive and zero values of purchases of durable
	goods on the part of individual households are observed, and
	no negative values, yet decisions on such purchases may be
	thought of as outcomes of an underlying, unobserved
	disposition to purchase that may be negative in some cases.
	For details see Greene's \emph{Econometric Analysis},
	Chapter 20.

Caminho de Menu:    /Model/Nonlinear models/Tobit

\subsection{tsls}
\hypertarget{cmd-tsls}{}

\begin{tabular}{ll}
Argumentos:  & \textsl{vari�vel-dependente} \textsl{vari�veis-independentes} \texttt{;} \textsl{instruments} \\
Op��es:      & \verb|--vcv| (mostrar matriz de covari�ncia) \\
 & \verb|--robust| (robust standard errors) \\
Exemplo:     & \texttt{tsls y1 0 y2 y3 x1 x2 ; 0 x1 x2 x3 x4 x5 x6} \\ 
\end{tabular}

	Computes two-stage least squares (TSLS or IV) estimates:
	\textsl{depvar} is the dependent variable,
	\textsl{vari�veis-independentes} is the list of independent variables
	(including right-hand side endogenous variables) in the
	structural equation for which TSLS estimates are needed; and
	\textsl{instruments} is the combined list of exogenous and
	predetermined variables in all the equations. If the
	\textsl{instruments} list is not at least as long as
	\textsl{vari�veis-independentes}, the model is not identified.

	In the above example, the \texttt{y}s are the endogenous
	variables and the \texttt{x}s are the exogenous and
	predetermined variables.  

	Output includes the Hausman test and, if the model is
	over-identified, the Sargan over-identification test.  In the
	Hausman test, the null hypothesis is that OLS estimates are
	consistent, or in other words estimation by means of
	instrumental variables is not required.  A model of this sort
	is over-identified if there are more instruments than are
	strictly required.  The Sargan test is based on an auxiliary
	regression of the residuals from the two-stage least squares
	model on the full list of instruments.  The null hypothesis is
	that all the instruments are valid, and suspicion is thrown on
	this hypothesis if the auxiliary regression has a significant
	degree of explanatory power.  Davidson and MacKinnon (2004,
	chapter 8) give a good explanation of both tests.

Caminho de Menu:    /Model/Other linear models/Two-Stage least Squares

\subsection{var}
\hypertarget{cmd-var}{}

\begin{tabular}{ll}
Argumentos:  & \textsl{ordem} \textsl{lista-de-vari�veis} \texttt{[} \texttt{;} \textsl{exolist} \texttt{]} \\
Op��es:      & \verb|--nc| (n�o incluir uma constante) \\
 & \verb|--seasonals| (incluir vari�veis sazonais auxiliares) \\
 & \verb|--robust| (robust standard errors) \\
 & \verb|--impulse-responses| (print impulse responses) \\
 & \verb|--variance-decomp| (print forecast variance decompositions) \\
 & \verb|--lagselect| (show information criteria for lag selection) \\
Exemplos:    & \texttt{var 4 x1 x2 x3 ; time mydum} \\ 
 & \verb@var 4 x1 x2 x3 --seasonals@ \\ 
 & \verb@var 12 x1 x2 x3 --lagselect@ \\ 
\end{tabular}

	Sets up and estimates (using OLS) a vector autoregression (VAR).
	The first argument specifies the lag order --- or the maximum
	lag order in case the \verb@--lagselect@ option is given (see
	below).  The order may be given numerically, or as the name of a
	pre-existing scalar variable. Then follows the setup for the first
	equation.  Don't include lags among the elements of
	\textsl{lista-de-vari�veis} --- they will be added automatically.
	The semi-colon separates the stochastic variables, for which
	\textsl{order} lags will be included, from any exogenous
	variables in \textsl{exolist}.  Note that a constant is
	included automatically unless you give the \verb@--nc@ flag, a
	trend can be added with the \verb@--trend@ flag, and seasonal
	dummy variables may be added using the \verb@--seasonals@
	flag.

	A separate regression is run for each variable in \textsl{lista-de-vari�veis}.
	Output for each equation includes F-tests for zero
	restrictions on all lags of each of the variables, an
	F-test for the significance of the maximum lag, and,
	if the \verb@--impulse-responses@ flag is given, forecast variance
	decompositions and impulse responses.

	Forecast variance decompositions and impulse responses are based on the
	Cholesky decomposition of the contemporaneous covariance matrix, and in
	this context the order in which the (stochastic) variables are given
	matters.  The first variable in the list is assumed to be ``most
	  exogenous'' within-period. The horizon for variance
	decompositions and impulse responses can be set using the \hyperlink{cmd-set}{set} command.

	If the \verb@--lagselect@ option is given, the first parameter to
	the \texttt{var} command is taken as the maximum lag order.  Output
	consists of a table showing the values of the Akaike (AIC), Schwartz
	(BIC) and Hannan--Quinn (HQC) information criteria computed from
	VARs of order 1 to the given maximum.  This is intended to help
	with the selection of the optimal lag order.  The usual VAR output is
	not presented.

Caminho de Menu:    /Modelo/S�rie temporal/Vector autoregression

\subsection{varlist}
\hypertarget{cmd-varlist}{}

	Prints a listing of variables currently available.
	list and ls are synonyms.  

\subsection{vartest}
\hypertarget{cmd-vartest}{}

\begin{tabular}{ll}
Argumentos:  & \textsl{var1} \textsl{var2} \\
\end{tabular}

	Calculates the F statistic for the null
	hypothesis that the population variances for the variables
	\textsl{var1} and \textsl{var2} are equal, and shows
	its p-value.

Caminho de Menu:    /Model/Bivariate tests/Difference of variances

\subsection{vecm}
\hypertarget{cmd-vecm}{}

\begin{tabular}{ll}
Argumentos:  & \textsl{ordem} \textsl{rank} \textsl{lista-de-vari�veis} \\
Op��es:      & \verb|--nc| (sem constante) \\
 & \verb|--rc| (constante restringida) \\
 & \verb|--crt| (constante e tend�ncia restringida) \\
 & \verb|--ct| (constante e tend�ncia n�o restringida) \\
 & \verb|--seasonals| (incluir auxiliares sazonais centradas) \\
 & \verb|--impulse-responses| (print impulse responses) \\
 & \verb|--variance-decomp| (print forecast variance decompositions) \\
Exemplos:    & \texttt{vecm 4 1 Y1 Y2 Y3} \\ 
 & \verb@vecm 3 2 Y1 Y2 Y3 --rc@ \\ 
 & Ver tamb�m\texttt{denmark.inp}, \texttt{hamilton.inp}
\end{tabular}

	A VECM is a form of vector autoregression or VAR (see \hyperlink{cmd-var}{var}), applicable where the variables in the model are
	individually integrated of order 1 (that is, are random walks, with or
	without drift), but exhibit cointegration.  This command is closely
	related to the Johansen test for cointegration (see \hyperlink{cmd-coint2}{coint2}).

	The \textsl{order} parameter to this command represents the lag
	order of the VAR system.  The number of lags in the VECM itself (where
	the dependent variable is given as a first difference) is one less than
	\textsl{order}.

	The \textsl{rank} parameter represents the cointegration rank, or in
	other words the number of cointegrating vectors.  This must be greater
	than zero and less than or equal to (generally, less than) the number of
	endogenous variables given in \textsl{lista-de-vari�veis}.

	\textsl{lista-de-vari�veis} supplies the list of endogenous variables, in
	levels. The inclusion of deterministic terms in the model is controlled
	by the option flags.  The default if no option is specified is to
	include an ``unrestricted constant'', which allows for the
	presence of a non-zero intercept in the cointegrating relations as well
	as a trend in the levels of the endogenous variables.  In the literature
	stemming from the work of Johensen (see for example his 1995 book) this
	is often referred to as ``case 3''.  The first four options
	given above, which are mutually exclusive, produce cases 1, 2, 4 and 5
	respectively.  The meaning of these cases and the criteria for selecting
	a case are explained in \GUG{}.

	The \verb@--seasonals@ option, which may be combined with any of the
	other options, specifies the inclusion of a set of centered seasonal
	dummy variables.  This option is available only for quarterly or monthly
	data.

	The first example above specifies a VECM with lag order 4 and a single
	cointegrating vector.  The endogenous variables are \texttt{Y1},
	\texttt{Y2} and \texttt{Y3}.  The second example uses the same
	variables but specifies a lag order of 3 and two cointegrating vectors;
	it also specifies a ``restricted constant'', which is
	appropriate if the cointegrating vectors may have a non-zero intercept
	but the \texttt{Y} variables have no trend.

Caminho de Menu:    /Modelo/S�rie temporal/VECM

\subsection{vif}
\hypertarget{cmd-vif}{}

	Must follow the estimation of a model which includes at least
	two independent variables. Calculates and displays the
	Variance Inflation Factors (VIFs) for the regressors.  The VIF
	for regressor j is defined as
	\[\frac{1}{1-R_j^2}\] where R\ensuremath{_{j}} is
	the coefficient of multiple correlation between regressor
	j and the other regressors. The factor has
	a minimum value of 1.0 when the variable in question is
	orthogonal to the other independent variables.  Neter,
	Wasserman, and Kutner (1990) suggest inspecting the largest
	VIF as a diagnostic for collinearity; a value greater than 10
	is sometimes taken as indicating a problematic degree of
	collinearity.

Caminho de Menu:    Janela do modelo, /Testes/Collinearity

\subsection{wls}
\hypertarget{cmd-wls}{}

\begin{tabular}{ll}
Argumentos:  & \textsl{wtvar} \textsl{vari�vel-dependente} \textsl{vari�veis-independentes} \\
Op��es:      & \verb|--vcv| (mostrar matriz de covari�ncia) \\
 & \verb|--robust| (robust standard errors) \\
 & \verb|--quiet| (n�o mostrar resultados da regress�o) \\
\end{tabular}

	Computes weighted least squares (WLS) estimates using
	\textsl{wtvar} as the weight, \textsl{depvar} as the
	dependent variable, and \textsl{vari�veis-independentes} as the list of
	independent variables.  Let \textsl{w} denote the positive
	square root of \texttt{wtvar}; then WLS is basically equivalent
	to an OLS regression of \textsl{w} \texttt{*}
	\textsl{depvar} on \textsl{w} \texttt{*}
	\textsl{vari�veis-independentes}.  The \emph{R}-squared,
	however, is calculated in a special manner, namely as
	\[R^2 = 1 - \frac{\rm ESS}{\rm WTSS}\] where ESS is the error sum of squares (sum of
	squared residuals) from the weighted regression and WTSS denotes
	the ``weighted total sum of squares'', which equals the
	sum of squared residuals from a regression of the weighted
	dependent variable on the weighted constant alone.

	If \textsl{wtvar} is a dummy variable, WLS estimation is
	equivalent to eliminating all observations with value zero for
	\textsl{wtvar}.

Caminho de Menu:    /Model/Other linear models/Weighted Least Squares

\subsection{xcorrgm}
\hypertarget{cmd-xcorrgm}{}

\begin{tabular}{ll}
Argumentos:  & \textsl{var1} \textsl{var2} \texttt{[} \textsl{maxlag} \texttt{]} \\
Exemplo:     & \texttt{xcorrgm x y 12} \\ 
\end{tabular}

	Prints and graphs the cross-correlogram for variables
	\textsl{var1} and \textsl{var2}, which may be specified by
	name or number.  The values are the sample correlation coefficients
	between the current value of \textsl{var1} and successive leads
	and lags of \textsl{var2}.

	If a \textsl{maxlag} value is specified the length of the
	cross-correlogram is limited to at most that number of leads and
	lags, otherwise the length is determined automatically, as a
	function of the frequency of the data and the number of
	observations.

Caminho de Menu:    /View/Cross-correlogram

Acesso alternativo: Menu de contexto da janela principal (multiple selection)

\subsection{xtab}
\hypertarget{cmd-xtab}{}

\begin{tabular}{ll}
Argumentos:  & \textsl{ylist} \texttt{;} \textsl{xlist} \\
Op��es:      & \verb|--row| (display row percentages) \\
 & \verb|--column| (display column percentages) \\
 & \verb|--zeros| (display zero entries) \\
\end{tabular}

	With no options given, displays a contingency table or
	cross-tabulation for each variable in \textsl{ylist} (by row)
	against each variable in \textsl{xlist} (by column).
	Variables in these lists can be referenced by name or by number.
	Note that all the variables must have been marked as discrete.

	The \verb@--row@ and \verb@--column@ options are
	mutually exclusive, and instead of the frequency count yield
	the percentages for each row or column, respectively. The
	\verb@--zeros@ option may be useful for importing the table
	into another program, such as a spreadsheet.

	Pearson's chi-square test for independence is displayed if the
	expected frequency under independence is at least 1.0e-7 for all
	cells.  A common rule of thumb for the validity of this statistic
	is that at least 80 percent of cells should have expected
	frequencies of 5 or greater; if this criterion is not met a
	warning is printed.



\section{Commands by topic}
\label{cmd-topics}

The following sections show the available commands grouped by topic.

%% auto-generated from gretl sources
\subsection{Estima��o}
\label{sect-Estimation}

{\small
\begin{longtable}{lp{2in}lp{2in}}
\cmd{\hyperlink{cmd-ar}{ar}} & Estima��o autoregressiva & 
\cmd{\hyperlink{cmd-arbond}{arbond}} & Modelos de painel din�mico \\

\cmd{\hyperlink{cmd-arima}{arima}} & Modelo ARMA & 
\cmd{\hyperlink{cmd-equation}{equation}} & Define equation within a system \\

\cmd{\hyperlink{cmd-estimate}{estimate}} & Estimate system of equations & 
\cmd{\hyperlink{cmd-garch}{garch}} & GARCH model \\

\cmd{\hyperlink{cmd-gmm}{gmm}} & GMM estimation & 
\cmd{\hyperlink{cmd-hccm}{hccm}} & HCCM estimation \\

\cmd{\hyperlink{cmd-heckit}{heckit}} & Heckman selection model & 
\cmd{\hyperlink{cmd-hsk}{hsk}} & Heteroskedasticity-corrected estimates \\

\cmd{\hyperlink{cmd-lad}{lad}} & Least Absolute Deviation estimation & 
\cmd{\hyperlink{cmd-logistic}{logistic}} & Logistic regression \\

\cmd{\hyperlink{cmd-logit}{logit}} & Logit regression & 
\cmd{\hyperlink{cmd-mle}{mle}} & Maximum likelihood estimation \\

\cmd{\hyperlink{cmd-mpols}{mpols}} & Multiple-precision OLS & 
\cmd{\hyperlink{cmd-nls}{nls}} & Nonlinear Least Squares \\

\cmd{\hyperlink{cmd-ols}{ols}} & Ordinary Least Squares & 
\cmd{\hyperlink{cmd-panel}{panel}} & Panel models \\

\cmd{\hyperlink{cmd-poisson}{poisson}} & Poisson estimation & 
\cmd{\hyperlink{cmd-probit}{probit}} & Probit model \\

\cmd{\hyperlink{cmd-system}{system}} & Systems of equations & 
\cmd{\hyperlink{cmd-tobit}{tobit}} & Tobit model \\

\cmd{\hyperlink{cmd-tsls}{tsls}} & Two-Stage Least Squares & 
\cmd{\hyperlink{cmd-var}{var}} & Vector Autoregression \\

\cmd{\hyperlink{cmd-vecm}{vecm}} & Vector Error Correction Model & 
\cmd{\hyperlink{cmd-wls}{wls}} & Weighted Least Squares \\

\end{longtable}
}

\subsection{Testes}
\label{sect-Tests}

{\small
\begin{longtable}{lp{2in}lp{2in}}
\cmd{\hyperlink{cmd-add}{add}} & Acrescentar vari�veis ao modelo & 
\cmd{\hyperlink{cmd-adf}{adf}} & Teste de Dickey-Fuller aumentado \\

\cmd{\hyperlink{cmd-arch}{arch}} & Teste ARCH & 
\cmd{\hyperlink{cmd-chow}{chow}} & Teste de Chow \\

\cmd{\hyperlink{cmd-coeffsum}{coeffsum}} & Soma de coeficientes & 
\cmd{\hyperlink{cmd-coint}{coint}} & Teste de cointegra��o Engle-Granger \\

\cmd{\hyperlink{cmd-coint2}{coint2}} & Teste de cointegra��o de Johansen & 
\cmd{\hyperlink{cmd-cusum}{cusum}} & Teste CUSUM \\

\cmd{\hyperlink{cmd-difftest}{difftest}} & Nonparametric tests for differences & 
\cmd{\hyperlink{cmd-hausman}{hausman}} & Panel diagnostics \\

\cmd{\hyperlink{cmd-kpss}{kpss}} & KPSS stationarity test & 
\cmd{\hyperlink{cmd-leverage}{leverage}} & Influential observations \\

\cmd{\hyperlink{cmd-lmtest}{lmtest}} & LM tests & 
\cmd{\hyperlink{cmd-meantest}{meantest}} & Difference of means \\

\cmd{\hyperlink{cmd-omit}{omit}} & Omit variables & 
\cmd{\hyperlink{cmd-qlrtest}{qlrtest}} & Quandt likelihood ratio test \\

\cmd{\hyperlink{cmd-reset}{reset}} & Ramsey's RESET & 
\cmd{\hyperlink{cmd-restrict}{restrict}} & Linear restrictions \\

\cmd{\hyperlink{cmd-runs}{runs}} & Runs test & 
\cmd{\hyperlink{cmd-testuhat}{testuhat}} & Normality of residual \\

\cmd{\hyperlink{cmd-vartest}{vartest}} & Difference of variances & 
\cmd{\hyperlink{cmd-vif}{vif}} & Variance Inflation Factors \\

\end{longtable}
}

\subsection{Transforma��es}
\label{sect-Transformations}

{\small
\begin{longtable}{lp{2in}lp{2in}}
\cmd{\hyperlink{cmd-diff}{diff}} & First differences & 
\cmd{\hyperlink{cmd-discrete}{discrete}} & Mark variables as discrete \\

\cmd{\hyperlink{cmd-dummify}{dummify}} & Create sets of dummies & 
\cmd{\hyperlink{cmd-lags}{lags}} & Create lags \\

\cmd{\hyperlink{cmd-ldiff}{ldiff}} & Log-differences & 
\cmd{\hyperlink{cmd-logs}{logs}} & Create logs \\

\cmd{\hyperlink{cmd-rhodiff}{rhodiff}} & Quasi-differencing & 
\cmd{\hyperlink{cmd-sdiff}{sdiff}} & Seasonal differencing \\

\cmd{\hyperlink{cmd-square}{square}} & Create squares of variables & 
\end{longtable}
}

\subsection{Estat�sticas}
\label{sect-Statistics}

{\small
\begin{longtable}{lp{2in}lp{2in}}
\cmd{\hyperlink{cmd-corr}{corr}} & Coeficientes de correla��o & 
\cmd{\hyperlink{cmd-corrgm}{corrgm}} & Correlograma \\

\cmd{\hyperlink{cmd-freq}{freq}} & Frequency distribution & 
\cmd{\hyperlink{cmd-hurst}{hurst}} & Hurst exponent \\

\cmd{\hyperlink{cmd-mahal}{mahal}} & Mahalanobis distances & 
\cmd{\hyperlink{cmd-pca}{pca}} & Principal Components Analysis \\

\cmd{\hyperlink{cmd-pergm}{pergm}} & Periodogram & 
\cmd{\hyperlink{cmd-spearman}{spearman}} & Spearmans's rank correlation \\

\cmd{\hyperlink{cmd-summary}{summary}} & Descriptive statistics & 
\cmd{\hyperlink{cmd-xcorrgm}{xcorrgm}} & Cross-correlogram \\

\cmd{\hyperlink{cmd-xtab}{xtab}} & Cross-tabulate variables & 
\end{longtable}
}

\subsection{Conjunto de Dados}
\label{sect-Dataset}

{\small
\begin{longtable}{lp{2in}lp{2in}}
\cmd{\hyperlink{cmd-append}{append}} & Acrescentar dados & 
\cmd{\hyperlink{cmd-data}{data}} & Import from database \\

\cmd{\hyperlink{cmd-dataset}{dataset}} & Manipulate the dataset & 
\cmd{\hyperlink{cmd-delete}{delete}} & Delete variables \\

\cmd{\hyperlink{cmd-genr}{genr}} & Generate a new variable & 
\cmd{\hyperlink{cmd-info}{info}} & Information on data set \\

\cmd{\hyperlink{cmd-labels}{labels}} & Print labels for variables & 
\cmd{\hyperlink{cmd-nulldata}{nulldata}} & Creating a blank dataset \\

\cmd{\hyperlink{cmd-open}{open}} & Open a data file & 
\cmd{\hyperlink{cmd-rename}{rename}} & Rename variables \\

\cmd{\hyperlink{cmd-setinfo}{setinfo}} & Edit attributes of variable & 
\cmd{\hyperlink{cmd-setobs}{setobs}} & Set frequency and starting observation \\

\cmd{\hyperlink{cmd-setmiss}{setmiss}} & Missing value code & 
\cmd{\hyperlink{cmd-smpl}{smpl}} & Set the sample range \\

\cmd{\hyperlink{cmd-store}{store}} & Save data & 
\cmd{\hyperlink{cmd-varlist}{varlist}} & Listing of variables \\

\end{longtable}
}

\subsection{Gr�ficos}
\label{sect-Graphs}

{\small
\begin{longtable}{lp{2in}lp{2in}}
\cmd{\hyperlink{cmd-boxplot}{boxplot}} & Gr�ficos caixa-com-bigodes & 
\cmd{\hyperlink{cmd-gnuplot}{gnuplot}} & Create a gnuplot graph \\

\cmd{\hyperlink{cmd-graph}{graph}} & Create ASCII graph & 
\cmd{\hyperlink{cmd-plot}{plot}} & ASCII plot \\

\cmd{\hyperlink{cmd-rmplot}{rmplot}} & Range-mean plot & 
\cmd{\hyperlink{cmd-scatters}{scatters}} & Multiple pairwise graphs \\

\end{longtable}
}

\subsection{Impress�o}
\label{sect-Printing}

{\small
\begin{longtable}{lp{2in}lp{2in}}
\cmd{\hyperlink{cmd-eqnprint}{eqnprint}} & Print model as equation & 
\cmd{\hyperlink{cmd-outfile}{outfile}} & Direct printing to file \\

\cmd{\hyperlink{cmd-print}{print}} & Print data or strings & 
\cmd{\hyperlink{cmd-printf}{printf}} & Formatted printing \\

\cmd{\hyperlink{cmd-sprintf}{sprintf}} & Printing to a string & 
\cmd{\hyperlink{cmd-tabprint}{tabprint}} & Print model in tabular form \\

\end{longtable}
}

\subsection{Predi��o}
\label{sect-Prediction}

{\small
\begin{longtable}{lp{2in}lp{2in}}
\cmd{\hyperlink{cmd-fcast}{fcast}} & Generate forecasts & 
\end{longtable}
}

\subsection{Programa��o}
\label{sect-Programming}

{\small
\begin{longtable}{lp{2in}lp{2in}}
\cmd{\hyperlink{cmd-break}{break}} & Sair de um ciclo & 
\cmd{\hyperlink{cmd-else}{else}} &  \\

\cmd{\hyperlink{cmd-end}{end}} & End block of commands & 
\cmd{\hyperlink{cmd-endif}{endif}} &  \\

\cmd{\hyperlink{cmd-endloop}{endloop}} & End a command loop & 
\cmd{\hyperlink{cmd-function}{function}} & Define a function \\

\cmd{\hyperlink{cmd-if}{if}} &  & 
\cmd{\hyperlink{cmd-include}{include}} & Include function definitions \\

\cmd{\hyperlink{cmd-loop}{loop}} & Start a command loop & 
\cmd{\hyperlink{cmd-run}{run}} & Execute a script \\

\cmd{\hyperlink{cmd-set}{set}} & Set program parameters & 
\end{longtable}
}

\subsection{Utilidades}
\label{sect-Utilities}

{\small
\begin{longtable}{lp{2in}lp{2in}}
\cmd{\hyperlink{cmd-criteria}{criteria}} & Crit�rio de selec��o de modelos & 
\cmd{\hyperlink{cmd-help}{help}} & Help on commands \\

\cmd{\hyperlink{cmd-modeltab}{modeltab}} & The model table & 
\cmd{\hyperlink{cmd-pvalue}{pvalue}} & Compute p-values \\

\cmd{\hyperlink{cmd-quit}{quit}} & Exit the program & 
\cmd{\hyperlink{cmd-shell}{shell}} & Execute shell commands \\

\end{longtable}
}



\section{Short-form command options}
\label{cmd-opt-short}

As can be seen from section~\ref{cmd-cmd}, the behavior of many gretl
commands can be modified via the use of option flags. These take the
form of two dashes followed by a string which is somewhat descriptive
of the effect of the option.

Some options require a parameter, which must be joined to the option
``flag'' with an equals sign. Among the options that do \emph{not}
require a parameter, certain common ones have a short form---a single
dash followed by a single letter---and it is considered idiomatic to
use the short forms in hansl scripts. The table below shows the
relevant mapping: for any command which supports the long-form option
in the first column, the short form in the second column is also
supported.

\begin{center}
\begin{tabular}{lc}
long form & short form \\ [4pt]
\option{verbose} & \texttt{-v} \\
\option{quiet}   & \texttt{-q} \\
\option{robust}  & \texttt{-r} \\
\option{hessian} & \texttt{-h} \\
\option{window}  & \texttt{-w} 
\end{tabular}
\end{center}



