\chapter{Model selection criteria}
\label{select-criteria}

\section{Introduction}
\label{select-intro}

In some contexts the econometrician chooses between alternative models
based on a formal hypothesis test.  For example, one might choose a
more general model over a more restricted one if the restriction in
question can be formulated as a testable null hypothesis, and the null
is rejected on an appropriate test.

In other contexts one sometimes seeks a criterion for model selection
that somehow measures the balance between goodness of fit or
likelihood, on the one hand, and parsimony on the other.  The
balancing is necessary because the addition of extra variables to a
model cannot reduce the degree of fit or likelihood, and is very
likely to increase it somewhat even if the additional variables are
not truly relevant to the data-generating process.

The best known such criterion, for linear models estimated via least
squares, is the adjusted $R^2$,
%
\[
\bar{R}^2 = 1 - \frac{{\rm SSR} / (n-k)}{{\rm TSS} / (n-1)}
\]
%
where $n$ is the number of observations in the sample, $k$ denotes the
number of parameters estimated, and SSR and TSS denote the sum of
squared residuals and the total sum of squares for the dependent
variable, respectively.  Compared to the ordinary coefficient of
determination or unadjusted $R^2$,
%
\[
R^2 = 1 - \frac{{\rm SSR}}{{\rm TSS}}
\]
%
the ``adjusted'' calculation penalizes the inclusion of additional
parameters, other things equal.  

\section{Information criteria}
\label{select-aic}

A more general criterion in a similar spirit is Akaike's (1974)
``Information Criterion'' (AIC).  The original formulation of this
measure is
%
\begin{equation}
\label{aic-orig}
{\rm AIC} = -2 \ell(\hat{\theta}) + 2k
\end{equation}
%
where $\ell(\hat{\theta})$ represents the maximum loglikelihood as a
function of the vector of parameter estimates, $\hat{\theta}$, and $k$
(as above) denotes the number of ``independently adjusted parameters
within the model.''  In this formulation, with AIC negatively related
to the likelihood and positively related to the number of parameters,
the researcher seeks the minimum AIC.

The AIC can be confusing, in that several variants of the calculation
are ``in circulation.''  For example, Davidson and MacKinnon (2004)
present a simplified version,
%
\[
{\rm AIC} = \ell(\hat{\theta}) - k
\]
%
which is just $-2$ times the original: in this case, obviously, one
wants to maximize AIC.

In the case of models estimated by least squares, the loglikelihood
can be written as
%
\begin{equation}
\label{ols-loglik}
\ell(\hat{\theta}) = -\frac{n}{2}(1 + \log 2\pi - \log n)
 - \frac{n}{2} \log {\rm SSR}
\end{equation}
%
Substituting (\ref{ols-loglik}) into (\ref{aic-orig}) we get
%
\[
{\rm AIC} = n(1 + \log 2\pi - \log n) + n\log {\rm SSR} + 2k
\]
%
which can also be written as
%
\begin{equation}
\label{full-aic-alt}
{\rm AIC} = n\log \left( \frac{\rm SSR}{n} \right) + 2k + 
  n(1 + \log 2\pi)
\end{equation}
%

Some authors simplify the formula for the case of models estimated
via least squares.  For instance, William Greene writes
%
\begin{equation}
\label{aic-greene}
{\rm AIC} = \log \left( \frac{\rm SSR}{n} \right) + \frac{2k}{n}
\end{equation}
%
This variant can be derived from (\ref{full-aic-alt}) by dividing
through by $n$ and subtracting the constant $1 + \log 2\pi$.  That is,
writing AIC$_G$ for the version given by Greene, we have
%
\[
{\rm AIC}_G = \frac{1}{n} {\rm AIC} - (1 + \log 2\pi)
\]
%

Finally, Ramanathan gives a further variant:
%
\[
{\rm AIC}_R = \left( \frac{\rm SSR}{n} \right) e^{2k/n}
\]
%
which is the exponential of the one given by Greene.  

Gretl began by using the Ramanathan variant, but since version 1.3.1
the program has used the original Akaike formula (\ref{aic-orig}), and
more specifically (\ref{full-aic-alt}) for models estimated via least
squares.

\vspace{1ex}

Although the Akaike criterion is designed to favor parsimony, arguably
it does not go far enough in that direction.  For instance, if we have
two nested models with $k-1$ and $k$ parameters respectively, and if
the null hypothesis that parameter $k$ equals 0 is true, in large
samples the AIC will nonetheless tend to select the less parsimonious
model about 16 percent of the time (see Davidson and MacKinnon, 2004,
chapter 15).

An alternative to the AIC which avoids this problem is the Schwarz
(1978) ``Bayesian information criterion'' (BIC).  The BIC can be
written (in line with Akaike's formulation of the AIC) as
%
\[
{\rm BIC} = -2 \ell(\hat{\theta}) + k \log n
\]
The multiplication of $k$ by $\log n$ in the BIC means that the
penalty for adding extra parameters grows with the sample size.  This
ensures that, asymptotically, one will not select a larger model over
a correctly specified parsimonious model.

A further alternative to AIC, which again tends to select more
parsimonious models than AIC, is the Hannan--Quinn criterion or HQC
(Hannan and Quinn, 1979).  Written consistently with the formulations
above, this is
%
\[
{\rm HQC} = -2 \ell(\hat{\theta}) + 2k \log \log n
\]
%
The Hannan--Quinn calculation is based on the law of the iterated
logarithm (note that the last term is the log of the log of the sample
size).  The authors argue that their procedure provides a ``strongly
consistent estimation procedure for the order of an autoregression'',
and that ``compared to other strongly consistent procedures this
procedure will underestimate the order to a lesser degree.''

\vspace{1ex}

Gretl reports the AIC, BIC and HQC (calculated as explained above) for
most sorts of models.  The key point in interpreting these values is
to know whether they are calculated such that smaller values are
better, or such that larger values are better.  In gretl, smaller
values are better: one wants to minimize the chosen criterion.





