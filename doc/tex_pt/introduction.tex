\chapter{Introduction}
\label{intro}


\section{Features at a glance}
\label{features}

\app{Gretl} is an econometrics package, including a shared library, a
command-line client program and a graphical user interface.
    
\begin{description}
\item[User-friendly] \app{Gretl} offers an intuitive user interface;
  it is very easy to get up and running with econometric analysis.
  Thanks to its association with the econometrics textbooks by Ramu
  Ramanathan, Jeffrey Wooldridge, and James Stock and Mark Watson, the
  package offers many practice data files and command scripts.  These
  are well annotated and accessible. Two other useful resources for gretl
  users are the available documentation and the
  \href{http://gretl.sourceforge.net/lists.html}{gretl-users} mailing
  list.
\item[Flexible] You can choose your preferred point on the spectrum
  from interactive point-and-click to batch processing, and can easily
  combine these approaches.
\item[Cross-platform] \app{Gretl}'s ``home'' platform is Linux but it
  is also available for MS Windows and Mac OS X, and should work on
  any unix-like system that has the appropriate basic libraries (see
  Appendix~\ref{app-build}).
\item[Open source] The full source code for \app{gretl} is available
  to anyone who wants to critique it, patch it, or extend it.
  See Appendix~\ref{app-build}.
\item[Sophisticated] \app{Gretl} offers a full range of least-squares
  based estimators, either for single equations and for systems,
  including vector autoregressions and vector error correction models.
  Several specific maximum likelihood estimators (e.g.\ probit, ARIMA,
  GARCH) are also provided natively; more advanced estimation methods
  can be implemented by the user via generic maximum likelihood or
  nonlinear GMM.
\item[Extendible] Users can enhance \app{gretl} by writing their own
  functions and procedures in \app{gretl}'s scripting language, which
  includes a reasonably wide range of matrix functions.
\item[Accurate] \app{Gretl} has been thoroughly tested on several
  benchmarks, among which the NIST reference datasets. See
  Appendix~\ref{app-accuracy}.
\item[Internet ready] \app{Gretl} can access and fetch databases from
  a server at Wake Forest University.  The MS Windows version comes
  with an updater program which will detect when a new version is
  available and offer the option of auto-updating.
\item[International] \app{Gretl} will produce its output in English,
  French, Italian, Spanish, Polish or German, depending on your
  computer's native language setting.
\end{description}


\section{Acknowledgements}
\label{ack}

The \app{gretl} code base originally derived from the program
\app{ESL} (``Econometrics Software Library''), written by Professor
Ramu Ramanathan of the University of California, San Diego.  
We are much in debt to Professor Ramanathan for making this code
available under the GNU General Public Licence and for helping to
steer \app{gretl}'s early development.

We are also grateful to the authors of several econometrics textbooks
for permission to package for \app{gretl} various datasets associated
with their texts.  This list currently includes William Greene, author
of \emph{Econometric Analysis}; Jeffrey Wooldridge (\emph{Introductory
  Econometrics: A Modern Approach}); James Stock and Mark Watson
(\emph{Introduction to Econometrics}); Damodar Gujarati (\emph{Basic
  Econometrics}); and Russell Davidson and James MacKinnon
(\emph{Econometric Theory and Methods}).  

GARCH estimation in \app{gretl} is based on code deposited in the
archive of the \emph{Journal of Applied Econometrics} by Professors
Fiorentini, Calzolari and Panattoni, and the code to generate
\emph{p}-values for Dickey--Fuller tests is due to James MacKinnon.  In
each case we are grateful to the authors for permission to use their
work.

With regard to the internationalization of \app{gretl}, thanks go to
Ignacio D�az-Emparanza (Spanish), Michel Robitaille and Florent
Bresson (French) , Cristian Rigamonti (Italian), Tadeusz Kufel and
Pawel Kufel (Polish), and Markus Hahn and Sven Schreiber (German).

\app{Gretl} has benefitted greatly from the work of numerous
developers of free, open-source software: for specifics please see
Appendix~\ref{app-build}.  Our thanks are due to Richard Stallman
of the Free Software Foundation, for his support of free software in
general and for agreeing to ``adopt'' \app{gretl} as a GNU program in
particular.

Many users of \app{gretl} have submitted useful suggestions and bug
reports.  In this connection particular thanks are due to Ignacio
D�az-Emparanza, Tadeusz Kufel, Pawel Kufel, Alan Isaac, Cri Rigamonti,
Sven Schreiber, Talha Yalta, and Dirk Eddelbuettel, who maintains the
\app{gretl} package for Debian GNU/Linux.


\section{Installing the programs}
\label{install}

\subsection{Linux}
\label{linux-install}

On the Linux\footnote{In this manual we use ``Linux'' as shorthand to
  refer to the GNU/Linux operating system.  What is said herein about
  Linux mostly applies to other unix-type systems too, though some
  local modifications may be needed.} platform you have the choice of
compiling the \app{gretl} code yourself or making use of a pre-built
package. Building \app{gretl} from the source is necessary if you want
to access the development version or customize \app{gretl} to your
needs, but this takes quite a few skills; most users will want to go
for a pre-built package.

Some Linux distributions feature \app{gretl} as part of their standard
offering: Debian, for example, or Ubuntu (in the \emph{universe}
repository). If this is the case, all you need to do is install
\app{gretl} through your package manager of choice (eg synaptic).

Ready-to-run packages are available in \app{rpm} format
(suitable for Red Hat Linux and related systems) on the \app{gretl}
webpage \url{http://gretl.sourceforge.net}.  
      
However, we're hopeful that some users with coding skills may consider
\app{gretl} sufficiently interesting to be worth improving and
extending.  The documentation of the libgretl API is by no means
complete, but you can find some details by following the link
``Libgretl API docs'' on the \app{gretl} homepage. People interested
in the \app{gretl} development are welcome to subscribe to the
\href{http://gretl.sourceforge.net/lists.html}{gretl-devel} mailing
list. 

If you prefer to compile your own (or are using a unix system for
which pre-built packages are not available), instructions on building
\app{gretl} can be found in Appendix~\ref{app-build}.

\subsection{MS Windows}
\label{windows-install}

The MS Windows version comes as a self-extracting executable.
Installation is just a matter of downloading \verb+gretl_install.exe+
and running this program. You will be prompted for a location to
install the package (the default is \verb+c:\userdata\gretl+).


\subsection{Updating}
\label{updating}

If your computer is connected to the Internet, then on start-up
\app{gretl} can query its home website at Wake Forest University to
see if any program updates are available; if so, a window will open up
informing you of that fact.  If you want to activate this feature,
check the box marked ``Tell me about gretl updates'' under
\app{gretl}'s ``Tools, Preferences, General'' menu.
      
The MS Windows version of the program goes a step further: it tells
you that you can update \app{gretl} automatically if you wish.  To do
this, follow the instructions in the popup window: close \app{gretl}
then run the program titled ``gretl updater'' (you should find this
along with the main \app{gretl} program item, under the Programs
heading in the Windows Start menu). Once the updater has completed its
work you may restart \app{gretl}.
      
%%% Local Variables: 
%%% mode: latex
%%% TeX-master: "gretl-guide"
%%% End: 
