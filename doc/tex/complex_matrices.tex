\documentclass{article}
\usepackage[letterpaper,body={6.0in,9.1in},top=0.7in,left=1.25in,nohead]{geometry}
\usepackage{verbatim,fancyvrb}
\usepackage{color,gretl}
\usepackage[authoryear]{natbib}
\usepackage[pdftex]{graphicx}
\usepackage[pdftex,hyperfootnotes=false]{hyperref}

\definecolor{steel}{rgb}{0.03,0.20,0.45}

\hypersetup{pdftitle={Complex matrices in gretl},
            pdfauthor={Allin Cottrell},
            colorlinks=true,
            linkcolor=blue,
            urlcolor=red,
            citecolor=steel,
            bookmarksnumbered=true,
            plainpages=false
}

\begin{document}

\setlength{\parindent}{0pt}
\setlength{\parskip}{1ex}
\setcounter{secnumdepth}{2}

\title{Complex matrices in gretl}
\author{Allin Cottrell}
\maketitle

\section{Introduction}
\label{sec:intro}

As of mid-August 2019, gretl has native support for complex
matrices. Not all of hansl's matrix functions accept complex input,
but we have enabled a sizable subset of these functions which we
believe should suffice for most econometric purposes. Complex support
may be extended in future as the need arises.

A first point to note is that complex values are treated as a special
case of the hansl \texttt{matrix} type; there's no \texttt{complex}
type as such. Complex scalars fall under the \texttt{matrix} type as
$1 \times 1$ matrices; the hansl \texttt{scalar} type is only for real
values.

This document explains how to create and manipulate complex matrices,
and discusses some questions of backward compatibility.

\section{Creating a complex matrix}
\label{sec:create}

The unique explicit constructor for complex matrices is the
\texttt{complex()} function which takes two arguments, giving the real
and imaginary parts respectively, and sticks them together, as in
\begin{code}
C = complex(A, B)
\end{code}
Three cases are supported, as follows.
\begin{itemize}
\item \texttt{A} and \texttt{B} are both $m \times n$ matrices
  (neither of which is in itself complex). Then \texttt{C} is an
  $m \times n$ complex matrix such that $c_{ij} = a_{ij} + b_{ij}i$.
\item \texttt{A} and \texttt{B} are both scalars: \texttt{C} is a
  $1 \times 1$ complex matrix such that $c = a + bi$.
\item \texttt{A} is an $m \times n$ matrix and \texttt{B} is a scalar:
  \texttt{C} is an $m \times n$ matrix such that
  $c_{ij} = a_{ij} + bi$.
\end{itemize}

In addition, complex matrices may naturally arise as the result of
various computations.

With both real and complex matrices in circulation, one may wish to
determine whether a particular matrix is complex. The function
\texttt{iscomplex()} can tell you. Passed an identifier, it returns 1
if it names a complex matrix, 0 if it names a real matrix, or
\texttt{NA} otherwise.

\section{Operators}
\label{sec:ops}

Most of the operators available for working with real matrices are
also available for complex ones; this includes the ``dot-operators''
which work element-wise or by ``broadcasting'' vectors. Moreover,
``mixed'' operands are accepted, as in \texttt{D = C + A} where
\texttt{C} is complex and \texttt{A} real. The result, \texttt{D},
will be complex. In such cases the real operand is treated as a
complex matrix with an all-zero imaginary part.

The operators \textit{not} accepted for complex matrices are:
``\texttt{\textbackslash}'' (left division); ``\texttt{/}'' (right
division); and the element-wise comparison operators that include an
inequality (such as ``\verb+.>+''), since complex values as such
cannot be compared as greater or lesser (though they can be compared
as equal or not equal).

\section{Functions}
\label{sec:funcs}

To give an idea of what works, and what doesn't work, for complex
matrices, we'll walk through the hansl function-space using the
categories employed in gretl's online ``Function reference'' (under the
\textsf{Help} menu in the GUI program).

\subsection{Linear algebra}

The only functions that accept complex arguments at present are:
\texttt{det}, \texttt{ldet}, \texttt{eigengen}, \texttt{eigensym} (for
Hermitian matrices), \texttt{fft}, \texttt{ffti}, \texttt{inv},
\texttt{ginv}, \texttt{hdprod}, \texttt{svd}, \texttt{tr}, and
\texttt{transp}.

In addition there's the complex-only function \texttt{ctrans()} which
gives the conjugate transpose; note that \texttt{transp} gives the
straight (non-conjugated) transpose of a complex matrix.

\subsection{Matrix building}

Given what was said in section~\ref{sec:create} above, several of the
functions in this category should be thought of as applying to the
real or imaginary part of a complex matrix (for example, \texttt{ones}
and \texttt{mnormal}), and are of course usable in that way.  However,
some of these functions can be applied to complex matrices as such,
namely, \texttt{diag}, \texttt{diagcat}, \texttt{vec}, \texttt{vech},
\texttt{unvech}, \texttt{lower} and \texttt{upper}.

The only functions not available for complex matrices are
\texttt{cnameset} and \texttt{rnameset}. That is, you cannot name the
columns or rows of such matrix (although this restriction could
probably be lifted without great difficulty).

\subsection{Matrix shaping}

The functions that accept complex input are: \texttt{cols},
\texttt{mreverse}, \texttt{mshape}, \texttt{rows}, \texttt{selifc},
\texttt{selifr} and \texttt{trimr}.

The functions \texttt{msortby}, \texttt{sort} and \texttt{dsort} are
excluded for the reason mentioned in section~\ref{sec:ops}.

\subsection{Statistical}

Supported for complex input: \texttt{meanr}, \texttt{meanc},
\texttt{sumc}, \texttt{sumr}, \texttt{prodr} and \texttt{prodc}. And
that's all.

\subsection{Mathematical}

In the matrix context, these are functions that are applied element by
element. For complex input the following are supported: \texttt{log},
\texttt{exp} and \texttt{sqrt}, plus all of the trigonometric
functions with the exception of \texttt{atan2}.

In addition there are the complex-only functions \texttt{cmod}
(complex modulus, also accessible via \texttt{abs}), \texttt{carg}
(complex ``argument''), \texttt{conj} (complex conjugate), \texttt{Re}
(real part) and \texttt{Im} (imaginary part).

\section{Backward compatibility}
\label{sec:compat}

Compatibility issues may arise in two contexts, both related to the
fact that gretl offered some degree of support for complex matrices
before they were full citizens of the hansl polity.

\begin{enumerate}
\item The functions \texttt{fft} (fast Fourier transform) and
  \texttt{eigengen} (eigenvalues and/or eigenvectors of a
  non-symmetric real matrix) returned complex matrices in what we'll
  call the ``legacy'' representation. In the case of \texttt{fft} and
  the eigenvalues from \texttt{eigengen} this took the form of a
  regular gretl matrix with real values in the first (or odd-numbered)
  column(s) and imaginary parts in the second (or even-numbered)
  column(s). Since calculating with such matrices using the standard
  matrix operators would result in nonsense, we provided the tailored
  functions \texttt{cmult} and \texttt{cdiv}.

  In the case of complex eigenvectors from \texttt{eigengen}---well,
  you probably don't want to know, but if you do, consult the help text
  for \texttt{eigengen}; they were not easy for a user to handle!
\item The function packages \texttt{cmatrix} and
  \texttt{ghosts}. These are relatively recent additions to gretl's
  functionality, designed to support frequency-domain analysis. Prior
  to the development of native complex-matrix functionality,
  \texttt{cmatrix} was needed as an dependency for \texttt{ghosts}
  (multivariate spectral analysis).
\end{enumerate}

So what happens with \texttt{fft} and \texttt{eigengen},
\texttt{cmatrix} and \texttt{ghosts} under the new regime?

The easier question is the one about \texttt{cmatrix} and
\texttt{ghosts}. The \texttt{cmatrix} package is no longer
required---or supported---and there will be (or already is) a version
of \texttt{ghosts} that relies on gretl's native complex-matrix
support.

As for \texttt{fft} and \texttt{eigengen}, \dots{} FIXME what do we
want to say here?

\end{document}
