\chapter{User-written functions}

Main structure:

\begin{flushleft}
\texttt{function \emph{type} \emph{funcname}(\emph{parameters})}\\
   \quad \ldots\\
   \quad \emph{function body}\\
   \quad \ldots \\
\texttt{end function}
\end{flushleft}

The opening line of a function definition contains these elements, in
strict order:

\begin{enumerate}
\item The keyword \texttt{function}.
\item \texttt{\emph{type}}, which states the type of value returned by the
  function, if any.  This must be one of \texttt{void} (if the
  function does not return anything), \texttt{scalar},
  \texttt{series}, \texttt{matrix}, \texttt{list} or \texttt{string}.
\item \texttt{\emph{funcname}}, the unique identifier for the function.
  Function names have a maximum length of 31 characters; they must
  start with a letter and can contain only letters, numerals and the
  underscore character.  You will get an error if you try to define a
  function having the same name as an existing gretl command.
\item The functions's \texttt{\emph{parameters}}, in the form of a
  comma-separated list enclosed in parentheses.
\end{enumerate}

Function parameters can be of any of the types shown below.

\begin{center}
\begin{tabular}{ll}
  \multicolumn{1}{c}{Type} & 
  \multicolumn{1}{c}{Description} \\ [4pt]
  \texttt{bool}   & scalar variable acting as a Boolean switch \\
  \texttt{int}    & scalar variable acting as an integer  \\
  \texttt{scalar} & scalar variable \\
  \texttt{series} & data series \\
  \texttt{list}   & named list of series \\
  \texttt{matrix} & matrix or vector \\
  \texttt{string} & string variable or string literal \\
  \texttt{bundle} & all-purpose container (see chapter~\ref{chap:bundles})
\end{tabular}
\end{center}

Each element in the listing of parameters must include two terms: a
type specifier, and the name by which the parameter shall be known
within the function.  An example follows:
%    
\begin{code}
function scalar myfunc (series y, list xvars, bool verbose)
\end{code}

Each of the type-specifiers, with the exception of \texttt{list} and
\texttt{string}, may be modified by prepending an asterisk to the
associated parameter name, as in
%    
\begin{code}
function scalar myfunc (series *y, scalar *b)
\end{code}

The meaning of this modification is explained below; it is related to
the use of pointer arguments in the C programming language.


\section{Parameter passing and return values}
\label{sec:params-returns}



\section{Recursion}

Obligatory factorial example
\begin{code}
function scalar factorial(scalar n)
    if (n<0) || (n>floor(n))
        # filter out everything that isn't a 
        # non-negative integer
        return NA
    elif n==0
        return 1
    else
        return n*factorial(n-1)
    endif
end function

loop i=0..6 --quiet
    printf "%d! = %d\n", i, factorial(i)
end loop
\end{code}

Note: this is fun, but in practice, you'll be much better off using
the pre-cooked gamma function (or, better still, its logarithm).

%%% Local Variables: 
%%% mode: latex
%%% TeX-master: "hansl-primer"
%%% End: 
