\chapter{User-written functions}
\label{chap:user-funcs}

Hansl natively provides a reasonably wide array of pre-defined
functions for manipulating variables of all kinds; the previous
chapters contain several examples. However, it is also possible to
extend hansl's native capabilities by defining additional
functions.

Here's what a user-defined function looks like:
\begin{flushleft}
\texttt{function \emph{type} \emph{funcname}(\emph{parameters})}\\
   \quad \ldots\\
   \quad \texttt{\emph{function body}}\\
   \quad \ldots \\
\texttt{end function}
\end{flushleft}

The opening line of a function definition contains these elements, in
strict order:

\begin{enumerate}
\item The keyword \texttt{function}.
\item \texttt{\emph{type}}, which states the type of value returned by
  the function, if any.  This must be one of \texttt{void} (if the
  function does not return anything), \texttt{scalar},
  \texttt{series}, \texttt{matrix}, \texttt{list}, \texttt{string} or
  \texttt{bundle}.
\item \texttt{\emph{funcname}}, the unique identifier for the
  function.  Function names have a maximum length of 31 characters;
  they must start with a letter and can contain only letters, numerals
  and the underscore character. They cannot coincide with the names of
  native commands or functions.
\item The function's \texttt{\emph{parameters}}, in the form of a
  comma-separated list enclosed in parentheses.\footnote{If you have
    to pass many parameters, you might want to consider wrapping them
    into a bundle to avoid syntax cluttering.} Note: parameters are
  the only way hansl function can receive anything from ``the
  outside''. In hansl there are no global variables.
\end{enumerate}

Function parameters can be of any of the types shown below.

\begin{center}
\begin{tabular}{ll}
  \multicolumn{1}{c}{Type} & 
  \multicolumn{1}{c}{Description} \\ [4pt]
  \texttt{bool}   & scalar variable acting as a Boolean switch \\
  \texttt{int}    & scalar variable acting as an integer  \\
  \texttt{scalar} & scalar variable \\
  \texttt{series} & data series (see section~\ref{sec:series})\\
  \texttt{list}   & named list of series  (see section~\ref{sec:lists})\\
  \texttt{matrix} & matrix or vector \\
  \texttt{string} & string variable or string literal \\
  \texttt{bundle} & all-purpose container
\end{tabular}
\end{center}

Each element in parameter list must include two terms: a type
specifier, and the name by which the parameter shall be known within
the function.

The \emph{function body} contains (almost) arbitrary hansl code, which
should compute the \emph{return value}, that is the value the function
is supposed to yield. Any variable declared inside the function is
\emph{local}, so it will cease to exist when the function ends.

The \cmd{return} command is used to stop execution of the code inside
the function and deliver its result to the calling code. This
typically happens at the end of the function body, but doesn't have
to. The function definition must end with the expression
\verb|end function|, on a line of its own.

\tip{Beware: unlike some other languages (e.g.\ Matlab or GAUSS), you
  cannot directly return multiple outputs from a function. However,
  you can return a bundle and stuff it with as many objects as you
  want.}

In order to get a feel for how functions work in practice, here's a
simple example:
\begin{code}
function scalar quasi_log(scalar x)
   /* popular approximation to the natural logarithm
     via Pad� polynomials 
   */
   if (x<0)
      scalar ret = NA
   else 
      scalar ret = 2*(x-1)/(x+1)
   endif
   return ret
end function

loop for (x=0.5; x<2; x+=0.1)
   printf "x = %4.2f; ln(x) = %g, approx = %g\n", x, ln(x), quasi_log(x)
endloop
\end{code}

The code above computes the rational function
\[
  f(x) = 2 \cdot \frac{x-1}{x+1} ,
\]
which provides a decent approximation to the natural logarithm in the
neighborhood of 1. Some comments on the code:

\begin{enumerate}
\item Since the function is meant to return a scalar, we put the
  keyword \texttt{scalar} after the intial \cmd{function}.
\item In this case the parameter list has only one element: it is
  named \texttt{x} and is specified to be a scalar.
\item On the next line the function definition begins; the body
  includes a comment and an \cmd{if} block.
\item The function ends by returning the computed value, \texttt{ret}.
\item The following lines give a simple example of usage. Note that in
  the \cmd{printf} command, the two functions \cmd{ln()} and
  \cmd{quasi\_log()} are indistinguishable from a purely syntactic
  viewpoint, although the former is native and the latter is
  user-defined.
\end{enumerate}

In ambitious uses of hansl you may end up writing several functions,
some of which may be quite long. In order to avoid cluttering your
script with function definitions, hansl provides the \cmd{include}
command: you can put your function definitions in a separate file (or
set of files) and read them in as needed.  For example, suppose you
saved the definition of \verb|quasi_log()| in a separate file called
\verb|quasilog_def.inp|: the code above could then be written more
compactly as
\begin{code}
include quasilog_def.inp

loop for (x=0.5; x<2; x+=0.1)
   printf "x = %4.2f; ln(x) = %g, approx = %g\n", x, ln(x), quasi_log(x)
endloop
\end{code}
Moreover, \cmd{include} commands can be nested.


\section{Parameter passing and return values}
\label{sec:params-returns}

In hansl, parameters are passed \emph{by value}, so what is used
inside the function is a copy of the original argument. You may modify
it, but you'll be just modifying the copy. The following example
should make this point clear:
\begin{code}
function void f(scalar x)
    x = x*2
    print x
end function

scalar x = 3
f(x)
print x
\end{code}
Running the above code yields
\begin{code}
              x =  6.0000000
              x =  3.0000000
\end{code}
The first \cmd{print} statement is executed inside the function, and
the displayed value is 6 because the input \verb|x| is doubled;
however, what really gets doubled is simply a \emph{copy} of the
original \texttt{x}: this is demonstrated by the second \cmd{print}
statement.  If you want a function to modify its arguments, you must
use pointers.

\subsection{Pointers}

Each of the type-specifiers, with the exception of \texttt{list} and
\texttt{string}, may be modified by prepending an asterisk to the
associated parameter name, as in
%    
\begin{code}
function scalar myfunc (matrix *y)
\end{code}
This indicates that the required argument is not a plain matrix but
rather a \emph{pointer-to-matrix}, or in other words the memory
address at which the variable is stored.

This can seem a bit mysterious to people unfamiliar with the C
programming language, so allow us to explain how pointers work by
analogy. Suppose you set up a barber shop. Ideally, your customers
would walk into your shop, sit on a chair and have their hair trimmed
or their beard shaved. However, local regulations forbid you to modify
anything coming in through your shop door. Of course, you wouldn't do
much business if people must leave your shop with their hair
untouched. Nevertheless, you have a simple way to get around this
limitation: your customers can come to your shop, tell you their home
address and walk out. Then, nobody stops you from going to their place
and exercising your fine profession. You're OK with the law, because
no modification of anything took place inside your shop.

While our imaginary restriction on the barber seems arbitrary, the
analogous restriction in a programming context is not: it prevents
functions from having unpredictable side effects. (You might be upset
if it turned out that your person was modified after visiting the
grocery store!)

In hansl (unlike C) you don't have to take any special care within the
function to distinguish the variable from its address,\footnote{In C,
  this would be called \emph{dereferencing} the pointer. The
  distinction is not required in hansl because there is no equivalent
  to operating on the supplied address itself, as in C.} you just use
the variable's name.  In order to supply the address of a variable
when you invoke the function, you use the ampersand (\verb|&|)
operator.

An example should make things clearer. The following code
\begin{code}
function void swap(scalar *a, scalar *b)
    scalar tmp = a
    a = b
    b = tmp
end function

scalar x = 0
scalar y = 1000000
swap(&x, &y)
print x y
\end{code}
gives the output
\begin{code}
              x =  1000000.0
              y =  0.0000000
\end{code}
So \texttt{x} and \texttt{y} have in fact been swapped. How?

First you have the function definition, in which the arguments are
pointers to scalars. Inside the function body, the distinction is
moot, as \verb|a| is taken to mean ``the scalar that you'll find at
the address given by the first argument'' (and likewise for
\verb|b|). The rest of the function simply swaps \texttt{a} and
\texttt{b} by means of a local temporary variable.

Outside the function, we first initialize the two scalars \texttt{x}
to 0 and \texttt{y} to a big number. When the function is called, it
is given as arguments \verb|&a| and \verb|&b|, which hansl identifies
as ``the address of'' the two scalars \texttt{a} and \texttt{b},
respectively.

Writing a function with pointer arguments has two main consequences:
first, as we just saw, it makes it possible to modify the function
arguments. Second, it avoids the computational cost of having to
allocate memory for a copy of the arguments and performing the copy
operation; such cost is proportional to the size of the argument.
Hence, for matrix arguments, this is a nice way to write faster
functions, as producing a copy of a large matrix can be quite
time-consuming.\footnote{However, the \cmd{const} qualifier achieves
  the same effect. See \GUG\ for further details.}

\subsection{Advanced parameter passing and optional arguments}

The parameters to a hansl function can be also specified in more
sophisticated ways than outlined above. There are three additional
features worth mentioning:
\begin{enumerate}
\item A descriptive string can be attached to each parameter for GUI usage.
\item For some parameter types, there is a special syntax construct
  for ensuring that its value is bounded; for example, you can
  stipulate a scalar argument to be positive, or constrained within a
  pre-specified range.
\item Some of the arguments can be made optional.
\end{enumerate}

A thorough discussion is too long to fit in this document, and the
interested reader should refer to the ``User-defined functions''
chapter of the \GUG. Here we'll just show you a simple, and hopefully
self-explanatory, example which combines features 2 and 3. Suppose
you have a function for producing smileys, defined as
\begin{code}
function void smileys(int times[0::1], bool frown[0])
    if frown
        string s = ":-("
    else
        string s = ":-)"
    endif
    
    loop times --quiet
        printf "%s ", s
    endloop
    
    printf "\n"
end function
\end{code}

Then, running
\begin{code}
smileys()
smileys(2, 1)
smileys(4)
\end{code}

produces

\begin{code}
:-) 
:-( :-( 
:-) :-) :-) :-) 
\end{code}

\section{Recursion}

Hansl functions can be recursive; what follows is the obligatory
factorial example:
\begin{code}
function scalar factorial(scalar n)
    if (n<0) || (n>floor(n))
        # filter out everything that isn't a 
        # non-negative integer
        return NA
    elif n==0
        return 1
    else
        return n * factorial(n-1)
    endif
end function

loop i=0..6 --quiet
    printf "%d! = %d\n", i, factorial(i)
endloop
\end{code}

Note: this is fun, but in practice, you'll be much better off using
the pre-cooked gamma function (or, better still, its logarithm).

%%% Local Variables: 
%%% mode: latex
%%% TeX-master: "hansl-primer"
%%% End: 
