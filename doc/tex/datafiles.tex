\chapter{Data files}
\label{datafiles}

\section{Native format}
\label{native-format}

\app{gretl} has its own format for data files.  Most users will
probably not want to read or write such files outside of \app{gretl}
itself, but occasionally this may be useful and full details on the
file formats are given in Appendix~\ref{app-datafile}.

\section{Other data file formats}
\label{other-formats}

\app{gretl} will read various other data formats.
    
\begin{itemize}
\item Plain text (ASCII) files.  These can be brought in using
  \app{gretl}'s ``File, Open Data, Import ASCII\dots{}'' menu item, or
  the \cmd{import} script command.  For details on what \app{gretl}
  expects of such files, see Section~\ref{scratch}.
\item Comma-Separated Values (CSV) files.  These can be imported using
  \app{gretl}'s ``File, Open Data, Import CSV\dots{}'' menu item, or
  the \cmd{import} script command. See also Section~\ref{scratch}.
\item Worksheets in the format of either MS \app{Excel} or
  \app{Gnumeric}.  These are also brought in using \app{gretl}'s
  ``File, Open Data, Import'' menu.  The requirements for such files
  are given in Section~\ref{scratch}.
\item BOX1 format data.  Large amounts of micro data are available
  (for free) in this format via the
  \href{http://www.census.gov/ftp/pub/DES/www/welcome.html}{Data
    Extraction Service} of the US Bureau of the Census. BOX1 data may
  be imported using the ``File, Open Data, Import BOX\dots{}'' menu
  item or the \cmd{import -o} script command.
\end{itemize}

When you import data from the ASCII, CSV or BOX formats, \app{gretl}
opens a ``diagnostic'' window, reporting on its progress in reading
the data.  If you encounter a problem with ill-formatted data, the
messages in this window should give you a handle on fixing the
problem.

For the convenience of anyone wanting to carry out more complex data
analysis, \app{gretl} has a facility for writing out data in the
native formats of GNU R and GNU Octave (see
Appendix~\ref{app-advanced}).  In the GUI client this option is found
under the ``File'' menu; in the command-line client use the
\cmd{store} command with the flag \cmd{-r} (R) or \cmd{-m} (Octave).

\section{Binary databases}
\label{dbase}

For working with large amounts of data \app{gretl} is supplied
with a database-handling routine.  A \emph{database}, as opposed to a
\emph{data file}, is not read directly into the program's workspace.
A database can contain series of mixed frequencies and sample ranges.
You open the database and select series to import into the working
dataset.  You can then save those series in a native format data file
if you wish. Databases can be accessed via \app{gretl}'s menu item
``File, Browse databases''.

For details on the format of \app{gretl} databases, see
Appendix~\ref{app-datafile}.

\subsection{Online access to databases}
\label{online-data}

As of version 0.40, \app{gretl} is able to access databases via the
internet.  Several databases are available from Wake Forest
University.  Your computer must be connected to the internet for this
option to work.  Please see the item on ``Online databases'' under
\app{gretl}'s Help menu.

\subsection{RATS 4 databases}
\label{RATS}

Thanks to Thomas Doan of \emph{Estima}, who provided me with the
specification of the database format used by RATS 4 (Regression
Analysis of Time Series), \app{gretl} can also handle such databases.
Well, actually, a subset of same: I have only worked on time-series
databases containing monthly and quarterly series.  My university has
the RATS G7 database containing data for the seven largest OECD
economies and \app{gretl} will read that OK.

\tip{Visit the \app{gretl}
  \href{http://gretl.sourceforge.net/gretl_data.html}{data page} for
  details and updates on available data.}

\section{Creating a data file from scratch}
\label{scratch}

There are five ways to do this:

\begin{enumerate}
\item Find, or create using a text editor, a plain text data file and
  open it with \app{gretl}'s ``Import ASCII'' option.
\item Use your favorite spreadsheet to establish the data file, save
  it in Comma Separated Values format if necessary (this should not be
  necessary if the spreadsheet program is MS Excel or Gnumeric), then
  use one of \app{gretl}'s ``Import'' options (CSV, Excel or Gnumeric,
  as the case may be).
\item Use \app{gretl}'s built-in spreadsheet.
\item Select data series from a suitable database.
\item Use your favorite text editor or other software tools to a
  create data file in \app{gretl} format independently.
\end{enumerate}

Here are a few comments and details on these methods.

\subsection{Common points on imported data}

Options (1) and (2) involve using \app{gretl}'s ``import'' mechanism.
For \app{gretl} to read such data successfully, certain general
conditions must be satisfied:

\begin{itemize}

\item The first row must contain valid variable names.  A valid
  variable name is of 8 characters maximum; starts with a letter; and
  contains nothing but letters, numbers and the underscore character,
  \verb+_+.  (Longer variable names will be truncated to 8
  characters.)  Qualifications to the above: First, in the case of an
  ASCII or CSV import, if the file contains no row with variable names
  the program will automatically add names, \verb+v1+, \verb+v2+ and
  so on.  Second, by ``the first row'' is meant the first
  \emph{relevant} row.  In the case of ASCII and CSV imports, blank
  rows and rows beginning with a hash mark, \verb+#+, are ignored.  In
  the case of Excel and Gnumeric imports, you are presented with a
  dialog box where you can select an offset into the spreadsheet, so
  that \app{gretl} will ignore a specified number of rows and/or
  columns.
          
\item Data values: these should constitute a rectangular block, with
  one variable per column (and one observation per row).  The number
  of variables (data columns) must match the number of variable names
  given. See also Section~\ref{missing-data}.  Numeric data are
  expected, but in the case of importing from ASCII/CSV, the program
  offers limited handling of character (string) data: if a given
  column contains character data only, consecutive numeric codes are
  substituted for the strings, and once the import is complete a table
  is printed showing the correspondence between the strings and the
  codes.
          
\item Dates (or observation labels): Optionally, the \emph{first}
  column may contain strings such as dates, or labels for
  cross-sectional observations.  Such strings have a maximum of 8
  characters (as with variable names, longer strings will be
  truncated).  A column of this sort should be headed with the string
  \verb+obs+ or \verb+date+, or the first row entry may be left
  blank

  For dates to be recognized as such, the date strings must adhere to
  one or other of a set of specific formats, as follows.  For
  \emph{annual} data: 4-digit years.  For \emph{quarterly} data: a
  4-digit year, followed by a separator (either a period, a colon, or
  the letter \verb+Q+), followed by a 1-digit quarter.  Examples:
  \verb+1997.1+, \verb+2002:3+, \verb+1947Q1+.  For \emph{monthly}
  data: a 4-digit year, followed by a period or a colon, followed by a
  two-digit month.  Examples: \verb+1997.01+, \verb+2002:10+.
          
\end{itemize}

CSV files can use comma, space or tab as the column separator.  When
you use the ``Import CSV'' menu item you are prompted to specify the
separator.  In the case of ``Import ASCII'' the program attempts to
auto-detect the separator that was used.If you use a spreadsheet to
prepare your data you are able to carry out various transformations of
the ``raw'' data with ease (adding things up, taking percentages or
whatever): note, however, that you can also do this sort of thing
easily --- perhaps more easily --- within \app{gretl}, by using the
tools under the ``Data, Add variables'' menu and/or ``Variable, define
new variable''.

\subsection{Appending imported data}

You may wish to establish a \app{gretl} dataset piece by piece, by
incremental importation of data from other sources.  This is supported
via the ``File, Append data'' menu items.  \app{gretl} will check the
new data for conformability with the existing dataset and, if
everything seems OK, will merge the data.  You can add new variables
in this way, provided the data frequency matches that of the existing
dataset.  Or you can append new observations for data series that are
already present; in this case the variable names must match up
correctly.  Note that by default (that is, if you choose ``Open data''
rather than ``Append data''), opening a new data file closes the
current one.

\subsection{Using the built-in spreadsheet}

Under \app{gretl}'s ``File, Create data set'' menu you can choose the
sort of dataset you want to establish (e.g.\ quarterly time series,
cross-sectional).  You will then be prompted for starting and ending
dates (or observation numbers) and the name of the first variable to
add to the dataset. After supplying this information you will be faced
with a simple spreadsheet into which you can type data values.  In the
spreadsheet window, clicking the right mouse button will invoke a
popup menu which enables you to add a new variable (column), to add an
observation (append a row at the foot of the sheet), or to insert an
observation at the selected point (move the data down and insert a
blank row.)

Once you have entered data into the spreadsheet you import these into
\app{gretl}'s workspace using the spreadsheet's ``Apply changes''
button.

Please note that \app{gretl}'s spreadsheet is quite basic and has no
support for functions or formulas.  Data transformations are done via
the ``Data'' or ``Variable'' menus in the main \app{gretl} window.

\subsection{Selecting from a database}

Another alternative is to establish your dataset by selecting
variables from a database.  \app{gretl} comes with a database of US
macroeconomic time series and, as mentioned above, the program will
reads RATS 4 databases.

Begin with \app{gretl}'s ``File, Browse databases'' menu item. This
has three forks: ``gretl native'', ``RATS 4'' and ``on database
server''.  You should be able to find the file \verb+fedstl.bin+ in
the file selector that opens if you choose the ``gretl native'' option
--- this file, which contains a large collection of US macroeconomic
time series, is supplied with the distribution.You won't find anything
under ``RATS 4'' unless you have purchased RATS data.\footnote{See
  \href{http://www.estima.com/}{www.estima.com}} If you do possess
RATS data you should go into \app{gretl}'s ``File, Preferences,
General'' dialog, select the Databases tab, and fill in the correct
path to your RATS files.

If your computer is connected to the internet you should find several
databases (at Wake Forest University) under ``on database server''.
You can browse these remotely; you also have the option of installing
them onto your own computer.  The initial remote databases window has
an item showing, for each file, whether it is already installed
locally (and if so, if the local version is up to date with the
version at Wake Forest).

Assuming you have managed to open a database you can import selected
series into \app{gretl}'s workspace by using the ``Import'' menu item
in the database window, or via the popup menu that appears if you
click the right mouse button, or by dragging the series into the
program's main window.

\subsection{Creating a gretl data file independently}

It is possible to create a data file in one or other of \app{gretl}'s
own formats using a text editor or software tools such as \app{awk},
\app{sed} or \app{perl}.  This may be a good choice if you have large
amounts of data already in machine readable form. You will, of course,
need to study the \app{gretl} data formats (XML format or
``traditional'' format) as described in Chapter~\ref{datafiles}.

\subsection{Further note}

\app{gretl} has no problem compacting data series of relatively high
frequency (e.g.\ monthly) to a lower frequency (e.g.\ quarterly): you
are given a choice of method (average, sum, start of period, or end of
period).  But it has no way of converting lower frequency data to
higher.  Therefore if you want to import series of various different
frequencies from a database into \app{gretl} \emph{you must start by
  importing a series of the lowest frequency you intend to use.} This
will initialize your \app{gretl} dataset to the low frequency, and
higher frequency data can be imported subsequently (they will be
compacted automatically).  If you start with a high frequency series
you will not be able to import any series of lower frequency.

\section{Missing data values}
\label{missing-data}

These are represented internally as \verb+DBL_MAX+, the largest
floating-point number that can be represented on the system (which is
likely to be at least 10 to the power 300, and so should not be
confused with legitimate data values).  In a native-format data file
they should be represented as \verb+NA+. When importing CSV data
\app{gretl} accepts several common representations of missing values
including $-$999, the string \verb+NA+ (in upper or lower case), a
single dot, or simply a blank cell.  Blank cells should, of course, be
properly delimited, e.g.\ \verb+120.6,,5.38+, in which the middle value
is presumed missing.

As for handling of missing values in the course of statistical
analysis, \app{gretl} does the following:

\begin{itemize}
\item In calculating descriptive statistics (mean, standard deviation,
  etc.) under the \cmd{summary} command, missing values are simply
  skipped and the sample size adjusted appropriately.
\item In running regressions \app{gretl} first adjusts the beginning
  and end of the sample range, truncating the sample if need be.
  Missing values at the beginning of the sample are common in time
  series work due to the inclusion of lags, first differences and so
  on; missing values at the end of the range are not uncommon due to
  differential updating of series and possibly the inclusion of leads.
\end{itemize}

If \app{gretl} detects any missing values ``inside'' the (possibly
truncated) sample range for a regression, the result depends on the
character of the dataset and the estimator chosen.  In many cases, the
program will automatically skip the missing observations when
calculating the regression results.  In this situation a message is
printed stating how many observations were dropped.  On the other
hand, the skipping of missing observations is not supported for all
procedures: exceptions include all autoregressive estimators, system
estimators such as SUR, and nonlinear least squares.  In the case of
panel data, the skipping of missing observations is supported only if
their omission leaves a balanced panel. If missing observations are
found in cases where they are not supported, \app{gretl} gives an
error message and refuses to produce estimates.

In case missing values in the middle of a dataset present a problem,
the \cmd{misszero} function (use with care!) is provided under the
\cmd{genr} command. By doing \cmd{genr foo = misszero(bar)} you can
produce a series \cmd{foo} which is identical to \cmd{bar} except that
any missing values become zeros.  Then you can use carefully
constructed dummy variables to, in effect, drop the missing
observations from the regression while retaining the surrounding
sample range.\footnote{\cmd{genr} also offers the inverse function to
  \cmd{misszero}, namely \cmd{zeromiss}, which replaces zeros in a
  given series with the missing observation code.}

\section{Data file collections}
\label{collections}

If you're using \app{gretl} in a teaching context you may be
interested in adding a collection of data files and/or scripts that
relate specifically to your course, in such a way that students can
browse and access them easily.

This is quite easy as of \app{gretl} version 1.2.1.  There are three
ways to access such collections of files:

\begin{itemize}
\item For data files: select the menu item ``File, Open data, sample
  file'', or click on the folder icon on the \app{gretl} toolbar.
\item For script files: select the menu item ``File, Open command
  file, practice file''.
\end{itemize}

When a user selects one of the items:

\begin{itemize}
\item The data or script files included in the gretl distribution are
  automatically shown (this includes files relating to Ramanathan's
  \emph{Introductory Econometrics} and Greene's \emph{Econometric
    Analysis}).
\item The program looks for certain known collections of data files
  available as optional extras, for instance the datafiles from
  various econometrics textbooks (Wooldridge, Gujarati, Stock and
  Watson) and the Penn World Table (PWT 5.6).  (See
  \href{http://gretl.sourceforge.net/gretl_data.html}{the data page}
  at the gretl website for information on these collections.)  If the
  additional files are found, they are added to the selection windows.
\item The program then searches for valid file collections (not
  necessarily known in advance) in these places: the ``system'' data
  directory, the system script directory, the user directory, and all
  first-level subdirectories of these.  (For reference, typical values
  for these directories are shown in Table~\ref{tab-colls}.)
\end{itemize}

\begin{table}[htbp]
    \begin{center}
    \begin{tabular}{lll}
      � & \multicolumn{1}{c}{\textit{Linux}} & 
        \multicolumn{1}{c}{\textit{MS Windows}} \\
        system data dir & \verb+/usr/share/gretl/data+ 
        & \verb+c:\userdata\gretl\data+\\
        system script dir & \verb+/usr/share/gretl/scripts+ 
        & \verb+c:\userdata\gretl\scripts+\\
        user dir & \verb+/home/me/gretl+ 
        & \verb+c:\userdata\gretl\user+\\
      \end{tabular}
    \end{center}
    \caption{Typical locations for file collections}
    \label{tab-colls}
  \end{table}

  Any valid collections will be added to the selection windows. So
  what constitutes a valid file collection?  This comprises either a
  set of data files in \app{gretl} XML format (with the \verb+.gdt+
  suffix) or a set of script files containing gretl commands (with
  \verb+.inp+ suffix), in each case accompanied by a ``master file''
  or catalog.  The \app{gretl} distribution contains several example
  catalog files, for instance the file \verb+descriptions+ in the
  \verb+misc+ sub-directory of the \app{gretl} data directory and
  \verb+ps_descriptions+ in the \verb+misc+ sub-directory of the
  scripts directory.  

  If you are adding your own collection, data catalogs should be named
  \verb+descriptions+ and script catalogs should be be named
  \verb+ps_descriptions+.  In each case the catalog should be placed
  (along with the associated data or script files) in its own specific
  sub-directory (e.g.\  \verb+/usr/share/gretl/data/mydata+ or
  \verb+c:\userdata\gretl\data\mydata+).

  The syntax of the (plain text) description files is straightforward.
  Here, for example, are the first few lines of gretl's ``misc'' data
  catalog:

\begin{code}
      # Gretl: various illustrative datafiles
      "arma","artificial data for ARMA script example"
      "ects_nls","Nonlinear least squares example"
      "hamilton","Prices and exchange rate, U.S. and Italy"
\end{code}

  The first line, which must start with a hash mark, contains a short
  name, here ``Gretl'', which will appear as the label for this
  collection's tab in the data browser window, followed by a colon,
  followed by an optional short description of the
  collection.

  Subsequent lines contain two elements, separated by a comma and
  wrapped in double quotation marks.  The first is a datafile name
  (leave off the \verb+.gdt+ suffix here) and the second is a short
  description of the content of that datafile.  There should be one
  such line for each datafile in the collection.

  A script catalog file looks very similar, except that there are
  three fields in the file lines: a filename (without its \verb+.inp+
  suffix), a brief description of the econometric point illustrated in
  the script, and a brief indication of the nature of the data used.
  Again, here are the first few lines of the supplied ``misc'' script
  catalog:

\begin{code}
      # Gretl: various sample scripts
      "arma.inp","ARMA modeling","artificial data"
      "ects_nls","Nonlinear least squares (Davidson)","artificial data"
      "leverage","Influential observations","artificial data"
      "longley","Multicollinearity","US employment"
\end{code}

  If you want to make your own data collection available to users,
  these are the steps:

  \begin{enumerate}
  \item Assemble the data, in whatever format is convenient.
  \item Convert the data to \app{gretl} format and save as \verb+gdt+
    files.  It is probably easiest to convert the data by importing
    them into the program from plain text, CSV, or a spreadsheet
    format (MS Excel or Gnumeric) then saving them. You may wish to
    add descriptions of the individual variables (the ``Variable, Edit
    attributes'' menu item), and add information on the source of the
    data (the ``Data, Edit info'' menu item).
  \item Write a descriptions file for the collection using a text
    editor.
  \item Put the datafiles plus the descriptions file in a subdirectory
    of the \app{gretl} data directory (or user directory).
  \item If the collection is to be distributed to other people,
    package the data files and catalog in some suitable manner, e.g.\
    as a zipfile.
  \end{enumerate}

  If you assemble such a collection, and the data are not proprietary,
  I would encourage you to submit the collection for packaging as a
  \app{gretl} optional extra.

%%% Local Variables: 
%%% mode: latex
%%% TeX-master: "gretl-guide"
%%% End: 

