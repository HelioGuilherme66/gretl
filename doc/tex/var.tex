\chapter{Multivariate time series models}
\label{chap:var}

\app{Gretl} provides a standard set of procedures for dealing with the
multivariate time-series models known as VARs (\emph{Vector
  AutoRegression}). More general models --- such as VARMAs, nonlinear
models or multivariate GARCH models --- are not provided as of now,
although it is entirely possible to estimate them by writing custom
procedures in the \app{gretl} scripting language. In this chapter, we
will briefly review \app{gretl}'s VAR toolbox.

\section{Notation}
\label{sec:var-def}

A VAR is a structure whose aim is to model the time persistence of a
vector of $n$ time series, $y_t$, via a multivariate autoregression,
as in
\begin{equation}
  \label{eq:VAR}
  y_t = A_1 y_{t-1} + A_2 y_{t-2} + \cdots + A_p y_{t-p} +
  B x_t + \epsilon_t 
\end{equation}
The number of lags $p$ is called the \emph{order} of the VAR. The
vector $x_t$, if present, typically contains a set of exogenous
variables, often including a constant, possibly with a time trend and
seasonal dummies. The vector $\epsilon_t$ is typically assumed to be a
vector white noise, with covariance matrix $\Sigma$.

Equation \eqref{eq:VAR} can be written more compactly as
\begin{equation}
  \label{eq:VARpoly}
  A(L) y_t = B x_t + \epsilon_t
\end{equation}
where $A(L)$ is a matrix polynomial in the lag operator, or as
\begin{equation}
  \label{eq:VARcompan}
  \left[\begin{array}{c} y_t \\ y_{t-1} \\ \cdots \\ y_{t-p-1} 
    \end{array} \right] = 
  \mathbf{A}
  \left[\begin{array}{c} y_{t-1} \\ y_{t-2} \\ \cdots \\ y_{t-p} 
    \end{array} \right] +
  \left[\begin{array}{c} B \\ 0 \\ \cdots \\ 0
    \end{array} \right] x_t +
  \left[\begin{array}{c} \epsilon_t \\ 0 \\ \cdots \\ 0 
    \end{array} \right]
\end{equation}
The matrix $\mathbf{A}$ is known as the ``companion matrix'' and equals
\[
\mathbf{A} =
  \left[\begin{array}{ccccc} 
      A_1 & A_2 & \cdots & A_p \\ 
      I & 0 & \cdots & 0 \\ 
      0 & I & \cdots & 0 \\ 
      \vdots & \vdots & \ddots & \vdots
    \end{array} \right]
\]
Equation \eqref{eq:VARcompan} is known as the ``companion form'' of
the VAR.

Another representation of interest is the so-called ``VMA
representation'', which is written in terms of a (typically infinite)
series of matrices $\Theta_i$ defined as
\begin{equation}
  \label{eq:VMA}
  \Theta_i = \frac{\partial y_t}{\partial \epsilon_{t-i}}
\end{equation}
The $\Theta_i$ matrices may be derived by recursive substitution in
equation \eqref{eq:VAR}: for example, assuming for simplicity that
$B=0$ and $p=1$, equation \eqref{eq:VAR} would become
\[
  y_t = A y_{t-1} + \epsilon_t
\]
which could be rewritten as
\[
  y_t = A^{n+1} y_{t-n-1} + \epsilon_t + A \epsilon_{t-1} + A^2
  \epsilon_{t-2} + \cdots + A^n \epsilon_{t-n}
\]
In this case $\Theta_i = A^i$. In general, it is possible to compute
$\Theta_i$ as the $n \times n$ north-west block of the $i$-th power of the
companion matrix $\mathbf{A}$ (so $\Theta_0$ is always an identity matrix).

The VAR is said to be \emph{stable} if all the eigenvalues of the
companion matrix $\mathbf{A}$ are smaller than 1 in absolute value, or
equivalently, if the matrix polynomial $A(L)$ in equation
\eqref{eq:VARpoly} is such that $|A(z)| = 0$ implies $|z|>1$. If this
is the case, $\lim_{n \to \infty} \Theta_n = 0$ and the vector $y_t$ is
stationary; as a consequence, the equation
\begin{equation}
  \label{eq:VMArep}
  y_t - E(y_t) = \sum_{i=0}^{\infty} \Theta_i \epsilon_{t-i}
\end{equation}
is a legitimate Wold representation. 

If the VAR is not stable, then the inferential procedures that are
called for become somewhat more specialized, except for some simple
cases. In particular, if the number of eigenvalues of $\mathbf{A}$
with modulus 1 is between 1 and $n-1$, the canonical tool to deal with
these models is the cointegrated VAR model, discussed in chapter
\ref{chap:vecm}.

\section{Estimation}
\label{sec:var-estim}

\begin{script}[htbp]
  \caption{Estimation of a VAR via OLS}
  \label{script:var-ols}
Input:
\begin{scodebit}
open sw_ch14.gdt
genr infl = 400*sdiff(log(PUNEW))

scalar p = 2
list X = LHUR infl
list Xlag = lags(p,X)

loop foreach i X
    ols $i const Xlag
end loop

var p X
\end{scodebit}
%$
Output (selected portions):
\begin{scodebit}
Model 1: OLS, using observations 1960:3-1999:4 (T = 158)
Dependent variable: LHUR

             coefficient   std. error   t-ratio   p-value 
  --------------------------------------------------------
  const       0.113673     0.0875210     1.299    0.1960  
  LHUR_1      1.54297      0.0680518    22.67     8.78e-51 ***
  LHUR_2     -0.583104     0.0645879    -9.028    7.00e-16 ***
  infl_1      0.0219040    0.00874581    2.505    0.0133   **
  infl_2     -0.0148408    0.00920536   -1.612    0.1090  

Mean dependent var   6.019198   S.D. dependent var   1.502549
Sum squared resid    8.654176   S.E. of regression   0.237830

...


VAR system, lag order 2
OLS estimates, observations 1960:3-1999:4 (T = 158)
Log-likelihood = -322.73663
Determinant of covariance matrix = 0.20382769
AIC = 4.2119
BIC = 4.4057
HQC = 4.2906
Portmanteau test: LB(39) = 226.984, df = 148 [0.0000]

Equation 1: LHUR

             coefficient   std. error   t-ratio   p-value 
  --------------------------------------------------------
  const       0.113673     0.0875210     1.299    0.1960  
  LHUR_1      1.54297      0.0680518    22.67     8.78e-51 ***
  LHUR_2     -0.583104     0.0645879    -9.028    7.00e-16 ***
  infl_1      0.0219040    0.00874581    2.505    0.0133   **
  infl_2     -0.0148408    0.00920536   -1.612    0.1090  

Mean dependent var   6.019198   S.D. dependent var   1.502549
Sum squared resid    8.654176   S.E. of regression   0.237830
\end{scodebit}
\end{script}

The \app{gretl} command for estimating a VAR is \cmd{var} which, in
the command line interface, is invoked in the following manner:
\begin{flushleft}
    \texttt{[ \emph{modelname} \textless - ] var \emph{p} \emph{Ylist} [;
    \emph{Xlist}]}
\end{flushleft}
where \texttt{p} is a scalar (the VAR order) and \texttt{Ylist} is a
list of variables describing the content of $y_t$. If the list
\texttt{Xlist} is absent, the vector $x_t$ is understood to contain a
constant only; if present, must be separated from \texttt{Ylist} by a
semi-colon and contains the other exogenous variables. Note, however,
that a few common choices can be obtained in a simpler way via
options: the options \app{gretl} provides are \option{trend},
\option{seasonals} and \option{nc} (no constant). Either
\texttt{Ylist} and \texttt{Xlist} may be named lists (see section
\ref{named-lists}). The ``\texttt{\textless -}'' construct can be used
to store the model under a name (see section
\ref{sect-script-objects}), if so desired. To estimate a VAR using the
graphical interface, choose ``Time Series, Vector Autoregression'',
under the Model menu.

The parameters in eq. \eqref{eq:VAR} are typically free from
restrictions, which implies that multivariate OLS provides a
consistent and asymptotically efficient estimator of all the
parameters.\footnote{In fact, under normality of $\epsilon_t$ OLS is
  indeed the conditional ML estimator. You may want to use other
  methods if you need to estimate a VAR in which some parameters are
  constrained.}  Given the simplicity of OLS, this is what every
software package, including \app{gretl}, uses: example script
\ref{script:var-ols} exemplifies the fact that the \cmd{var} command
gives you exactly the output you would have from a battery of OLS
regressions. The advantage of using the dedicated command is that,
after estimation is done, it makes it much easier to access certain
quantities and manage certain tasks. For example, the \dollar{coeff}
accessor returns the estimated coefficients as a matrix with $n$
columns and \dollar{sigma} returns an estimate of the matrix $\Sigma$,
the covariance matrix of $\epsilon_t$.

\begin{table}[htbp]
  \centering
  \begin{tabular}{rl}
    \hline
    Periodicity & horizon \\
    \hline
    Quarterly & 20 (5 years) \\
    Monthly & 24 (2 years) \\
    Daily & 3 weeks \\
    All other cases & 10 \\
    \hline
  \end{tabular}
  \caption{VMA horizon as a function of the dataset periodicity}
  \label{tab:var-horizon}
\end{table}

In addition, two accessors become available for the companion matrix
(\dollar{compan}) and the VMA representation (\dollar{vma}). The
latter deserves a detailed description: since the VMA representation
\eqref{eq:VMArep} is of infinite order, \app{gretl} defines a
\emph{horizon} up to which the $\Theta_i$ matrices are computed
automatically. By default, this is a function of the periodicity of
the data (see table \ref{tab:var-horizon}), but it can be set by the
user to any desired value via the \cmd{set} command with the
\cmd{horizon} parameter, as in
\begin{code}
set horizon 30
\end{code}
Calling the horizon $h$, the \dollar{vma} accessor returns an $(h+1)
\times n^2$ matrix, in which the $(i+1)$-th row is the vectorized form
of $\Theta_i$.

\section{VAR order selection}
\label{sec:lagselect}

\hrulefill FIXME: todo \hrulefill
\begin{flushleft}
    \texttt{var \emph{p} \emph{Ylist} [; \emph{Xlist}]} \option{lagselect}
\end{flushleft}

Here \texttt{p} is the \emph{maximum} lag.

Information crieria (see chapter \ref{select-criteria})

\hrulefill

\section{Structural VARs}
\label{sec:svar}

As of today, \app{Gretl} does not provide a native implementation for
the class of models known as ``Structural VARs''; however, it does
provide an implementation of the most classic, and certainly most
popular, version. 

\subsection{IRF and FEVD}

Assume that the disturbance in equation \eqref{eq:VAR} can be thought
of as a linear function of a vector of \emph{structural shocks} $u_t$,
which are assumed to have unit variance and to be incorrelated between
one another, so $V(u_t) = I$. If $\epsilon_t = K u_t$, it follows that
$\Sigma = V(\epsilon_t) = KK'$.

The main object of interest in this setting the sequence of matrices
\begin{equation}
  \label{eq:svma}
  C_k = \frac{\partial y_t}{\partial u_{t-i}} = \Theta_k K, 
\end{equation}
known as the structural VMA representation. From the $C_k$ matrices
defined in equation \eqref{eq:svma} two quantities of interest may be
derived: the \textbf{Impulse Response Function} (IRF) and the
\textbf{Forecast Error Variance Decomposition} (FEVD).

The IRF of variable $i$ to shock $j$ is simply the sequence of the
elements in row $i$ and column $j$ of the $C_k$ matrices. In formulae:
\[
  \mathcal{I}_{i,j,k} = \pder{y_{i,t}}{u_{j, t-k}}
\]
As a rule, Impulse Response Functions are plotted as a function of
$k$, and are interpreted as the effect that a shock has on an
observable variable through time. Of course, what we observe are the
estimated IRFs, so it is natural to endow them with confidence
intervals: following common practice among econometric software
\app{gretl} computes the confidence intervals by using the
bootstrap\footnote{It is possible, in principle, to compute analytical
  confidence intervals via an asymptotic approximation, but this is
  not a very popular choice: asymptotic formulae are known to often
  give a very poor approximation of the finite-sample properties.};
details are later in this section.

Another quantity of interest that may be computed from the structural
VMA representation is the Forecast Error Variance Decomposition
(FEVD). The forecast error variance after $h$ steps is given by
\[
  \Omega_h = \sum_{k=0}^h C_k C_k'
\]
hence the variance for variable $i$ is
\[
  \omega^2_i = \left[ \Omega_h \right]_{i,i} = \sum_{k=0}^h
  \mathrm{diag}(C_k C_k')_i =
  \sum_{k=0}^h \sum_{l=1}^n ({}_kc_{i.l})^2 
\]
where ${}_kc_{i.l}$ is, trivially, the $i,l$ element of $C_k$. As a
consequence, the share of uncertainty on variable $i$ that can be
attributed to the $j$-th shock after $h$ periods equals
\[
  \mathcal{VD}_{i,j,h} =
  \frac{\sum_{k=0}^h ({}_kc_{i.j})^2 }{  \sum_{k=0}^h \sum_{l=1}^n
    ({}_kc_{i.l})^2 } .
\]
This makes it possible to quantify which shocks are most important to
determine a certain variable in the short and/or in the long run.

\subsection{Triangularization}

The formula \ref{eq:svma} takes $K$ as known, while of course it has
to be estimated. The estimation problem has been the subject of an
enormous body of literature we will not even attempt to summarize
here: see for example \cite[chapter 9]{LKBook05}.

Suffice it to say that the most popular choice dates back to
\cite{sims80}, and consists in assuming that $K$ is lower triangular,
so its estimate is simply the Cholesky deconposition of the estimate of
$\Sigma$.

 \ldots Cholesky $\to$ ``most exogenous first'' mantra
\ldots

\subsection{IRF bootstrap}

FIXME: todo 


%%% Local Variables: 
%%% mode: latex
%%% TeX-master: "gretl-guide"
%%% End: 
