\chapter{Multivariate time series models}
\label{chap:var}

\app{Gretl} provides a standard set of procedures for dealing with the
multivariate time-series models known as VARs (\emph{Vector
  AutoRegression}). More general models, such as VARMAs, nonlinear
models or multivariate GARCH models, are not provided as of now,
although it is entirely possible to estimate them by writing custom
procedures in the \app{gretl} scripting language. In this chapter, we
will briefly review \app{gretl}'s VAR toolbox.

\section{Notation}
\label{sec:var-def}

A VAR of order $p$ is a structure whose aim is to model the time
persistence of a vector of $n$ time series via a multivariate
autoregression, as in
\begin{equation}
  \label{eq:VAR}
  y_t = A_1 y_{t-1} + A_2 y_{t-2} + \cdots + A_p y_{t-p} +
  B x_t + \epsilon_t 
\end{equation}
The number of lags $p$ is called the \emph{order} of the VAR; the
vector $x_t$ typically contains a vector of exogenous variables:
typical choices are a constant, possibly with a time trend and
seasonal dummies. The vector $\epsilon_t$ is typically assumed to be a
vector white noise, with covariance matrix $\Sigma$.

Equation \eqref{eq:VAR} can be written more compactly as
\begin{equation}
  \label{eq:VARpoly}
  A(L) y_t = B x_t + \epsilon_t ,
\end{equation}
where $A(L)$ is a matrix polynomial in the lag operator, or as
\begin{equation}
  \label{eq:VARcompan}
  \left[\begin{array}{c} y_t \\ y_{t-1} \\ \cdots \\ y_{t-p-1} 
    \end{array} \right] = 
  \mathbf{A}
  \left[\begin{array}{c} y_{t-1} \\ y_{t-2} \\ \cdots \\ y_{t-p} 
    \end{array} \right] +
  \left[\begin{array}{c} B \\ 0 \\ \cdots \\ 0
    \end{array} \right] x_t +
  \left[\begin{array}{c} \epsilon_t \\ 0 \\ \cdots \\ 0 
    \end{array} \right]
\end{equation}
where the matrix $\mathbf{A}$ is called ``companion matrix'' and equals
\[
\mathbf{A} =
  \left[\begin{array}{ccccc} 
      A_1 & A_2 & \cdots & A_p \\ 
      I & 0 & \cdots & 0 \\ 
      0 & I & \cdots & 0 \\ 
      \vdots & \vdots & \ddots & \vdots
    \end{array} \right] ;
\]
equation \eqref{eq:VARcompan} is known as the ``companion form'' of
the VAR.

Another representation of interest is the so-called ``VMA
representation'', which is written in terms of a (typically infinite)
series of matrices $C_i$ defined as
\begin{equation}
  \label{eq:VMA}
  C_i = \frac{\partial y_t}{\partial \epsilon_{t-i}}
\end{equation}
The $C_i$ matrices may be derived by recursive substitution in
equation \eqref{eq:VAR}: for example, assuming for simplicity that
$B=0$ and $p=1$, equation \eqref{eq:VAR} would become
\[
  y_t = A y_{t-1} + \epsilon_t ,
\]
which could be rewritten as
\[
  y_t = A^{n+1} y_{t-n-1} + \epsilon_t + A \epsilon_{t-1} + A^2
  \epsilon_{t-2} + \cdots + A^n \epsilon_{t-n} ;
\]
so in this case $C_i = A^i$. In general, it is possible to compute
$C_i$ as the $n \times n$ north-west block of the $i$-th power of the
companion matrix $\mathbf{A}$ (so $C_0$ is always an identity matrix).

The VAR is said to be \emph{stable} if all the eigenvalues of the
companion matrix $\mathbf{A}$ are smaller than 1 in absolute value, or
equivalently, if the matrix polynomial $A(L)$ in equation
\eqref{eq:VARpoly} is such that $|A(z)| = 0$ implies $|z|>1$. If this
is the case, $\lim_{n \to \infty} C_n = 0$ and the vector $y_t$ is
stationary; as a consequence, the equation
\begin{equation}
  \label{eq:VMArep}
  y_t - E(y_t) = \sum_{i=0}^{\infty} C_i \epsilon_{t-i}
\end{equation}
is a legitimate Wold representation. 

If the VAR is not stable, then the inferential procedures that are
called for become somewhat more specialized except for some simple
cases. In particular, if the number of eigenvalues of $\mathbf{A}$
with modulus 1 is between 1 and $n-1$, the canonical tool to deal with
these models is the cointegrated VAR model, for which you ought to
refer to chapter \ref{chap:vecm}.

\section{Estimation}
\label{sec:var-estim}

\begin{script}[htbp]
  \caption{Estimation of a VAR via OLS}
  \label{script:var-ols}
Input:
\begin{scodebit}
open sw_ch14.gdt
genr infl = 400*sdiff(log(PUNEW))

scalar p = 2
list X = LHUR infl
list Xlag = lags(p,X)

loop foreach i X
    ols $i const Xlag
end loop

var p X
\end{scodebit}
%$
Output (selected portions):
\begin{scodebit}
Model 1: OLS, using observations 1960:3-1999:4 (T = 158)
Dependent variable: LHUR

             coefficient   std. error   t-ratio   p-value 
  --------------------------------------------------------
  const       0.113673     0.0875210     1.299    0.1960  
  LHUR_1      1.54297      0.0680518    22.67     8.78e-51 ***
  LHUR_2     -0.583104     0.0645879    -9.028    7.00e-16 ***
  infl_1      0.0219040    0.00874581    2.505    0.0133   **
  infl_2     -0.0148408    0.00920536   -1.612    0.1090  

Mean dependent var   6.019198   S.D. dependent var   1.502549
Sum squared resid    8.654176   S.E. of regression   0.237830

...


VAR system, lag order 2
OLS estimates, observations 1960:3-1999:4 (T = 158)
Log-likelihood = -322.73663
Determinant of covariance matrix = 0.20382769
AIC = 4.2119
BIC = 4.4057
HQC = 4.2906
Portmanteau test: LB(39) = 226.984, df = 148 [0.0000]

Equation 1: LHUR

             coefficient   std. error   t-ratio   p-value 
  --------------------------------------------------------
  const       0.113673     0.0875210     1.299    0.1960  
  LHUR_1      1.54297      0.0680518    22.67     8.78e-51 ***
  LHUR_2     -0.583104     0.0645879    -9.028    7.00e-16 ***
  infl_1      0.0219040    0.00874581    2.505    0.0133   **
  infl_2     -0.0148408    0.00920536   -1.612    0.1090  

Mean dependent var   6.019198   S.D. dependent var   1.502549
Sum squared resid    8.654176   S.E. of regression   0.237830
\end{scodebit}
\end{script}

The \app{gretl} command for estimating a VAR is \cmd{var}. The
parameters in eq. \eqref{eq:VAR} are typically free from restrictions,
which implies that multivariate OLS provides a consistent and
asymptotically efficient estimator of all the parameters.\footnote{In
  fact, under normality of $\epsilon_t$ OLS is indeed the conditional
  ML estimator. You may want to use other methods if you need to
  estimate a VAR in which some parameters are constrained.}

Given the simplicity of OLS, this is what \app{gretl}
uses: example script \ref{script:var-ols} exemplifies the fact that
the \cmd{var} command gives you exactly the output you would have from
a battery of OLS regressions. The advantage of using the dedicated
command is that it makes it much easier to access quantities and
manage tasks that you normally want to, after estimation is done. For
example, the \dollar{coeff} returns the estimated coefficients as a
matrix with $n$ columns. In addition, two accessors become available
for the companion matrix (\dollar{compan}) and the VMA
representation (\dollar{vma}). 

\begin{table}[htbp]
  \centering
  \begin{tabular}{rl}
    \hline
    Periodicity & horizon \\
    \hline
    Quarterly & 20 (5 years) \\
    Monthly & 24 (2 years) \\
    Daily & 3 weeks \\
    All other cases & 10 \\
    \hline
  \end{tabular}
  \caption{VMA horizon as a function of the dataset periodicity}
  \label{tab:var-horizon}
\end{table}

The latter deserves a detailed
description: since the VMA representation \eqref{eq:VMArep} is of
infinite order, \app{gretl} defines a \emph{horizon} up to which the
$C_i$ matrices are computed automatically. By default, this is a
function of the periodicity of the data (see table
\ref{tab:var-horizon}), but it can be set by the user to any desired
value via the \cmd{set horizon} command. Calling the horizon $h$, the
\dollar{vma} accessor returns an $(h+1) \times n^2$ matrix, in which
the $(i+1)$-th row is the vectorized form of $C_i$.


\section{Structural VARs}
\label{sec:svar}

To be written. 

%%% Local Variables: 
%%% mode: latex
%%% TeX-master: "gretl-guide"
%%% End: 
