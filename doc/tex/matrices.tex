\chapter{Matrix manipulation}
\label{chap-matrices}

\section{Introduction}
\label{matrix-intro}

As of version 1.5.1, gretl offers the facility of creating and
manipulating user-defined matrices.  This is currently experimental;
it is not very advanced and some aspects of the syntax are liable to
change.  Nonetheless we set out the current situation here.

\section{Creating matrices}
\label{matrix-create}

Matrices can be created in any one of four ways:

\begin{enumerate}
\item By full specification of the scalar values that compose the
  matrix, in numerical form or by reference to pre-existing
  scalar variables, or both; or
\item by providing a list of data series; or
\item by providing a \textit{named list} of series; or
\item using a formula of the same general type that is used
  with the \texttt{genr} command, whereby a new matrix is defined
  in terms of existing matrices and/or scalars, or via some
  special functions.
\end{enumerate}

At present these methods cannot be mixed in the specification of a
given matrix.  Examples of each follow.

To specify a matrix directly in terms of scalars, the syntax is,
for example: 

\begin{code}
matrix A = { 1, 2, 3 ; 4, 5, 6 }
\end{code}

The matrix is defined by rows; the elements on each row are separated
by commas and the rows are separated by semi-colons.  The whole
expression must be wrapped in braces.  Spaces within the braces are
not significant.  The above expression defines a $2\times3$ matrix.
Each element must be either a numerical value or the name of a
pre-existing scalar variable.  Directly after the closing brace you
can append a single quote (\texttt{'}) to obtain the transpose.

To specify a matrix in terms of data series the syntax is, for
example,
%
\begin{code}
matrix A = { x1, x2, x3 }
\end{code}
%
where the names of the variables are separated by commas.  By default,
each variable occupies a column (and there can only be one variable
per column).  The range of data values included in the matrix depends
on the current setting of the sample range.

Please note that while gretl's built-in statistical functions are
capable of handling missing values, the matrix arithmetic functions
are not, and you will get an error if you try to build a matrix from
series that include missing values.

Instead of giving an explicit list of variables, you may instead
provide the name of a saved list, as in
%
\begin{code}
list xlist = x1 x2 x3
matrix A = { xlist }
\end{code}
%
When you provide a named list, the data series are by default placed
in columns, as is natural in an econometric context: if you want them
in rows, append the transpose symbol.

\section{Manipulating matrices}
\label{matrix-manip}

You can create new matrices or replace existing matrices by means of
various transformations, in a manner similar to the \texttt{genr}
command for scalars and data series.  To get a matrix result, however,
the command must start with the keyword \texttt{matrix}, not
\texttt{genr}.

The following operators are available for matrices:

\begin{center}
\begin{tabular}{ll}
\texttt{+} & addition \\
\texttt{-} & subtraction \\
\texttt{*} & ordinary matrix multiplication \\
\texttt{/} & matrix ``division'' (see below) \\
\texttt{.*} & element-wise multiplication \\
\texttt{./} & element-wise division \\
\verb+.^+ & element-wise exponentiation \\
\verb+~+ & column-wise concatenation \\
\texttt{**} & Kronecker product 
\end{tabular}
\end{center}

Here are explanations of the less obvious cases.  First, in matrix
``division'', $A/B$ is algebraically equivalent to $B^{-1}A$
(pre-multiplication by the inverse of the ``divisor'').  Therefore the
following two expressions are equivalent in principle:
%
\begin{code}
matrix C = A / B
matrix C = inv(B) * A
\end{code}
%
where \texttt{inv()} is the matrix inversion function (see below for
more on matrix functions).  The first form, however, may be more
accurate than the second; the solution is obtained via LU
decomposition, without the explicit calculation of the inverse matrix.

In element-wise multiplication if we write
%
\begin{code}
matrix C = A .* B
\end{code}
% 
then $c_{ij} = a_{ij} \times b_{ij}$.  Likewise with element-wise
division, $c_{ij} = a_{ij}/b_{ij}$. Element-wise exponentiation, as in
%
\begin{code}
C = A .^ k
\end{code}
% 
produces $c_{ij} = a_{ij}^k$.  The variable $k$ must be a scalar or
$1\times 1$ matrix.

In column-wise concatenation of an $m\times n$ matrix $A$ and
an $m\times p$ matrix $B$, the result is an $m\times (n+p)$ matrix.
That is,
%
\begin{code}
C = A ~ B
\end{code}
% 
produces $C = \left[ \begin{array}{cc} A & B \end{array} \right]$.

The following functions are available for element-by-element
transformations of matrices: \texttt{log}, \texttt{exp}, \texttt{sin},
\texttt{cos}, \texttt{tan}, \texttt{atan}, \texttt{int}, \texttt{abs},
\texttt{sqrt}, \texttt{dnorm}, \texttt{cnorm}, \texttt{qnorm},
\texttt{gamma} and \texttt{lngamma}.  These functions have the same
meanings as in \texttt{genr}.  For example, if a matrix \texttt{A} is
already defined, then
%
\begin{code}
matrix B = sqrt(A)
\end{code}
%
generates a matrix such that $b_{ij} = \sqrt{a_{ij}}$.  All of these
functions require a single matrix as argument, or an expression which
evaluates to a single matrix.

In addition these matrix-specific functions are available:

\begin{center}
\begin{tabular}{ll}
\texttt{det()} & determinant \\
\texttt{ldet()} & log-determinant \\
\texttt{inv()} & inverse \\
\texttt{tr()} & trace \\
\texttt{rows()} & number of rows \\
\texttt{cols()} & number of columns \\
\texttt{diag()} & extract principal diagonal as column vector \\
\texttt{transp()} & transpose 
\end{tabular}
\end{center}

These functions also require a single matrix (or an expression which
evaluates to a single matrix) as argument.

Finally, there are some functions which generate certain sorts of
matrices from scratch, given a specification of the desired
dimensions.

\begin{center}
\begin{tabular}{ll}
\texttt{I(}\textsl{n}\texttt{)} & $n\times n$ identity matrix \\
\texttt{zeros(}\textsl{m}\texttt{,}\textsl{n}\texttt{)} & 
   $m\times n$ zero matrix \\
\texttt{ones(}\textsl{m}\texttt{,}\textsl{n}\texttt{)} &
   $m\times n$ matrix filled with 1s \\
\texttt{uniform(}\textsl{m}\texttt{,}\textsl{n}\texttt{)} &
   $m\times n$ matrix filled with uniform random values \\
\texttt{normal(}\textsl{m}\texttt{,}\textsl{n}\texttt{)} &
   $m\times n$ matrix filled with normal random values \\
\end{tabular}
\end{center}

The values \textsl{m} and \textsl{n} may be given numerically, or by
reference to pre-existing scalar variables, as in
%
\begin{code}
scalar m = 4
scalar n = 5
matrix A = normal(m,n)
\end{code}
%
The \texttt{uniform()} and \texttt{normal()} matrix functions fill the
matrix with drawings from the uniform (0--1) distribution and the
standard normal distribution respectively.

\section{Selecting sub-matrices}
\label{matrix-sub}

You can select sub-matrices of a given matrix using the syntax

\texttt{A[}\textsl{rows},\textsl{cols}\texttt{]}

where \textsl{rows} can take one of four forms:

\begin{center}
\begin{tabular}{ll}
empty & selects all rows \\
a single integer & selects the single specified row \\
two integers separated by a colon & selects a range of rows \\
the name of a matrix & selects the specified rows
\end{tabular}
\end{center}

With regard to the last option, the matrix given in the \textsl{rows}
field must be either $p\times 1$ or $1\times p$, and should contain
integer values in the range 1 to $n$, where $n$ is the number of rows
in the matrix from which the selection is to be made.

The \textsl{cols} specification works in the same way, \textit{mutatis
  mutandis}.  Here are some examples.
%
\begin{code}
matrix B = A[1,]
matrix B = A[2:3,3:5]
matrix B = A[2,2]
matrix idx = { 1, 2, 6 }
matrix B = A[idx,]
\end{code}
%
The first example selects row 1 from matrix \texttt{A}; the second
selects a $2\times 3$ submatrix; the third selects a scalar; and
the fourth selects rows 1, 2, and 6 from matrix \texttt{A}.

You can use selections of this sort on either the right-hand side of
a matrix-generating formula or the left.  Here is an example of use of
a selection on the right, to extract a $2\times 2$ submatrix $B$ from a
$3\times 3$ matrix $A$:
%
\begin{code}
matrix A = { 1, 2, 3; 4, 5, 6; 7, 8, 9 }
matrix B = A[1:2,2:3]
\end{code}
%
And here is an example of a selection on the left, to write a $2\times
2$ identity matrix into the bottom right corner of a $3\times 3$
matrix $A$:
%
\begin{code}
matrix A = { 1, 2, 3; 4, 5, 6; 7, 8, 9 }
matrix A[2:3,2:3] = I(2)
\end{code}

\section{Extended example}
\label{matrix-example}

Example \ref{examp-matrix} shows how matrix methods can be used to
replicate gretl's built-in OLS functionality.  The example illustrates
various additional points.  First, if you just write \texttt{matrix
  A}, where a matrix \texttt{A} is already defined, the effect is to
print the matrix.  Second, there is some ``cross over'' between matrix
expressions and \texttt{genr} (actually the synonym \texttt{scalar} is
used in the script).  In a \texttt{genr} formula, you can use matrix
functions that produce scalar results (e.g.\ \texttt{rows()}).  You
can also reference $1\times 1$ matrices as if they were ordinary
scalars (e.g.\ \texttt{SSR}).  And in a \texttt{matrix} formula you
can reference scalar variables where appropriate.  Note, however, that
ordinary data series cannot be used in \texttt{matrix} expressions,
other than in the special case of defining a matrix from a list of
series as in section~\ref{matrix-create} above.

\begin{script}[htbp]
  \caption{OLS via matrix methods}
  \label{examp-matrix}
\begin{code}
  open data4-1
  matrix X = { const, sqft }
  matrix y = { price }
  matrix b = inv(X'*X) * X'*y
  printf "estimated coefficient vector\n"
  matrix b
  matrix uh = y - X*b
  matrix SSR = uh'*uh
  matrix SSR
  scalar s2 = SSR / (rows(X) - rows(b))
  matrix V = s2 * inv(X'*X)
  matrix V
  matrix se = sqrt(diag(V))
  printf "estimated standard errors\n"
  matrix se
  # compare with built-in function
  ols price const sqft --vcv
\end{code}
\end{script}



















