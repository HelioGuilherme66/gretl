\chapter{Introduction}
\label{intro}

\section{Name me}
%\label{}

Hansl is gretl's scripting language; the name expands, recursively, to
``hansl's a neat scripting language''. Although the name is new (we
got tired of saying ``gretl's scripting language'' and wanted a
shorthand) the concept is not: gretl has supported scripting since its
inception. In this Guide we explain how to make the most of gretl via
scripting.

So what exactly is a hansl script? It is a plain text file containing
a set of \textsl{statements}, to be executed by gretl. These
statements fall into three broad classes: commands, function calls,
and syntactical ``glue'' that enables you to make parts of a script
conditional (\texttt{if} \dots{} \texttt{endif}), to repeat sections
of the script (\texttt{loop} \dots{} \texttt{endloop}), to define your
own functions, and so on.

      
%%% Local Variables: 
%%% mode: latex
%%% TeX-master: "hansl-guide"
%%% End: 
