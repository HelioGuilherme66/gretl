\chapter{Nice-looking output}
\label{chap:formatting}

\section{Formatted output}
\label{sec:printf}

To give you more control over output, hansl provides you with the
\cmd{printf} command, which works in a very similar way as in a great
number of programming languages.

Its syntax goes like this:
\begin{flushleft}
  \texttt{printf \emph{format string}, \emph{arguments}}
\end{flushleft}

The \cmd{printf} command gives you extreme flexibility in what to
print and how. The \emph{format string} is used to specify the precise
way in which you want the \emph{arguments} to be printed.

\subsection{The format string}
\label{sec:fmtstring}

Technically, the \cmd{printf} command prints out the format string;
this must be an expression that evaluates to a string, but in most
cases will just be a string literal, that is an alphanumeric sequence
surrounded by double quotes. However, some character sequences in the
format string have a special meaning: those beginning with the percent
character (\texttt{\%}) are interpreted as ``placeholders'' for the
items contained in the argument list; moreover, special characters
such as the newline character are represented via a combination
beginning with the backslash (\verb|\|).

For example,
\begin{code}
printf "The square root of %d is (roughly) %6.4f.\n", 5, sqrt(5)
\end{code}
will print 
\begin{code}
The square root of 5 is (roughly) 2.2361.
\end{code}

Let's see how:
\begin{itemize}
\item The first special sequence is \verb|%d|: this indicates that we
  want an integer number at that place in the output string; since it
  is the leftmost ``percent'' expression, it is matched to the first
  argument, that is 5.
\item The second special sequence is \verb|%6.4f|, which stands for a
  decimal value with 4 digits after the decimal separator\footnote{The
    decimal separator is the dot in English, but may be different in
    other locales.} and at least 6 digits wide; this will be matched
  to the second argument. Note that arguments are separated by
  commas. Also note that the second argument is not a scalar constant
  nor a scalar variable, but an expression that evaluates to a scalar.
\item The format string ends with the sequence \verb|\n|, which
  inserts a newline.
\end{itemize}

The escape sequences \verb|\n| (newline), \verb|\t| (tab), \verb|\v|
(vertical tab) and \verb|\\| (literal backslash) are recognized. To
print a literal percent sign, use \verb|%%|.

Apart from those shown in the above example, recognized numeric
formats are \verb|%e|, \verb|%E|, \verb|%f|, \verb|%g|, and \verb|%G|,
in each case with the various modifiers available in C. The format
\verb|%s| should be used for strings. As in C, numerical values that
form part of the format (width and or precision) may be given directly
as numbers, as in \verb|%10.4f|, or they may be given as variables. In
the latter case, one puts asterisks into the format string and
supplies corresponding arguments in order. For example,

\begin{code}
  scalar width = 12 
  scalar precision = 6 
  printf "x = %*.*f\n", width, precision, x
\end{code}

If a matrix argument is given in association with a numeric format,
the entire matrix is printed using the specified format for each
element. A few more examples are given in table \ref{tab:printf-ex}.
\begin{table}[htbp]
  \centering
  \begin{footnotesize}
    \begin{tabular}{p{0.6\textwidth}p{0.3\textwidth}}
      \hline
      The command & produces \\
      \hline
      \verb|printf "%12.3f", $pi| & 3.142 \\
      \verb|printf "%12.7f", $pi| & 3.1415927 \\
      \verb|printf "%6s%12.5f%12.5f %d\n", "alpha", 3.5, 9.1, 3| &
      \verb| alpha     3.50000     9.10000 3| \\
      \verb|printf "%6s%12.5f%12.5f\t%d\n", "beta", 1.2345, 1123.432, 11| &
      \verb|  beta     1.23450  1123.43200 11| \\
      \verb|printf "%d, %10d, %04d\n", 1,2,3| & 
      \verb|1,          2, 0003| \\
      \verb|printf "%6.0f (%5.2f%%)\n", 32, 11.232| & \verb|32 (11.23%)| \\
      \hline
    \end{tabular}
  \end{footnotesize}
  \caption{Print format examples}
  \label{tab:printf-ex}
\end{table}

\subsection{Output to a string}
\label{sec:sprintf}

A closely related effect can be achieved via the \cmd{sprintf}
function: instead of printing out the result, it stores it to a
string, as in
\begin{code}
  string G = sprintf("x = %*.*f\n", width, precision, x)
\end{code}
after which the variable \texttt{G} can be the object of further
processing.

\tip{Both \cmd{printf} and \cmd{sprintf} started their career as
  commands. Both of them can now be used as functions too. This is
  going to go at some point: in the future, \cmd{sprintf} will
  retain the function form only; as for \cmd{printf}, we haven't made
  a final decision yet.}

\subsection{Output to a file}
\label{sec:outfile}

If you want to write your output to a file, the \cmd{outfile} command
diverts output to filename, until further notice, so you'll want to
put it before the commands which print out the material you want to
save. Use the flag \option{append} to append output to an existing
file or \option{write} to start a new file (or overwrite an existing
one). Only one file can be opened in this way at any given time. The
\option{close} flag is used to close an output file that was
previously opened as above. Output will then revert to the default
stream.

For example:
\begin{code}
  printf "One!\n"
  outfile "myfile.txt" --write
  printf "Two!\n"
  outfile "myfile.txt" --close
  printf "Three!\n"
  outfile "myfile.txt" --append
  printf "Four!\n"
  outfile "myfile.txt" --close
  printf "Five!\n"
\end{code}
will result in the file \texttt{myfile.txt} containing the lines
\begin{code}
Two!
Four!  
\end{code}

Three special variants on the above are available. If you give the
keyword \texttt{null} in place of a real filename along with the write
option, the effect is to suppress all printed output until redirection
is ended.\footnote{For Unix geeks: if either of the keywords
  \texttt{stdout} or \texttt{stderr} are given in place of a regular
  filename the effect is to redirect output to standard output or
  standard error output, respectively.}

The \option{quiet} option is for use with \option{write} or
\option{append}: its effect is to turn off the echoing of commands and
the printing of auxiliary messages while output is redirected. It is
equivalent to doing
\begin{code}
  set echo off 
  set messages off
\end{code}
except that when redirection is ended the original values of the echo
and messages variables are restored.

\section{Graphics}

The primary graphing command in hansl is \texttt{gnuplot} which, as
the name suggests, provides an interface to the \textsf{gnuplot}
program. It is used for plotting series in a dataset (see
part~\ref{part:hp-data}) or columns in a matrix. For discussion of
this command (and some other more specialized ones, such as
\texttt{boxplot} and \texttt{qqplot}), see the \GUG.

%%% Local Variables: 
%%% mode: latex
%%% TeX-master: "hansl-primer"
%%% End: 
