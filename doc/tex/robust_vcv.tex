\chapter{Robust covariance matrix estimation}
\label{chap-robust-vcv}

\section{Introduction}
\label{vcv-intro}

Consider (once again) the linear regression model
%
\begin{equation}
\label{eq:ols-again}
y = X\beta + u
\end{equation}
%
where $y$ and $u$ are $T$-vectors, $X$ is a $T \times k$ matrix of
regressors, and $\beta$ is a $k$-vector of parameters.  As is well
known, the estimator of $\beta$ given by Ordinary Least Squares (OLS)
is
%
\begin{equation}
\label{eq:ols-betahat}
\hat{\beta} = (X'X)^{-1} X'y
\end{equation}
%
If the condition $E(u|X) = 0$ is satisfied, this is an unbiased
estimator; under somewhat weaker conditions the estimator is biased
but consistent.  It is straightforward to show that when the OLS
estimator is unbiased (that is, when $E(\hat{\beta}-\beta) = 0$), its
variance is
%
\begin{equation}
\label{eq:ols-varb}
\mbox{Var}(\hat{\beta}) = 
  E\left((\hat{\beta}-\beta)(\hat{\beta}-\beta)'\right) 
  = (X'X)^{-1} X' \Omega X (X'X)^{-1}
\end{equation}
%
where $\Omega = E(uu')$ is the covariance matrix of the error terms.

Under the assumption that the error terms are independently and
identically distributed (iid) we can write $\Omega = \sigma^2 I$,
where $\sigma^2$ is the (common) variance of the errors (and the
covariances are zero).  In that case (\ref{eq:ols-varb}) simplifies to
the ``classical'' formula,
%
\begin{equation}
\label{eq:ols-classical-varb}
\mbox{Var}(\hat{\beta}) = \sigma^2(X'X)^{-1}
\end{equation}

If the iid assumption is not satisfied, two things follow.  First, it
is possible in principle to construct a more efficient estimator than
OLS --- for instance some sort of Feasible Generalized Least Squares
(FGLS).  Second, the simple ``classical'' formula for the variance of
the least squares estimator is no longer correct, and hence the
conventional OLS standard errors --- which are just the square roots
of the diagonal elements of the matrix defined by
(\ref{eq:ols-classical-varb}) --- do not provide valid means of
statistical inference.

In the recent history of econometrics there are broadly two approaches
to the problem of non-iid errors.  The ``traditional'' approach is to
use an FGLS estimator.  For example, if the departure from the iid
condition takes the form of time-series dependence, and if one
believes that this could be modeled as a case of first-order
autocorrelation, one might employ an AR(1) estimation method such as
Cochrane--Orcutt, Hildreth--Lu, or Prais--Winsten.  If the problem is
that the error variance is non-constant across observations, one might
estimate the variance as a function of the independent variables and
then perform weighted least squares, using as weights the reciprocals
of the estimated variances.

While these methods are still in use, an alternative approach has
found increasing favor: that is, use OLS but compute standard errors
(or more generally, covariance matrices) that are robust with respect
to deviations from the iid assumption.  This is typically combined
with an emphasis on using large datasets --- large enough that the
researcher can place some reliance on the (asymptotic) consistency
property of OLS.  This approach has been enabled by the availability
of cheap computing power.  The computation of robust standard errors
and the handling of very large datasets were daunting tasks at one
time, but now they are unproblematic.  The other point favoring the
newer methodology is that while FGLS offers an efficiency advantage in
principle, it often involves making additional statistical assumptions
which may or may not be justified, which may not be easy to test
rigorously, and which may threaten the consistency of the estimator
--- for example, the ``common factor restriction'' that is implied by
traditional FGLS ``corrections'' for autocorrelated errors.

James Stock and Mark Watson's \textit{Introduction to Econometrics}
illustrates this approach at the level of undergraduate instruction:
many of the datasets they use comprise thousands or tens of thousands
of observations; FGLS is downplayed; and robust standard errors are
reported as a matter of course.  In fact, the discussion of the
classical standard errors (labeled ``homoskedasticity-only'') is
confined to an Appendix.

Against this background it may be useful to set out and discuss all
the various options offered by \app{gretl} in respect of robust
covariance matrix estimation.  The first point to notice is that
\app{gretl} produces ``classical'' standard errors by default (in all
cases apart from GMM estimation).  In script mode you can get robust
standard errors by appending the \verb|--robust| flag to estimation
commands.  In the GUI program the model specification dialog usually
contains a ``Robust standard errors'' check box, along with a
``configure'' button that is activated when the box is checked.  The
configure button takes you to a configuration dialog (which can also
be reached from the main menu bar: Tools $\rightarrow$ Preferences
$\rightarrow$ General $\rightarrow$ HCCME).  There you can select from
a set of possible robust estimation variants, and can also choose to
make robust estimation the default.

The specifics of the available options depend on the nature of the
data under consideration --- cross-sectional, time series or panel ---
and also to some extent the choice of estimator.  (Although we
introduced robust standard errors in the context of OLS above, they
may be used in conjunction with other estimators too.)  The following
three sections of this chapter deal with matters that are specific to
the three sorts of data just mentioned.  Note that additional details
regarding covariance matrix estimation in the context of GMM are given
in chapter~\ref{chap:gmm}.

We close this introduction with a brief statement of what ``robust
standard errors'' can and cannot achieve.  They can provide for
asymptotically valid statistical inference in models that are
basically correctly specified, but in which the errors are not iid.
The ``asymptotic'' part means that they may be of little use in small
samples.  The ``correct specification'' part means that they are not a
magic bullet: if the error term is correlated with the regressors, so
that the parameter estimates themselves are biased and inconsistent,
robust standard errors will not save the day.


\section{Cross-sectional data and the HCCME}
\label{vcv-hccme}

With cross-sectional data, the most likely departure from iid errors
is heteroskedasticity (non-constant variance).\footnote{In some
  specialized contexts spatial autocorrelation may be an issue.  Gretl
  does not have any built-in methods to handle this and we will not
  discuss it here.}  In some cases one may be able to arrive at a
judgment regarding the likely form of the heteroskedasticity, and
hence to apply a specific correction.  The more common case, however,
is where the heteroskedasticity is of unknown form.  We seek an
estimator of the covariance matrix of the parameter estimates that
retains its validity, at least asymptotically, in face of unspecified
heteroskedasticity.  It is not obvious, a priori, that this should be
possible, but White (1980) showed that
%
\begin{equation}
\label{eq:ols-varb-h}
\widehat{\mbox{Var}}_{\rm h}(\hat{\beta}) = 
       (X'X)^{-1} X' \hat{\Omega} X (X'X)^{-1}
\end{equation}
%
does the trick. (As usual in statistics, we need to say ``under
certain conditions'', but the conditions are not very restrictive.)
$\hat{\Omega}$ is in this context a diagonal matrix, whose non-zero
elements may be estimated using squared OLS residuals.  White referred
to (\ref{eq:ols-varb-h}) as a heteroskedasticity-consistent covariance
matrix estimator (HCCME).

Davidson and MacKinnon (2004, chapter 5) offer a useful discussion of
several variants on White's HCCME theme. They refer to the original
variant of (\ref{eq:ols-varb-h}) --- in which the diagonal elements of
$\hat{\Omega}$ are estimated directly by the squared OLS residuals,
$\hat{u}^2_t$ --- as HC$_0$.  (The associated standard errors are
often called ``White's standard errors''.)  The various refinements of
White's proposal share a common point of departure, namely the idea
that the squared OLS residuals are likely to be ``too small'' on
average.  This point is quite intuitive.  The OLS parameter estimates,
$\hat{\beta}$, satisfy by design the criterion that the sum of
squared residuals,
%
\[
\sum \hat{u}^2_t = \sum \left( y_t - X_t \hat{\beta} \right)^2
\]
%
is minimized for given $X$ and $y$.  Suppose that $\hat{\beta} \neq
\beta$.  This is almost certain to be the case: even is OLS is not
biased, it would be a miracle if the $\hat{\beta}$ calculated from any
finite sample were exactly equal to $\beta$.  But in that case the sum
of squares of the true, unobserved \textit{errors}, $\sum u^2_t = \sum
(y_t - X_t \beta)^2$ is bound to be greater than $\sum \hat{u}^2_t$.
The elaborated variants on HC$_0$ take this point on board as follows:
%
\begin{itemize}
\item HC$_1$: Applies a degrees-of-freedom correction, multiplying the
  HC$_0$ matrix by $T/(T-k)$.
\item HC$_2$: Instead of using $\hat{u}^2_t$ for the diagonal elements
  of $\hat{\Omega}$, uses $\hat{u}^2_t/(1-h_t)$, where $h_t =
  X_t(X'X)^{-1}X'_t$, the $t^{\rm th}$ diagonal element of the ``hat''
  matrix, $P$, which has the property that $P\cdot X = \hat{y}$.  The
  relevance of $h_t$ is that if the variance of all the $u_t$ is
  $\sigma^2$, the expectation of $\hat{u}^2_t$ is $\sigma^2(1-h_t)$,
  or in other words, the ratio $\hat{u}^2_t/(1-h_t)$ has expectation
  $\sigma^2$.  As Davidson and MacKinnon show, $0\leq h_t <1$ for all
  $t$, so this adjustment cannot reduce the the diagonal elements of
  $\hat{\Omega}$ and in general revises them upward.
\item HC$_3$: Uses $\hat{u}^2_t/(1-h_t)^2$.  The additional factor of
  $(1-h_t)$ in the denominator, relative to HC$_2$, may be justified
  on the grounds that observations with large variances tend to exert
  a lot of influence on the OLS estimates, so that the corresponding
  residuals tend to be under-estimated.  See Davidson and MacKinnon
  for more details.  
\end{itemize}

The relative merits of these variants have been explored by means of
both simulations and theoretical analysis.  Unfortunately there is not
a clear consensus on which is ``best''.  Davidson and MacKinnon argue
that the original HC$_0$ is likely to perform worse than the others;
nonetheless, ``White's standard errors'' are reported more often than
the more sophisticated variants and therefore, for reasons of
comparability, HC$_0$ is the default HCCME in \app{gretl}.

If you wish to use HC$_1$, HC$_2$ or HC$_3$ you can arrange for this
in either of two ways.  In script mode, you can do, for example,
%
\begin{code}
set hc_version 2
\end{code}
%
In the GUI program you can go to the HCCME configuration dialog, as
noted above, and choose any of these variants to be the default.


\section{Time series data and HAC covariance matrices}
\label{vcv-hac}

Heteroskedasticity may be an issue with time series data too, but it
is unlikely to be the only, or even the primary, concern.  

One form of heteroskedasticity is common in macroeconomic time series,
but is fairly easily dealt with.  That is, in the case of strongly
trending series such as Gross Domestic Product, aggregate consumption,
aggregate investment, and so on, higher levels of the variable in
question are likely to be associated with higher variability in
absolute terms.  The obvious ``fix'', employed in many
macroeconometric studies, is to use the logs of such series rather
than the raw levels.  Provided the \textit{proportional} variability
of such series remains roughly constant over time, the log
transformation is effective.  

Other forms of heteroskedasticity may resist the log transformation,
but may demand a special treatment distinct from the calculation of
robust standard errors.  We have in mind here \textit{autoregressive
  conditional} heteroskedasticity, for example in the behavior of
asset prices, where large disturbances to the market may usher in
periods of increased volatility.  Such phenomena call for specific
estimation strategies, such as GARCH (see chapter~\ref{chap:timeser}).

Despite the points made above, some residual degree of
heteroskedasticity may be present in time series data: the key point
is that in most cases it is likely to be combined with serial
correlation (autocorrelation), hence demanding a special treatment.
In White's approach, $\hat{\Omega}$, the estimated covariance matrix
of the $u_t$, remains conveniently diagonal: the variances,
$E(u^2_t)$, may differ by $t$ but the covariances, $E(u_t u_s)$, are
all zero.  Autocorrelation in time series data means that at least
some of the the diagonal elements of $\hat{\Omega}$ should be
non-zero.  This introduces a substantial complication and requires
another piece of terminology; estimates of the covariance matrix that
are asymptotically valid in face of both heteroskedasticity and
autocorrelation of the error process are termed HAC
(heteroskedasticity and autocorrelation consistent).

The issue of HAC estimation is treated in more technical terms in
chapter~\ref{chap:gmm}.  Here we try to convey some of the intuition
at a more basic level.

The ``obvious'' extension of White's HCCME to the case of
autocorrelated errors would seem to be this: estimate the off-diagonal
elements (autocovariances) of $\hat{\Omega}$ using, once again, the
appropriate OLS residuals: $\hat{\omega}_{ts} = \hat{u}_t \hat{u}_s$.
This is basically right, but demands an important amendment.  We seek
a \textit{consistent} estimator, one that converges towards the true
$\Omega$ as the sample size tends towards infinity.  This can't work
if we allow unbounded serial dependence.  Bigger samples will enable
us to estimate more of the true $\omega_{ts}$ elements (that is, for
$t$ and $s$ more widely separated in time) but will \textit{not}
contribute ever-increasing information regarding the maximally
separated $\omega_{ts}$ pairs, since the maximal accessible separation
itself grows with the sample size.  To ensure consistency, we have to
confine our attention to processes exhibiting temporally limited
dependence, or in other words cut off the computation of the
$\hat{\omega}_{ts}$ values at some maximum value of $p = t-s$ (where
$p$ may be treated as an increasing function of the sample size, $T$,
though it cannot increase in proportion to $T$).

The simplest variant of this idea is to truncate the computation at
some finite lag order $p$, where $p$ grows as, say, $T^{1/4}$.  The
trouble with this is that the resulting $\hat{\Omega}$ may not be a
positive definite matrix.  In practical terms, we may end up with
negative estimated variances, in which case the robust standard errors
that we seek are undefined (being the square roots of negative
numbers).  The Newey--West estimator works around this problem by
assigning declining weights to the sample autocovariances as the
temporal separation increases.



\section{Special issues with panel data}
\label{vcv-panel}




%%% Local Variables: 
%%% mode: latex
%%% TeX-master: "gretl-guide"
%%% End: 