\chapter{Named lists and strings}
\label{chap-persist}

%%% Should this chapter deal with saving models too?

\section{Named lists}
\label{named-lists}

Many \app{gretl} commands take one or more lists of series as
arguments.  To make this easier to handle in the context of command
scripts, and in particular within user-defined functions, \app{gretl}
offers the possibility of \textit{named lists}.  

\subsection{Creating and modifying named lists}

A named list is created using the keyword \texttt{list}, followed by
the name of the list, an equals sign, and an expression that forms a
list.  The most basic sort of expression that works in this context
is a space-separated list of variables, given either by name or
by ID number.  For example,
%
\begin{code}
list xlist = 1 2 3 4
list reglist = income price 
\end{code}

Note that the variables in question must be of the series type: you
can't include scalars in a named list.

Two special forms are available:
\begin{itemize}
\item If you use the keyword \texttt{null} on the right-hand side,
  you get an empty list.
\item If you use the keyword \texttt{dataset} on the right, you get
  a list containing all the series in the current dataset (except
  the pre-defined \texttt{const}).
\end{itemize}

The name of the list must start with a letter, and must be composed
entirely of letters, numbers or the underscore character.  The maximum
length of the name is 15 characters; list names cannot contain
spaces.  

Once a named list has been created, it will be ``remembered'' for the
duration of the \app{gretl} session, and can be used in the context of
any \app{gretl} command where a list of variables is expected.  One
simple example is the specification of a list of regressors:
%
\begin{code}
list xlist = x1 x2 x3 x4
ols y 0 xlist
\end{code}

Lists can be modified in two ways.  To \textit{redefine} an existing
list altogether, use the same syntax as for creating a list.  For
example
%
\begin{code}
list xlist = 1 2 3
xlist = 4 5 6
\end{code}

After the second assignment, \texttt{xlist} contains just variables 4,
5 and 6.

To \textit{append} or \textit{prepend} variables to an existing list,
we can make use of the fact that a named list stands in for a
``longhand'' list.  For example, we can do
%
\begin{code}
list xlist = xlist 5 6 7
xlist = 9 10 xlist 11 12
\end{code}
%
Another option for appending a term (or a list) to an existing list is
to use \texttt{+=}, as in
%
\begin{code}
xlist += cpi
\end{code}
%
To drop a variable from a list, use \texttt{-=}:
%
\begin{code}
xlist -= cpi
\end{code}
%

In most contexts where lists are used in \app{gretl}, it is expected
that they do not contain any duplicated elements.  If you form a new
list by simple concatenation, as in \texttt{list L3 = L1 L2}
(where \texttt{L1} and \texttt{L2} are existing lists), it's possible
that the result may contain duplicates.  To guard against this you can
form a new list as the union of two existing ones:
%
\begin{code}
list L3 = L1 || L2
\end{code}
%
The result is a list that contains all the members of \texttt{L1},
plus any members of \texttt{L2} that are not already in \texttt{L1}.

In the same vein, you can construct a new list as the intersection of
two existing ones:
%
\begin{code}
list L3 = L1 && L2
\end{code}
%
Here \texttt{L3} contains all the elements that are present in both
\texttt{L1} and \texttt{L2}.


\subsection{Lists and matrices}

Another way of forming a list is by assignment from a matrix.  The
matrix in question must be interpretable as a vector containing ID
numbers of (series) variables.  It may be either a row or a column
vector, and each of its elements must have an integer part that is
no greater than the number of variables in the data set.  For example:
%
\begin{code}
matrix m = {1,2,3,4}
list L = m
\end{code}
%
The above is OK provided the data set contains at least 4 variables.

\subsection{Querying a list}

You can determine whether an unknown variable actually represents a list
using the function \texttt{islist()}.
%
\begin{code}
series xl1 = log(x1)
series xl2 = log(x2)
list xlogs = xl1 xl2
genr is1 = islist(xlogs)
genr is2 = islist(xl1)
\end{code}

The first \texttt{genr} command above will assign a value of 1 to
\texttt{is1} since \texttt{xlogs} is in fact a named list.  The second
genr will assign 0 to \texttt{is2} since \texttt{xl1} is a data
series, not a list.  

You can also determine the number of variables or elements in a list
using the function \texttt{nelem()}.
%
\begin{code}
list xlist = 1 2 3
nl = nelem(xlist)
\end{code}

The (scalar) variable \texttt{nl} will be assigned a value of 3 since
\texttt{xlist} contains 3 members.

You can display the membership of a named list just by giving its name,
as illustrated in this interactive session:
%
\begin{code}
? list xlist = x1 x2 x3
Added list 'xlist'
? xlist
 x1 x2 x3
\end{code}
%
Note that \texttt{print xlist} will do something different, namely
print the values of all the variables in \texttt{xlist} (as should be
expected).

\subsection{Generating lists of transformed variables}

Given a named list of variables, you are able to generate lists of
transformations of these variables using the functions
commands \texttt{log}, \texttt{lags}, \texttt{diff}, \texttt{ldiff}
or \texttt{sdiff}.  For example
%
\begin{code}
list xlist = x1 x2 x3
list lxlist = log(xlist)
list difflist = diff(xlist)
\end{code}

When generating a list of \textit{lags} in this way, you specify the
maximum lag order inside the parentheses, before the list name and
separated by a comma.  For example
%
\begin{code}
list xlist = x1 x2 x3
list laglist = lags(2, xlist)
\end{code}
%
or
%
\begin{code}
scalar order = 4
list laglist = lags(order, xlist)
\end{code}

These commands will populate \texttt{laglist} with the specified
number of lags of the variables in \texttt{xlist}.  You can give the
name of a single series in place of a list as the second argument to
\texttt{lags}: this is equivalent to giving a list with just one
member.


\subsection{Checking for missing values}

\app{Gretl} offers several functions for recognizing and handling
missing values (see the \GCR{} for details). In this context it is
worth remarking that the \texttt{ok()} function can be used with a
list argument.  For example,
%
\begin{code}
list xlist = x1 x2 x3
series xok = ok(xlist)
\end{code}
%
After these commands, the series \texttt{xok} will have value 1 for
observations where none of \texttt{x1}, \texttt{x2}, or
\texttt{x3} has a missing value, and value 0 for any observations
where this condition is not met.


\section{Named strings}
\label{named-strings}

For some purposes it may be useful to save a string (that is, a
sequence of characters) as a named variable that can be reused.
Versions of \app{gretl} higher than 1.6.0 offer this facility, but
some of the refinements noted below are available only in \app{gretl}
1.7.2 and higher.

To \textit{define} a string variable, you can use either of two
commands, \texttt{string} or \texttt{sprintf}.  The \texttt{string}
command is simpler: you can type, for example,
%
\begin{code}
string s1 = "some stuff I want to save"
string s2 = getenv("HOME")
string s3 = @foostr + 10
\end{code}
%
The first field after \texttt{string} is the name under which the
string should be saved, then comes an equals sign, then comes a
specification of the string to be saved. This can be the keyword
\texttt{null}, to produce an empty string, or may take any of the 
following forms:

\begin{itemize}
\item a string literal (enclosed in double quotes); or
\item the name of an existing string variable, with \verb|@|
  prepended; or
\item a function that returns a string (see below); or
\item any of the above followed by \texttt{+} and an integer offset;
  or
\item any of the above, followed by a space, followed by another
  such field.
\end{itemize}

The role of the integer offset is to use a substring of the preceding
element, starting at the given character offset.  An error is flagged
is the offset is greater than the length of the string in question.

If more than one such component is present, the resulting string
is the concatenation of the elements.

The \texttt{sprintf} command is more flexible.  It works exactly as
\app{gretl}'s \texttt{printf} command except that the ``format''
string must be preceded by the name of a string variable.  For
example,
%
\begin{code}
scalar x = 8
sprintf foo "var%d", x
\end{code}

To \textit{retrieve the value} of a string variable, you give the name
of the variable preceded by the ``at'' sign, \verb|@|.  

In most contexts, the \verb|@| notation is treated as a ``macro''.
That is, if a sequence of characters in a \app{gretl} command
following the symbol \verb|@| is recognized as the name of a string
variable, the value of that variable is sustituted literally into the
command line before the regular parsing of the command is
carried out.  This is illustrated in the following interactive
session:
%
\begin{code}
? scalar x = 8
 scalar x = 8
Generated scalar x (ID 2) = 8
? sprintf foo "var%d", x
Saved string as 'foo'
? print "@foo"
var8
\end{code}
%
Note the effect of the quotation marks in the line 
\verb|print "@foo"|.  The line
%
\begin{code}
? print @foo
\end{code}
%
would \textit{not} print a literal ``\texttt{var8}'' as above.  After
pre-processing the line would read
%
\begin{code}
print var8
\end{code}
%
It would therefore print the value(s) of the variable \texttt{var8},
if such a variable exists, or would generate an error otherwise.

In certain specific contexts, however, it is natural to treat
\verb|@|-variables as variables in their own right, and \app{gretl}
does so.  These contexts are:
\begin{itemize}
\item When they appear among the arguments to the commands \texttt{printf} and
  \texttt{sprintf}.
\item On the right-hand side of a \texttt{string} assignment.
\item When they appear as an argument to the function
  \texttt{isstring} (see below).
\end{itemize}

Here is an illustration of the use of named string arguments with
\texttt{printf}:
%
\begin{code}
? string vstr = "variance"
Saved string as 'vstr'
? printf "vstr: %12s\n", @vstr
vstr:     variance
\end{code}
%
Note that \verb|@vstr| should not be put in quotes in this context.
Similarly with
\begin{code}
? string copy = @vstr
\end{code}

\subsection{Built-in strings}

Apart from any strings that the user may define, some string variables
are defined by \app{gretl} itself.  These may be useful for people
writing functions that include shell commands.  The built-in strings
are as shown in Table~\ref{tab-strings}.

\begin{table}[htbp]
\centering
\begin{tabular}{ll}
  \verb|@gretldir| & the \app{gretl} installation directory \\
  \verb|@workdir| & user's current \app{gretl} working directory \\
  \verb|@gnuplot| & path to, or name of, the \app{gnuplot} executable \\
  \verb|@tramo|& path to, or name of, the \app{tramo} executable \\
  \verb|@x12a| & path to, or name of, the \app{x-12-arima} executable \\
  \verb|@tramodir| & \app{tramo} data directory \\
  \verb|@x12adir| & \app{x-12-arima} data directory \\
\end{tabular}
\caption{Built-in string variables}
\label{tab-strings}
\end{table}

\subsection{Reading strings from the environment}

In addition, it is possible to read into \app{gretl}'s named strings,
values that are defined in the external environment.  To do this you
use the function \texttt{getenv}, which takes the name of an environment
variable as its argument.  For example:
%
\begin{code}
? string user = getenv("USER")
Saved string as 'user'
? string home = getenv("HOME")
Saved string as 'home'
? print "@user's home directory is @home"
cottrell's home directory is /home/cottrell
\end{code}
%
To check whether you got a non-empty value from a given call to
\texttt{getenv}, you can use the function \texttt{isstring}, as in
%
\begin{code}
? string temp = getenv("TEMP")
Saved empty string as 'temp'
? scalar x = isstring(@temp)
Generated scalar x (ID 2) = 0
\end{code}
%
Note that \texttt{isstring} is really shorthand for ``is a string that
actually contains something''.  

At present the \texttt{getenv} function can only be used on the
right-hand side of a \texttt{string} assignment, as in the above
illustrations.

\subsection{Capturing strings via the shell}

If shell commands are enabled in \app{gretl}, you can capture the
output from such commands using the syntax 

\texttt{string} \textsl{stringname} = \texttt{\$(}\textsl{shellcommand}\texttt{)}

That is, you enclose a shell command in parentheses, preceded by
a dollar sign.

\subsection{Reading from a file into a string}

You can read the content of a file into a string variable using
the syntax

\texttt{string} \textsl{stringname} = \texttt{readfile(}\textsl{filename}\texttt{)}

The \textsl{filename} field may include components that are
string variables.  For example

\verb|string foo = readfile(@x12adir/QNC.rts)|

\subsection{The \texttt{strstr} function}

Invocation of this function takes the form

\texttt{string} \textsl{stringname} = 
\texttt{strstr(}\textsl{s1}\texttt{,} \textsl{s2}\texttt{)}

The effect is to search \textsl{s1} for the first occurrence of
\textsl{s2}.  If no such occurrence is found, an empty string is
returned; otherwise the portion of \textsl{s1} starting with
\textsl{s2} is returned.  For example:
%
\begin{code}
? string hw = "hello world"
Saved string as 'hw'
? string w = strstr(@hw, "o")
Saved string as 'w'
? print "@w"
o world
\end{code}
%


    
%%% Local Variables: 
%%% mode: latex
%%% TeX-master: "gretl-guide"
%%% End: 

