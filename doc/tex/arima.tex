\chapter{ARMA models}
\label{arma-estimation}

\section{Representation and syntax}
\label{arma-repr}

The \cmd{arma} command performs estimation of autoregressive-moving
average (ARMA) models; the most general representation of a model
estimable by \app{gretl} is as follows:
\begin{equation}
  \label{eq:general-arma}
  A(L) B(L^s) y_t = x_t \beta + C(L) D(L^s) \epsilon_t ,
\end{equation}
where $L$ is the lag operator ($L^n x_t = x_{t-n}$), $s$ is the
number of subperiods for seasonal time series (for example, 12 for
monthly series), $x_t$ is a vector of exogenous variables and
$\epsilon_t$ is a white noise process.

The ``basic'' ARMA model is obtained when $x_t = 1$ and there are no
seasonal operators. In this case, $B(L^s) = D(L^s) = 1$ and the model
becomes
\begin{equation}
  \label{eq:plain-arma}
  A(L) y_t = \mu + C(L) \epsilon_t ,
\end{equation}
where, in customary notation, the vector $\beta$ reduces to the
intercept $\mu$. It is possible to write the above equation more
explicitly as
\[
  y_t = \mu + \phi_1 y_{t-1} + \ldots + \phi_p y_{t-p} + 
  \epsilon_t + \theta_1 \epsilon_{t-1} + \ldots + \theta_q
  \epsilon_{t-q} ;
\]
the corresponding \app{gretl} syntax is simply
\begin{code}
  gretl p q ; y
\end{code}
where \verb|p| and \verb|q| are the desired lag orders; these can be
either numbers or pre-defined scalars. The parameter $\mu$ can be
dropped if necessary by appending the option \cmd{--nc} to the command.

If you want to estimate a model with explanatory variables, the above
syntax must be extended to
\begin{code}
  gretl p q ; y const x1 x2
\end{code}
This command would estimate the following model:
\[
  y_t = \beta_0 + x_{1,t} \beta_1 + x_{2,t} \beta_2 + 
  \phi_1 y_{t-1} + \ldots + \phi_p y_{t-p} + 
  \epsilon_t + \theta_1 \epsilon_{t-1} + \ldots + \theta_q \epsilon_{t-q} .
\]
What \app{gretl} estimates in this case is known as an ARMAX (ARMA +
eXogenous variables) model, which is different from what some other packages
call ``regression model with ARMA errors''. The difference is apparent
by considering the model \app{gretl} estimates:
\begin{equation}
  \label{eq:armax}
  A(L) y_t = x_t \beta + C(L) \epsilon_t ,
\end{equation}
and a regression model with ARMA errors, that would be
\begin{eqnarray}
  \label{eq:reg-arma}
  y_t & = & x_t \beta + u_t \\
  A(L) u_t & = & C(L) \epsilon_t ;
\end{eqnarray}
the latter would translate into
\[
  A(L) y_t = A(L) \left(x_t \beta \right) + C(L) \epsilon_t ;
\]
the ARMAX formulation has the advantage that the coefficient $\beta$
can be immediately interpreted as the marginal effects of the $x_t$
variables on the conditional mean of $y_t$.

The structure of the \cmd{arma} command does not let you specify
models with gaps in the lag structure\footnote{Actually, this
  limitation can be circumvented for the autoregressive part by
  including lags of the dependent variable in the exogenous list.
  This, however, despite leading to correct estimation, would break
  the existing mechanism for forecasting.}. A more flexible lag
structure is especially desirable when analyzing time series that
display strong seasonal patterns. In these cases, the full model
(\ref{eq:general-arma}) can be used. For example, the syntax
\begin{code}
  arma 1 1 ; 1 1 ; y
\end{code}
would be used to estimate a model with four parameters:
\[
  ( 1 - \phi L )  ( 1 - \Phi L^s ) y_t = \mu + ( 1 + \theta L ) ( 1 + \Theta L^s ) \epsilon_t;
\]
assuming that $y_t$ is a quarterly series (and therefore $s=4$), the
above equation can be written more explicitly as
\[
  y_t = \mu + \phi y_{t-1} + \Phi y_{t-4} - (\phi \cdot \Phi) y_{t-5} + 
  \epsilon_t + \theta \epsilon_{t-1} + \Theta \epsilon_{t-4} +
  (\theta \cdot \Theta) \epsilon_{t-5} .
\]

\section{Estimation}
\label{arma-est}

The algorithm \app{gretl} uses to estimate the parameters of an ARMA
model is conditional maximum likelihood, also known as ``conditional
sum of squares'' (see Hamilton, 1994, page 132). This method was
exemplified in the script \ref{jack-arma}, and only a brief
description will be given here.

\ldots

\section{Forecasting}
\label{arma-fcast}


%%% Local Variables: 
%%% mode: latex
%%% TeX-master: "gretl-guide"
%%% End: 

