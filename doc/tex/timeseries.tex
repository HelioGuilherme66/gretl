\chapter{Time series models}
\label{chap:timeser}

\section{ARIMA models}
\label{arma-estimation}

\subsection{Representation and syntax}
\label{arma-repr}

The \cmd{arma} command performs estimation of autoregressive-moving
average (ARMA) models; the most general representation of an ARMA
model that may be estimated by \app{gretl} is as follows:
\begin{equation}
  \label{eq:general-arma}
  A(L) B(L^s) y_t = x_t \beta + C(L) D(L^s) \epsilon_t ,
\end{equation}
where $L$ is the lag operator ($L^n x_t = x_{t-n}$), $s$ is the
number of subperiods for seasonal time series (for example, 12 for
monthly series), $x_t$ is a vector of exogenous variables and
$\epsilon_t$ is a white noise process.

The basic ARMA model is obtained when $x_t = 1$ and there are no
seasonal operators. In this case, $B(L^s) = D(L^s) = 1$ and the model
becomes
\begin{equation}
  \label{eq:plain-arma}
  A(L) y_t = \mu + C(L) \epsilon_t ,
\end{equation}
where, in customary notation, the vector $\beta$ reduces to the
intercept $\mu$.  The equation above may be written more explicitly
as
\[
  y_t = \mu + \phi_1 y_{t-1} + \ldots + \phi_p y_{t-p} + 
  \epsilon_t + \theta_1 \epsilon_{t-1} + \ldots + \theta_q
  \epsilon_{t-q} 
\]
The corresponding \app{gretl} syntax is simply
\begin{code}
  arma p q ; y
\end{code}
where \verb|p| and \verb|q| are the desired lag orders; these can be
either numbers or pre-defined scalars. The parameter $\mu$ can be
dropped if necessary by appending the option \cmd{--nc} to the command.

If you want to estimate a model with explanatory variables, the above
syntax must be extended to
\begin{code}
  arma p q ; y const x1 x2
\end{code}
This command would estimate the following model:
\[
  y_t = \beta_0 + x_{1,t} \beta_1 + x_{2,t} \beta_2 + 
  \phi_1 y_{t-1} + \ldots + \phi_p y_{t-p} + 
  \epsilon_t + \theta_1 \epsilon_{t-1} + \ldots + \theta_q \epsilon_{t-q} .
\]
What \app{gretl} estimates in this case is known as an ARMAX (ARMA +
eXogenous variables) model, which is different from what some other packages
call ``regression model with ARMA errors''. The difference is apparent
by considering the model \app{gretl} estimates:
\begin{equation}
  \label{eq:armax}
  A(L) y_t = x_t \beta + C(L) \epsilon_t ,
\end{equation}
and a regression model with ARMA disturbances
\begin{eqnarray}
  \label{eq:reg-arma}
  y_t & = & x_t \beta + u_t \\
  A(L) u_t & = & C(L) \epsilon_t ;
\end{eqnarray}
the latter would translate into
\[
  A(L) y_t = A(L) \left(x_t \beta \right) + C(L) \epsilon_t ;
\]
the ARMAX formulation has the advantage that $\beta$ can be
immediately interpreted as the vector of marginal effects of the $x_t$
variables on the conditional mean of $y_t$.

The \cmd{arma} command can be therefore used also for estimating
\emph{Transfer Function Models}, a type of generalization of
autoregressive-moving average (ARMA) models adding the effects of
exogenous variable distributed along time, as in:
\begin{equation}
  \label{eq:tfunc-model}
  \phi (L) \cdot \Phi (L^s) y_t = \sum_{i=1}^kv_{i}(L)x_{it} + \theta (L)\cdot \Theta (L^s) \epsilon_t ,
\end{equation}

The structure of the \cmd{arma} command does not let you specify
models with gaps in the lag structure. A more flexible lag
structure is especially desirable when analyzing time series that
display strong seasonal patterns. In these cases, the full model
(\ref{eq:general-arma}) can be used. For example, the syntax
\begin{code}
  arma 1 1 ; 1 1 ; y
\end{code}
would be used to estimate a model with four parameters:
\[
  ( 1 - \phi L )  ( 1 - \Phi L^s ) y_t = \mu + ( 1 + \theta L ) ( 1 + \Theta L^s ) \epsilon_t;
\]
assuming that $y_t$ is a quarterly series (and therefore $s=4$), the
above equation can be written more explicitly as
\[
  y_t = \mu + \phi y_{t-1} + \Phi y_{t-4} - (\phi \cdot \Phi) y_{t-5} + 
  \epsilon_t + \theta \epsilon_{t-1} + \Theta \epsilon_{t-4} +
  (\theta \cdot \Theta) \epsilon_{t-5} .
\]
Such a model is known as a ``multiplicative seasonal ARMA model''.

For more general models, this limitation can be circumvented for the
autoregressive part by including lags of the dependent variable in the
exogenous list. As an example, the following command 
\begin{code}
  arma 0 0 ; 0 1 ; y const y(-2)
\end{code}
on a quarterly series would estimate the parameters of the model
\[
  y_t = \mu + \phi y_{t-2} + \epsilon_t + \Theta \epsilon_{t-4}.
\]
However, this workaround is not recommended: although this would
deliver correct estimates, it would break the existing mechanism for
forecasting.

The above discussion presupposes that the time series $y_t$ has
already been subjected to all the transformations deemed necessary for
ensuring stationarity. Differencing is the most common of these
transformations, and \app{gretl} provides a mechanism to include this
step into the \cmd{arma} command: the syntax
\begin{code}
  arma p d q ; y 
\end{code}
would estimate and ARMA(p,q) model on $\Delta^d y_t$, and is
functionally equivalent to 
\begin{code}
  series tmp = y
  loop for i=1..d
    tmp = diff(tmp)
  end loop
  arma p q ; tmp 
\end{code}
Such a model is known as an ARIMA (AutoRegressive Integrated
Moving-Average) model; for this reason, \app{gretl} provides the
\cmd{arima} command as an alias for \cmd{arma}. Seasonal differencing
is handled similarly, with the syntax
\begin{code}
  arma p d q ; P D Q ; y 
\end{code}
Thus, the command 
\begin{code}
  arma 1 0 0 ; 1 1 1 ; y 
\end{code}
would produce the same results as
\begin{code}
  genr dsy = sdiff(y)
  arma 1 0 ; 1 1 ; dsy 
\end{code}

\subsection{Estimation}
\label{arma-est}

The algorithm \app{gretl} uses to estimate the parameters of an ARMA
model is conditional maximum likelihood (CML), also known as
``conditional sum of squares'' --- see Hamilton (1994, p.\ 132).  This
method was exemplified in the script \ref{jack-arma}, and only a brief
description will be given here.  Given a sample of size $T$, the CML
method minimizes the sum of squared one-step-ahead prediction errors
generated by the model for the observations $t_0, \ldots, T$. The
starting point $t_0$ depends on the orders of the AR polynomials in the
model.

This method is nearly equivalent to maximum likelihood under the
hypothesis of normality; the difference is that the first $(t_0 -
1)$ observations are considered fixed and only enter the likelihood
function as conditioning variables. As a consequence, the two methods
are asymptotically equivalent under standard conditions.

The numerical method used for maximizing the log-likelihood is BHHH.
The covariance matrix for the parameters (hence the standard errors)
are computed via the OPG (Outer Product of the Gradients) method.

Full ML estimation under normality is available in \app{gretl} via
the \verb|x-12| plugin, which is automatically used if the option
\verb|--x-12-arima| is appendend to the \cmd{arma} command. For
example, the following code
\begin{code}
  open data10-1
  arma 1 1 ; r
  arma 1 1 ; r --x-12-arima
\end{code}
produces the estimates below:
\begin{center}
  \begin{tabular}{crrrr}
    \hline
    Parameter & \multicolumn{2}{c}{\textrm{CML}} &
    \multicolumn{2}{c}{\textrm{ML (x-12 plugin)}} \\
    \hline 
    $\mu$ & 1.07322 & 0.488661 &  1.00232 & 0.133002 \\
    $\phi$ & 0.852772 & 0.0450252 & 0.855373  & 0.0496304 \\
    $\theta$ & 0.591838 & 0.0456662 & 0.587986 & 0.0799962 \\
    \hline
  \end{tabular}
\end{center}

For comparability with \app{gretl}'s own estimates, you can instruct
\app{X-12-ARIMA} to produce CML estimates by appending the option
\verb|--conditional| along with \verb|--x-12-arima|.

\subsection{Forecasting}
\label{arma-fcast}

To be written

\section{Unit root tests}
\label{sec:uroot}

To be completed

\subsection{The ADF test}
\label{sec:ADFtest}

The ADF (Augmented Dickey-Fuller) test is, as implemented in
\app{gretl}, the $t$-statistic on $\varphi$ in the following regression:
\begin{equation}
  \label{eq:ADFtest}
  \Delta y_t = \mu_t + \varphi y_{t-1} + \sum_{i=1}^p \gamma_i \Delta
  y_{t-i} + \epsilon_t
\end{equation}
\begin{itemize}
\item $H_0$: $y_t$ is I(1);
\item one-sided test: $H_0$ is rejected if $t_{\varphi}$ is ``small'';
\item nonstandard distribution, also depends on the contents of
  $\mu_t$ --- p-values by MacKinnon.
\end{itemize}
Syntax:
\begin{code}
  adf n x1
\end{code}
\begin{itemize}
\item Determinstic kernel options;
\item auto-selection of $p$ via negative \verb|n|.
\end{itemize}
\subsection{The KPSS test}
\label{sec:KPSStest}

\begin{equation}
  \label{eq:KPSStest}
  \eta = \frac{\sum_{i=1}^T S_t^2 }{ T^2 \bar{\sigma}^2 }
\end{equation}
where $S_t = \sum_{s=1}^t e_s$ and $\bar{\sigma}^2$ is an
estimate of the long-run variance of $e_t = (y_t - \bar{y})$.
\begin{itemize}
\item $H_0$: $y_t$ is I(0);
\item one-sided test: $H_0$ is rejected if $\eta$ is ``big'';
\item extension to deterministic trend ($e_t$ are the residuals from
  an OLS regression of $y_t$ on a constant and a linear trend);
\item nonstandard distribution --- we provide 90\%, 95\%, 97.5\% and
  99\% quantiles.
\end{itemize}
Syntax:
\begin{code}
  kpss n x1
\end{code}
\begin{itemize}
\item \verb|--trend| option;
\item \verb|n| is used for estimating $\bar{\sigma}^2$; in the GUI,
  default is the integer part of $4 \left( \frac{T}{100}
  \right)^{1/4}$.
\end{itemize}

\subsection{The Johansen tests}
\label{sec:Joh-test}

Strictly speaking, these are tests for cointegration. However, they
can be used as multivariate unit-root tests since they are the
multivariate generalization of the ADF test.
\begin{equation}
  \label{eq:Joh-tests}
  \Delta y_t = \mu_t + \Pi y_{t-1} + \sum_{i=1}^p \Gamma_i \Delta
  y_{t-i} + \epsilon_t
\end{equation}
If the rank of $\Pi$ is 0, the processes are all I(1); If the rank of
$\Pi$ is full, the processes are all I(0); in between, you have
cointegration.

The rank of $\Pi$ is investigated by computing the eigenvalues of a
closely related matrix (call it $M$) whose rank is the same as $\Pi$:
tests on the rank of $\Pi$ can therefore be carried out by testing how
many eigenvalues of $M$ are 0. The two Johansen tests are the
``$\lambda$-max'' test, for hypotheses on individual eigenvalues, and
the ``trace'' test, for joint hypotheses.

The \app{gretl} command \cmd{coint2} performs these two tests.
Example:
\begin{code}
  coint2 n x1 x2 x3
\end{code}
sets $p$ in equation (\ref{eq:Joh-tests}) to $(n-1)$ and prints a table
with the two test batteries for the three series \verb|x1|, \verb|x2|
and \verb|x3|.
 
\ldots Options for the deterministic kernel \ldots
More on this in section \ref{sec:johansen-test}

\section{ARCH and GARCH}
\label{sec:arch}

Heteroskedasticity means a non-constant variance of the error term in
a regression model.  Autoregressive Conditional Heteroskedasticity
(ARCH) is a phenomenon specific to time series models, whereby the
variance of the error displays autoregressive behavior; for instance,
the time series exhibits successive periods where the error variance
is relatively large, and successive periods where it is relatively
small.  This sort of behavior is reckoned to be quite common in asset
markets: an unsettling piece of news can lead to a period of increased
volatility in the market.

An ARCH error process of order $q$ can be represented as
\[
u_t = \sigma_t \varepsilon_t; \qquad
\sigma^2_t \equiv {\rm E}(u^2_t|\Omega_{t-1}) = 
\alpha_0 + \sum_{i=1}^q \alpha_i u^2_{t-i}
\]
where the $\varepsilon_t$s are independently and identically
distributed (iid) with mean zero and variance 1, and where $\sigma_t$
is taken to be the positive square root of $\sigma^2_t$.
$\Omega_{t-1}$ denotes the information set as of time $t-1$ and
$\sigma^2_t$is known as the conditional variance: that is, the
variance conditional on information dated $t-1$ and earlier.

It is important to notice the difference between ARCH and an ordinary
autoregressive error process.  The simplest (first-order) case of the
latter can be written as
\[
u_t = \rho u_{t-1} + \varepsilon_t; \qquad -1 < \rho < 1
\]
where the $\varepsilon_t$s are iid with mean zero and constant
variance $\sigma^2$.  With an AR(1) error, if $\rho$ is positive then
a positive value of $u_t$ will tend to be followed, with probability
greater than 0.5, by a positive $u_{t+1}$.  With an ARCH error
process, a disturbance $u_t$ of large absolute value will tend to be
followed by further large absolute values, but with no presumption
that the successive values will be of the same sign.  ARCH in asset
prices is a ``stylized fact'' and is consistent with market
efficiency; on the other hand autoregressive behavior of asset prices
would violate market efficiency.

One can test for ARCH of order $q$ in the following
way:
\begin{enumerate}
\item Estimate the model of interest via OLS and save the squared
  residuals, $\hat{u}^2_t$.
\item Perform an auxiliary regression in which the current squared
  residual is regressed on a constant and $q$ lags of itself.
\item Find the $TR^2$ value (sample size times unadjusted $R^2$) for
  the auxiliary regression.
\item Refer the $TR^2$ value to the $\chi^2$ distribution with $q$
  degrees of freedom, and if the p-value is ``small enough'' reject
  the null hypothesis of homoskedasticity in favor of the alternative
  of ARCH($q$).
\end{enumerate}

This test is implemented in \app{gretl} via the \cmd{arch} command.
This command may be issued following the estimation of a time-series
model by OLS, or by selection from the ``Tests'' menu in the model
window (again, following OLS estimation).  The result of the test is
reported and if the $TR^2$ from the auxiliary regression has a p-value
less than 0.10, ARCH estimates are also reported.  These estimates
take the form of Generalized Least Squares (GLS), specifically
weighted least squares, using weights that are inversely proportional
to the predicted standard deviations of the disturbances,
$\hat{\sigma}_t$, derived from the auxiliary regression.

In addition, the ARCH test is available after estimating a vector
autoregression (VAR).  In this case, however, there is no provision to
re-estimate the model via GLS.

\subsection{GARCH}
\label{subsec:garch}

The simple ARCH($q$) process is useful for introducing the general
concept of conditional heteroskedasticity in time series, but it has
been found to be insufficient in empirical work.  The dynamics of the
error variance permitted by ARCH($q$) are not rich enough to represent 
the patterns found in financial data.  The generalized ARCH or GARCH
model is now more widely used.  

The representation of the variance of a process in the GARCH model is
somewhat (but not exactly) analogous to the ARMA representation of the
level of a time series.  The variance at time $t$ is allowed
to depend on both past values of the variance and past values of the
realized squared disturbance, as shown in the following system
of equations:
\begin{eqnarray}
  \label{eq:garch-meaneq}
  y_t &  = & X_t \beta + u_t \\
  \label{eq:garch-epseq}
  u_t &  = & \sigma_t \varepsilon_t \\
  \label{eq:garch-vareq}
  \sigma^2_t & = & \alpha_0 + \sum_{i=1}^q \alpha_i u^2_{t-i} +
	  \sum_{j=1}^p \beta_i \sigma^2_{t-j}
\end{eqnarray}
As above, $\varepsilon_t$ is an iid sequence with unit variance.
$X_t$ is a matrix of regressors (or in the simplest case,
just a vector of 1s allowing for a non-zero mean of $y_t$).  Note that
if $p=0$, GARCH collapses to ARCH($q$): the generalization is embodied
in the $\beta_i$ terms that multiply previous values of the error
variance.

In principle the underlying innovation, $\varepsilon_t$, could follow
any suitable probability distribution, and besides the obvious
candidate of the normal or Gaussian distribution the $t$ distribution
has been used in this context.  Currently \app{gretl} only handles the
case where $\varepsilon_t$ is assumed to be Gaussian.  However, the
\verb|--robust| option to the \cmd{garch} command computes the
covariance matrix of the parameters via the Bollerslev--Wooldridge
``sandwich'' estimator, so that the estimator \app{gretl} uses can be
considered Quasi-Maximum Likelihood (QML) even with non-normal
disturbances.

Example:
\begin{code}
  garch p q ; y const x
\end{code}
where \verb|p| $\ge 0$ and \verb|q| $>0$ denote the respective lag
orders as shown in equation (\ref{eq:garch-vareq}).  These values
can be supplied in numerical form or as the names of pre-defined
scalar variables.

\subsection{GARCH estimation}
\label{subsec:garch-est}

Estimation of the parameters of a GARCH model is by no means a
straightforward task.  In \app{gretl}, this is done using the specific
Maximum Likelihood method proposed by Fiorentini, Calzolari and
Panattoni (1996).\footnote{The algorithm is based on Fortran code
  deposited in the archive of the \textit{Journal of Applied
    Econometrics} by the authors, and is used by kind permission of
  Professor Fiorentini.}  This method was adopted as a benchmark in
the study of GARCH results by McCullough and Renfro (1998).  It
employs analytical first and second derivatives of the log-likelihood,
and uses a mixed-gradient algorithm, exploiting the information matrix
in the early iterations and then switching to the Hessian in the
neighborhood of the maximum likelihood.  (This progress can be
observed if you append the \verb|--verbose| option to \app{gretl}'s
\cmd{garch} command.)

It is not uncommon, when one estimates a GARCH model for an arbitrary
time series, to find that the iterative calculation of the estimates
fails to converge.  For the GARCH model to make sense, there are
strong restrictions on the admissible parameter values, and it is not
always the case that there exists a set of values inside the
admissible parameter space for which the likelihood is maximized.  

The restrictions in question can be explained by reference to the
simplest (and much the most common) instance of the GARCH model, where
$p = q = 1$.  In the GARCH(1, 1) model the conditional variance is
\begin{equation}
\label{eq:condvar}
\sigma^2_t = \alpha_0 + \alpha_1 u^2_{t-1} + \beta_1 \sigma^2_{t-1}
\end{equation}
Taking the unconditional expectation of (\ref{eq:condvar}) we get
\[
\sigma^2 = \alpha_0 + \alpha_1 \sigma^2 + \beta_1 \sigma^2
\]
so that
\[
\sigma^2 = \frac{\alpha_0}{1 - \alpha_1 - \beta_1}
\]
For this unconditional variance to exist, we require that $\alpha_1 +
\beta_1 < 1$, and for it to be positive we require that $\alpha_0 > 0$.

A common reason for non-convergence of GARCH estimates (that is, a
common reason for the non-existence of $\alpha_i$ and $\beta_i$ values
that satisfy the above requirements and at the same time maximize the
likelihood of the data) is misspecification of the model.  It is
important to realize that GARCH, in itself, allows \textit{only} for
time-varying volatility in the data.  If the \textit{mean} of the
series in question is not constant, or if the error process is not
only heteroskedastic but also autoregressive, it is necessary to take
this into account when formulating a suitable model.  For example, it
may be necessary to take the first difference of the variable in
question and/or to add suitable regressors, $X_t$, as in
(\ref{eq:garch-meaneq}).

\section{Cointegration and Vector Error Correction Models}
\label{vecm-explanation}

\subsection{The Johansen cointegration test}
\label{sec:johansen-test}

The Johansen test for cointegration has to take into account what
hypotheses one is willing to make on the deterministic terms, which
leads to the famous ``five cases.'' A full and general illustration of
the five cases requires a fair amount of matrix algebra, but an
intuitive understanding of the issue can be gained by means of a
simple example.
    
Consider a series $x_t$ which behaves as follows
%      
\[ x_t = m + x_{t-1} + \varepsilon_t \] 
%
where $m$ is a real number and $\varepsilon_t$ is a white noise
process. As is easy to show, $x_t$ is a random walk which fluctuates
around a deterministic trend with slope $m$. In the special case $m$ =
0, the deterministic trend disappears and $x_t$ is a pure random walk.
    
Consider now another process $y_t$, defined by
%      
\[ y_t = k + x_t + u_t \] 
%
where, again, $k$ is a real number and $u_t$ is a white noise process.
Since $u_t$ is stationary by definition, $x_t$ and $y_t$ cointegrate:
that is, their difference
%      
\[ z_t = y_t - x_t = k + u_t \]
%	
is a stationary process. For $k$ = 0, $z_t$ is simple zero-mean white
noise, whereas for $k$ $\ne$ 0 the process $z_t$ is white noise with a
non-zero mean.
  
After some simple substitutions, the two equations above can be
represented jointly as a VAR(1) system
%      
\[ \left[ \begin{array}{c} y_t \\ x_t \end{array} \right] = \left[
  \begin{array}{c} k + m \\ m \end{array} \right] + \left[
  \begin{array}{rr} 0 & 1 \\ 0 & 1 \end{array} \right] \left[
  \begin{array}{c} y_{t-1} \\ x_{t-1} \end{array} \right] + \left[
  \begin{array}{c} u_t + \varepsilon_t \\ \varepsilon_t \end{array}
\right] \]
%	
or in VECM form
%      
\begin{eqnarray*}
  \left[  \begin{array}{c} \Delta y_t \\ \Delta x_t \end{array} \right]  & = & 
  \left[  \begin{array}{c} k + m \\ m \end{array} \right] +
  \left[  \begin{array}{rr} -1 & 1 \\ 0 & 0 \end{array} \right] 
  \left[  \begin{array}{c} y_{t-1} \\ x_{t-1} \end{array} \right] + 
  \left[  \begin{array}{c} u_t + \varepsilon_t \\ \varepsilon_t \end{array} \right] = \\
  & = & 
  \left[  \begin{array}{c} k + m \\ m \end{array} \right] +
  \left[  \begin{array}{r} -1 \\ 0 \end{array} \right]
  \left[  \begin{array}{rr} 1 & -1 \end{array} \right] 
  \left[  \begin{array}{c} y_{t-1} \\ x_{t-1} \end{array} \right] + 
  \left[  \begin{array}{c} u_t + \varepsilon_t \\ \varepsilon_t \end{array} \right] = \\
  & = & 
  \mu_0 + \alpha \beta^{\prime} \left[  \begin{array}{c} y_{t-1} \\ x_{t-1} \end{array} \right] + \eta_t = 
  \mu_0 + \alpha z_{t-1} + \eta_t ,
\end{eqnarray*}
%	
where $\beta$ is the cointegration vector and $\alpha$ is the
``loadings'' or ``adjustments'' vector.
     
We are now in a position to consider three possible cases:
    
\begin{enumerate}
\item $m$ $\ne$ 0: In this case $x_t$ is trended, as we just saw; it
  follows that $y_t$ also follows a linear trend because on average it
  keeps at a distance $k$ from $x_t$. The vector
  $\mu_0$ is unrestricted.  This case is the default
  for gretl's \cmd{vecm} command.
	
\item $m$ = 0 and $k$ $\ne$ 0: In this case, $x_t$ is not trended and
  as a consequence neither is $y_t$. However, the mean distance
  between $y_t$ and $x_t$ is non-zero. The vector
  $\mu_0$ is given by
%	  
  \[
  \mu_0 = \left[ \begin{array}{c} k \\ 0 \end{array} \right]
  \]
%	    
  which is not null and therefore the VECM shown above does have a
  constant term. The constant, however, is subject to the restriction
  that its second element must be 0. More generally,
  $\mu_0$ is a multiple of the vector $\alpha$. Note
  that the VECM could also be written as
%	  
  \[
  \left[ \begin{array}{c} \Delta y_t \\ \Delta x_t \end{array} \right]
  = \left[ \begin{array}{r} -1 \\ 0 \end{array} \right] \left[
    \begin{array}{rrr} 1 & -1 & -k \end{array} \right] \left[
    \begin{array}{c} y_{t-1} \\ x_{t-1} \\ 1 \end{array} \right] +
  \left[ \begin{array}{c} u_t + \varepsilon_t \\ \varepsilon_t
    \end{array} \right]
  \]
%	   
  which incorporates the intercept into the cointegration vector. This
  is known as the ``restricted constant'' case; it may be specified in
  gretl's \cmd{vecm} command using the option flag \verb+--rc+.
	
\item $m$ = 0 and $k$ = 0: This case is the most restrictive: clearly,
  neither $x_t$ nor $y_t$ are trended, and the mean distance between
  them is zero. The vector $\mu_0$ is also 0, which
  explains why this case is referred to as ``no constant.''  This case
  is specified using the option flag \verb+--nc+ with \cmd{vecm}.
	
\end{enumerate}

In most cases, the choice between the three possibilities is based on
a mix of empirical observation and economic reasoning. If the
variables under consideration seem to follow a linear trend then we
should not place any restriction on the intercept. Otherwise, the
question arises of whether it makes sense to specify a cointegration
relationship which includes a non-zero intercept. One example where
this is appropriate is the relationship between two interest rates:
generally these are not trended, but the VAR might still have an
intercept because the difference between the two (the ``interest rate
spread'') might be stationary around a non-zero mean (for example,
because of a risk or liquidity premium).
    
The previous example can be generalized in three directions:
    
\begin{enumerate}
\item If a VAR of order greater than 1 is considered, the algebra gets
  more convoluted but the conclusions are identical.
\item If the VAR includes more than two endogenous variables the
  cointegration rank $r$ can be greater than 1. In this case, $\alpha$
  is a matrix with $r$ columns, and the case with restricted constant
  entails the restriction that $\mu_0$ should be some
  linear combination of the columns of $\alpha$.
\item If a linear trend is included in the model, the deterministic
  part of the VAR becomes $\mu_0 + \mu_1 t$. The reasoning is
  practically the same as above except that the focus now centers on
  $\mu_1$ rather than $\mu_0$.  The counterpart to the ``restricted
  constant'' case discussed above is a ``restricted trend'' case, such
  that the cointegration relationships include a trend but the first
  differences of the variables in question do not.  In the case of an
  unrestricted trend, the trend appears in both the cointegration
  relationships and the first differences, which corresponds to the
  presence of a quadratic trend in the variables themselves (in
  levels).  These two cases are specified by the option flags
  \verb+--crt+ and \verb+--ct+, respectively, with the \cmd{vecm}
  command.
\end{enumerate}


%%% Local Variables: 
%%% mode: latex
%%% TeX-master: "gretl-guide"
%%% End: 

