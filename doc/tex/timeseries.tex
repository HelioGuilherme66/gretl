\chapter{Time series models}
\label{chap:timeser}

\section{ARIMA models}
\label{arma-estimation}

\subsection{Representation and syntax}
\label{arma-repr}

The \cmd{arma} command performs estimation of AutoRegressive,
Integrated, Moving Average (ARIMA) models.  These are models that can
be written in the form
\begin{equation}
  \label{eq:plain-0-arma}
  \phi(L) y_t = \theta(L) \epsilon_t
\end{equation}
where $\phi(L)$, and $\theta(L)$ are polynomials in the lag operator,
$L$, defined such that $L^n x_t = x_{t-n}$, and $\epsilon_t$ is a
white noise process. The exact content of $y_t$, of the AR polynomial
$\phi()$, and of the MA polynomial $\theta()$, will be explained in the
following.

\subsection{Mean terms}
\label{sec:arma-nonzeromean}

The process $y_t$ as written in equation (\ref{eq:plain-0-arma}) has,
without further qualifications, mean zero. If the model is to be
applied to real data, it is necessary to include some term to handle
the possibility that $y_t$ has non-zero mean. There are two possible
ways to represent processes with nonzero mean: one is to define $\mu_t$
as the \emph{unconditional} mean of $y_t$, namely the central value of
its marginal distribution. Therefore, the series $\tilde{y}_t = y_t -
\mu_t$ has mean 0, and the model (\ref{eq:plain-0-arma}) applies to
$\tilde{y}_t$. In practice, assuming that $\mu_t$ is a linear function
of some observable variables $x_t$, the model becomes
\begin{equation}
  \label{eq:arma-with-x}
  \phi(L) (y_t - x_t \beta) = \theta(L) \epsilon_t
\end{equation}
This is sometimes known as a ``regression model with ARMA errors'';
its structure may be more apparent if we represent it using two
equations:
\begin{eqnarray*}
  y_t & = & x_t \beta + u_t \\
  \phi(L) u_t & = & \theta(L) \epsilon_t
\end{eqnarray*}

The model just presented is also sometimes known as ``ARMAX'' (ARMA +
eXogenous variables).  It seems to us, however, that this label is
more appropriately applied to a different model: another way to
include a mean term in (\ref{eq:plain-0-arma}) is to base the
representation on the \emph{conditional} mean of $y_t$, that is the
central value of the distribution of $y_t$ \emph{given its own past}.
Assuming, again, that this can be represented as a linear combination
of some observable variables $z_t$, the model would expand to
\begin{equation}
  \label{eq:arma-with-z}
  \phi(L) y_t = z_t \gamma + \theta(L) \epsilon_t
\end{equation}
The formulation (\ref{eq:arma-with-z}) has the advantage that $\gamma$
can be immediately interpreted as the vector of marginal effects of
the $z_t$ variables on the conditional mean of $y_t$.  And by adding
lags of $z_t$ to this specification one can estimate \emph{Transfer
  Function models} (which generalize ARMA by adding the effects of
exogenous variable distributed across time).

\app{Gretl} provides a way to estimate both forms. Models written as
in (\ref{eq:arma-with-x}) are estimated by maximum likelihood; models
written as in (\ref{eq:arma-with-z}) are estimated by conditional
maximum likelihood. (For more on these options see the section on
``Estimation'' below.)  

In the special case when $x_t = z_t = 1$ (that is, the models include
a constant but no exogenous variables) the two specifications discussed
above reduce to
\begin{equation}
  \phi(L) (y_t - \mu) = \theta(L) \epsilon_t
  \label{eq:arma-with-xconst} 
\end{equation}
and
\begin{equation}
  \phi(L) y_t = \alpha + \theta(L) \epsilon_t
  \label{eq:arma-with-zconst}
\end{equation}
respectively.  These formulations are essentially equivalent, but if
they represent one and the same process $\mu$ and $\alpha$ are, fairly
obviously, not numerically identical; rather
\[
\alpha = \left(1 - \phi_1 - \ldots - \phi_p\right) \mu
\]

The \app{gretl} syntax for estimating (\ref{eq:arma-with-xconst}) is simply
\begin{code}
  arma p q ; y
\end{code}
The AR and MA lag orders, \verb|p| and \verb|q|, can be given either as
numbers or as pre-defined scalars. The parameter $\mu$ can be dropped
if necessary by appending the option \cmd{--nc} (``no constant'') to
the command. If estimation of (\ref{eq:arma-with-zconst}) is needed,
the switch \texttt{--conditional} must be appended to the command, as
in 
\begin{code}
  arma p q ; y --conditional
\end{code}

Generalizing this principle to the estimation of
(\ref{eq:arma-with-x}) or (\ref{eq:arma-with-z}), you get that
\begin{code}
  arma p q ; y const x1 x2
\end{code}
would estimate the following model:
\[
  y_t - x_t \beta = \phi_1 \left(y_{t-1} - x_{t-1} \beta \right) + \ldots + 
   \phi_p \left( y_{t-p} - x_{t-p} \beta \right) + 
  \epsilon_t + \theta_1 \epsilon_{t-1} + \ldots + \theta_q \epsilon_{t-q}
\]
where in this instance $x_t \beta = \beta_0 + x_{t,1} \beta_1 +
x_{t,2} \beta_2$. Appending the \texttt{--conditional} switch, as in 
\begin{code}
  arma p q ; y const x1 x2 --conditional
\end{code}
would estimate the following model:
\[
  y_t = x_t \gamma + \phi_1 y_{t-1} + \ldots +  \phi_p y_{t-p} + 
  \epsilon_t + \theta_1 \epsilon_{t-1} + \ldots + \theta_q \epsilon_{t-q}
\]

Ideally, the issue broached above could be made moot by writing a more
general specification that nests the alternatives; that is
\begin{equation}
 \label{armax-general}
  \phi(L) \left(y_t - x_t \beta\right) = z_t \gamma  + \theta(L) \epsilon_t ;
\end{equation}
we would like to generalize the \cmd{arma} command so that
the user could specify, for any estimation method, whether certain
exogenous variables should be treated as $x_t$s or $z_t$s, but we're
not yet at that point (and neither are most other software packages).

\subsection{Seasonal models}

A more flexible lag structure is desirable when analyzing time series
that display strong seasonal patterns. Model (\ref{eq:plain-0-arma})
can be expanded to
\begin{equation}
  \label{eq:seasonal-arma}
  \phi(L) \Phi(L^s) y_t = \theta(L) \Theta(L^s) \epsilon_t .
\end{equation}
For such cases, a fuller form of the syntax is available, namely,
\begin{code}
  arma p q ; P Q ; y
\end{code}
where \texttt{p} and \texttt{q} represent the non-seasonal AR and MA
orders, and \texttt{P} and \texttt{Q} the seasonal orders.  For
example,
\begin{code}
  arma 1 1 ; 1 1 ; y
\end{code}
would be used to estimate the following model:
\[
  (1 -\phi L)(1 -\Phi L^s) (y_t - \mu) = (1 + \theta L)(1 + \Theta L^s) \epsilon_t
\]
If $y_t$ is a quarterly series (and therefore $s=4$), the above
equation can be written more explicitly as
\[
y_t - \mu = \phi (y_{t-1} - \mu) + \Phi (y_{t-4} - \mu) - (\phi
  \cdot \Phi) (y_{t-5} - \mu) + \epsilon_t + \theta \epsilon_{t-1} + \Theta
  \epsilon_{t-4} + (\theta \cdot \Theta) \epsilon_{t-5}
\]
Such a model is known as a ``multiplicative seasonal ARMA model''.

\subsection{Differencing and ARIMA}

The above discussion presupposes that the time series $y_t$ has
already been subjected to all the transformations deemed necessary for
ensuring stationarity (see also section \ref{sec:uroot}). Differencing
is the most common of these transformations, and \app{gretl} provides
a mechanism to include this step into the \cmd{arma} command: the
syntax
\begin{code}
  arma p d q ; y 
\end{code}
would estimate an ARMA$(p,q)$ model on $\Delta^d y_t$. It is
functionally equivalent to 
\begin{code}
  series tmp = y
  loop for i=1..d
    tmp = diff(tmp)
  end loop
  arma p q ; tmp 
\end{code}
except with regard to forecasting after estimation (see below).

When the series $y_t$ is differenced before performing the analysis
the model is known as ARIMA (``I'' for Integrated); for this reason,
\app{gretl} provides the \cmd{arima} command as an alias for
\cmd{arma}.

Seasonal differencing is handled similarly, with the syntax
\begin{code}
  arma p d q ; P D Q ; y 
\end{code}
where \texttt{D} is the order for seasonal differencing.  Thus, the
command
\begin{code}
  arma 1 0 0 ; 1 1 1 ; y 
\end{code}
would produce the same parameter estimates as
\begin{code}
  genr dsy = sdiff(y)
  arma 1 0 ; 1 1 ; dsy 
\end{code}
where we use the \texttt{sdiff} function to create a seasonal
difference (e.g.\ for quarterly data, $y_t - y_{t-4}$).

\subsection{Estimation}
\label{arma-est}

The default estimation method for ARMA models is exact maximum
likelihood estimation (under the assumption that the error term is
normally distributed), using the Kalman filter in conjunction with the
BFGS maximization algorithm.  The gradient of the log-likelihood with
respect to the parameter estimates is approximated numerically.  This
method produces results that are directly comparable with many other
software packages.  The constant, and any exogenous variables, are
treated as in equation (\ref{eq:arma-with-x}).  The covariance matrix
for the parameters is computed using a numerical approximation to the
Hessian at convergence.

The alternative method, invoked with the \verb|--conditional| switch,
is conditional maximum likelihood (CML), also known as ``conditional
sum of squares'' --- see Hamilton (1994, p.\ 132).  This method was
exemplified in the script~\ref{jack-arma}, and only a brief
description will be given here.  Given a sample of size $T$, the CML
method minimizes the sum of squared one-step-ahead prediction errors
generated by the model for the observations $t_0, \ldots, T$.  The
starting point $t_0$ depends on the orders of the AR polynomials in
the model.  The numerical maximization method used is BHHH, and the
covariance matrix is computed using a Gauss--Newton regression.

The CML method is nearly equivalent to maximum likelihood under the
hypothesis of normality; the difference is that the first $(t_0 - 1)$
observations are considered fixed and only enter the likelihood
function as conditioning variables. As a consequence, the two methods
are asymptotically equivalent under standard conditions --- except for
the fact, discussed above, that our CML implementation treats the
constant and exogenous variables as per equation (\ref{eq:arma-with-z}).

The two methods can be compared as in the following example
\begin{code}
  open data10-1
  arma 1 1 ; r
  arma 1 1 ; r --conditional
\end{code}
which produces the estimates shown in Table~\ref{tab:ml-cml}.  As you
can see, the estimates of $\phi$ and $\theta$ are quite similar.  The
reported constants differ widely, as expected --- see the discussion
following equations (\ref{eq:arma-with-xconst}) and
(\ref{eq:arma-with-zconst}).  However, dividing the CML constant by
$1-\phi$ we get 7.38, which is not far from the ML estimate of 6.93.

\begin{table}[htbp]
\caption{ML and CML estimates}
\label{tab:ml-cml}
\begin{center}
  \begin{tabular}{crrrr}
    \hline
    Parameter & \multicolumn{2}{c}{ML} &
    \multicolumn{2}{c}{CML} \\
    \hline 
    $\mu$ & 6.93042 & (0.673202) & 1.07322 & (0.488661) \\
    $\phi$ & 0.855360 & (0.0512026) & 0.852772 & (0.0450252) \\
    $\theta$ & 0.588056 & (0.0809769) & 0.591838 & (0.0456662) \\
    \hline
  \end{tabular}
\end{center}
\end{table}

\subsection{Convergence and initialization}

The numerical methods used to maximize the likelihood for ARMA models
are not guaranteed to converge.  Whether or not convergence is
achieved, and whether or not the true maximum of the likelihood
function is attained, may depend on the starting values for the
parameters.  \app{Gretl} employs one of the following two
initialization mechanisms, depending on the specification of the model
and the estimation method chosen.

\begin{enumerate}
\item Estimate a pure AR model by Least Squares (nonlinear least
  squares if the model requires it, otherwise OLS).  Set the AR
  parameter values based on this regression and set the MA
  parameters to a small positive value (0.0001).
\item The Hannan--Rissanen method: First estimate an autoregressive
  model by OLS and save the residuals.  Then in a second OLS pass add
  appropriate lags of the first-round residuals to the model, to
  obtain estimates of the MA parameters.
\end{enumerate}

To see the details of the ARMA estimation procedure, add the
\verb|--verbose| option to the command.  This prints a notice of the
initialization method used, as well as the parameter values and
log-likelihood at each iteration.

Besides the build-in initialization mechanisms, the user has the
option of specifying a set of starting values manually.  This is done
via the \cmd{set} command: the first argument should be the keyword
\texttt{initvals} and the second should be the name of a pre-specified
matrix containing starting values.  For example
\begin{code}
  matrix start = { 0, 0.85, 0.34 }
  set initvals start
  arma 1 1 ; y
\end{code}
The specified matrix should have just as many parameters as the model:
in the example above there are three parameters, since the model
implicitly includes a constant.  The constant, if present, is always
given first; otherwise the order in which the parameters are
expected is the same as the order of specification in the \cmd{arma}
or \cmd{arima} command.  In the example the constant is set to zero,
$\phi_1$ to 0.85, and $\theta_1$ to 0.34.

You can get \app{gretl} to revert to automatic initialization via
the command \cmd{set initvals auto}.

\subsection{Estimation via X-12-ARIMA}

As an alternative to estimating ARMA models using ``native'' code,
\app{gretl} offers the option of using the external program
\app{X-12-ARIMA}.  This is the seasonal adjustment software produced
and maintained by the U.S. Census Bureau; it is used for all official
seasonal adjustments at the Bureau.

\app{Gretl} includes a module which interfaces with \app{X-12-ARIMA}:
it translates \cmd{arma} commands using the syntax outlined above into
a form recognized by \app{X-12-ARIMA}, executes the program, and
retrieves the results for viewing and further analysis within
\app{gretl}.  To use this facility you have to install
\app{X-12-ARIMA} separately.  Packages for both MS Windows and
GNU/Linux are available from the \app{gretl} website,
\url{http://gretl.sourceforge.net/}.

To invoke \app{X-12-ARIMA} as the estimation engine, append the flag
\verb|--x-12-arima|, as in
\begin{code}
  arma p q ; y --x-12-arima
\end{code}
As with native estimation, the default is to use exact ML but there is
the option of using conditional ML with the \verb|--conditional| flag.
However, please note that when \app{X-12-ARIMA} is used in conditional
ML mode, the comments above regarding the variant treatments of the
mean of the process $y_t$ \textit{do not apply}.  That is, when you
use \app{X-12-ARIMA} the model that is estimated is
(\ref{eq:arma-with-x}), regardless of whether estimation is by exact
ML or conditional ML.


\subsection{Forecasting}
\label{arma-fcast}

ARMA models are often used for forecasting purposes.  The
autoregressive component, in particular, offers the possibility of
forecasting a process ``out of sample'' over a substantial time
horizon.

\app{Gretl} supports forecasting on the basis of ARMA models using the
method set out by Box and Jenkins (1976).\footnote{See in particular
  their ``Program 4'' on p.\ 505ff.}  The Box and Jenkins algorithm
produces a set of integrated AR coefficients which take into account
any differencing of the dependent variable (seasonal and/or
non-seasonal) in the ARIMA context, thus making it possible to
generate a forecast for the level of the original variable.  By
contrast, if you first difference a series manually and then apply
ARMA to the differenced series, forecasts will be for the differenced
series, not the level.  This point is illustrated
in Example~\ref{arima-fcast-script}.  The parameter estimates are identical
for the two models.  The forecasts differ but are mutually consistent:
the variable \texttt{fcdiff} emulates the ARMA forecast (static,
one step ahead within the sample range, and dynamic out of sample).

\begin{script}[htbp]
  \caption{ARIMA forecasting}
  \label{arima-fcast-script}
\begin{code}
  open greene18_2.gdt
  # log of quarterly U.S. nominal GNP, 1950:1 to 1983:4
  genr y = log(Y)
  # and its first difference
  genr dy = diff(y)
  # reserve 2 years for out-of-sample forecast
  smpl ; 1981:4
  # Estimate using ARIMA
  arima 1 1 1 ; y 
  # forecast over full period
  smpl --full
  fcast fc1
  # Return to sub-sample and run ARMA on the first difference of y
  smpl ; 1981:4
  arma 1 1 ; dy
  smpl --full
  fcast fc2
  genr fcdiff = (t<=1982:1)*(fc1 - y(-1)) + (t>1982:1)*(fc1 - fc1(-1))
  # compare the forecasts over the later period
  smpl 1981:1 1983:4
  print y fc1 fc2 fcdiff --byobs
\end{code}
The output from the last command is:
\begin{code}
                    y          fc1          fc2       fcdiff

  1981:1      7.964086     7.940930      0.02668      0.02668
  1981:2      7.978654     7.997576      0.03349      0.03349
  1981:3      8.009463     7.997503      0.01885      0.01885
  1981:4      8.015625     8.033695      0.02423      0.02423
  1982:1      8.014997     8.029698      0.01407      0.01407
  1982:2      8.026562     8.046037      0.01634      0.01634
  1982:3      8.032717     8.063636      0.01760      0.01760
  1982:4      8.042249     8.081935      0.01830      0.01830
  1983:1      8.062685     8.100623      0.01869      0.01869
  1983:2      8.091627     8.119528      0.01891      0.01891
  1983:3      8.115700     8.138554      0.01903      0.01903
  1983:4      8.140811     8.157646      0.01909      0.01909
\end{code}
\end{script}


\subsection{Limitations}

The structure of \app{gretl}'s \cmd{arma} command does not allow you
to specify models with gaps in the lag structure, other than via the
seasonal specification discussed above.  For example, if you have a
monthly time series, you cannot estimate an ARMA model with AR terms
(or MA terms) at just lags 1, 3 and 5.

At a pinch, you could circumvent this limitation in respect of the AR
part of the specification by the trick of including lags of the
dependent variable in the list of ``exogenous'' variables.  For
example, the following command
\begin{code}
  arma 0 0 ; 0 1 ; y const y(-2)
\end{code}
on a quarterly series would estimate the parameters of the model
\[
  y_t - \mu = \phi \left(y_{t-2} - \mu\right) + \epsilon_t + \Theta \epsilon_{t-4}
\]
However, this workaround is not really recommended: it should deliver
correct estimates, but will break the existing mechanism for
forecasting.

\section{Unit root tests}
\label{sec:uroot}

\subsection{The ADF test}
\label{sec:ADFtest}

The ADF (Augmented Dickey-Fuller) test is, as implemented in
\app{gretl}, the $t$-statistic on $\varphi$ in the following regression:
\begin{equation}
  \label{eq:ADFtest}
  \Delta y_t = \mu_t + \varphi y_{t-1} + \sum_{i=1}^p \gamma_i \Delta
  y_{t-i} + \epsilon_t .
\end{equation}

This test statistic is probably the best-known and most widely used
unit root test. It is a one-sided test whose null hypothesis is
$\varphi = 0$ versus the alternative $\varphi < 0$. Under the null,
$y_t$ must be differenced at least once to achieve stationarity;
under the alternative, $y_t$ is already stationary and no differencing
is required. Hence, large negative values of the test statistic lead
to the rejection of the null.

One peculiar aspect of this test is that its limit distribution is
non-standard under the null hypothesis: moreover, the shape of the
distribution, and consequently the critical values for the test,
depends on the form of the $\mu_t$ term.  A full analysis of the
various cases is inappropriate here: Hamilton (1994) contains an
excellent discussion, but any recent time series textbook covers
this topic. Suffice it to say that \app{gretl} allows the user to
choose the specification for $\mu_t$ among four different
alternatives:

\begin{center}
  \begin{tabular}{cc}
    \hline
    $\mu_t$ & command option \\
    \hline
    0 & \verb|--nc| \\
    $\mu_0$ &  \verb|--c| \\
    $\mu_0 + \mu_1 t$ &  \verb|--ct| \\
    $\mu_0 + \mu_1 t + \mu_1 t^2$ &  \verb|--ctt| \\
    \hline
  \end{tabular}
\end{center}

These options are not mutually exclusive; when they are used together
the statistic will be reported separately for each case.  By default,
\app{gretl} uses by default the combination \verb|--c --ct --ctt|. For
each case, approximate p-values are calculated by means of the
algorithm developed in MacKinnon (1996).

The \app{gretl} command used to perform the test is \cmd{adf}; for example
\begin{code}
  adf 4 x1 --c --ct
\end{code}
would compute the test statistic as the t-statistic for $\varphi$ in
equation \ref{eq:ADFtest} with $p=4$ in the two cases $\mu_t = \mu_0$
and $\mu_t = \mu_0 + \mu_1 t$.

The number of lags ($p$ in equation \ref{eq:ADFtest}) should be chosen
as to ensure that (\ref{eq:ADFtest}) is a parametrization flexible
enough to represent adequately the short-run persistence of $\Delta
y_t$. Setting $p$ too low results in size distortions in the test,
whereas setting $p$ too high would lead to low power. As a convenience
to the user, the parameter $p$ can be automatically determined.
Setting $p$ to a negative number triggers a sequential procedure that
starts with $p$ lags and decrements $p$ until the $t$-statistic for
the parameter $\gamma_p$ exceeds 1.645 in absolute value.

\subsection{The KPSS test}
\label{sec:KPSStest}

The KPSS test (Kwiatkowski, Phillips, Schmidt and Shin, 1992) is a
unit root test in which the null hypothesis is opposite to that in the
ADF test: under the null, the series in question is stationary; the
alternative is that the series is $I(1)$.

The basic intuition behind this test statistic is very simple: if
$y_t$ can be written as $y_t = \mu + u_t$, where $u_t$ is some
zero-mean stationary process, then not only does the sample average of
the $y_t$'s provide a consistent estimator of $\mu$, but the long-run
variance of $u_t$ is a well-defined, finite number. Neither of these
properties hold under the alternative.

The test itself is based on the following statistic:
\begin{equation}
  \label{eq:KPSStest}
  \eta = \frac{\sum_{i=1}^T S_t^2 }{ T^2 \bar{\sigma}^2 }
\end{equation}
where $S_t = \sum_{s=1}^t e_s$ and $\bar{\sigma}^2$ is an estimate of
the long-run variance of $e_t = (y_t - \bar{y})$. Under the null, this
statistic has a well-defined (nonstandard) asymptotic distribution,
which is free of nuisance parameters and has been tabulated by
simulation. Under the alternative, the statistic diverges.

As a consequence, it is possible to construct a one-sided test based
on $\eta$, where $H_0$ is rejected if $\eta$ is bigger than the
appropriate critical value; \app{gretl} provides the 90\%, 95\%,
97.5\% and 99\% quantiles.

Usage example:
\begin{code}
  kpss m y
\end{code}
where \verb|m| is an integer representing the bandwidth or window
size used in the formula for estimating the long run variance:
\[
  \bar{\sigma}^2 = \sum_{i=-m}^m \left( 1 - \frac{|i|}{m+1} \right) \hat{\gamma}_i
\]
The $\hat{\gamma}_i$ terms denote the empirical autocovariances of
$e_t$ from order $-m$ through $m$.  For this estimator to be
consistent, $m$ must be large enough to accommodate the short-run
persistence of $e_t$, but not too large compared to the sample size
$T$.  In the GUI interface of \app{gretl}, this value defaults to the
integer part of $4 \left( \frac{T}{100} \right)^{1/4}$.

The above concept can be generalized to the case where $y_t$ is
thought to be stationary around a deterministic trend. In this case,
formula (\ref{eq:KPSStest}) remains unchanged, but the series $e_t$ is
defined as the residuals from an OLS regression of $y_t$ on a constant
and a linear trend. This second form of the test is obtained by
appending the \verb|--trend| option to the \cmd{kpss} command:
\begin{code}
  kpss n y --trend
\end{code}
Note that in this case the asymptotic distribution of the test is
different and the critical values reported by \app{gretl} differ
accordingly.


\subsection{The Johansen tests}
\label{sec:Joh-test}

Strictly speaking, these are tests for cointegration. However, they
can be used as multivariate unit-root tests since they are the
multivariate generalization of the ADF test. See section
\ref{sec:VECM-rep} for more details.

\section{ARCH and GARCH}
\label{sec:arch-garch}

Heteroskedasticity means a non-constant variance of the error term in
a regression model.  Autoregressive Conditional Heteroskedasticity
(ARCH) is a phenomenon specific to time series models, whereby the
variance of the error displays autoregressive behavior; for instance,
the time series exhibits successive periods where the error variance
is relatively large, and successive periods where it is relatively
small.  This sort of behavior is reckoned to be quite common in asset
markets: an unsettling piece of news can lead to a period of increased
volatility in the market.

An ARCH error process of order $q$ can be represented as
\[
u_t = \sigma_t \varepsilon_t; \qquad
\sigma^2_t \equiv {\rm E}(u^2_t|\Omega_{t-1}) = 
\alpha_0 + \sum_{i=1}^q \alpha_i u^2_{t-i}
\]
where the $\varepsilon_t$s are independently and identically
distributed (iid) with mean zero and variance 1, and where $\sigma_t$
is taken to be the positive square root of $\sigma^2_t$.
$\Omega_{t-1}$ denotes the information set as of time $t-1$ and
$\sigma^2_t$ is the conditional variance: that is, the
variance conditional on information dated $t-1$ and earlier.

It is important to notice the difference between ARCH and an ordinary
autoregressive error process.  The simplest (first-order) case of the
latter can be written as
\[
u_t = \rho u_{t-1} + \varepsilon_t; \qquad -1 < \rho < 1
\]
where the $\varepsilon_t$s are independently and identically
distributed with mean zero and variance $\sigma^2$.  With an
AR(1) error, if $\rho$ is positive then a positive value of $u_t$ will
tend to be followed, with probability greater than 0.5, by a positive
$u_{t+1}$.  With an ARCH error process, a disturbance $u_t$ of large
absolute value will tend to be followed by further large absolute
values, but with no presumption that the successive values will be of
the same sign.  ARCH in asset prices is a ``stylized fact'' and is
consistent with market efficiency; on the other hand autoregressive
behavior of asset prices would violate market efficiency.

One can test for ARCH of order $q$ in the following
way:
\begin{enumerate}
\item Estimate the model of interest via OLS and save the squared
  residuals, $\hat{u}^2_t$.
\item Perform an auxiliary regression in which the current squared
  residual is regressed on a constant and $q$ lags of itself.
\item Find the $TR^2$ value (sample size times unadjusted $R^2$) for
  the auxiliary regression.
\item Refer the $TR^2$ value to the $\chi^2$ distribution with $q$
  degrees of freedom, and if the p-value is ``small enough'' reject
  the null hypothesis of homoskedasticity in favor of the alternative
  of ARCH($q$).
\end{enumerate}

This test is implemented in \app{gretl} via the \cmd{arch} command.
This command may be issued following the estimation of a time-series
model by OLS, or by selection from the ``Tests'' menu in the model
window (again, following OLS estimation).  The result of the test is
reported and if the $TR^2$ from the auxiliary regression has a p-value
less than 0.10, ARCH estimates are also reported.  These estimates
take the form of Generalized Least Squares (GLS), specifically
weighted least squares, using weights that are inversely proportional
to the predicted variances of the disturbances, $\hat{\sigma}_t$,
derived from the auxiliary regression.

In addition, the ARCH test is available after estimating a vector
autoregression (VAR).  In this case, however, there is no provision to
re-estimate the model via GLS.

\subsection{GARCH}
\label{subsec:garch}

The simple ARCH($q$) process is useful for introducing the general
concept of conditional heteroskedasticity in time series, but it has
been found to be insufficient in empirical work.  The dynamics of the
error variance permitted by ARCH($q$) are not rich enough to represent 
the patterns found in financial data.  The generalized ARCH or GARCH
model is now more widely used.  

The representation of the variance of a process in the GARCH model is
somewhat (but not exactly) analogous to the ARMA representation of the
level of a time series.  The variance at time $t$ is allowed
to depend on both past values of the variance and past values of the
realized squared disturbance, as shown in the following system
of equations:
\begin{eqnarray}
  \label{eq:garch-meaneq}
  y_t &  = & X_t \beta + u_t \\
  \label{eq:garch-epseq}
  u_t &  = & \sigma_t \varepsilon_t \\
  \label{eq:garch-vareq}
  \sigma^2_t & = & \alpha_0 + \sum_{i=1}^q \alpha_i u^2_{t-i} +
	  \sum_{j=1}^p \delta_i \sigma^2_{t-j}
\end{eqnarray}
As above, $\varepsilon_t$ is an iid sequence with unit variance.
$X_t$ is a matrix of regressors (or in the simplest case,
just a vector of 1s allowing for a non-zero mean of $y_t$).  Note that
if $p=0$, GARCH collapses to ARCH($q$): the generalization is embodied
in the $\delta_i$ terms that multiply previous values of the error
variance.

In principle the underlying innovation, $\varepsilon_t$, could follow
any suitable probability distribution, and besides the obvious
candidate of the normal or Gaussian distribution the $t$ distribution
has been used in this context.  Currently \app{gretl} only handles the
case where $\varepsilon_t$ is assumed to be Gaussian.  However, when
the \verb|--robust| option to the \cmd{garch} command is given, the
estimator \app{gretl} uses for the covariance matrix can be considered
Quasi-Maximum Likelihood even with non-normal disturbances.  See below
for more on the options regarding the GARCH covariance matrix.

Example:
\begin{code}
  garch p q ; y const x
\end{code}
where \verb|p| $\ge 0$ and \verb|q| $>0$ denote the respective lag
orders as shown in equation (\ref{eq:garch-vareq}).  These values
can be supplied in numerical form or as the names of pre-defined
scalar variables.

\subsection{GARCH estimation}
\label{subsec:garch-est}

Estimation of the parameters of a GARCH model is by no means a
straightforward task.  (Consider equation~\ref{eq:garch-vareq}: the
conditional variance at any point in time, $\sigma^2_t$, depends on
the conditional variance in earlier periods, but $\sigma^2_t$ is not
observed, and must be inferred by some sort of Maximum Likelihood
procedure.)  \app{Gretl} uses the method proposed by Fiorentini,
Calzolari and Panattoni (1996),\footnote{The algorithm is based on
  Fortran code deposited in the archive of the \textit{Journal of
    Applied Econometrics} by the authors, and is used by kind
  permission of Professor Fiorentini.}  which was adopted as a
benchmark in the study of GARCH results by McCullough and Renfro
(1998).  It employs analytical first and second derivatives of the
log-likelihood, and uses a mixed-gradient algorithm, exploiting the
information matrix in the early iterations and then switching to the
Hessian in the neighborhood of the maximum likelihood.  (This progress
can be observed if you append the \verb|--verbose| option to
\app{gretl}'s \cmd{garch} command.)

Several options are available for computing the covariance matrix of
the parameter estimates in connection with the \cmd{garch} command.
At a first level, one can choose between a ``standard'' and a
``robust'' estimator.  By default, the Hessian is used unless the
\verb|--robust| option is given, in which case the QML estimator is
used.  A finer choice is available via the \cmd{set} command, as
shown in Table~\ref{tab:garch-vcv}.

\begin{table}[htbp]
\caption{Options for the GARCH covariance matrix}
\label{tab:garch-vcv}
\begin{center}
\begin{tabular}{ll}
\multicolumn{1}{c}{\textit{command}} &
\multicolumn{1}{c}{\textit{effect}} \\ [4pt]
\texttt{set garch\_vcv hessian} & Use the Hessian \\
\texttt{set garch\_vcv im} & Use the Information Matrix \\
\texttt{set garch\_vcv op} & Use the Outer Product of the Gradient \\
\texttt{set garch\_vcv qml} & QML estimator \\
\texttt{set garch\_vcv bw} & Bollerslev--Wooldridge ``sandwich'' estimator
\end{tabular}
\end{center}
\end{table}

It is not uncommon, when one estimates a GARCH model for an arbitrary
time series, to find that the iterative calculation of the estimates
fails to converge.  For the GARCH model to make sense, there are
strong restrictions on the admissible parameter values, and it is not
always the case that there exists a set of values inside the
admissible parameter space for which the likelihood is maximized.  

The restrictions in question can be explained by reference to the
simplest (and much the most common) instance of the GARCH model, where
$p = q = 1$.  In the GARCH(1, 1) model the conditional variance is
\begin{equation}
\label{eq:condvar}
\sigma^2_t = \alpha_0 + \alpha_1 u^2_{t-1} + \delta_1 \sigma^2_{t-1}
\end{equation}
Taking the unconditional expectation of (\ref{eq:condvar}) we get
\[
\sigma^2 = \alpha_0 + \alpha_1 \sigma^2 + \delta_1 \sigma^2
\]
so that
\[
\sigma^2 = \frac{\alpha_0}{1 - \alpha_1 - \delta_1}
\]
For this unconditional variance to exist, we require that $\alpha_1 +
\delta_1 < 1$, and for it to be positive we require that $\alpha_0 > 0$.

A common reason for non-convergence of GARCH estimates (that is, a
common reason for the non-existence of $\alpha_i$ and $\delta_i$ values
that satisfy the above requirements and at the same time maximize the
likelihood of the data) is misspecification of the model.  It is
important to realize that GARCH, in itself, allows \textit{only} for
time-varying volatility in the data.  If the \textit{mean} of the
series in question is not constant, or if the error process is not
only heteroskedastic but also autoregressive, it is necessary to take
this into account when formulating an appropriate model.  For example,
it may be necessary to take the first difference of the variable in
question and/or to add suitable regressors, $X_t$, as in
(\ref{eq:garch-meaneq}).

\section{Cointegration and Vector Error Correction Models}
\label{vecm-explanation}

\subsection{Vector Error Correction Models as representation of a
  cointegrated system}
\label{sec:VECM-rep}

Consider a VAR of order $p$ with a deterministic part given by $\mu_t$
(typically, a polynomial in time). Then, it is possible to write the $n$-variate
process $y_t$ as 
\[
  y_t = \mu_t + A_1 y_{t-1} + A_2 y_{t-2} + \cdots + A_p y_{t-p} +
  \epsilon_t ;
\]
however, this model can be re-cast in a form more suitable to analyze
the phenomenon of cointegration. Since $y_t = y_{t-1} - \Delta y_t$
and $y_{t-i} = y_{t-1} - (\Delta y_{t-1} + \Delta y_{t-2} + \cdots +
\Delta y_{t-i+1}$, the \emph{Vector Error Correction} form of the
previous model is given by
\begin{equation}
  \label{eq:Joh-tests}
  \Delta y_t = \mu_t + \Pi y_{t-1} + \sum_{i=1}^p \Gamma_i \Delta
  y_{t-i} + \epsilon_t ,
\end{equation}
where $\Pi = \sum_{i=1}^p A_i$ and $\Gamma_k = -\sum_{i=k}^p A_i$.

If the rank of $\Pi$ is 0, the processes are all I(1); if the rank of
$\Pi$ is full, the processes are all I(0); in between, $\Pi$ can be
written as $\alpha \beta'$ and you have cointegration.

The rank of $\Pi$ is investigated by computing the eigenvalues of a
closely related matrix (call it $M$) whose rank is the same as $\Pi$:
however, $M$ is by construction symmetric and positive semidefinite.
As a consequence, all its eigenvalues are real and non-negative; tests
on the rank of $\Pi$ can therefore be carried out by testing how many
eigenvalues of $M$ are 0.

If all the eigenvalues are significantly different from 0, then all the
processes are stationary. If, on the contrary, there is at least one
zero eigenvalue, then the $y_t$ process is integrated, although some
linear combination $\beta'y_t$ might be stationary. On the other
extreme, if no eigenvalues are significantly different from 0, then not
only the process $y_t$ is non-stationary, but the same holds for any
linear combination $\beta'y_t$; in other words, no cointegration
occurs.

The two Johansen tests are the ``$\lambda$-max'' test, for hypotheses
on individual eigenvalues, and the ``trace'' test, for joint
hypotheses.  The \app{gretl} command \cmd{coint2} performs these two
tests.  

As in the ADF test, the asymptotic distribution of the tests varies
with the deterministic kernel $\mu_t$ one includes in the VAR.
\app{gretl} provides the following options (for a short discussion of
the meaning of the five options, see section \ref{sec:johansen-test} below):
\begin{center}
  \begin{tabular}{cc}
    \hline
    $\mu_t$ & command option \\
    \hline
    0 & \verb|--nc| \\
    $\mu_0, \alpha_{\perp}'\mu_0 = 0 $ &  \verb|--rc| \\
    $\mu_0$ &  default \\
    $\mu_0 + \mu_1 t , \alpha_{\perp}'\mu_1 = 0$ &  \verb|--crt| \\
    $\mu_0 + \mu_1 t$ &  \verb|--ct| \\
    \hline
  \end{tabular}
\end{center}
Note that for this command the above options are mutually exclusive.
In addition, you have the option of using the \verb|--seasonal|
options, for augmenting $\mu_t$ with centered seasonal dummies.  In
each case, p-values are computed via the approximations by Doornik
(1998).

The following code uses the \cmd{denmark} database, supplied with
\app{gretl}, to replicate Johansen's example found in his 1995 book.
\begin{code}
  open denmark
  coint2 2 LRM LRY IBO IDE --rc --seasonal
\end{code}
In this case, the vector $y_t$ in equation (\ref{eq:Joh-tests})
comprises the four variables \cmd{LRM}, \cmd{LRY}, \cmd{IBO},
\cmd{IDE}. The number of lags equals $p$ in (\ref{eq:Joh-tests}) plus
one. Part of the output is reported below:
\begin{center}
\begin{code}
Johansen test:
Number of equations = 4
Lag order = 2
Estimation period: 1974:3 - 1987:3 (T = 53)

Case 2: Restricted constant
Rank Eigenvalue Trace test p-value   Lmax test  p-value
   0    0.43317     49.144 [0.1284]     30.087 [0.0286]
   1    0.17758     19.057 [0.7833]     10.362 [0.8017]
   2    0.11279     8.6950 [0.7645]     6.3427 [0.7483]
   3   0.043411     2.3522 [0.7088]     2.3522 [0.7076]

eigenvalue     0.43317      0.17758      0.11279     0.043411 
\end{code}
\end{center}

Since both the trace and $\lambda$-max accept the null hypothesis that
the smallest eigenvalue is in fact 0, we may conclude that the series
are in fact non-stationary. However, some linear combination may be
$I(0)$, as indicated by the rejection of the $\lambda$-max of the
hypothesis that the rank of $\Pi$ is 0 (the trace test gives less
clear-cut evidence for this).

\subsection{The Johansen cointegration test}
\label{sec:johansen-test}

The Johansen test for cointegration is used to establish the rank of
$\beta$; in other words, how many cointegration vectors the system
has.  This test has to take into account what hypotheses one is
willing to make on the deterministic terms, which leads to the famous
``five cases.'' A full and general illustration of the five cases
requires a fair amount of matrix algebra, but an intuitive
understanding of the issue can be gained by means of a simple example.
    
Consider a series $x_t$ which behaves as follows
%      
\[ x_t = m + x_{t-1} + \varepsilon_t \] 
%
where $m$ is a real number and $\varepsilon_t$ is a white noise
process. As is easy to show, $x_t$ is a random walk which fluctuates
around a deterministic trend with slope $m$. In the special case $m$ =
0, the deterministic trend disappears and $x_t$ is a pure random walk.
    
Consider now another process $y_t$, defined by
%      
\[ y_t = k + x_t + u_t \] 
%
where, again, $k$ is a real number and $u_t$ is a white noise process.
Since $u_t$ is stationary by definition, $x_t$ and $y_t$ cointegrate:
that is, their difference
%      
\[ z_t = y_t - x_t = k + u_t \]
%	
is a stationary process. For $k$ = 0, $z_t$ is simple zero-mean white
noise, whereas for $k$ $\ne$ 0 the process $z_t$ is white noise with a
non-zero mean.
  
After some simple substitutions, the two equations above can be
represented jointly as a VAR(1) system
%      
\[ \left[ \begin{array}{c} y_t \\ x_t \end{array} \right] = \left[
  \begin{array}{c} k + m \\ m \end{array} \right] + \left[
  \begin{array}{rr} 0 & 1 \\ 0 & 1 \end{array} \right] \left[
  \begin{array}{c} y_{t-1} \\ x_{t-1} \end{array} \right] + \left[
  \begin{array}{c} u_t + \varepsilon_t \\ \varepsilon_t \end{array}
\right] \]
%	
or in VECM form
%      
\begin{eqnarray*}
  \left[  \begin{array}{c} \Delta y_t \\ \Delta x_t \end{array} \right]  & = & 
  \left[  \begin{array}{c} k + m \\ m \end{array} \right] +
  \left[  \begin{array}{rr} -1 & 1 \\ 0 & 0 \end{array} \right] 
  \left[  \begin{array}{c} y_{t-1} \\ x_{t-1} \end{array} \right] + 
  \left[  \begin{array}{c} u_t + \varepsilon_t \\ \varepsilon_t \end{array} \right] = \\
  & = & 
  \left[  \begin{array}{c} k + m \\ m \end{array} \right] +
  \left[  \begin{array}{r} -1 \\ 0 \end{array} \right]
  \left[  \begin{array}{rr} 1 & -1 \end{array} \right] 
  \left[  \begin{array}{c} y_{t-1} \\ x_{t-1} \end{array} \right] + 
  \left[  \begin{array}{c} u_t + \varepsilon_t \\ \varepsilon_t \end{array} \right] = \\
  & = & 
  \mu_0 + \alpha \beta^{\prime} \left[  \begin{array}{c} y_{t-1} \\ x_{t-1} \end{array} \right] + \eta_t = 
  \mu_0 + \alpha z_{t-1} + \eta_t ,
\end{eqnarray*}
%	
where $\beta$ is the cointegration vector and $\alpha$ is the
``loadings'' or ``adjustments'' vector.
     
We are now in a position to consider three possible cases:
    
\begin{enumerate}
\item $m$ $\ne$ 0: In this case $x_t$ is trended, as we just saw; it
  follows that $y_t$ also follows a linear trend because on average it
  keeps at a distance $k$ from $x_t$. The vector
  $\mu_0$ is unrestricted.  This case is the default
  for gretl's \cmd{vecm} command.
	
\item $m$ = 0 and $k$ $\ne$ 0: In this case, $x_t$ is not trended and
  as a consequence neither is $y_t$. However, the mean distance
  between $y_t$ and $x_t$ is non-zero. The vector
  $\mu_0$ is given by
%	  
  \[
  \mu_0 = \left[ \begin{array}{c} k \\ 0 \end{array} \right]
  \]
%	    
  which is not null and therefore the VECM shown above does have a
  constant term. The constant, however, is subject to the restriction
  that its second element must be 0. More generally,
  $\mu_0$ is a multiple of the vector $\alpha$. Note
  that the VECM could also be written as
%	  
  \[
  \left[ \begin{array}{c} \Delta y_t \\ \Delta x_t \end{array} \right]
  = \left[ \begin{array}{r} -1 \\ 0 \end{array} \right] \left[
    \begin{array}{rrr} 1 & -1 & -k \end{array} \right] \left[
    \begin{array}{c} y_{t-1} \\ x_{t-1} \\ 1 \end{array} \right] +
  \left[ \begin{array}{c} u_t + \varepsilon_t \\ \varepsilon_t
    \end{array} \right]
  \]
%	   
  which incorporates the intercept into the cointegration vector. This
  is known as the ``restricted constant'' case; it may be specified in
  gretl's \cmd{vecm} command using the option flag \verb+--rc+.
	
\item $m$ = 0 and $k$ = 0: This case is the most restrictive: clearly,
  neither $x_t$ nor $y_t$ are trended, and the mean distance between
  them is zero. The vector $\mu_0$ is also 0, which
  explains why this case is referred to as ``no constant.''  This case
  is specified using the option flag \verb+--nc+ with \cmd{vecm}.
	
\end{enumerate}

In most cases, the choice between the three possibilities is based on
a mix of empirical observation and economic reasoning. If the
variables under consideration seem to follow a linear trend then we
should not place any restriction on the intercept. Otherwise, the
question arises of whether it makes sense to specify a cointegration
relationship which includes a non-zero intercept. One example where
this is appropriate is the relationship between two interest rates:
generally these are not trended, but the VAR might still have an
intercept because the difference between the two (the ``interest rate
spread'') might be stationary around a non-zero mean (for example,
because of a risk or liquidity premium).
    
The previous example can be generalized in three directions:
    
\begin{enumerate}
\item If a VAR of order greater than 1 is considered, the algebra gets
  more convoluted but the conclusions are identical.
\item If the VAR includes more than two endogenous variables the
  cointegration rank $r$ can be greater than 1. In this case, $\alpha$
  is a matrix with $r$ columns, and the case with restricted constant
  entails the restriction that $\mu_0$ should be some
  linear combination of the columns of $\alpha$.
\item If a linear trend is included in the model, the deterministic
  part of the VAR becomes $\mu_0 + \mu_1 t$. The reasoning is
  practically the same as above except that the focus now centers on
  $\mu_1$ rather than $\mu_0$.  The counterpart to the ``restricted
  constant'' case discussed above is a ``restricted trend'' case, such
  that the cointegration relationships include a trend but the first
  differences of the variables in question do not.  In the case of an
  unrestricted trend, the trend appears in both the cointegration
  relationships and the first differences, which corresponds to the
  presence of a quadratic trend in the variables themselves (in
  levels).  These two cases are specified by the option flags
  \verb+--crt+ and \verb+--ct+, respectively, with the \cmd{vecm}
  command.
\end{enumerate}

\subsection{Identification of the cointegration vectors}
\label{sec:johansen-ident}

\textbf{FIXME: this is but a stub}

Maximum likelihood estimation of $\beta$ can be shown to be equivalent
to solving an eigenvector problem. In practice, if the cointegration
rank is known to be $r$, the $n \times r$ matrix $\beta$ is estimated
as the solution to the following matrix equation:
\[
  M \beta = \beta \langle \lambda \rangle
\]
where $\langle \lambda \rangle$ is a diagonal matrix containing the
eigenvalues. Notice, however, that if all columns of $\beta$ are
cointegration vectors, then any arbitrary linear combinations of those
is a cointegration vector too. This means that the matrix $\beta$ is
under-identified. As a consequence, its elements do not have a proper
covariance matrix, but this difficulty can be circumvented by imposing
an appropriate number of restrictions. The method \app{gretl} uses is
known as the ``Phillips normalization''. The starting point is writing
$\beta$ in partitioned form as in
\[
  \beta = \left[
    \begin{array}{c} \beta_1 \\ \beta_2  \end{array}
    \right] ,
\]
where $\beta_1$ is an $r \times r$ matrix and  $\beta_2$ is $(n-r)
\times r$. Assuming that $\beta_1$ has full rank, $beta$ can be
post-multiplied by $\beta_1^{-1}$, giving
\[
  \hat{\beta} = \left[
    \begin{array}{c} I \\ \beta_2 \beta_1^{-1}  \end{array}
    \right] =
    \left[
    \begin{array}{c} I \\ \hat{\beta_2} \end{array}
  \right]  ,
\]

The coefficients that \app{gretl} produces are $\hat{\beta}$, with
$\hat{\beta_2}$ known as the matrix of unrestricted coefficients.

%%% Local Variables: 
%%% mode: latex
%%% TeX-master: "gretl-guide"
%%% End: 

