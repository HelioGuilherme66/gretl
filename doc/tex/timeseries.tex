\chapter{Univariate time series models}
\label{chap:timeseries}

\section{Introduction}
\label{sec:tsintro}

Time series models are discussed in this chapter and the next two.
Here we concentrate on ARIMA models, unit root tests, and GARCH. The
following chapter deals with VARs, and chapter~\ref{chap:vecm} with
cointegration and error correction.

\section{ARIMA models}
\label{arma-estimation}

\subsection{Representation and syntax}
\label{arma-repr}

The \cmd{arma} command performs estimation of AutoRegressive,
Integrated, Moving Average (ARIMA) models.  These are models that can
be written in the form
\begin{equation}
  \label{eq:plain-0-arma}
  \phi(L) y_t = \theta(L) \epsilon_t
\end{equation}
where $\phi(L)$, and $\theta(L)$ are polynomials in the lag operator,
$L$, defined such that $L^n x_t = x_{t-n}$, and $\epsilon_t$ is a
white noise process. The exact content of $y_t$, of the AR polynomial
$\phi()$, and of the MA polynomial $\theta()$, will be explained in the
following.

\subsection{Mean terms}
\label{sec:arma-nonzeromean}

The process $y_t$ as written in equation (\ref{eq:plain-0-arma}) has,
without further qualifications, mean zero. If the model is to be
applied to real data, it is necessary to include some term to handle
the possibility that $y_t$ has non-zero mean. There are two possible
ways to represent processes with nonzero mean: one is to define $\mu_t$
as the \emph{unconditional} mean of $y_t$, namely the central value of
its marginal distribution. Therefore, the series $\tilde{y}_t = y_t -
\mu_t$ has mean 0, and the model (\ref{eq:plain-0-arma}) applies to
$\tilde{y}_t$. In practice, assuming that $\mu_t$ is a linear function
of some observable variables $x_t$, the model becomes
\begin{equation}
  \label{eq:arma-with-x}
  \phi(L) (y_t - x_t \beta) = \theta(L) \epsilon_t
\end{equation}
This is sometimes known as a ``regression model with ARMA errors'';
its structure may be more apparent if we represent it using two
equations:
\begin{eqnarray*}
  y_t & = & x_t \beta + u_t \\
  \phi(L) u_t & = & \theta(L) \epsilon_t
\end{eqnarray*}

The model just presented is also sometimes known as ``ARMAX'' (ARMA +
eXogenous variables).  It seems to us, however, that this label is
more appropriately applied to a different model: another way to
include a mean term in (\ref{eq:plain-0-arma}) is to base the
representation on the \emph{conditional} mean of $y_t$, that is the
central value of the distribution of $y_t$ \emph{given its own past}.
Assuming, again, that this can be represented as a linear combination
of some observable variables $z_t$, the model would expand to
\begin{equation}
  \label{eq:arma-with-z}
  \phi(L) y_t = z_t \gamma + \theta(L) \epsilon_t
\end{equation}
The formulation (\ref{eq:arma-with-z}) has the advantage that $\gamma$
can be immediately interpreted as the vector of marginal effects of
the $z_t$ variables on the conditional mean of $y_t$.  And by adding
lags of $z_t$ to this specification one can estimate \emph{Transfer
  Function models} (which generalize ARMA by adding the effects of
exogenous variable distributed across time).

Gretl provides a way to estimate both forms. Models written as
in (\ref{eq:arma-with-x}) are estimated by maximum likelihood; models
written as in (\ref{eq:arma-with-z}) are estimated by conditional
maximum likelihood. (For more on these options see the section on
``Estimation'' below.)  

In the special case when $x_t = z_t = 1$ (that is, the models include
a constant but no exogenous variables) the two specifications discussed
above reduce to
\begin{equation}
  \phi(L) (y_t - \mu) = \theta(L) \epsilon_t
  \label{eq:arma-with-xconst} 
\end{equation}
and
\begin{equation}
  \phi(L) y_t = \alpha + \theta(L) \epsilon_t
  \label{eq:arma-with-zconst}
\end{equation}
respectively.  These formulations are essentially equivalent, but if
they represent one and the same process $\mu$ and $\alpha$ are, fairly
obviously, not numerically identical; rather
\[
\alpha = \left(1 - \phi_1 - \ldots - \phi_p\right) \mu
\]

The gretl syntax for estimating (\ref{eq:arma-with-xconst}) is simply
\begin{code}
arma p q ; y
\end{code}
The AR and MA lag orders, \texttt{p} and \texttt{q}, can be given either as
numbers or as pre-defined scalars. The parameter $\mu$ can be dropped
if necessary by appending the option \cmd{--nc} (``no constant'') to
the command. If estimation of (\ref{eq:arma-with-zconst}) is needed,
the switch \option{conditional} must be appended to the command, as
in 
\begin{code}
arma p q ; y --conditional
\end{code}

Generalizing this principle to the estimation of
(\ref{eq:arma-with-x}) or (\ref{eq:arma-with-z}), you get that
%
\begin{code}
arma p q ; y const x1 x2
\end{code}
%
would estimate the following model:
\[
  y_t - x_t \beta = \phi_1 \left(y_{t-1} - x_{t-1} \beta \right) + \ldots + 
   \phi_p \left( y_{t-p} - x_{t-p} \beta \right) + 
  \epsilon_t + \theta_1 \epsilon_{t-1} + \ldots + \theta_q \epsilon_{t-q}
\]
where in this instance $x_t \beta = \beta_0 + x_{t,1} \beta_1 +
x_{t,2} \beta_2$. Appending the \option{conditional} switch, as in 
\begin{code}
arma p q ; y const x1 x2 --conditional
\end{code}
would estimate the following model:
\[
  y_t = x_t \gamma + \phi_1 y_{t-1} + \ldots +  \phi_p y_{t-p} + 
  \epsilon_t + \theta_1 \epsilon_{t-1} + \ldots + \theta_q \epsilon_{t-q}
\]

Ideally, the issue broached above could be made moot by writing a more
general specification that nests the alternatives; that is
\begin{equation}
 \label{armax-general}
  \phi(L) \left(y_t - x_t \beta\right) = z_t \gamma  + \theta(L) \epsilon_t ;
\end{equation}
we would like to generalize the \cmd{arma} command so that
the user could specify, for any estimation method, whether certain
exogenous variables should be treated as $x_t$s or $z_t$s, but we're
not yet at that point (and neither are most other software packages).


\subsection{Seasonal models}

A more flexible lag structure is desirable when analyzing time series
that display strong seasonal patterns. Model (\ref{eq:plain-0-arma})
can be expanded to
\begin{equation}
  \label{eq:seasonal-arma}
  \phi(L) \Phi(L^s) y_t = \theta(L) \Theta(L^s) \epsilon_t .
\end{equation}
For such cases, a fuller form of the syntax is available, namely,
\begin{code}
arma p q ; P Q ; y
\end{code}
where \texttt{p} and \texttt{q} represent the non-seasonal AR and MA
orders, and \texttt{P} and \texttt{Q} the seasonal orders.  For
example,
\begin{code}
arma 1 1 ; 1 1 ; y
\end{code}
would be used to estimate the following model:
\[
  (1 -\phi L)(1 -\Phi L^s) (y_t - \mu) = (1 + \theta L)(1 + \Theta L^s) \epsilon_t
\]
If $y_t$ is a quarterly series (and therefore $s=4$), the above
equation can be written more explicitly as
\[
y_t - \mu = \phi (y_{t-1} - \mu) + \Phi (y_{t-4} - \mu) - (\phi
  \cdot \Phi) (y_{t-5} - \mu) + \epsilon_t + \theta \epsilon_{t-1} + \Theta
  \epsilon_{t-4} + (\theta \cdot \Theta) \epsilon_{t-5}
\]
Such a model is known as a ``multiplicative seasonal ARMA model''.


\subsection{Gaps in the lag structure}

The standard way to specify an ARMA model in gretl is via the AR
and MA orders, $p$ and $q$ respectively.  In this case all lags from 1
to the given order are included.  In some cases one may wish to
include only certain specific AR and/or MA lags.  This can be done in
either of two ways.
%
\begin{itemize}
\item One can construct a matrix containing the desired lags (positive
  integer values) and supply the name of this matrix in place of $p$
  or $q$.
\item One can give a comma-separated list of lags, enclosed in braces,
  in place of $p$ or $q$.
\end{itemize}
%
The following code illustrates these options:
%
\begin{code}
matrix pvec = {1,4}
arma pvec 1 ; y
arma {1,4} 1 ; y
\end{code}
%
Both forms above specify an ARMA model in which AR lags 1 and 4 are
used (but not 2 and 3). 

This facility is available only for the non-seasonal component of
the ARMA specification.

\subsection{Differencing and ARIMA}

The above discussion presupposes that the time series $y_t$ has
already been subjected to all the transformations deemed necessary for
ensuring stationarity (see also section \ref{sec:uroot}). Differencing
is the most common of these transformations, and gretl provides
a mechanism to include this step into the \cmd{arma} command: the
syntax
\begin{code}
arma p d q ; y 
\end{code}
would estimate an ARMA$(p,q)$ model on $\Delta^d y_t$. It is
functionally equivalent to 
\begin{code}
series tmp = y
loop i=1..d
  tmp = diff(tmp)
endloop
arma p q ; tmp 
\end{code}
except with regard to forecasting after estimation (see below).

When the series $y_t$ is differenced before performing the analysis
the model is known as ARIMA (``I'' for Integrated); for this reason,
gretl provides the \cmd{arima} command as an alias for
\cmd{arma}.

Seasonal differencing is handled similarly, with the syntax
\begin{code}
arma p d q ; P D Q ; y 
\end{code}
where \texttt{D} is the order for seasonal differencing.  Thus, the
command
\begin{code}
arma 1 0 0 ; 1 1 1 ; y 
\end{code}
would produce the same parameter estimates as
\begin{code}
series dsy = sdiff(y)
arma 1 0 ; 1 1 ; dsy 
\end{code}
where we use the \texttt{sdiff} function to create a seasonal
difference (e.g.\ for quarterly data, $y_t - y_{t-4}$).

In specifying an ARIMA model with exogenous regressors we face a
choice which relates back to the discussion of the variant models
(\ref{eq:arma-with-x}) and (\ref{eq:arma-with-z}) above.  If we choose
model (\ref{eq:arma-with-x}), the ``regression model with ARMA
errors'', how should this be extended to the case of ARIMA?  The issue
is whether or not the differencing that is applied to the dependent
variable should also be applied to the regressors.  Consider the
simplest case, ARIMA with non-seasonal differencing of order 1. We may
estimate either
\begin{equation}
  \label{eq:arima-with-x}
  \phi(L) (1-L) (y_t - X_t\beta) = \theta(L) \epsilon_t
\end{equation}
or
\begin{equation}
  \label{eq:arima-with-z}
  \phi(L) \left((1-L) y_t - X_t\beta\right) = \theta(L) \epsilon_t
\end{equation}
%
The first of these formulations can be described as a regression model
with ARIMA errors, while the second preserves the levels of the $X$
variables.  As of gretl version 1.8.6, the default model is
(\ref{eq:arima-with-x}), in which differencing is applied to both
$y_t$ and $X_t$.  However, when using the default estimation method
(native exact ML, see below), the option \option{y-diff-only} may be
given, in which case gretl estimates
(\ref{eq:arima-with-z}).\footnote{Prior to gretl 1.8.6, the
  default model was (\ref{eq:arima-with-z}).  We changed this for the
  sake of consistency with other software.}

\subsection{Estimation}
\label{arma-est}

The default estimation method for ARMA models is exact maximum
likelihood estimation (under the assumption that the error term is
normally distributed), using a variety of techniques: the main
algorithm for evaluating the log-likelihood is AS197 by
\cite{melard84}. Maximization is performed via BFGS and the score is
approximated numerically.  This method produces results that are
directly comparable with many other software packages.  The constant,
and any exogenous variables, are treated as in equation
(\ref{eq:arma-with-x}).  The covariance matrix for the parameters is
computed using a numerical approximation to the Hessian at
convergence.

The alternative method, invoked with the \option{conditional} switch,
is conditional maximum likelihood (CML), also known as ``conditional
sum of squares'' \citep[see][p.\ 132]{hamilton94}.  This method was
exemplified in the script~\ref{jack-arma}, and only a brief
description will be given here.  Given a sample of size $T$, the CML
method minimizes the sum of squared one-step-ahead prediction errors
generated by the model for the observations $t_0, \ldots, T$.  The
starting point $t_0$ depends on the orders of the AR polynomials in
the model.  The numerical maximization method used is BHHH, and the
covariance matrix is computed using a Gauss--Newton regression.

The CML method is nearly equivalent to maximum likelihood under the
hypothesis of normality; the difference is that the first $(t_0 - 1)$
observations are considered fixed and only enter the likelihood
function as conditioning variables. As a consequence, the two methods
are asymptotically equivalent under standard conditions---except for
the fact, discussed above, that our CML implementation treats the
constant and exogenous variables as per equation (\ref{eq:arma-with-z}).

The two methods can be compared as in the following example
\begin{code}
open data10-1
arma 1 1 ; r
arma 1 1 ; r --conditional
\end{code}
which produces the estimates shown in Table~\ref{tab:ml-cml}.  As you
can see, the estimates of $\phi$ and $\theta$ are quite similar.  The
reported constants differ widely, as expected---see the discussion
following equations (\ref{eq:arma-with-xconst}) and
(\ref{eq:arma-with-zconst}).  However, dividing the CML constant by
$1-\phi$ we get 7.38, which is not far from the ML estimate of 6.93.

\begin{table}[htbp]
\caption{ML and CML estimates}
\label{tab:ml-cml}
\begin{center}
  \begin{tabular}{crrrr}
    \hline
    Parameter & \multicolumn{2}{c}{ML} &
    \multicolumn{2}{c}{CML} \\
    \hline 
    $\mu$ & 6.93042 & (0.923882) & 1.07322 & (0.488661) \\
    $\phi$ & 0.855360 & (0.0511842) & 0.852772 & (0.0450252) \\
    $\theta$ & 0.588056 & (0.0986096) & 0.591838 & (0.0456662) \\
    \hline
  \end{tabular}
\end{center}
\end{table}

\subsection{Convergence and initialization}

The numerical methods used to maximize the likelihood for ARMA models
are not guaranteed to converge.  Whether or not convergence is
achieved, and whether or not the true maximum of the likelihood
function is attained, may depend on the starting values for the
parameters.  Gretl employs one of the following two
initialization mechanisms, depending on the specification of the model
and the estimation method chosen.

\begin{enumerate}
\item Estimate a pure AR model by Least Squares (nonlinear least
  squares if the model requires it, otherwise OLS).  Set the AR
  parameter values based on this regression and set the MA
  parameters to a small positive value (0.0001).
\item The Hannan--Rissanen method: First estimate an autoregressive
  model by OLS and save the residuals.  Then in a second OLS pass add
  appropriate lags of the first-round residuals to the model, to
  obtain estimates of the MA parameters.
\end{enumerate}

To see the details of the ARMA estimation procedure, add the
\option{verbose} option to the command.  This prints a notice of the
initialization method used, as well as the parameter values and
log-likelihood at each iteration.

Besides the built-in initialization mechanisms, the user has the
option of specifying a set of starting values manually.  This is done
via the \cmd{set} command: the first argument should be the keyword
\texttt{initvals} and the second should be the name of a pre-specified
matrix containing starting values.  For example
\begin{code}
matrix start = { 0, 0.85, 0.34 }
set initvals start
arma 1 1 ; y
\end{code}
The specified matrix should have just as many parameters as the model:
in the example above there are three parameters, since the model
implicitly includes a constant.  The constant, if present, is always
given first; otherwise the order in which the parameters are
expected is the same as the order of specification in the \cmd{arma}
or \cmd{arima} command.  In the example the constant is set to zero,
$\phi_1$ to 0.85, and $\theta_1$ to 0.34.

You can get gretl to revert to automatic initialization via
the command \cmd{set initvals auto}.

Two variants of the BFGS algorithm are available in gretl.  In
general we recommend the default variant, which is based on an
implementation by \cite{nash90}, but for some problems the
alternative, limited-memory version (L-BFGS-B, see
\citealp{byrd-etal95}) may increase the chances of convergence on the
ML solution.  This can be selected via the \option{lbfgs} option to
the \texttt{arma} command.

\subsection{Estimation via X-12-ARIMA}

As an alternative to estimating ARMA models using ``native'' code,
gretl offers the option of using the external program
\app{X-12-ARIMA}.  This is the seasonal adjustment software produced
and maintained by the U.S. Census Bureau; it is used for all official
seasonal adjustments at the Bureau.

Gretl includes a module which interfaces with \app{X-12-ARIMA}:
it translates \cmd{arma} commands using the syntax outlined above into
a form recognized by \app{X-12-ARIMA}, executes the program, and
retrieves the results for viewing and further analysis within
gretl.  To use this facility you have to install
\app{X-12-ARIMA} separately.  Packages for both MS Windows and
GNU/Linux are available from the gretl website,
\url{http://gretl.sourceforge.net/}.

To invoke \app{X-12-ARIMA} as the estimation engine, append the flag
\verb|--x-12-arima|, as in
\begin{code}
arma p q ; y --x-12-arima
\end{code}
As with native estimation, the default is to use exact ML but there is
the option of using conditional ML with the \option{conditional} flag.
However, please note that when \app{X-12-ARIMA} is used in conditional
ML mode, the comments above regarding the variant treatments of the
mean of the process $y_t$ \textit{do not apply}.  That is, when you
use \app{X-12-ARIMA} the model that is estimated is
(\ref{eq:arma-with-x}), regardless of whether estimation is by exact
ML or conditional ML.  In addition, the treatment of exogenous
regressors in the context of ARIMA differencing is always that shown
in equation (\ref{eq:arima-with-x}).


\subsection{Forecasting}
\label{arma-fcast}

ARMA models are often used for forecasting purposes.  The
autoregressive component, in particular, offers the possibility of
forecasting a process ``out of sample'' over a substantial time
horizon.

Gretl supports forecasting on the basis of ARMA models using the
method set out by \cite{box-jenkins76}.\footnote{See in particular
  their ``Program 4'' on p.\ 505ff.}  The Box and Jenkins algorithm
produces a set of integrated AR coefficients which take into account
any differencing of the dependent variable (seasonal and/or
non-seasonal) in the ARIMA context, thus making it possible to
generate a forecast for the level of the original variable.  By
contrast, if you first difference a series manually and then apply
ARMA to the differenced series, forecasts will be for the differenced
series, not the level.  This point is illustrated
in Listing~\ref{arima-fcast-script}.  The parameter estimates are identical
for the two models.  The forecasts differ but are mutually consistent:
the variable \texttt{fcdiff} emulates the ARMA forecast (static,
one step ahead within the sample range, and dynamic out of sample).

\begin{script}[htbp]
  \caption{ARIMA forecasting}
  \label{arima-fcast-script}
\begin{scode}
open greene18_2.gdt
# log of quarterly U.S. nominal GNP, 1950:1 to 1983:4
series y = log(Y)
# and its first difference
series dy = diff(y)
# reserve 2 years for out-of-sample forecast
smpl ; 1981:4
# Estimate using ARIMA
arima 1 1 1 ; y 
# forecast over full period
smpl --full
fcast fc1
# Return to sub-sample and run ARMA on the first difference of y
smpl ; 1981:4
arma 1 1 ; dy
smpl --full
fcast fc2
series fcdiff = (t<=1982:1)? (fc1 - y(-1)) : (fc1 - fc1(-1))
# compare the forecasts over the later period
smpl 1981:1 1983:4
print y fc1 fc2 fcdiff --byobs
\end{scode}
\scriptURL
The output from the last command is:
%
\begin{code}
                  y          fc1          fc2       fcdiff
1981:1      7.964086     7.940930      0.02668      0.02668
1981:2      7.978654     7.997576      0.03349      0.03349
1981:3      8.009463     7.997503      0.01885      0.01885
1981:4      8.015625     8.033695      0.02423      0.02423
1982:1      8.014997     8.029698      0.01407      0.01407
1982:2      8.026562     8.046037      0.01634      0.01634
1982:3      8.032717     8.063636      0.01760      0.01760
1982:4      8.042249     8.081935      0.01830      0.01830
1983:1      8.062685     8.100623      0.01869      0.01869
1983:2      8.091627     8.119528      0.01891      0.01891
1983:3      8.115700     8.138554      0.01903      0.01903
1983:4      8.140811     8.157646      0.01909      0.01909
\end{code}
\end{script}


\section{Unit root tests}
\label{sec:uroot}

\subsection{The ADF test}
\label{sec:ADFtest}

The Augmented Dickey--Fuller (ADF) test is, as implemented in
gretl, the $t$-statistic on $\varphi$ in the following regression:
\begin{equation}
  \label{eq:ADFtest}
  \Delta y_t = \mu_t + \varphi y_{t-1} + \sum_{i=1}^p \gamma_i \Delta
  y_{t-i} + \epsilon_t .
\end{equation}

This test statistic is probably the best-known and most widely used
unit root test. It is a one-sided test whose null hypothesis is
$\varphi = 0$ versus the alternative $\varphi < 0$ (and hence large
negative values of the test statistic lead to the rejection of the
null).  Under the null, $y_t$ must be differenced at least once to
achieve stationarity; under the alternative, $y_t$ is already
stationary and no differencing is required.

One peculiar aspect of this test is that its limit distribution is
non-standard under the null hypothesis: moreover, the shape of the
distribution, and consequently the critical values for the test,
depends on the form of the $\mu_t$ term.  A full analysis of the
various cases is inappropriate here: \cite{hamilton94} contains an
excellent discussion, but any recent time series textbook covers
this topic. Suffice it to say that gretl allows the user to
choose the specification for $\mu_t$ among four different
alternatives:

\begin{center}
  \begin{tabular}{cc}
    \hline
    $\mu_t$ & command option \\
    \hline
    0 & \option{nc} \\
    $\mu_0$ &  \option{c} \\
    $\mu_0 + \mu_1 t$ &  \option{ct} \\
    $\mu_0 + \mu_1 t + \mu_1 t^2$ &  \option{ctt} \\
    \hline
  \end{tabular}
\end{center}

These option flags are not mutually exclusive; when they are used
together the statistic will be reported separately for each selected
case.  By default, gretl uses the combination \verb|--c --ct|.
For each case, approximate p-values are calculated by means of the
algorithm developed in \cite{mackinnon96}.

The gretl command used to perform the test is \cmd{adf}; for example
\begin{code}
adf 4 x1
\end{code}
would compute the test statistic as the $t$-statistic for $\varphi$ in
equation \ref{eq:ADFtest} with $p=4$ in the two cases $\mu_t = \mu_0$
and $\mu_t = \mu_0 + \mu_1 t$.

The number of lags ($p$ in equation \ref{eq:ADFtest}) should be chosen
as to ensure that (\ref{eq:ADFtest}) is a parametrization flexible
enough to represent adequately the short-run persistence of $\Delta
y_t$. Setting $p$ too low results in size distortions in the test,
whereas setting $p$ too high leads to low power. As a convenience
to the user, the parameter $p$ can be automatically determined.
Setting $p$ to a negative number triggers a sequential procedure that
starts with $p$ lags and decrements $p$ until the $t$-statistic for
the parameter $\gamma_p$ exceeds 1.645 in absolute value.

\subsection{The ADF-GLS test}
\label{sec:ADF-GLS}

\citet*{ERS96} proposed a variant of the ADF test which involves an
alternative method of handling the parameters pertaining to the
deterministic term $\mu_t$: these are estimated first via Generalized
Least Squares, and in a second stage an ADF regression is performed
using the GLS residuals. This variant offers greater power than the
regular ADF test for the cases $\mu_t = \mu_0$ and $\mu_t = \mu_0 +
\mu_1 t$. 

The ADF-GLS test is available in gretl via the \verb|--gls|
option to the \texttt{adf} command. When this option is selected the
\verb|--nc| and \verb|--ctt| options become unavailable, and only one
case can be selected at a time; by default the constant-only model is
used but a trend can be added using the \verb|--ct| flag.  When a
trend is present in this test MacKinnon-type p-values are not
available; instead we show critical values from Table 1 in
\cite{ERS96}.


\subsection{The KPSS test}
\label{sec:KPSStest}

The KPSS test \citep*{KPSS92} is a unit root test in which the null
hypothesis is opposite to that in the ADF test: under the null, the
series in question is stationary; the alternative is that the series
is $I(1)$.

The basic intuition behind this test statistic is very simple: if
$y_t$ can be written as $y_t = \mu + u_t$, where $u_t$ is some
zero-mean stationary process, then not only does the sample average of
the $y_t$s provide a consistent estimator of $\mu$, but the long-run
variance of $u_t$ is a well-defined, finite number. Neither of these
properties hold under the alternative.

The test itself is based on the following statistic:
\begin{equation}
  \label{eq:KPSStest}
  \eta = \frac{\sum_{i=1}^T S_t^2 }{ T^2 \bar{\sigma}^2 }
\end{equation}
where $S_t = \sum_{s=1}^t e_s$ and $\bar{\sigma}^2$ is an estimate of
the long-run variance of $e_t = (y_t - \bar{y})$. Under the null, this
statistic has a well-defined (nonstandard) asymptotic distribution,
which is free of nuisance parameters and has been tabulated by
simulation. Under the alternative, the statistic diverges.

As a consequence, it is possible to construct a one-sided test based
on $\eta$, where $H_0$ is rejected if $\eta$ is bigger than the
appropriate critical value; gretl provides the 90, 95 and 99
percent quantiles. The critical values are computed via the method
presented by \cite{sephton95}, which offers greater accuracy than the
values tabulated in \cite{KPSS92}.

Usage example:
\begin{code}
kpss m y
\end{code}
where \verb|m| is an integer representing the bandwidth or window
size used in the formula for estimating the long run variance:
\[
  \bar{\sigma}^2 = \sum_{i=-m}^m \left( 1 - \frac{|i|}{m+1} \right) \hat{\gamma}_i
\]
The $\hat{\gamma}_i$ terms denote the empirical autocovariances of
$e_t$ from order $-m$ through $m$.  For this estimator to be
consistent, $m$ must be large enough to accommodate the short-run
persistence of $e_t$, but not too large compared to the sample size
$T$.  If the supplied \verb|m| is non-positive a default value is
computed, namely the integer part of $4 \left( \frac{T}{100}
\right)^{1/4}$.

The above concept can be generalized to the case where $y_t$ is
thought to be stationary around a deterministic trend. In this case,
formula (\ref{eq:KPSStest}) remains unchanged, but the series $e_t$ is
defined as the residuals from an OLS regression of $y_t$ on a constant
and a linear trend. This second form of the test is obtained by
appending the \option{trend} option to the \cmd{kpss} command:
\begin{code}
kpss n y --trend
\end{code}
Note that in this case the asymptotic distribution of the test is
different and the critical values reported by gretl differ
accordingly.


\subsection{Panel unit root tests}
\label{sec:panel-uroot}

The most commonly used unit root tests for panel data involve a
generalization of the ADF procedure, in which the joint null
hypothesis is that a given times series is non-stationary for all
individuals in the panel.

In this context the ADF regression (\ref{eq:ADFtest}) can be rewritten
as

\begin{equation}
  \label{eq:panelADF}
  \Delta y_{it} = \mu_{it} + \varphi_i y_{i,t-1} + \sum_{j=1}^{p_i} \gamma_{ij} \Delta
  y_{i,t-j} + \epsilon_{it}
\end{equation}

The model (\ref{eq:panelADF}) allows for maximal heterogeneity across
the individuals in the panel: the parameters of the deterministic
term, the autoregressive coefficient $\varphi$, and the lag order $p$
are all specific to the individual, indexed by $i$.

One possible modification of this model is to impose the assumption
that $\varphi_i = \varphi$ for all $i$; that is, the individual time
series share a common autoregressive root (although they may differ in
respect of other statistical properties). The choice of whether or not
to impose this assumption has an important bearing on the hypotheses
under test. Under model (\ref{eq:panelADF}) the joint null is
$\varphi_i = 0$ for all $i$, meaning that all the individual time
series are non-stationary, and the alternative (simply the negation of
the null) is that \emph{at least one} individual time series is
stationary.  When a common $\varphi$ is assumed, the null is that
$\varphi = 0$ and the alternative is that $\varphi < 0$. The null
still says that all the individual series are non-stationary, but the
alternative now says that they are \emph{all} stationary.  The choice
of model should take this point into account, as well as the gain in
power from forming a pooled estimate of $\varphi$ and, of course, the
plausibility of assuming a common AR(1) coefficient.\footnote{If the
  assumption of a common $\varphi$ seems excessively restrictive, bear
  in mind that we routinely assume common slope coefficients when
  estimating panel models, even if this is unlikely to be literally
  true.}

In gretl, the formulation (\ref{eq:panelADF}) is used
automatically when the \texttt{adf} command is used on panel data. The
joint test statistic is formed using the method of \citet*{IPS03}. In
this context the behavior of \texttt{adf} differs from regular
time-series data: only one case of the deterministic term is handled
per invocation of the command; the default is that $\mu_{it}$ includes
just a constant but the \verb|--nc| and \verb|--ct| flags can be used
to suppress the constant or to include a trend, respectively; and the
quadratic trend option \verb|--ctt| is not available.

The alternative that imposes a common value of $\varphi$ is
implemented via the \texttt{levinlin} command. The test statistic is
computed as per \citet*{LLC2002}.  As with the \texttt{adf} command, the
first argument is the lag order and the second is the name of the
series to test; and the default case for the deterministic component
is a constant only. The options \verb|--nc| and \verb|--ct| have the
same effect as with \texttt{adf}. One refinement is that the lag order
may be given in either of two forms: if a scalar is given, this is
taken to represent a common value of $p$ for all individuals, but you
may instead provide a vector holding a set of $p_i$ values, hence
allowing the order of autocorrelation of the series to differ by
individual. So, for example, given
\begin{code}
levinlin 2 y
levinlin {2,2,3,3,4,4} y
\end{code}
the first command runs a joint ADF test with a common lag order of 2,
while the second (which assumes a panel with six individuals) allows
for differing short-run dynamics. The first argument to
\texttt{levinlin} can be given as a set of comma-separated integers
enclosed in braces, as shown above, or as the name of an appropriately
dimensioned pre-defined matrix (see chapter~\ref{chap:matrices}).

Besides variants of the ADF test, the KPSS test also can be used with
panel data via the \texttt{kpss} command. In this case the test (of
the null hypothesis that the given time series is \emph{stationary}
for all individuals) is implemented using the method of
\cite{choi01}. This is an application of \emph{meta-analysis}, the
statistical technique whereby an overall or composite p-value for the
test of a given null hypothesis can be computed from the p-values of a
set of separate tests. Unfortunately, in the case of the KPSS test we
are limited by the unavailability of precise p-values, although if an
individual test statistic falls between the 10 percent and 1 percent
critical values we are able to interpolate with a fair degree of
confidence. This gives rise to four cases.

\begin{enumerate}
\item All the individual KPSS test statistics fall between the 10
  percent and 1 percent critical values: the Choi method gives us
  a plausible composite p-value.
\item Some of the KPSS test statistics exceed the 1 percent value and
  none fall short of the 10 percent value: we can give an upper bound
  for the composite p-value by setting the unknown p-values to 0.01.
\item Some of the KPSS test statistics fall short of the 10 percent
  critical value but none exceed the 1 percent value: we can give a
  lower bound to the composite p-value by setting the unknown p-values
  to 0.10.
\item None of the above conditions are satisfied: the Choi method
  fails to produce any result for the composite KPSS test.
\end{enumerate}


\section{Cointegration test}
\label{sec:coint-test}

The generally recommended test for cointegration is the Johansen test,
which is discussed in detail in chapter~\ref{chap:vecm}. In this
context we just offer a few remarks on the cointegration test of
\cite{engle-granger87}, because it builds on the univariate ADF test 
discussed above (section~\ref{sec:ADFtest}).

For the Engle--Granger test, the procedure is:
\begin{enumerate}
\item Test each series for a unit root using an ADF test.
\item Run a ``cointegrating regression'' via OLS. For this we select
  one of the potentially cointegrated variables as dependent, and
  include the other potentially cointegrated variables as regressors.
\item Perform an ADF test on the residuals from the cointegrating
  regression.
\end{enumerate}

The idea is that cointegration is supported if (a) the null of
non-stationarity is \emph{not} rejected for each of the series
individually, in step 1, while (b) the null \emph{is} rejected for the
residuals at step 3. That is, each of the individual series is $I(1)$
but some linear combination of the series is $I(0)$.  

This test is implemented in gretl by the \texttt{coint} command,
which requires an integer lag order (for the ADF tests) followed by a
list of variables to be tested, the first of which will be taken as
dependent in the cointegrating regression. Please see the online help
for \texttt{coint}, or the \textit{Gretl Command Reference}, for
further details.


\section{ARCH and GARCH}
\label{sec:arch-garch}

Heteroskedasticity means a non-constant variance of the error term in
a regression model.  Autoregressive Conditional Heteroskedasticity
(ARCH) is a phenomenon specific to time series models, whereby the
variance of the error displays autoregressive behavior; for instance,
the time series exhibits successive periods where the error variance
is relatively large, and successive periods where it is relatively
small.  This sort of behavior is reckoned to be common in asset
markets: an unsettling piece of news can lead to a period of increased
volatility in the market.

An ARCH error process of order $q$ can be represented as
\[
u_t = \sigma_t \varepsilon_t; \qquad
\sigma^2_t \equiv {\rm E}(u^2_t|\Omega_{t-1}) = 
\alpha_0 + \sum_{i=1}^q \alpha_i u^2_{t-i}
\]
where the $\varepsilon_t$s are independently and identically
distributed (iid) with mean zero and variance 1, and where $\sigma_t$
is taken to be the positive square root of $\sigma^2_t$.
$\Omega_{t-1}$ denotes the information set as of time $t-1$ and
$\sigma^2_t$ is the conditional variance: that is, the
variance conditional on information dated $t-1$ and earlier.

It is important to notice the difference between ARCH and an ordinary
autoregressive error process.  The simplest (first-order) case of the
latter can be written as
\[
u_t = \rho u_{t-1} + \varepsilon_t; \qquad -1 < \rho < 1
\]
where the $\varepsilon_t$s are independently and identically
distributed with mean zero and variance $\sigma^2$.  With an AR(1)
error, if $\rho$ is positive then a positive value of $u_t$ will tend
to be followed by a positive $u_{t+1}$.  With an ARCH error process, a
disturbance $u_t$ of large absolute value will tend to be followed by
further large absolute values, but with no presumption that the
successive values will be of the same sign.  ARCH in asset prices is a
``stylized fact'' and is consistent with market efficiency; on the
other hand autoregressive behavior of asset prices would violate
market efficiency.

One can test for ARCH of order $q$ in the following
way:
\begin{enumerate}
\item Estimate the model of interest via OLS and save the squared
  residuals, $\hat{u}^2_t$.
\item Perform an auxiliary regression in which the current squared
  residual is regressed on a constant and $q$ lags of itself.
\item Find the $TR^2$ value (sample size times unadjusted $R^2$) for
  the auxiliary regression.
\item Refer the $TR^2$ value to the $\chi^2$ distribution with $q$
  degrees of freedom, and if the p-value is ``small enough'' reject
  the null hypothesis of homoskedasticity in favor of the alternative
  of ARCH($q$).
\end{enumerate}

This test is implemented in gretl via the \cmd{modtest} command
with the \option{arch} option, which must follow estimation of a
time-series model by OLS (either a single-equation model or a
VAR). For example,
%
\begin{code}
ols y 0 x
modtest 4 --arch
\end{code}

This example specifies an ARCH order of $q=4$; if the order argument
is omitted, $q$ is set equal to the periodicity of the data.  In the
graphical interface, the ARCH test is accessible from the ``Tests''
menu in the model window (again, for single-equation OLS or VARs).

\subsection{GARCH}
\label{subsec:garch}

The simple ARCH($q$) process is useful for introducing the general
concept of conditional heteroskedasticity in time series, but it has
been found to be insufficient in empirical work.  The dynamics of the
error variance permitted by ARCH($q$) are not rich enough to represent 
the patterns found in financial data.  The generalized ARCH or GARCH
model is now more widely used.  

The representation of the variance of a process in the GARCH model is
somewhat (but not exactly) analogous to the ARMA representation of the
level of a time series.  The variance at time $t$ is allowed
to depend on both past values of the variance and past values of the
realized squared disturbance, as shown in the following system
of equations:
\begin{eqnarray}
  \label{eq:garch-meaneq}
  y_t &  = & X_t \beta + u_t \\
  \label{eq:garch-epseq}
  u_t &  = & \sigma_t \varepsilon_t \\
  \label{eq:garch-vareq}
  \sigma^2_t & = & \alpha_0 + \sum_{i=1}^q \alpha_i u^2_{t-i} +
	  \sum_{j=1}^p \delta_j \sigma^2_{t-j}
\end{eqnarray}
As above, $\varepsilon_t$ is an iid sequence with unit variance.
$X_t$ is a matrix of regressors (or in the simplest case,
just a vector of 1s allowing for a non-zero mean of $y_t$).  Note that
if $p=0$, GARCH collapses to ARCH($q$): the generalization is embodied
in the $\delta_j$ terms that multiply previous values of the error
variance.

In principle the underlying innovation, $\varepsilon_t$, could follow
any suitable probability distribution, and besides the obvious
candidate of the normal or Gaussian distribution the Student's $t$
distribution has been used in this context.  Currently gretl
only handles the case where $\varepsilon_t$ is assumed to be Gaussian.
However, when the \option{robust} option to the \cmd{garch} command is
given, the estimator gretl uses for the covariance matrix can be
considered Quasi-Maximum Likelihood even with non-normal disturbances.
See below for more on the options regarding the GARCH covariance
matrix.

Example:
\begin{code}
garch p q ; y const x
\end{code}
where \verb|p| $\ge 0$ and \verb|q| $>0$ denote the respective lag
orders as shown in equation (\ref{eq:garch-vareq}).  These values
can be supplied in numerical form or as the names of pre-defined
scalar variables.

\subsection{GARCH estimation}
\label{subsec:garch-est}

Estimation of the parameters of a GARCH model is by no means a
straightforward task.  (Consider equation~\ref{eq:garch-vareq}: the
conditional variance at any point in time, $\sigma^2_t$, depends on
the conditional variance in earlier periods, but $\sigma^2_t$ is not
observed, and must be inferred by some sort of Maximum Likelihood
procedure.)  By default gretl uses native code that employs the
BFGS maximizer; you also have the option (activated by the
\verb|--fcp| command-line switch) of using the method proposed by
\cite{fiorentini96},\footnote{The algorithm is based on Fortran code
  deposited in the archive of the \textit{Journal of Applied
    Econometrics} by the authors, and is used by kind permission of
  Professor Fiorentini.} which was adopted as a benchmark in the study
of GARCH results by \cite{mccullough98}.  It employs analytical first
and second derivatives of the log-likelihood, and uses a
mixed-gradient algorithm, exploiting the information matrix in the
early iterations and then switching to the Hessian in the neighborhood
of the maximum likelihood.  (This progress can be observed if you
append the \option{verbose} option to gretl's \cmd{garch}
command.)

Several options are available for computing the covariance matrix of
the parameter estimates in connection with the \cmd{garch} command.
At a first level, one can choose between a ``standard'' and a
``robust'' estimator.  By default, the Hessian is used unless the
\option{robust} option is given, in which case the QML estimator is
used.  A finer choice is available via the \cmd{set} command, as
shown in Table~\ref{tab:garch-vcv}.

\begin{table}[htbp]
\caption{Options for the GARCH covariance matrix}
\label{tab:garch-vcv}
\begin{center}
\begin{tabular}{ll}
\multicolumn{1}{c}{\textit{command}} &
\multicolumn{1}{c}{\textit{effect}} \\ [4pt]
\texttt{set garch\_vcv hessian} & Use the Hessian \\
\texttt{set garch\_vcv im} & Use the Information Matrix \\
\texttt{set garch\_vcv op} & Use the Outer Product of the Gradient \\
\texttt{set garch\_vcv qml} & QML estimator \\
\texttt{set garch\_vcv bw} & Bollerslev--Wooldridge ``sandwich'' estimator
\end{tabular}
\end{center}
\end{table}

It is not uncommon, when one estimates a GARCH model for an arbitrary
time series, to find that the iterative calculation of the estimates
fails to converge.  For the GARCH model to make sense, there are
strong restrictions on the admissible parameter values, and it is not
always the case that there exists a set of values inside the
admissible parameter space for which the likelihood is maximized.  

The restrictions in question can be explained by reference to the
simplest (and much the most common) instance of the GARCH model, where
$p = q = 1$.  In the GARCH(1, 1) model the conditional variance is
\begin{equation}
\label{eq:condvar}
\sigma^2_t = \alpha_0 + \alpha_1 u^2_{t-1} + \delta_1 \sigma^2_{t-1}
\end{equation}
Taking the unconditional expectation of (\ref{eq:condvar}) we get
\[
\sigma^2 = \alpha_0 + \alpha_1 \sigma^2 + \delta_1 \sigma^2
\]
so that
\[
\sigma^2 = \frac{\alpha_0}{1 - \alpha_1 - \delta_1}
\]
For this unconditional variance to exist, we require that $\alpha_1 +
\delta_1 < 1$, and for it to be positive we require that $\alpha_0 > 0$.

A common reason for non-convergence of GARCH estimates (that is, a
common reason for the non-existence of $\alpha_i$ and $\delta_i$ values
that satisfy the above requirements and at the same time maximize the
likelihood of the data) is misspecification of the model.  It is
important to realize that GARCH, in itself, allows \textit{only} for
time-varying volatility in the data.  If the \textit{mean} of the
series in question is not constant, or if the error process is not
only heteroskedastic but also autoregressive, it is necessary to take
this into account when formulating an appropriate model.  For example,
it may be necessary to take the first difference of the variable in
question and/or to add suitable regressors, $X_t$, as in
(\ref{eq:garch-meaneq}).

%%% Local Variables: 
%%% mode: latex
%%% TeX-master: "gretl-guide"
%%% End: 

