\chapter{MIDAS models}
\label{chap:MIDAS}

The acronym MIDAS stands for ``Mixed Data Sampling''. MIDAS models can
essentially be described as models where one or more independent
variables are observed at a higher frequency than the dependent
variable, and possibly an ad-hoc parsimonious parameterization is
adopted. See \citealp{ghysels04}; \citealp{ghysels15};
\citealp{armesto10} for a fuller introduction. Naturally, these models
require easy handling of multiple-frequency data. The way this is done
in gretl is explained in Chapter \ref{chap:mixed-frequency}; in
this chapter, we concentrate on the numerical aspects of estimation.

\section{Parsimonious parameterizations}
\label{sec:hparams}

The simplest MIDAS regression specification---known as ``unrestricted
MIDAS'' or U-MIDAS---simply includes $p$ lags of a high-frequency
regressor, each with its own parameter to be estimated (which can be
done via OLS). A typical case can be written as
\begin{equation}
  \label{eq:UMIDAS}
  y_t = \beta_0 + \alpha y_{t-1} + \sum_{i=0}^k \delta_i x_{\tau-i} + \varepsilon_t, 
\end{equation}
where $\tau$ represents ``high-frequency time''. Obvious
generalizations of this specification include a higher AR order for
$y$ and inclusion of additional high-frequency regressors.

Estimation of \eqref{eq:UMIDAS} can easily be accomplished via
OLS. However, it is more common to enforce parsimony by making the
individual coefficients on lagged high-frequency terms a function of a
relatively small number of hyperparameters, as in
\begin{equation}
  \label{eq:MIDAS}
  y_t = \beta_0 + \alpha y_{t-1} + \gamma W(x_{\tau}, x_{\tau-1},\dots,
  x_{\tau-k}; \theta) + \varepsilon_t
\end{equation}
where $W(\cdot)$ is a weighting function associated with a given MIDAS
specification, $\theta$ is a vector of hyperparameters.

This presents a couple of computational questions: how to calculate
the per-lag coefficients given the values of the hyperparameters, and
how best to estimate the value of the hyperparameters?  Hansl is
functional enough to allow a savvy user to address these questions
from scratch, but of course it's helpful to have some built-in
high-level functionality. At present gretl can handle natively four
commonly used parameterizations: normalized exponential Almon,
normalized beta (with or without a zero last coefficient), and plain
(non-normalized) Almon polynomial. The Almon variants take one or more
parameters (two being a common choice). The beta variants take either
two or three parameters. Full details on the forms taken by the
$W(\cdot)$ function are provided in section \ref{sec:midas-param}.

All variants are handled by the functions \texttt{mweights} and
\texttt{mgradient}. These functions work as follows.
\begin{itemize}
\item \texttt{mweights} takes three arguments: the number of lags
  required ($p$), the $k$-vector of hyperparameters ($\theta$), and an
  integer code or string indicating the method (see
  Table~\ref{tab:midas-parm}). It returns a $p$-vector containing the
  coefficients.
\item \texttt{mgradient} takes three arguments, just like
  \texttt{mweights}. However, this function returns a $p \times k$
  matrix holding the (analytical) gradient of the $p$ coefficients or
  weights with respect to the $k$ elements of $\theta$.
\end{itemize}

\begin{table}[htbp]
  \centering
  \begin{tabular}{lcl}
    \textit{Parameterization} & 
      \textit{code} & \textit{string} \\[4pt]
    Normalized exponential Almon & 1 & \verb|"nealmon"| \\
    Normalized beta, zero last lag & 2 & \verb|"beta0"| \\
    Normalized beta, non-zero last lag & 3 & \verb|"betan"| \\
    Almon polynomial & 4 & \verb|"almonp"| \\
    One-parameter beta & 5 & \verb|"beta1"|
  \end{tabular}
  \caption{MIDAS parameterizations}
  \label{tab:midas-parm}
\end{table}
In the case of the non-normalized Almon polynomial the $\gamma$
coefficient in (\ref{eq:MIDAS}) is identically 1.0 and is omitted.
The \verb|"beta1"| case is the the same as the two-parameter
\verb|"beta0"| except that $\theta_1$ is constrained to equal 1,
leaving $\theta_2$ as the only free parameter. \cite{ghysels-qian16}
make a case for use of this particularly parsimonious
version.\footnote{Note, however, that at present \verb|"beta1"| cannot
  be mixed with other parameterizations in a single model.}

An additional function is provided for convenience: it is named
\texttt{mlincomb} and it combines \texttt{mweights} with the
long-standing \texttt{lincomb} function, which takes a list (of
series) argument followed by a vector of coefficients and produces a
series result, namely a linear combination of the elements of the
list. If we have a suitable list \texttt{X} available, we can
do, for example,
\begin{code}
series foo = mlincomb(X, theta, "beta0")
\end{code}
This is equivalent to
\begin{code}
series foo = lincomb(X, mweights(nelem(X), theta, "beta0"))
\end{code}
but saves a little typing and some CPU cycles.

\section{Estimating MIDAS models}
\label{sec:estimation}

Gretl offers a dedicated command, \texttt{midasreg}, for estimation of
MIDAS models. (There's a corresponding item, \textsf{MIDAS}, under the
\textsf{Time series} section of the \textsf{Model} menu in the gretl
GUI.) We begin by discussing that, then move on to possibilities for
defining your own estimator.

The syntax of \texttt{midasreg} looks like this:

\texttt{midasreg \textsl{depvar} \textsl{xlist} ;
\textsl{midas-terms} [ \textsl{options} ]}

The \texttt{\textsl{depvar}} slot takes the name (or series ID number)
of the dependent variable, and \texttt{\textsl{xlist}} is the list of
regressors that are observed at the same frequency as the dependent
variable; this list may contain lags of the dependent variable. The
\texttt{\textsl{midas-terms}} slot accepts one or more specification(s)
for high-frequency terms. Each of these specifications must conform to
one or other of the following patterns:

\begin{tabular}{ll}
1 & \texttt{mds(\textsl{mlist}, \textsl{minlag}, 
   \textsl{maxlag}, \textsl{type}, \textsl{theta})} \\
2 & \texttt{mdsl(\textsl{llist}, \textsl{type}, \textsl{theta})}
\end{tabular}

In case 1 \texttt{\textsl{mlist}} must be a \textbf{MIDAS list}, as
defined above, which contains a full set of per-period series (but no
lags). Lags will be generated automatically, governed by the
\texttt{\textsl{minlag}} and \texttt{\textsl{maxlag}} (integer)
arguments, which may be given as numerical values or the names of
predefined scalar variables. The integer (or string)
\texttt{\textsl{type}} argument represents the type of
parameterization; in addition to the values 1 to 4 defined in
Table~\ref{tab:midas-parm} a value of 0 (or the string
\verb|"umidas"|) indicates unrestricted MIDAS.

In case 2 \texttt{\textsl{llist}} is assumed to be a list that
already contains the required set of high-frequency lags---as may be
obtained via the \texttt{hflags} function described in
section~\ref{sec:hflags}---hence \texttt{\textsl{minlag}} and
\texttt{\textsl{maxlag}} are not wanted.

The final \texttt{\textsl{theta}} argument is optional in most cases
(implying an automatic initialization of the hyperparameters). If this
argument is given it must take one of the following forms:
\begin{enumerate}
\item The name of a matrix (vector) holding initial values for the
  hyperparameters, or a simple expression which defines a matrix
  using scalars, such as \texttt{\{1, 5\}}.
\item The keyword \texttt{null}, indicating that an automatic
  initialization should be used (as happens when this argument is
  omitted).
\item An integer value (in numerical form), indicating how many
  hyperparameters should be used (which again calls for
  automatic initialization).
\end{enumerate}
The third of these forms is required if you want automatic
initialization in the Almon polynomial case, since we need to know how
many terms you wish to include. (In the normalized exponential Almon
case we default to the usual two hyperparameters if
\texttt{\textsl{theta}} is omitted or given as \texttt{null}.)

The \texttt{midasreg} syntax allows the user to specify multiple
high-frequency predictors, if wanted: these can have different lag
specifications, different parameterizations and/or different
frequencies.

The options accepted by \texttt{midasreg} include \option{quiet}
(suppress printed output), \option{verbose} (show detail of
iterations, if applicable) and \option{robust} (use a HAC estimator of
the Newey--West type in computing standard errors).  Two additional
specialized options are described below.

\subsection{Examples of usage}

Suppose we have a dependent variable named \texttt{dy} and a MIDAS
list named \texttt{dX}, and we wish to run a MIDAS regression using
one lag of the dependent variable and high-frequency lags 1 to 10 of
the series in \texttt{dX}. The following will produce U-MIDAS
estimates:
%
\begin{code}
midasreg dy const dy(-1) ; mds(dX, 1, 10, 0)
\end{code}
%
The next lines will produce estimates for the normalized exponential
Almon parameterization with two coefficients, both initialized to
zero:
%
\begin{code}
midasreg dy const dy(-1) ; mds(dX, 1, 10, "nealmon", {0,0})
\end{code}
%
In the examples above, the required lags will be added to the dataset
automatically then deleted after use. If you are estimating several
models using a single set of MIDAS lags it is more efficient to create
the lags once and use the \texttt{mdsl} specifier.  For example, the
following estimates three variant parameterizations (exponential
Almon, beta with zero last lag, and beta with non-zero last lag) on
the same data:
\begin{code}
list dXL = hflags(1, 10, dX)
midasreg dy 0 dy(-1) ; mdsl(dXL, "nealmon", {0,0})
midasreg dy 0 dy(-1) ; mdsl(dXL, "beta0", {1,5})
midasreg dy 0 dy(-1) ; mdsl(dXL, "betan", {1,1,0})
\end{code}

Any additional MIDAS terms should be separated by spaces, as in
\begin{code}
midasreg dy const dy(-1) ; mds(dX,1,9,1,theta1) mds(Z,1,6,3,theta2)
\end{code}

\subsection{Replication exercise}

We give a substantive illustration of \texttt{midasreg} in
Listing~\ref{lst:midasreg-ghysels}. This replicates the first
practical example discussed by Ghysels in the user's guide titled
\textit{MIDAS Matlab Toolbox},\footnote{See \cite{ghysels15}. This
  document announces itself as Version 2.0 of the guide and is dated
  November 1, 2015. The example we're looking at appears on pages
  24--26; the associated \textsf{Matlab} code can be found in the
  program \texttt{appADLMIDAS1.m}.}
The dependent variable is the quarterly log-difference of
real GDP, named \texttt{dy} in our script. The independent variables
are the first lag of \texttt{dy} and monthly lags 3 to 11 of the
monthly log-difference of non-farm payroll employment (named
\texttt{dXL} in our script). Therefore, in this case equation
\eqref{eq:MIDAS} becomes
\[
  y_t = \alpha + \beta y_{t-1} + \gamma W(x_{\tau-3}, x_{\tau-4},\dots,
  x_{\tau-11}; \theta) + \varepsilon_t
\]
and in the U-MIDAS case the model comes down to
\[
  y_t = \alpha + \beta y_{t-1} + \sum_{i=1}^9 \delta_i x_{\tau-i-2}
  + \varepsilon_t
\]

The script exercises all five of the parameterizations mentioned
above,\footnote{The \textsf{Matlab} program includes an additional
  parameterization not supported by gretl, namely a
  step-function.}  and in each case the results of 9
pseudo-out-of-sample forecasts are recorded so that their Root Mean
Square Errors can be compared.

\begin{script}[p]
  \caption{Script to replicate results given by Ghysels}
  \label{lst:midasreg-ghysels}
\begin{scode}
set verbose off
open gdp_midas.gdt --quiet

# form the dependent variable
series dy = 100 * ldiff(qgdp)
# form list of high-frequency lagged log differences
list X = payems*
list dXL = hflags(3, 11, hfldiff(X, 100))
# initialize matrix to collect forecasts
matrix FC = {}

# estimation sample
smpl 1985:1 2009:1

print "=== unrestricted MIDAS (umidas) ==="
midasreg dy 0 dy(-1) ; mdsl(dXL, 0)
fcast --out-of-sample --static --quiet
FC ~= $fcast

print "=== normalized beta with zero last lag (beta0) ==="
midasreg dy 0 dy(-1) ; mdsl(dXL, 2, {1,5})
fcast --out-of-sample --static --quiet
FC ~= $fcast

print "=== normalized beta, non-zero last lag (betan) ==="
midasreg dy 0 dy(-1) ; mdsl(dXL, 3, {1,1,0})
fcast --out-of-sample --static --quiet
FC ~= $fcast

print "=== normalized exponential Almon (nealmon) ==="
midasreg dy 0 dy(-1) ; mdsl(dXL, 1, {0,0})
fcast --out-of-sample --static --quiet
FC ~= $fcast

print "=== Almon polynomial (almonp) ==="
midasreg dy 0 dy(-1) ; mdsl(dXL, 4, 4)
fcast --out-of-sample --static --quiet
FC ~= $fcast

smpl 2009:2 2011:2
matrix my = {dy}
print "Forecast RMSEs:"
printf "  umidas  %.4f\n", fcstats(my, FC[,1])[2]
printf "  beta0   %.4f\n", fcstats(my, FC[,2])[2]
printf "  betan   %.4f\n", fcstats(my, FC[,3])[2]
printf "  nealmon %.4f\n", fcstats(my, FC[,4])[2]
printf "  almonp  %.4f\n", fcstats(my, FC[,5])[2]
\end{scode}
\end{script}

\begin{script}[p]
  \caption{Replication of Ghysels' results, partial output}
  \label{ghysels-out}
\begin{scode}
=== normalized beta, non-zero last lag (betan) ===
Model 3: MIDAS (NLS), using observations 1985:1-2009:1 (T = 97)
Using L-BFGS-B with conditional OLS
Dependent variable: dy

              estimate    std. error   t-ratio   p-value 
  -------------------------------------------------------
  const       0.748578    0.146404      5.113    1.74e-06 ***
  dy_1        0.248055    0.118903      2.086    0.0398   **

        MIDAS list dXL, high-frequency lags 3 to 11
   
  HF_slope    1.72167     0.582076      2.958    0.0039   ***
  Beta1       0.998501    0.0269479    37.05     1.10e-56 ***
  Beta2       2.95148     2.93404       1.006    0.3171  
  Beta3      -0.0743143   0.0271273    -2.739    0.0074   ***

Sum squared resid    28.78262   S.E. of regression   0.562399
R-squared            0.356376   Adjusted R-squared   0.321012
Log-likelihood      -78.71248   Akaike criterion     169.4250
Schwarz criterion    184.8732   Hannan-Quinn         175.6715

=== Almon polynomial (almonp) ===
Model 5: MIDAS (NLS), using observations 1985:1-2009:1 (T = 97)
Using Levenberg-Marquardt algorithm
Dependent variable: dy

              estimate    std. error   t-ratio   p-value 
  -------------------------------------------------------
  const       0.741403    0.146433      5.063    2.14e-06 ***
  dy_1        0.255099    0.119139      2.141    0.0349   **

        MIDAS list dXL, high-frequency lags 3 to 11

  Almon0      1.06035     1.53491       0.6908   0.4914  
  Almon1      0.193615    1.30812       0.1480   0.8827  
  Almon2     -0.140466    0.299446     -0.4691   0.6401  
  Almon3      0.0116034   0.0198686     0.5840   0.5607  

Sum squared resid    28.66623   S.E. of regression   0.561261
R-squared            0.358979   Adjusted R-squared   0.323758
Log-likelihood      -78.51596   Akaike criterion     169.0319
Schwarz criterion    184.4802   Hannan-Quinn         175.2784

Forecast RMSEs:
  umidas  0.5424
  beta0   0.5650
  betan   0.5210
  nealmon 0.5642
  almonp  0.5329
\end{scode}
\end{script}

The data file used in the replication, \texttt{gdp\_midas.gdt}, was
contructed as described in section~\ref{sec:data-merge} (and as noted
there, it is included in the current gretl package). Part of the
output from the replication script is shown in
Listing~\ref{ghysels-out}. The $\gamma$ coefficient is labeled
\texttt{HF\_slope} in the gretl output.

For reference, output from \textsf{Matlab} (version R2016a for Linux)
is available at
\url{http://gretl.sourceforge.net/midas/matlab_output.txt}. For the
most part (in respect of regression coefficients and auxiliary
statistics such as $R^2$ and forecast RMSEs), gretl's output agrees
with that of \textsf{Matlab} to the extent that one can reasonably
expect on nonlinear problems---that is, to at least 4 significant
digits in all but a few instances.\footnote{Nonlinear results, even
  for a given software package, are subject to slight variation
  depending on the compiler used and the exact versions of supporting
  numerical libraries.}  Standard errors are not quite so close across
the two programs, particularly for the hyperparameters of the beta and
exponential Almon functions. We show these in Table~\ref{tab:stderrs}.

\begin{table}[hbtp]
  \centering
  \begin{tabular}{rcccccc}
 & \multicolumn{2}{c}{2-param beta \qquad} & 
  \multicolumn{2}{c}{3-param beta \qquad} &
  \multicolumn{2}{c}{Exp Almon \qquad} \\[4pt]
 & \textsf{Matlab} & \textsf{gretl} & 
   \textsf{Matlab} & \textsf{gretl} &
   \textsf{Matlab} & \textsf{gretl} \\
const    & 0.135 & 0.140 & 0.143 & 0.146 & 0.135 & 0.140 \\
dy(-1)   & 0.116 & 0.118 & 0.116 & 0.119 & 0.116 & 0.119 \\
HF slope & 0.559 & 0.575 & 0.566 & 0.582 & 0.562 & 0.575 \\
$\theta_1$ & 0.067 & 0.106 & 0.022 & 0.027 & 2.695 & 6.263 \\
$\theta_2$ & 9.662 & 17.140 & 1.884 & 2.934 & 0.586 & 1.655 \\
$\theta_3$ &        &	     & 0.022 & 0.027 \\
\end{tabular}
  \caption{Comparison of standard errors from MIDAS regressions}
  \label{tab:stderrs}
\end{table}

Differences of this order are not unexpected, however, when different
methods are used to calculate the covariance matrix for a nonlinear
regression. The \textsf{Matlab} standard errors are based on a
numerical approximation to the Hessian at convergence, while those
produced by gretl are based on a Gauss--Newton Regression, as
discussed and recommended in \citet[chapter 6]{davidson-mackinnon04}.

\subsection{Underlying methods}

The \texttt{midasreg} command calls one of several possible estimation
methods in the background, depending on the MIDAS specification(s). As
shown in Listing~\ref{ghysels-out}, this is flagged in a line of
output immediately preceding the ``\texttt{Dependent variable}'' line.
If the only specification type is U-MIDAS, the method is
OLS. Otherwise it is one of three variants of Nonlinear Least Squares.
\begin{itemize}
\item Levenberg--Marquardt. This is the back-end for gretl's
  \texttt{nls} command.
\item L-BFGS-B with conditional OLS. L-BFGS is a ``limited memory''
  version of the BFGS optimizer and the trailing ``-B'' means that it
  supports bounds on the parameters, which is useful for reasons given
  below.
\item Golden Section search with conditional OLS. This is a line
  search method, used only when there is a just a single
  hyperparameter to estimate.
\end{itemize}

Levenberg--Marquardt is the default NLS method, but if the MIDAS
specifications include any of the beta variants or normalized
exponential Almon we switch to L-BFGS-B, \textit{unless} the user
gives the \option{levenberg} option. The ability to set bounds on the
hyperparameters via L-BFGS-B is helpful, first because the beta
parameters (other than the third one, if applicable) must be
non-negative but also because one is liable to run into numerical
problems (in calculating the weights and/or gradient) if their values
become too extreme. For example, we have found it useful to place
bounds of $-2$ and $+2$ on the exponential Almon parameters.

Here's what we mean by ``conditional OLS'' in the context of L-BFGS-B
and line search: the search algorithm itself is only responsible for
optimizing the MIDAS hyperparameters, and when the algorithm calls for
calculation of the sum of squared residuals given a certain
hyperparameter vector we optimize the remaining parameters
(coefficients on base-frequency regressors, slopes with respect to
MIDAS terms) via OLS.

\subsection{Testing for a structural break}

The \option{breaktest} option can be used to carry out the Quandt
Likelihood Ratio (QLR) test for a structural break at the stage of
running the final Gauss--Newton regression (to check for convergence
and calculate the covariance matrix of the parameter estimates).  This
can be a useful aid to diagnosis, since non-homogeneity of the data
over the estimation period can lead to numerical problems in nonlinear
estimation, besides compromising the forecasting capacity of the
resulting equation. For example, when this option is given with the
command to estimate the ``\texttt{betan}'' model shown in
Listing~\ref{ghysels-out}, the following result is appended to the
standard output:
%
\begin{code}
QLR test for structural break -
  Null hypothesis: no structural break
  Test statistic: chi-square(6) = 35.1745 at observation 2005:2
  with asymptotic p-value = 0.000127727
\end{code}
%
Despite the strong evidence for a structural break, in this case the
nonlinear estimator appears to converge successfully. But one might
wonder if a shorter estimation period could provide better
out-of-sample forecasts.

\subsection{Defining your own MIDAS estimator}

As explained above, the \texttt{midasreg} command is in effect a
``wrapper'' for various underlying methods. Some users may wish to
undo the wrapping. (This would be required if you wish to introduce
any nonlinearity other than that associated with the stock MIDAS
parameterizations, or to define your own MIDAS parameterization).

Anyone with ambitions in this direction will presumably be quite
familiar with the commands and functions available in hansl, gretl's
scripting language, so we will not say much here beyond presenting a
couple of examples. First we show how the \texttt{nls} command can be
used, along with the MIDAS-related functions described in
section~\ref{sec:hparams}, to estimate a model with the exponential
Almon specification.
%
\begin{code}
open gdp_midas.gdt --quiet
series dy = 100 * ldiff(qgdp)
series dy1 = dy(-1)
list X = payems*
list dXL = hflags(3, 11, hfldiff(X, 100))

smpl 1985:1 2009:1

# initialization via OLS
series mdX = mean(dXL)
ols dy 0 dy1 mdX --quiet
matrix b = $coeff | {0,0}'
scalar p = nelem(dXL)

# convenience matrix for computing gradient
matrix mdXL = {dXL}

# normalized exponential Almon via nls
nls dy = b[1] + b[2]*dy1 + b[3]*mdx
  series mdx = mlincomb(dXL, b[4:], 1)
  matrix grad = mgradient(p, b[4:], 1)
  deriv b = {const, dy1, mdx} ~ (b[3] * mdXL * grad)
  param_names "const dy(-1) HF_slope Almon1 Almon2"
end nls
\end{code}

Listing~\ref{lst:manual-beta1} presents a more ambitious example: we
use \texttt{GSSmin} (Golden Section minimizer) to estimate a MIDAS
model with the ``one-parameter beta'' specification (that is, the
two-parameter beta with $\theta_1$ clamped at 1). Note that while the
function named \texttt{beta1\_SSR} is specialized to the given
parameterization, \texttt{midas\_GNR} is a fairly general means of
calculating the Gauss--Newton regression for an ADL(1) MIDAS
model, and it could be generalized further without much difficulty.

\begin{script}[p]
  \caption{Manual MIDAS: one-parameter beta specification}
  \label{lst:manual-beta1}
\begin{scode}
set verbose off

function scalar beta1_SSR (scalar th2, const series y,
                           const series x, list L)
  matrix theta = {1, th2}
  series mdx = mlincomb(L, theta, 2)
  # run OLS conditional on theta
  ols y 0 x mdx --quiet
  return $ess
end function

function matrix midas_GNR (const matrix theta, const series y,
                           const series x, list L, int type)
  # Gauss-Newton regression
  series mdx = mlincomb(L, theta, type)
  ols y 0 x mdx --quiet
  matrix b = $coeff
  matrix u = {$uhat}
  matrix mgrad = mgradient(nelem(L), theta, type)
  matrix M = {const, x, mdx} ~ (b[3] * {L} * mgrad)
  matrix V
  set svd on # in case of strong collinearity
  mols(u, M, null, &V)
  return (b | theta) ~ sqrt(diag(V))
end function

/* main */

open gdp_midas.gdt --quiet

series dy = 100 * ldiff(qgdp)
series dy1 = dy(-1)
list dX = ld_payem*
list dXL = hflags(3, 11, dX)

# estimation sample
smpl 1985:1 2009:1

matrix b = {0, 1.01, 100}
# use Golden Section minimizer
SSR = GSSmin(b, beta1_SSR(b[1], dy, dy1, dXL), 1.0e-6)
printf "SSR (GSS) = %.15g\n", SSR
matrix theta = {1, b[1]}' # column vector needed
matrix bse = midas_GNR(theta, dy, dy1, dXL, 2)
bse[4,2] = $nan # mask std error of clamped coefficient
modprint bse "const dy(-1) HF_slope Beta1 Beta2"
\end{scode}
\end{script}

\subsection{Plot of coefficients}

At times, it may be useful to plot the ``gross'' coefficients on the
lags of the high-frequency series in a MIDAS regression---that is, the
normalized weights multiplied by the \texttt{HF\_slope} coefficient
(the $\gamma$ in \ref{eq:MIDAS}). After estimation of a MIDAS model in
the gretl GUI this is available via the item \textsf{MIDAS
  coefficients} under the \textsf{Graphs} menu in the model window. It
is also easily generated via script, since the \dollar{model} bundle
that becomes available following the \texttt{midasreg} command
contains a matrix, \texttt{midas\_coeffs}, holding these
coefficients. So the following is sufficient to display the plot:
%
\begin{code}
matrix m = $model.midas_coeffs
plot m
   options with-lp fit=none
   literal set title "MIDAS coefficients"
   literal set ylabel ''
end plot --output=display
\end{code}

Caveat: this feature is at present available only for models with a
single MIDAS specification.

\section{Parameterization functions}
\label{sec:midas-param}

Here we give some more detail of the MIDAS parameterizations supported
by gretl.

\vspace{1ex}

In general the normalized coefficient or weight $i$ ($i=1,\ldots,p$)
is given by
\begin{equation}
\label{eq:MIDASgeneral}
  w_i = \frac{f(i,\theta)}
  {\sum_{k=1}^pf(k,\theta)}
\end{equation}
such that the coefficients sum to unity.

In the \textbf{normalized exponential Almon} case with $m$ parameters
the function $f(\cdot)$ is
\begin{equation}
f(i,\theta) = \exp\left(\sum_{j=1}^m \theta_j i^j\right)
\end{equation}
So in the usual two-parameter case we have
\[
w_i =
  \frac{\exp\left(\theta_1 i + \theta_2 i^2\right)}
  {\sum_{k=1}^p \exp\left(\theta_1 k + \theta_2 k^2\right)}
\]
and equal weighting is obtained when $\theta_1 = \theta_2 = 0$.

\vspace{1ex}

In the standard, two-parameter \textbf{normalized beta} case we have
\begin{equation}
\label{eq:fbeta}
  f(i, \theta) = (i^-/p^-)^{\theta_1-1} \cdot (1-i^-/p^-)^{\theta_2-1}
\end{equation}
where $p^- = p-1$, and $i^- = i-1$ except at the end-points, $i=1$ and
$i=p$, where we add and subtract, respectively, machine epsilon to
avoid numerical problems.  This formulation constrains the coefficient
on the last lag to be zero---provided that the weights are declining
at higher lags, a condition that is ensured if $\theta_2$ is greater
than $\theta_1$ by a sufficient margin. The special case of
$\theta_1 = \theta_2 = 1$ yields equal weights at all lags. A third
parameter can be used to allow a non-zero final weight, even in the
case of declining weights.  Let $w_i$ denote the normalized weight
obtained by using (\ref{eq:fbeta}) in (\ref{eq:MIDASgeneral}). Then the
modified variant with additional parameter $\theta_3$ can be written
as
\[
w^{(3)}_i = \frac{w_i + \theta_3}{1 + p\theta_3}
\]
That is, we add $\theta_3$ to each weight then renormalize so that the
$w^{(3)}_i$ values again sum to unity.

In Eric Ghysels' Matlab code the two beta variants are labeled
``normalized beta density with a zero last lag'' and ``normalized beta
density with a non-zero last lag'' respectively.  Note that while the
two basic beta parameters must be positive, the third additive
parameter may be positive, negative or zero.

\vspace{1ex}

In the case of the plain \textbf{Almon polynomial} of order $m$,
coefficient $i$ is given by
\[
w_i = \sum_{j=1}^m \theta_j i^{j-1}
\]
Note that no normalization is applied in this case, so no additional
coefficient should be placed before the MIDAS lags term in the context
of a regression.

\subsection*{Analytical gradients}

Here we set out the expressions for the analytical gradients produced
by the \texttt{mgradient} function, and also used internally by the
\texttt{midasreg} command. In these expressions $f(i,\theta)$ should
be understood as referring back to the specific forms noted above
for the exponential Almon and beta distributions. The summation
$\sum_k$ should be understood as running from 1 to $p$.

For the normalized exponential Almon case, the gradient is
\begin{align*}
\frac{dw_i}{d\theta_j} &= 
\frac{f(i, \theta) i^j}{\sum_kf(k, \theta)} - 
\frac{f(i, \theta)}{\left[\sum_kf(k, \theta)\right]^2}
\, \sum_k\left[f(k, \theta) k^j\right] \\[4pt]
 &= w_i \left(i^j - 
\frac{\sum_k\left[f(k,\theta)k^j\right]}{\sum_k f(k, \theta)}\right)
\end{align*}

For the two-parameter normalized beta case it is
\begin{align*}
\frac{dw_i}{d\theta_1} &=
\frac{f(i,\theta) \log(i^-/p^-)}{\sum_k f(k, \theta)} -
\frac{f(i,\theta)}{\left[\sum_k f(k,\theta)\right]^2}
\sum_k\left[f(k,\theta) \log(k^-/p^-)\right] \\[4pt]
&= w_i \left(\log(i^-/p^-) - 
\frac{\sum_k\left[f(k,\theta) \log(k^-/p^-)\right]}{\sum_k 
 f(k,\theta)}\right) \\[8pt]
\frac{dw_i}{d\theta_2} &=
\frac{f(i,\theta) \log(1 - i^-/p^-)}{\sum_k f(k, \theta)} -
\frac{f(i,\theta)}{\left[\sum_k f(k,\theta)\right]^2}
\sum_k\left[f(k,\theta) \log(1 - k^-/p^-)\right] \\[4pt]
&= w_i \left(\log(1 - i^-/p^-) - 
\frac{\sum_k\left[f(k,\theta) \log(1 - k^-/p^-)\right]}{\sum_k 
 f(k,\theta)}\right)
\end{align*}

And for the three-parameter beta, we have
\begin{align*}
\frac{dw^{(3)}_i}{d\theta_1} &= 
  \frac{1}{1+p\theta_3} \, \frac{dw_i}{d\theta_1} \\
\frac{dw^{(3)}_i}{d\theta_2} &= 
  \frac{1}{1+p\theta_3} \, \frac{dw_i}{d\theta_2} \\
\frac{dw^{(3)}_i}{d\theta_3} &=
\frac{1}{1+p\theta_3} - \frac{p(w_i + \theta_3)}{(1+p\theta_3)^2}
\end{align*}

For the (non-normalized) Almon polynomial the gradient is simply
\[
\frac{dw_i}{d\theta_j} = i^{j-1}
\]

%%% Local Variables:
%%% mode: latex
%%% TeX-master: "gretl-guide"
%%% End:
