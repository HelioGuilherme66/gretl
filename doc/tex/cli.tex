\chapter{The command line interface}
\label{cli}

\section{Gretl at the console}
\label{cli-console}

The \app{gretl} package includes the command-line program
\app{gretlcli}.  On Linux it can be run from a terminal window (xterm,
rxvt, or similar), or at the text console.  Under MS Windows it can be
run in a console window (sometimes inaccurately called a ``DOS box'').
\app{gretlcli} has its own help file, which may be accessed by typing
``help'' at the prompt. It can be run in batch mode, sending output
directly to a file (see also the \emph{Gretl Command Reference}).
    
If \app{gretlcli} is linked to the \app{readline} library (this is
automatically the case in the MS Windows version; also see
Appendix~\ref{app-technote}), the command line is recallable and
editable, and offers command completion.  You can use the Up and Down
arrow keys to cycle through previously typed commands.  On a given
command line, you can use the arrow keys to move around, in
conjunction with Emacs editing keystokes.\footnote{Actually, the key
  bindings shown below are only the defaults; they can be customized.
  See the
  \href{http://cnswww.cns.cwru.edu/~chet/readline/readline.html}{readline
    manual}.} The most common of these are:
%    
\begin{center}
  \begin{tabular}{cl}
    \textit{Keystroke} & \multicolumn{1}{c}{\textit{Effect}}\\
    \verb+Ctrl-a+ & go to start of line\\
    \verb+Ctrl-e+ & go to end of line\\
    \verb+Ctrl-d+ & delete character to right\\
  \end{tabular}
\end{center}
%
where ``\verb+Ctrl-a+'' means press the ``\verb+a+'' key while the
``\verb+Ctrl+'' key is also depressed.  Thus if you want to change
something at the beginning of a command, you \emph{don't} have to
backspace over the whole line, erasing as you go.  Just hop to the
start and add or delete characters.  If you type the first letters of
a command name then press the Tab key, readline will attempt to
complete the command name for you.  If there's a unique completion it
will be put in place automatically.  If there's more than one
completion, pressing Tab a second time brings up a list.

\section{CLI syntax}
\label{cli-syntax}

Probably the most useful mode for heavy-duty work with \app{gretlcli}
is batch (non-interactive) mode, in which the program reads and
processes a script, and sends the output to file.  For example
\begin{code}
    gretlcli -b scriptfile > outputfile
\end{code}

The \textsl{scriptfile} is treated as a program argument; it should
specify a data file to use internally, using the syntax \cmd{open
  datafile}.  Don't forget the \texttt{-b} (batch) switch, otherwise
the program will wait for user input after executing the script.

    
%%% Local Variables: 
%%% mode: latex
%%% TeX-master: "gretl-guide"
%%% End: 

