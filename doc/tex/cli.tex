\chapter{The command line interface}
\label{cli}



\section{Gretl at the console}
\label{cli-console}


      The \app{gretl} package includes the
      command-line program \app{gretlcli}. On
      Linux it can be run from the console, or in an xterm (or
      similar).  Under MS Windows it can be run in a console window
      (sometimes inaccurately called a ``DOS box'').
      \app{gretlcli} has its own help file, which
      may be accessed by typing ``help'' at the prompt. It
      can be run in batch mode, sending outout directly to a file (see
      alse the \emph{Gretl Command Reference}).
    
      If \app{gretlcli} is linked to the
      \app{readline} library (this is
      automatically the case in the MS Windows version; also see
      [?]), the command line is recallable
      and editable, and offers command completion.  You can use the Up
      and Down arrow keys to cycle through previously typed commands.
      On a given command line, you can use the arrow keys to move
      around, in conjunction with Emacs editing
      keystokes.\footnote{Actually, the key bindings shown below
	  are only the defaults; they can be customized.  See the
	  \href{http://cnswww.cns.cwru.edu/~chet/readline/readline.html}{readline 
	    manual}.} The most common of these are:
    

\begin{center}
\begin{tabular}{ll}
Keystroke & Effect\\
\verb+Ctrl-a+ & go to start of line\\
\verb+Ctrl-e+ & go to end of line\\
\verb+Ctrl-d+ & delete character to right\\
\end{tabular}
\end{center}


      where ``\verb+Ctrl-a+'' means press the
      ``\verb+a+'' key while the
      ``\verb+Ctrl+'' key is also depressed.
      Thus if you want to change something at the beginning of a
      command, you \emph{don't} have to backspace over
      the whole line, erasing as you go.  Just hop to the start and
      add or delete characters.
    If you type the first letters of a command name then press
      the Tab key, readline will attempt to complete the command name
      for you.  If there's a unique completion it will be put in place
      automatically.  If there's more than one completion, pressing
      Tab a second time brings up a list.

\section{Changes from Ramanathan's ESL}
\label{cli-syntax}

\app{gretlcli} inherits its basic
      command syntax from Ramu Ramanathan's
      \app{ESL}, and command scripts developed
      for \app{ESL} should be usable with few or
      no changes: the only things to watch for are multi-line commands
      and the \cmd{freq} command.
    
\begin{itemize}
\item In \app{ESL}, a semicolon is used
	  as a terminator for many commands.  I decided to remove this
	  in \app{gretlcli}. Semicolons are
	  simply ignored, apart from a few special cases where they
	  have a definite meaning: as a separator for two lists in the
	  \cmd{ar} and \cmd{tsls} commands,
	  and as a marker for an unchanged starting or ending
	  observation in the \cmd{smpl} command. In
	  \app{ESL} semicolon termination gives
	  the possibility of breaking long commands over more than one
	  line; in \app{gretlcli} this is done by
	  putting a trailing backslash \verb+\+ at the end
	  of a line that is to be continued.
	
\item With \cmd{freq}, you can't at present
	  specify user-defined ranges as in
	  \app{ESL}.  A chi-square test for
	  normality has been added to the output of this command.
	
\end{itemize}


      Note also that the command-line syntax for running a batch job
      is simplified. For \app{ESL} you typed,
      e.g. 
      
\begin{verbatim} 
        esl -b datafile < inputfile > outputfile
      \end{verbatim}
 while for \app{gretlcli}
      you type: 
      
\begin{verbatim}
	gretlcli -b inputfile > outputfile
      \end{verbatim}
 The inputfile is treated as a program
      argument; it should specify a datafile to use internally, using
      the syntax \cmd{open datafile}  or the special
      comment \verb+(* !+
      datafile \verb+*)+
    
%%% Local Variables: 
%%% mode: latex
%%% TeX-master: "gretl-guide"
%%% End: 

