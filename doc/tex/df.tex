\documentclass{article}
\usepackage{lucidabr}
\begin{document}

\newcommand{\subml}{\textrm{\scriptsize ML}}

\section{Biased and unbiased estimators of variance}

It's well known that given a sample, $\{x_i\}$, of size $n$ from a
normally distributed population, the Maximum Likelihood (ML) estimator
of the population variance, $\sigma^2$, is
%
\begin{equation}
\label{eq:sigma-hat}
\hat{\sigma}_{\subml}^2 = \frac{1}{n} \sum_{i=1}^n (x_i - \bar{x})^2
\end{equation}
%
where $\bar{x}$ is the sample mean, $n^{-1} \sum_{i=1}^n x_i$.
It's also well known that $\hat{\sigma}_{\subml}^2$, while consistent,
is a biased estimator, and is commonly replaced by the ``sample
variance'', namely,
%
\begin{equation}
\label{eq:sample-variance}
s^2 = \frac{1}{n-1} \sum_{i=1}^n (x_i - \bar{x})^2
\end{equation}

The intuition behind the bias in (\ref{eq:sigma-hat}) is
straightforward.  First, the quantity we seek to estimate is defined
as
%
\[
\sigma^2 = \frac{1}{N} \sum_{i=1}^n (x_i - \mu)^2
\]
%
for a population of size $N$ with mean $\mu$.  Second, $\bar{x}$ is an
estimator of $\mu$; it is easily shown to be the least-squares
estimator, and also (assuming normality) the ML estimator.  It is
unbiased, but is of course subject to sampling error; in any given
sample it is highly unlikely that $\bar{x} = \mu$.  Given that
$\bar{x}$ is the least-squares estimator, the sum of squared
deviations of the $x_i$ from \textit{any} value other than $\bar{x}$
must be greater than the summation in (\ref{eq:sigma-hat}).  But since
$\mu$ is almost certainly not equal to $\bar{x}$, the sum of squared
deviations of the $x_i$ from $\mu$ will surely be greater than the sum
of squared deviations in (\ref{eq:sigma-hat}). It follows that the
expected value of $\hat{\sigma}_{\subml}^2$ falls short of the
population variance.

The proof that $s^2$ is indeed the unbiased estimator can be found in
any good statistics textbook, where we also learn that the magnitude
$n-1$ in (\ref{eq:sample-variance}) can be brought under a general
description as the ``degrees of freedom'' of the calculation at
hand. Given $\bar{x}$, the $n$ sample values provide only $n-1$ pieces
of information since the $n$th value can always be deduced via the
formula for $\bar{x}$.

\section{Application to OLS regression}

The argument above carries over into the usual calculation of standard
errors in the context of OLS regression as applied to the linear model
$y = X\beta + u$.  If the disturbances, $u$, are assumed to be
independently and identically distributed (IID), then the variance of
the OLS estimator, $\hat\beta$, is given by
%
\[
\mbox{Var}\left(\hat\beta\right) = \sigma^2 (X'X)^{-1}
\]
%
where $\sigma^2$ is the variance of the error term and $X$ is the
$n\times k$ matrix of regressors.  But how should the unknown
$\sigma^2$ be estimated?  The ML estimator is
%
\begin{equation}
\label{eq:ols-sigma2}
\hat\sigma^2 = \frac{1}{n} \sum_{i=1}^n \hat{u}^2_i
\end{equation}
%
where the $\hat{u}^2_i$ are squared residuals, $u_i = y_i - X_i\beta$.
But this estimator is biased and we typically use the unbiased
counterpart
%
\begin{equation}
\label{eq:ols-s2}
s^2 = \frac{1}{n-k} \sum_{i=1}^n \hat{u}_i^2
\end{equation}
%
in which $n - k$ is the number of degrees of freedom given $n$
residuals from a regression where $k$ parameters are estimated.

The ``classical'' estimator of the variance of $\hat\beta$ in the
context of OLS is then $V = s^2 (X'X)^{-1}$.  And the standard errors
of the individual parameter estimates, $s_{\hat{\beta}_i}$, being the
square roots of the diagonal elements of $V$, inherit a ``degrees of
freedom correction'' or df correction from the estimator $s^2$.

Going one step further, consider hypothesis testing in the context of
OLS.  Since the variance of $\hat\beta$ is unknown and must itself be
estimated, the sampling distribution of the OLS coefficients is not,
strictly speaking, normal.  But if the \textit{disturbances} are
normally distributed (besides being IID) then, even in small samples,
the parameter estimates will follow a distribution that can be
specified exactly, namely the Student's $t$ distribution with degrees
of freedom equal to the value given above, $\nu = n-k$.

That is, besides using a df correction in computing the standard
errors of the OLS coefficients, one uses the same $\nu$ in selecting
the particular distribution to which the ``$t$-ratio'',
$(\hat{\beta}_i-\beta^0)/s_{\hat{\beta}_i}$, should be referred in
order to determine the marginal significance level or $p$-value for
the null hypothesis $H_0: \beta_i = \beta^0$.

So far, so good.  Everyone expects df correction in OLS standard
errors, just as we expect division by $n-1$ in the sample variance.
And users of econometric software expect that the $p$-values
reported for OLS coefficients will be based on the appropriate
$t$ distribution.

\section{Beyond OLS}

The situation is different when we move beyond estimation of
the classical linear model via OLS.  We may wish to estimate nonlinear
models (sometimes by least squares), and many models of interest to
econometricians are commonly estimated via maximization of a
likelihood function, or via the generalized method of moments (GMM).

In such cases we do not, in general, have exact small-sample results
to rely upon; in particular, we cannot assume that coefficient
estimates follow the Student's $t$ distribution.  Rather, we typically
appeal to asymptotic results in statistical theory.  We seek
\textit{consistent} estimators which, although they may be biased,
nonetheless converge in probability to the corresponding parameter
values as the sample size goes to infinity.  Under the right
conditions, laws of large numbers and central limit theorems entitle
us to expect that test statistics will converge to the normal
distribution, or the $\chi^2$ distribution for multivariate tests,
given big enough samples.

\subsection{To ``correct'' or not?}

The question arises, should we or should we not apply a df
``correction'' in reporting variance estimates and standard errors
for models that depart from the classical linear specification?

One argument against applying df adjustment is that it lacks a
theoretical basis in relation to hypothesis testing, in that it does
not produce test statistics that follow any known distribution in
small samples.  In addition, if parameter estimates are obtained via
ML, it makes sense to report ML estimates of variances even if these
are biased, since it is the ML quantities that are used in computing
the criterion function and in forming likelihood-ratio tests of
hypotheses.

On the other hand, pragmatic arguments for doing df adjustment are (a)
that it makes for closer comparability between regular OLS estimates
and nonlinear ones, and (b) that it provides a ``pinch of salt'' in
relation to small-sample results, even if it lacks rigorous
justification.  

Note that even for fairly small samples, the difference between the
biased and unbiased estimators in equations (\ref{eq:sigma-hat}) and
(\ref{eq:sample-variance}) above will be small.  For example, if
$n=30$ then $s^2 = \frac{30}{29} \hat\sigma^2$.  In econometric
modelling proper, however, the difference can be quite substantial.
If $n=50$ and $k=10$, the $s^2$ defined in (\ref{eq:ols-s2}) will be
$50/40 = 1.25$ as large as the $\hat\sigma^2$ in
(\ref{eq:ols-sigma2}), and standard errors will be about 12 percent
larger.  One might therefore make a case for ``inflating'' the
standard errors obtained via nonlinear estimators as a precaution
against taking results to be ``more precise than they really
are''.

The force of the last point depends, of course, on the sophistication
of the investigator.  One might equally say that savvy econometricians
should know to apply a discount factor (albeit an imprecise one) to
small-sample estimates outside of the classical, normal linear model---or
even that they should know to distrust such results and insist on
large samples before making inferences.  From this point
of view one could argue that df adjustment provides a false sense
of security.

\section{Consistency and ``grey'' cases}

Consistency (in the sense of uniformity) is a bugbear when dealing
with this issue.  To give a simple example, suppose an econometrics
program follows the policy of applying df correction for OLS
estimation but not for ML estimation.  One is, of course, free to
estimate the classical, normal linear model via ML, in which case
$\hat\beta$ should be numerically identical to that obtained via OLS.
But the user of the software will obtain two different sets of
standard errors depending on the estimation method.  This example is
not very troublesome, however; presumably one would apply ML to the
classical linear model only to make a pedagogical point.

Here is a more awkward case.  An unrestricted vector autoregression
(VAR) is a system of equations, but the ML estimate of this system,
given normal errors, is equivalent to equation-by-equation OLS.
Should df correction be applied to VARs? Consistency with OLS argues
Yes. But wait---a popular extension of the VAR methodology is the
vector error-correction model (VECM).  VECMs are closely related to
VARs and one might well be interested in making comparisons across the
two, but a VECM is a nonlinear system and the cointegrating vectors
that lie at the heart of this model must be estimated via ML.  So
perhaps VAR results should \textit{not} be df adjusted, for
comparability with VECMs.

Another ``grey area'' is the class of Feasible Generalized Least
Squares (FGLS) estimators---for example, weighted least squares
or AR(1) estimators such as Cochrane--Orcutt.  These depart from
the classical linear model, yet according to econometric tradition 
standard errors are generally df adjusted.  





\end{document}

