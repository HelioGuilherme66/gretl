\documentclass{article}
\usepackage{doc,url,verbatim,fancyvrb}
\usepackage{pifont}
\usepackage[authoryear]{natbib}
\usepackage{gretl}
\usepackage[letterpaper,body={6.3in,9.15in},top=.8in,left=1.1in]{geometry}
\usepackage[pdftex,hyperfootnotes=false]{hyperref}

\usepackage[utf8]{inputenc}
\DeclareUnicodeCharacter{2212}{-}

\bibliographystyle{gretl}

% \usepackage[a4paper,body={6.1in,9.7in},top=.8in,left=1.1in]{geometry}

\begin{document}

\setlength{\parindent}{0pt}
\setlength{\parskip}{1ex}
\setcounter{secnumdepth}{1}

\newenvironment{funcdoc}
{\noindent\hrulefill\\[-10pt]}
{\medskip}

\newcommand{\argname}[1]{\textsl{#1}}
  
\title{gretl + lpsolve}
\author{Allin Cottrell}
\maketitle

\section{Introduction}
\label{sec:intro}

As of the 2021d release gretl offers support for \textsf{lpsolve}, an
open-source library for solving linear programming problems (see
\url{https://sourceforge.net/projects/lpsolve/}).\footnote{For
  technical reasons this is not currently available in the 32-bit
  build of gretl for MS Windows. That may follow if we're able to
  resolve the problems we've encountered.}

While linear programming is not part of the standard econometrics
toolkit it surely falls under the rubric adopted in the founding of
the Econometrics Society in 1930, namely ``to promote studies that aim
at a unification between the theoretical-quantitative and the
empirical-quantitative approach to economic problems.''\footnote{See 
  \url{https://www.econometricsociety.org/society/about}.} And
some well known econometric studies have used linear programs as a
complement to estimation methodology---see for example
\cite{ferrier-lovell90}.

Section~\ref{sec:platforms} gives specifics per platform. Note that on
Linux you have to install the \textsf{lpsolve} package to activate
gretl's support.

\section{Implementation}
\label{sec:implement}

There are two main ways of using \textsf{lpsolve} from within gretl.
\begin{itemize}
\item In the GUI program you can load or compose a linear program in
  the native format of \textsf{lpsolve}, send it for solution, and
  retrieve the results as native gretl objects.
\item Using either the GUI program or the command-line program
  \texttt{gretlcli}, you can formulate a linear program in various
  ways and call the \texttt{lpsolve()} function to obtain the solution
  in the form of a gretl bundle.
\end{itemize}

The following two subsections give details of the two methods.

\subsection{Via the script editor}

\subsection{Via the \texttt{lpsolve} function}

The \texttt{lpsolve()} function has this signature:
\begin{verbatim}
  bundle lpsolve (const bundle specs)
\end{verbatim}

The input bundle, \texttt{specs}, must hold a specification of the
problem to be solved and may hold some optional elements.  The
problem specification itself may take any of these three forms:
\begin{itemize}
\item Under the key \texttt{lp\_filename}, the name of an \texttt{lp}
  file, as described under the heading ``Via the script editor.''
\item Under the key \texttt{lp\_buffer}, a multi-line string in
  \texttt{lp} format. Such a string can be composed using gretl's
  \texttt{sprintf} function and string concatenation. You can think of
  this as an \texttt{lp} file existing in memory rather than on disk.
\item Under the key \texttt{lp\_bundle}, a gretl bundle in which a
  linear program is represented by (primarily) a set of matrices. The
  requirements for such a bundle are set out in
  section~\ref{sec:lp-bundle} below.
\end{itemize}

\section{Top-level options}
\label{sec:gen-opts}

By ``top-level'' we mean that these options should be included
directly in the \texttt{specs} bundle passed to
\texttt{lpsolve()}. They apply regardless of the method by which the
linear program is specified.

\begin{itemize}
\item \texttt{verbose} (boolean, default 0): If this has a non-zero
  value output similar to that from the \texttt{lp\_solve} executable
  is printed. Otherwise \texttt{lpsolve()} runs silently unless the
  solver fails.
\item \texttt{sensitivity} (boolean, default 0): If this has a
  non-zero value additional output is provided as described in
  section~\ref{sec:retval}.
\item \texttt{model\_name} (string): If this is included it is shown
  when verbose output is produced.
\end{itemize}

\section{Contents of the return bundle}
\label{sec:retval}

Let $n$ and $m$ denote, respectively, the number of variables whose
values are to be optimized and the number of linear constraints.

The bundle returned by the \textsf{lpsolve} function includes the
following:
\begin{itemize}
\item \texttt{objective}: a scalar, the optimized value of the objective
  function.
\item \texttt{variable\_values}: an $n$-vector holding the optimized
  values of the variables.
\item \texttt{constraint\_values}: an $m$-vector holding the values of
  the constrained magnitudes at the solution.
\item \texttt{shadow\_prices}: an $m$-vector holding the Lagrange
  multipliers associated with the constraints.
\end{itemize}

In addition, if the \textsf{lpsolve} library version is 5.5.2.11 or
higher a scalar \texttt{accuracy} value is included. This gives the
accuracy with which the constraints are met; it is 0 at an exact
solution, larger values indicate approximation.

If the sensitivity option is activated the bundle contains an
additional $(n+m) \times 3$ matrix under the name
\texttt{sensitivity}. This holds all the dual variables along with the
lower and upper limits of their validity.

\section{Specifying a linear program via bundle}
\label{sec:lp-bundle}

A bundle given as \texttt{lp\_bundle} must have at least three
required, as follows.
\begin{itemize}
\item \texttt{objective}: a row vector of length $n$ holding the
  coefficients on the variables in the linear objective
  function. Zeros must be inserted for variables that do not feature
  in the objective. If column names are attached to
  \texttt{objective} the variables are named accordingly in the
  printed output (if any) and the rows of \texttt{variable\_values}
  in the output bundle are named.
\item \texttt{constraints}: a matrix of dimension $m \times
  (n+1)$. Row $i$ holds the coefficient on each of the variables in
  constraint $i$ (again, with zeros inserted for any variables not
  referenced in the constraint), followed by the right-hand side
  value. If row names are attached to \texttt{constraints}, these
  names are used in the output.
\item \texttt{ctypes}: a specification of the types of the constraints
  (inequalities or equalities). It can take the form of an array of
  $m$ strings, each of which must be ``\texttt{<=}'', ``\texttt{>=}''
  or ``\texttt{=}'', or alternatively an $m$-vector in which these
  types are encoded as 1, 2 or 3, respectively.
\end{itemize}

The following optional elements of \texttt{lp\_bundle} are supported:
\begin{itemize}
\item \texttt{minimize} (boolean, default 0): If this has a non-zero
  value the objective is minimized; by default it is maximized.
\item \texttt{intvars} (matrix): If one or more of the variables can
  take on only integer values, this should be indicated by supplying a
  vector holding the indices of the variables in question.
\end{itemize}

\section{Platform specifics}
\label{sec:platforms}

In the gretl packages for 64-bit MS Windows and macOS, the
\textsf{lpsolve} library (as of this writing, version 5.5.2.11) is
included; no additional installation is required.

On Linux, however, it's easy to install the \textsf{lpsolve} library
via the package manager so we leave that up to the user. Here are the
commands that should be used on three Linux distributions:
\begin{center}
\begin{tabular}{ll}
  Fedora & \texttt{sudo dnf install lpsolve} \\
  Arch & \texttt{sudo pacman -S lpsolve} \\
  Debian & \texttt{sudo apt-get install lp-solve}
\end{tabular}
\end{center}
After executing the relevant command it may be necessary to follow up
with
\begin{verbatim}
sudo ldconfig
\end{verbatim}
to ensure that the newly installed library can be found by gretl.

\bibliography{gretl}

\end{document}
