\chapter{Introduction}
\label{intro}


\section{Features at a glance}
\label{features}

Gretl is an econometrics package, including a shared library, a
command-line client program and a graphical user interface.
    
\begin{description}
\item[User-friendly] Gretl offers an intuitive user interface;
  it is very easy to get up and running with econometric analysis.
  Thanks to its association with the econometrics textbooks by Ramu
  Ramanathan, Jeffrey Wooldridge, and James Stock and Mark Watson, the
  package offers many practice data files and command scripts.  These
  are well annotated and accessible. Two other useful resources for gretl
  users are the available documentation and the
  \href{http://gretl.sourceforge.net/lists.html}{gretl-users} mailing
  list.
\item[Flexible] You can choose your preferred point on the spectrum
  from interactive point-and-click to complex scripting, and can easily
  combine these approaches.
\item[Cross-platform] Gretl's ``home'' platform is Linux but it
  is also available for MS Windows and Mac OS X, and should work on
  any unix-like system that has the appropriate basic libraries (see
  Appendix~\ref{app-build}).
\item[Open source] The full source code for gretl is available
  to anyone who wants to critique it, patch it, or extend it.
  See Appendix~\ref{app-build}.
\item[Sophisticated] Gretl offers a full range of least-squares
  based estimators, either for single equations and for systems,
  including vector autoregressions and vector error correction models.
  Several specific maximum likelihood estimators (e.g.\ probit, ARIMA,
  GARCH) are also provided natively; more advanced estimation methods
  can be implemented by the user via generic maximum likelihood or
  nonlinear GMM.
\item[Extensible] Users can enhance gretl by writing their own
  functions and procedures in gretl's scripting language, which
  includes a wide range of matrix functions.
\item[Accurate] Gretl has been thoroughly tested on several
  benchmarks, among which the NIST reference datasets. See
  Appendix~\ref{app-accuracy}.
\item[Internet ready] Gretl can fetch materials such databases,
  collections of textbook datafiles and add-on packages over the
  internet.
\item[International] Gretl will produce its output in English,
  French, Italian, Spanish, Polish, Portuguese, German, Basque,
  Turkish, Russian, Albanian or Greek depending on your computer's
  native language setting.
\end{description}


\section{Acknowledgements}
\label{ack}

The gretl code base originally derived from the program
\app{ESL} (``Econometrics Software Library''), written by Professor
Ramu Ramanathan of the University of California, San Diego.  
We are much in debt to Professor Ramanathan for making this code
available under the GNU General Public Licence and for helping to
steer gretl's early development.

We are also grateful to the authors of several econometrics textbooks
for permission to package for gretl various datasets associated
with their texts.  This list currently includes William Greene, author
of \emph{Econometric Analysis}; Jeffrey Wooldridge (\emph{Introductory
  Econometrics: A Modern Approach}); James Stock and Mark Watson
(\emph{Introduction to Econometrics}); Damodar Gujarati (\emph{Basic
  Econometrics}); Russell Davidson and James MacKinnon
(\emph{Econometric Theory and Methods}); and Marno Verbeek (\emph{A
  Guide to Modern Econometrics}).

GARCH estimation in gretl is based on code deposited in the
archive of the \emph{Journal of Applied Econometrics} by Professors
Fiorentini, Calzolari and Panattoni, and the code to generate
\emph{p}-values for Dickey--Fuller tests is due to James MacKinnon.  In
each case we are grateful to the authors for permission to use their
work.

With regard to the internationalization of gretl, thanks go to
Ignacio D�az-Emparanza (Spanish), Michel Robitaille and Florent
Bresson (French), Cristian Rigamonti (Italian), Tadeusz Kufel and
Pawel Kufel (Polish), Markus Hahn and Sven Schreiber (German), H�lio
Guilherme and Henrique Andrade (Portuguese), Susan Orbe (Basque),
Talha Yalta (Turkish) and Alexander Gedranovich (Russian).

Gretl has benefitted greatly from the work of numerous
developers of free, open-source software: for specifics please see
Appendix~\ref{app-build}.  Our thanks are due to Richard Stallman
of the Free Software Foundation, for his support of free software in
general and for agreeing to ``adopt'' gretl as a GNU program in
particular.

Many users of gretl have submitted useful suggestions and bug
reports.  In this connection particular thanks are due to Ignacio
D�az-Emparanza, Tadeusz Kufel, Pawel Kufel, Alan Isaac, Cri Rigamonti,
Sven Schreiber, Talha Yalta, Andreas Rosenblad, and Dirk Eddelbuettel,
who maintains the gretl package for Debian GNU/Linux.


\section{Installing the programs}
\label{install}

\subsection{Linux}
\label{linux-install}

On the Linux\footnote{In this manual we use ``Linux'' as shorthand to
  refer to the GNU/Linux operating system.  What is said herein about
  Linux mostly applies to other unix-type systems too, though some
  local modifications may be needed.} platform you have the choice of
compiling the gretl code yourself or making use of a pre-built
package. Building gretl from the source is necessary if you want
to access the development version or customize gretl to your
needs, but this takes quite a few skills; most users will want to go
for a pre-built package.

Some Linux distributions feature gretl as part of their standard
offering: Debian, Ubuntu and Fedora, for example.  If this is the
case, all you need to do is install gretl through your package
manager of choice. In addition the gretl webpage at
\url{http://gretl.sourceforge.net} offers a ``generic'' package in
\app{rpm} format for modern Linux systems.

If you prefer to compile your own (or are using a unix system for
which pre-built packages are not available), instructions on building
gretl can be found in Appendix~\ref{app-build}.

\subsection{MS Windows}
\label{windows-install}

The MS Windows version comes as a self-extracting executable.
Installation is just a matter of downloading \verb+gretl_install.exe+
and running this program. You will be prompted for a location to
install the package.

\subsection{Mac OS X}
\label{mac-install}

The Mac version comes as a gzipped disk image. Installation is a
matter of downloading the image file, opening it in the Finder, and
dragging \texttt{Gretl.app} to the Applications folder. However, when
installing for the first time two prerequisite packages must be
put in place first; details are given on the gretl website.

%%% Local Variables: 
%%% mode: latex
%%% TeX-master: "gretl-guide"
%%% End: 
