\chapter{Introduction}
\label{intro}


\section{Features at a glance}
\label{features}

\app{Gretl} is an econometrics package, including a shared library, a
command-line client program and a graphical user interface.
    
\begin{description}
\item[User-friendly] \app{Gretl} offers an intuitive user interface;
  it is very easy to get up and running with econometric analysis.
  Thanks to its association with the econometrics textbooks by Ramu
  Ramanathan, Jeffrey Wooldridge, and James Stock and Mark Watson the
  package offers many practice data files and command scripts.  These
  are well annotated and accessible.
\item[Flexible] You can choose your preferred point on the spectrum
  from interactive point-and-click to batch processing, and can easily
  combine these approaches.
\item[Cross-platform] \app{Gretl}'s ``home'' platform is Linux but it
  is also available for MS Windows and Mac OS X, and should work on
  any unix-like system that has the appropriate basic libraries (see
  Appendix~\ref{app-technote}).
\item[Open source] The full source code for \app{gretl} is available
  to anyone who wants to critique it, patch it, or extend it.
\item[Reasonably sophisticated] \app{Gretl} offers a full range of
  least-squares based estimators, including two-stage least squares
  and nonlinear least squares.  It also offers several specific
  maximum-likelihood estimators (e.g.\ logit, probit, tobit) as well
  as generic maximum likelihood estimation.  The program supports
  estimation of systems of simultaneous equations, GARCH, ARIMA, vector
  autoregressions and vector error correction models.
\item[Accurate] \app{Gretl} has been thoroughly tested on the NIST
  reference datasets. See Appendix~\ref{app-accuracy}.
\item[Internet ready] \app{Gretl} can access and fetch databases from
  a server at Wake Forest University.  The MS Windows version comes
  with an updater program which will detect when a new version is
  available and offer the option of auto-updating.
\item[International] \app{Gretl} will produce its output in English,
  French, Italian, Spanish, Polish or German, depending on your
  computer's native language setting.
\end{description}


\section{Acknowledgements}
\label{ack}

The \app{gretl} code base originally derived from the program
\app{ESL} (``Econometrics Software Library''), written by Professor
Ramu Ramanathan of the University of California, San Diego.  
We are much in debt to Professor Ramanathan for making this code
available under the GNU General Public Licence and for helping to
steer \app{gretl}'s development.

We are also grateful to the authors of several econometrics textbooks
for permission to package for \app{gretl} various datasets associated
with their texts.  This list currently includes William Greene, author
of \emph{Econometric Analysis}; Jeffrey Wooldridge (\emph{Introductory
  Econometrics: A Modern Approach}); James Stock and Mark Watson
(\emph{Introduction to Econometrics}); Damodar Gujarati (\emph{Basic
  Econometrics}); and Russell Davidson and James MacKinnon
(\emph{Econometric Theory and Methods}).  

GARCH estimation in \app{gretl} is based on code deposited in the
archive of the \emph{Journal of Applied Econometrics} by Professors
Fiorentini, Calzolari and Panattoni, and the code to generate
\emph{p}-values for Dickey--Fuller tests is due to James MacKinnon.  In
each case we are grateful to the authors for permission to use their
work.

With regard to the internationalization of \app{gretl}, thanks go to
Ignacio D�az-Emparanza (Spanish), Michel Robitaille and Florent
Bresson (French) , Cristian Rigamonti (Italian), Tadeusz Kufel and
Pawel Kufel (Polish), and Markus Hahn and Sven Schreiber (German).

\app{Gretl} has benefitted greatly from the work of numerous
developers of free, open-source software: for specifics please see
Appendix~\ref{app-technote}.  Our thanks are due to Richard Stallman
of the Free Software Foundation, for his support of free software in
general and for agreeing to ``adopt'' \app{gretl} as a GNU program in
particular.

Many users of \app{gretl} have submitted useful suggestions and bug
reports.  In this connection particular thanks are due to Ignacio
D�az-Emparanza, Tadeusz Kufel, Pawel Kufel, Alan Isaac, Cri Rigamonti
and Dirk Eddelbuettel, who maintains the \app{gretl} package for
Debian GNU/Linux.  


\section{Installing the programs}
\label{install}

\subsection{Linux}
\label{linux-install}

On the Linux\footnote{In this manual we use ``Linux'' as shorthand to
  refer to the GNU/Linux operating system.  What is said herein about
  Linux mostly applies to other unix-type systems too, though some
  local modifications may be needed.} platform you have the choice of
compiling the \app{gretl} code yourself or making use of a pre-built
package. Ready-to-run packages are available in \app{rpm} format
(suitable for Red Hat Linux and related systems) and also \app{deb}
format (Debian GNU/Linux).  If you prefer to compile your own (or are
using a unix system for which pre-built packages are not available)
here is what to do.
      
\begin{enumerate}
\item Download the latest \app{gretl} source package from
  \href{http://gretl.sourceforge.net/}{gretl.sourceforge.net}.
\item Unzip and untar the package.  On a system with the GNU utilities
  available, the command would be \cmd{tar xvfz gretl-N.tar.gz}
  (replace \cmd{N} with the specific version number of the file you
  downloaded at step 1).
\item Change directory to the gretl source directory created at step 2
  (e.g.\ \verb+gretl-1.1.5+).
          
\item The basic routine is then
            
\begin{code}
    ./configure 
    make 
    make check
    make install
\end{code}
However, you should probably read the \verb+INSTALL+ file first,
and/or do
\begin{code}
    ./configure --help
\end{code}
first to see what options are available. One option you way wish to
tweak is \cmd{--prefix}.  By default the installation goes under
\verb+/usr/local+ but you can change this.  For example
\begin{code}
    ./configure --prefix=/usr
\end{code}
will put everything under the \verb+/usr+ tree.  In the event that a
required library is not found on your system, so that the configure
process fails, please see Appendix~\ref{app-technote}.
          
\end{enumerate}

\app{Gretl} offers support for the \app{gnome} desktop.  To take
advantage of this you should compile the program yourself (as
described above).  If you want to suppress the \app{gnome}-specific
features you can pass the option \verb+--without-gnome+ to
\cmd{configure}.


\subsection{MS Windows}
\label{windows-install}

The MS Windows version comes as a self-extracting executable.
Installation is just a matter of downloading \verb+gretl_install.exe+
and running this program. You will be prompted for a location to
install the package (the default is \verb+c:\userdata\gretl+).


\subsection{Updating}
\label{updating}

If your computer is connected to the Internet, then on start-up
\app{gretl} can query its home website at Wake Forest University to
see if any program updates are available; if so, a window will open up
informing you of that fact.  If you want to activate this feature,
check the box marked ``Tell me about gretl updates'' under
\app{gretl}'s ``Tools, Preferences, General'' menu.
      
The MS Windows version of the program goes a step further: it tells
you that you can update \app{gretl} automatically if you wish.  To do
this, follow the instructions in the popup window: close \app{gretl}
then run the program titled ``gretl updater'' (you should find this
along with the main \app{gretl} program item, under the Programs
heading in the Windows Start menu). Once the updater has completed its
work you may restart \app{gretl}.
      
%%% Local Variables: 
%%% mode: latex
%%% TeX-master: "gretl-guide"
%%% End: 
