\chapter{Introduction}
\label{intro}


\section{Features at a glance}
\label{features}

\app{gretl} is an econometrics package, including a shared library, a
command-line client program and a graphical user interface.
    
\begin{description}
\item[User-friendly] \app{gretl} offers an intuitive user interface;
  it is very easy to get up and running with econometric analysis.
  Thanks to its association with the econometrics textbooks by Ramu
  Ramanathan, Jeffrey Wooldridge, and James Stock and Mark Watson the
  package offers many practice data files and command scripts.  These
  are well annotated and accessible.
\item[Flexible] You can choose your preferred point on the spectrum
  from interactive point-and-click to batch processing, and can easily
  combine these approaches.
\item[Cross-platform] \app{gretl}'s home platform is Linux, but it is
  also available for MS Windows.  I have compiled it on Mac OS X and
  AIX and it should work on any unix-like system that has the
  appropriate basic libraries (see Appendix~\ref{app-technote}).
\item[Open source] The full source code for \app{gretl} is available
  to anyone who wants to critique it, patch it, or extend it.  The
  author welcomes any bug reports.
\item[Reasonably sophisticated] \app{gretl} offers a full range of
  least-squares based estimators, including two-stage least squares
  and nonlinear least squares.  It also offers (binomial) logit and
  probit estimation, and has a loop construct for running Monte Carlo
  analyses or iterative estimation of non-linear models. While it does
  not include all the estimators and tests that a professional
  econometrician might require, it supports the export of data to the
  formats of (\href{http://www.r-project.org/}{GNU R}) and
  (\href{http://www.octave.org/}{GNU Octave}) for further analysis
  (see Appendix~\ref{app-advanced}).
\item[Accurate] \app{gretl} has been thoroughly tested on the NIST
  reference datasets. See Appendix~\ref{app-accuracy}.
\item[Internet ready] \app{gretl} can access and fetch databases from
  a server at Wake Forest University.  The MS Windows version comes
  with an updater program which will detect when a new version is
  available and offer the option of auto-updating.
\item[International] \app{gretl} will produce its output in English,
  French, Italian, Spanish or Polish, depending on your computer's
  native language setting.
\end{description}


\section{Acknowledgements}
\label{ack}

My primary debt is to Professor Ramu Ramanathan of the University of
California, San Diego.  A few years back he was kind enough to provide
me with the source code for his program \app{ESL} (``Econometrics
Software Library''), which I ported to Linux, and since then I have
collaborated with him on updating and extending the program.  For the
\app{gretl} project I have made extensive changes to the original
\app{ESL} code.  New econometric functionality has been added, and the
graphical interface is entirely new. Please note that Professor
Ramanathan is not responsible for any bugs in \app{gretl}.
    
I am grateful to the authors of several econometrics textbooks for
permission to package for \app{gretl} various datasets associated with
their texts.  This list currently includes William Greene, author of
\emph{Econometric Analysis}; Jeffrey Wooldridge (\emph{Introductory
  Econometrics: A Modern Approach}); James Stock and Mark Watson
(\emph{Introduction to Econometrics}); Damodar Gujarati (\emph{Basic
  Econometrics}); and Russell Davidson and James MacKinnon
(\emph{Econometric Theory and Methods}).  GARCH estimation in
\app{gretl} is based on code deposited in the archive of the
\emph{Journal of Applied Econometrics} by Professors Fiorentini,
Calzolari and Panattoni, and the code to generate \emph{p}-values for
Dickey Fuller tests is due to James MacKinnon.  In each case I am
grateful to the authors for permission to use their work.  

With regard to the internationalization of \app{gretl}, I wish to
thank Ignacio D�az-Emparanza, Michel Robitaille, Cristian Rigamonti
and Tadeusz Kufel, who prepared the Spanish, French, Italian and
Polish translations respectively.

I have benefitted greatly from the work of numerous developers of
open-source software: for specifics please see
Appendix~\ref{app-technote}. My thanks are due to Richard Stallman of
the Free Software Foundation, for his support of free software in
general and for agreeing to ``adopt'' \app{gretl} as a GNU program in
particular.

Many users of \app{gretl} have submitted useful suggestions and bug
reports.  In this connection particular thanks are due to Ignacio
D�az-Emparanza, Tadeusz Kufel, Pawel Kufel, Alan Isaac, Cri Rigamonti
and Dirk Eddelbuettel, who maintains the \app{gretl} package for
Debian GNU/Linux.  Besides making good suggestions, Riccardo ``Jack''
Lucchetti has contributed substantial code and econometric expertise.


\section{Installing the programs}
\label{install}

\subsection{Linux}
\label{linux-install}

On the Linux\footnote{Terminology is a bit of a problem here, but in
  this manual I will use ``Linux'' as shorthand to refer to the
  GNU/Linux operating system.  What is said herein about Linux mostly
  applies to other unix-type systems too, though some local
  modifications may be needed.} platform you have the choice of
compiling the \app{gretl} code yourself or making use of a pre-built
package. Ready-to-run packages are available in \app{rpm} format
(suitable for Red Hat Linux and related systems) and also \app{deb}
format (Debian GNU/Linux).  I am grateful to Dirk Eddelb�ttel for
making the latter.  If you prefer to compile your own (or are using a
unix system for which pre-built packages are not available) here is
what to do.
      
\begin{enumerate}
\item Download the latest \app{gretl} source package from
  \href{http://gretl.sourceforge.net/}{gretl.sourceforge.net}.
          
\item Unzip and untar the package.  On a system with the GNU utilities
  available, the command would be \cmd{tar xvfz gretl-N.tar.gz}
  (replace \cmd{N} with the specific version number of the file you
  downloaded at step 1).
\item Change directory to the gretl source directory created at step 2
  (e.g.\ \verb+gretl-1.1.5+).
          
\item The basic routine is then
            
\begin{code}
    ./configure 
    make 
    make check
    make install
\end{code}
However, you should probably read the \verb+INSTALL+ file first,
and/or do
\begin{code}
    ./configure --help
\end{code}
first to see what options are available. One option you way wish to
tweak is \cmd{--prefix}.  By default the installation goes under
\verb+/usr/local+ but you can change this.  For example
\begin{code}
    ./configure --prefix=/usr
\end{code}
will put everything under the \verb+/usr+ tree.  In the event that a
required library is not found on your system, so that the configure
process fails, please see Appendix~\ref{app-technote}.
          
\end{enumerate}

As of version 0.97 \app{gretl} offers support for the \app{gnome}
desktop.  To take advantage of this you should compile the program
yourself (as described above).  If you want to suppress the
\app{gnome}-specific features you can pass the option
\verb+--without-gnome+ to \cmd{configure}.


\subsection{MS Windows}
\label{windows-install}

The MS Windows version comes as a self-extracting executable.
Installation is just a matter of downloading \verb+gretl_install.exe+
and running this program. You will be prompted for a location to
install the package (the default is \verb+c:\userdata\gretl+).


\subsection{Updating}
\label{updating}

If your computer is connected to the Internet, then on start-up
\app{gretl} can query its home website at Wake Forest University to
see if any program updates are available; if so, a window will open up
informing you of that fact.  If you want to activate this feature,
check the box marked ``Tell me about gretl updates'' under
\app{gretl}'s ``File, Preferences, General'' menu.
      
The MS Windows version of the program goes a step further: it tells
you that you can update \app{gretl} automatically if you wish.  To do
this, follow the instructions in the popup window: close \app{gretl}
then run the program titled ``gretl updater'' (you should find this
along with the main \app{gretl} program item, under the Programs
heading in the Windows Start menu). Once the updater has completed its
work you may restart \app{gretl}.
      
%%% Local Variables: 
%%% mode: latex
%%% TeX-master: "gretl-guide"
%%% End: 
