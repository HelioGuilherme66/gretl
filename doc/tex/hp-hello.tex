\chapter{Hello, world!}
\label{chap:hello}

We begin with the time-honored ``Hello, world'' program, the
obligatory first step in any programming language. It's actually very
simple in hansl:
\begin{code}
  # First example
  print "Hello, world!"
\end{code}

There are several ways to run the above example: you can put it in a
text file \texttt{first\_ex.inp} and have gretl execute it from the
command line through the command
\begin{code}
  gretlcli -b first_ex.inp
\end{code}
or you could just copy its contents in the editor window of a GUI
gretl session and click on the ``gears'' icon. It's up to you; use
whatever you like best.

From a syntactical point of view, allow us to draw attention on
the following points:
\begin{enumerate}
\item The line that begins with a hash mark (\texttt{\#}) is a
  comment: if a hash mark is encountered, everything from that point
  to the end of the current line is treated as a comment, and ignored
  by the interpreter.
\item The next line contains a \emph{command} (\cmd{print}) followed
  by an \emph{argument}; this is fairly typical of hansl: many jobs
  are carried out by calling commands.
\item The quotation character is a straight double-quote.
\item Hansl does not have an explicit command terminator such as the
  ``\texttt{;}'' character in the C language family (C++, Java, C\#,
  \ldots) or GAUSS; instead it uses the newline character as an
  implicit terminator. So at the end of a command, you \emph{must}
  insert a newline. Conversely, you can't split a single command
  over more than one line unless (a) the line to be continued ends
  with a comma or (b) you insert a ``\textbackslash'' (backslash)
  character, which causes gretl to ignore the following line break.
\end{enumerate}

Note also that the \cmd{print} command automatically appends a line
break, and does not recognize ``escape'' sequences such as
``\verb|\n|''. Such sequences are just printed literally---with a
single exception, namely that a backslash immediately followed by
double-quote produces an embedded double-quote. The \cmd{printf}
command can be used for greater control over output; see chapter
\ref{chap:formatting}.

Let's now examine a simple variant of the above:
\begin{code}
  /*
    Second example
  */
  string foo = "Hello, world"
  print foo
\end{code}

In this example, the comment is written using the convention adopted
in the C programming language: everything between ``\verb|/*|'' and
``\verb|*/|'' is ignored.\footnote{Each type of comment can be masked
  by the other:
\begin{itemize}
\item If \texttt{/*} follows \texttt{\#} on a given line which does
  not already start in ignore mode, then there's nothing special about
  \texttt{/*}, it's just part of a \texttt{\#}-style comment.
\item If \texttt{\#} occurs when we're already in comment mode, it is
  just part of a comment.
\end{itemize}} Comments of this type cannot be nested.

Then we have the line
\begin{code}
  string foo = "Hello, world"
\end{code}
In this line, we assign the value ``\texttt{Hello, world}'' to the
variable named \texttt{foo}. Note that
\begin{enumerate}
\item The assignment operator is the equals sign (\texttt{=}).
\item The name of the variable (its \emph{identifier}) must follow the
  following convention: identifiers can be at most 31 characters long
  and must be plain ASCII. They must start with a letter, and can
  contain only letters, numbers and the underscore
  character.\footnote{Actually one exception to this rule is
    supported: identifiers taking the form of a single Greek
    letter. See chapter~\ref{chap:greeks} for details.} Identifiers in
  hansl are case-sensitive, so \texttt{foo}, \texttt{Foo} and
  \texttt{FOO} are three distinct names. Of course, some words are
  reserved and can't be used as identifiers (however, nearly all
  reserved words only contain lowercase characters).
\item The string delimiter is the double quote (\verb|"|). 
\end{enumerate}

In hansl, a variable has to be of one of these types: \texttt{scalar},
\texttt{series}, \texttt{matrix}, \texttt{list}, \texttt{string},
\texttt{bundle} or \texttt{array}. As we've just seen, string
variables are used to hold sequences of alphanumeric characters. We'll
introduce the other ones gradually; for example, the \texttt{matrix}
type will be the object of the next chapter.

The reader may have noticed that the line 
\begin{code}
  string foo = "Hello, world"
\end{code}
implicitly performs two tasks: it \emph{declares} \texttt{foo} as a
variable of type \texttt{string} and, at the same time, \emph{assigns}
a value to \texttt{foo}. The declaration component is not strictly
required. In most cases gretl is able to figure out by itself what
type a newly introduced variable should have, and the line
\verb|foo = "Hello, world"| (without a type specifier) would have
worked just fine.  However, it is more elegant (and leads to more
legible and maintainable code) to use a type specifier at least the
first time you introduce a variable.
  
In the next example, we will use a variable of the \texttt{scalar}
type:
\begin{code}
  scalar x = 42
  print x
\end{code}
A \texttt{scalar} is a double-precision floating point number, so
\texttt{42} is the same as \texttt{42.0} or \texttt{4.20000E+01}. Note
that hansl doesn't have a separate variable type for integers.

An important detail to note is that, contrary to most other
matrix-oriented languages in use in the econometrics community, hansl
is \emph{strongly typed}. That is, you cannot assign a value of one
type to a variable that has already been declared as having a
different type. For example, this will return an error:
\begin{code}
  string a = "zoo"
  a = 3.14 # no, no, no!
\end{code}
If you try running the example above, an error will be
flagged. However, it is acceptable to destroy the original variable,
via the \cmd{delete} command, and then re-declare it, as in
\begin{code}
  scalar X = 3.1415
  delete X
  string X = "apple pie"
\end{code}

There is no ``type-casting'' as in C, but some automatic type
conversions are possible (more on this later).

Many commands can take more than one argument, as in
\begin{code}
  set verbose off

  scalar x = 42
  string foo = "not bad"
  print x foo 
\end{code}
In this example, one \texttt{print} is used to print the values of two
variables; more generally, \texttt{print} can be followed by as many
arguments as desired. The other difference with respect to the
previous code examples is in the use of the \texttt{set}
command. Describing this command in detail would lead us to an overly
long diversion; suffice it to say that it is used to set the values of
various ``state variables'' that influence the behavior of the
program; here it is used as a way to silence unwanted output. See the
\GCR{} for more on \texttt{set}.

% There was a reference here to a {chap:settings}, but it has not
% been written at this point

The \cmd{eval} command is useful when you want to look at the result
of an expression without assigning it to a variable; for example
\begin{code}
  eval 2+3*4
\end{code}
will print the number 14. This is most useful when running gretl
interactively, like a calculator, but it is usable in a hansl script
for checking purposes, as in the following (rather silly) example:
\begin{code}
  scalar a = 1
  scalar b = -1
  # this ought to be 0
  eval a+b
\end{code}

\section{Manipulation of scalars}

Algebraic operations work in the obvious way, with the classic
algebraic operators having their traditional precedence rules. The
caret (\verb|^|) is used for exponentiation. For example,
\begin{code}
  scalar phi = exp(-0.5 * (x-m)^2 / s2) / sqrt(2 * $pi * s2)
\end{code}
%$
in which we assume that \texttt{x}, \texttt{m} and \texttt{s2} are
pre-existing scalars. The example above contains two noteworthy
points:
\begin{itemize}
\item The usage of the \cmd{exp} (exponential) and \cmd{sqrt} (square
  root) functions; it goes without saying that hansl possesses a
  reasonably wide repertoire of such functions. See the \GCR{} for the
  complete list.
\item The usage of \verb|$pi| for the constant $\pi$. While
  user-specified identifiers must begin with a letter, built-in
  identifiers for internal objects typically have a ``dollar'' prefix;
  these are known as \emph{accessors} (basically, read-only
  variables).  Most accessors are defined in the context of an open
  dataset (see part~\ref{part:hp-data}), but some represent
  pre-defined constants, such as $\pi$. Again, see the \GCR{} for a
  comprehensive list.
\end{itemize}

Hansl does not possess a specific Boolean type, but scalars can be
used for holding true/false values. It follows that you can also use
the logical operators \emph{and} (\verb|&&|), \emph{or} (\verb+||+),
and \emph{not} (\verb|!|) with scalars, as in the following example:
\begin{code}
  a = 1
  b = 0
  c = !(a && b) 
\end{code}
In the example above, \texttt{c} will equal 1 (true), since
\verb|(a && b)| is false, and the exclamation mark is the negation
operator.  Note that 0 evaluates to false, and anything else (not
necessarily 1) evaluates to true.

A few constructs are taken from the C language family: one is the
postfix increment operator:
\begin{code}
  a = 5
  b = a++
  print a b
\end{code}
the second line is equivalent to \texttt{b = a}, followed by
\texttt{a++}, which in turn is shorthand for \texttt{a = a+1}, so
running the code above will result in \texttt{b} containing 5 and
\texttt{a} containing 6. Postfix subtraction is also supported; prefix
operators, however, are not supported. Another C borrowing is
inflected assignment, as in \texttt{a += b}, which is equivalent to
\texttt{a = a + b}; several other similar operators are available,
such as \texttt{-=}, \texttt{*=} and more. See the \GCR{} for details.

The internal representation for a missing value is \texttt{NaN} (``not
a number''), as defined by the IEEE 754 floating point standard.  This
is what you get if you try to compute quantities like the square root
or the logarithm of a negative number. You can also set a value to
``missing'' directly using the keyword \texttt{NA}.  The complementary
functions \cmd{missing} and \cmd{ok} can be used to determine whether
a scalar is \texttt{NA}. In the following example a value of zero is
assigned to the variable named \texttt{test}:
\begin{code}
  scalar not_really = NA
  scalar test = ok(not_really)
\end{code}
Note that you cannot test for equality to \texttt{NA}, as in
\begin{code}
  if x == NA ... # wrong!
\end{code}
because a missing value is taken as indeterminate and hence not equal
to anything. This last example, despite being wrong, illustrates a
point worth noting: the test-for-equality operator in hansl is the
double equals sign, ``\texttt{==}'' (as opposed to plain
``\texttt{=}'' which indicates assignment).

\section{Manipulation of strings}

Most of the previous section applies, with obvious modifications, to
strings: you may manipulate strings via operators and/or
functions. Hansl's repertoire of functions for manipulating strings
offers all the standard capabilities one would expect, such as
\cmd{toupper}, \cmd{tolower}, \cmd{strlen}, etc., plus some more
specialized ones. Again, see the \GCR\ for a complete list.

In order to access part of a string, you may use the \cmd{substr}
function,\footnote{Actually, there is a cooler method, which uses the
  same syntax as matrix slicing (see chapter \ref{chap:matrices}):
  \cmd{substr(s, 3, 5)} is functionally equivalent to \cmd{s[3:5]}.} as in
\begin{code}
  string s = "endogenous"
  string pet = substr(s, 3, 5)
\end{code}
which would result to assigning the value \texttt{dog} to the variable
\texttt{pet}.

The following are useful operators for strings:
\begin{itemize}
\item the \verb|~| operator, to join two or more strings, as
  in\footnote{On some national keyboards, you don't have the tilde
    (\texttt{\~}) character. In gretl's script editor, this can be
    obtained via its Unicode representation: type Ctrl-Shift-U,
    followed by \texttt{7e}.}
  \begin{code}
    string s1 = "sweet"
    string s2 = "Home, " ~ s1 ~ " home."
  \end{code}
\item the closely related \verb|~=| operator, which acts as an
  inflected assignment operator (so \verb|a ~= "_ij"| is equivalent to
  \verb|a = a ~ "_ij"|);
\item the offset operator \texttt{+}, which yields a substring of the
  preceding element, starting at the given character offset.  An empty
  string is returned if the offset is greater than the length of the
  string in question.
\end{itemize}

A noteworthy point: strings may be (almost) arbitrarily long;
moreover, they can contain special characters such as line breaks and
tabs. It is therefore possible to use hansl for performing rather
complex operations on text files by loading them into memory as a very
long string and then operating on that; interested readers should take
a look at the \cmd{readfile}, \cmd{getline}, \cmd{strsub} and
\cmd{regsub} functions in the \GCR.\footnote{We are not claiming that
  hansl would be the tool of choice for text processing in
  general. Nonetheless the functions mentioned here can be very useful
  for tasks such as pre-processing plain text data files that do not
  meet the requirements for direct importation into gretl.}

For \emph{creating} complex strings, the most flexible tool is the
\cmd{sprintf} function. Its usage is illustrated in
Chapter~\ref{chap:formatting}.

% Finally, it is quite common to use \emph{string
%   substitution} in hansl sripts; however, this is another topic that
% deserves special treatment so we defer its description to section
% \ref{sec:stringsub}.

%%% Local Variables: 
%%% mode: latex
%%% TeX-master: "hansl-primer"
%%% End: 
