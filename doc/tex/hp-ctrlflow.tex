\chapter{Control flow}
\label{chap:hp-ctrlflow}

The primary means for controlling the flow of execution in a hansl
script are the the \cmd{loop} statement (repeated execution), \cmd{if}
statement (conditional execution), the \cmd{catch} modifier (which
enables the trapping of errors that would otherwise halt execution)
and the \cmd{quit} command (which forces termination).

\section{Loops}
\label{sec:hr-loops}

The basic hansl command for looping is (surprise) \cmd{loop}, and
takes the form
\begin{code}
loop <control-expression> <options>
    ...
endloop
\end{code}
In other words, the pair of statements \cmd{loop} and \cmd{endloop}
enclose the statements to repeat. Of course, loops can be nested.
Several variants of the \texttt{<control-expression>} for a loop are
supported, as follows:
\begin{enumerate}
\item unconditional loop
\item while loop
\item index loop
\item foreach loop
\item for loop.
\end{enumerate}
These variants are briefly described below.

\subsection{Unconditional loop}

This is the simplest variant. It takes the form
\begin{code}
loop <times>
   ...
endloop
\end{code}
where \texttt{<times>} is any expression that evaluates to a scalar,
namely the required number of iterations. This is only evaluated at
the beginning of the loop, so the number of iterations cannot be
changed from within the loop itself.

\subsection{Index loop}

The syntax is
\begin{code}
loop <counter>=<min>..<max>
   ...
endloop
\end{code}
The limits \texttt{<min>} and \texttt{<max>} must evaluate to scalars;
they are automatically turned into integers if they have a fractional
part. The \texttt{<counter>} variable is started at \texttt{<min>} and
incremented by 1 on each iteration until it equals \texttt{<max>}.

The counter is ``read-only'' inside the loop. You can access either
its numerical value or (by prepending a dollar sign) its string
representation. For example:
\begin{code}
scalar Max = 4
loop i=1..Max --quiet
    printf "$i squared = %d\n", i*i
endloop
\end{code}
%$

\subsection{While loop}

Here you have
\begin{code}
loop while <condition>
   ...
endloop
\end{code}
where \texttt{<condition>} should evaluate to a scalar, which is
re-evaluated at each iteration. Looping stops as soon as
\texttt{<condition>} becomes false (0). If \texttt{<condition>}
becomes \texttt{NA}, an error is flagged and execution stops.  By
default, \texttt{while} loops cannot exceed 1000 iterations. This is
intended as a safeguard against potentially infinite loops. This
setting can be overridden if necessary by setting the
\texttt{loop\_maxiter} state variable to a different value (see
chapter \ref{chap:settings}).

\subsection{Foreach loop}
\label{sec:loop-foreach}

In this case the syntax is
\begin{code}
loop foreach <counter> <catalogue>
   ...
endloop
\end{code}
where \texttt{<catalogue>} can be either a variable of type
\texttt{list} (i.e.\ a list of series), or a collection of
space-separated strings. The counter variable automatically takes on
the numerical values 1, 2, 3, \dots{} as execution proceeds, but its
string value (accessed by prepending a dollar sign) shadows the names
of the series in the list or the space-separated strings; this sort of
loop is designed for string substitution.

Here is an example in which the \texttt{<catalogue>} is a collection
of function names: 
\begin{code}
loop foreach f sqrt exp ln
    printf "$f(1) = %g\n", $f(1)
endloop
\end{code}

\subsection{For loop}

The final form of loop control emulates the \cmd{for} statement in the
C programming language.  The syntax is \texttt{loop for}, followed by
three component expressions, separated by semicolons and surrounded by
parentheses, that is
\begin{code}
loop for ( <init> ; <cont> ; <modifier> )
   ...
endloop
\end{code}

The three components are as follows:
\begin{enumerate}
\item Initialization (\texttt{<init>}): This is evaluated once at the
  start of the loop.
\item Continuation condition (\texttt{<cont>}): this is evaluated at
  the top of each iteration (including the first).  If the expression
  evaluates as true (non-zero), iteration continues, otherwise it
  stops. 
\item Modifier (\texttt{<modifier>}): an expression which modifies the
  value of some variable.  This is evaluated prior to checking the
  continuation condition, on each iteration after the first.
\end{enumerate}

Here's a simple example:
%
\begin{code}
loop for (r=0.01; r<.991; r+=.01)
   ...
endloop
\end{code}

Be aware of the limited precision of floating-point arithmetic. For
example, the code snippet below will iterate forever on most platforms
because \texttt{x} will never equal \textit{exactly} 0.01, even though
it might seem that it should.
\begin{code}
loop for (x=1; x!=0.01; x=x*0.1)
    printf "x = .18g\n", x
endloop  
\end{code}
However, if you replace the condition \texttt{x!=0.01} with
\texttt{x>=0.01}, the code will run as (probably) intended.
 
\vspace{1ex}

\subsection{Loop options}

Two options can be given to the \cmd{loop} statement. One is
\option{quiet}. This has simply the effect of suppressing certain
lines of output that gretl prints by default to signal progress and
completion of the loop; it has no other effect and the semantics of
the loop contents remain unchanged. This option is commonly used in
complex scripts.

The \option{progressive} option is mostly used as a quick and
efficient way to set up simulation studies. When this option is given,
a few commands (notably \cmd{print} and \cmd{store}) are given a
special, \emph{ad hoc} meaning. Please refer to the \GUG\ for more
information.
 
\subsection{Breaking out of loops}
\label{sec:loop-break}

The \cmd{break} command makes it possible to break out of a loop if
necessary. Note that
\begin{itemize}
\item if you nest loops, \cmd{break} in the innermost loop will
  interrupt that loop only and not the outer ones; and
\item hansl does not provide any equivalent to the C \texttt{continue}
  statement.
\end{itemize}

\section{The \cmd{if} statement}

Conditional execution in hansl uses the \cmd{if} keyword. Its fullest
usage is as follows
\begin{code}
if <condition>
   ...
elif <condition>
   ...
else 
   ...
endif  
\end{code}

Points to note:
\begin{itemize}
\item The \texttt{<condition>} can be any expression that evaluates to a
  scalar: 0 is interpreted as ``false'', non-zero is interpreted as
  ``true''; \texttt{NA} generates an error.
\item The \cmd{elif} and \cmd{else} clauses are optional: the minimal
  form is just \texttt{if} \dots{} \texttt{endif}.
\item The \cmd{elif} keyword can be repeated, making hansl's \cmd{if}
  statement a multi-way branch statement. There is no separate
  \cmd{switch} or \cmd{case} statement in hansl.
\item Conditional blocks of this sort can be nested up to a maximum of
  1024.
\end{itemize}

Example:
\begin{code}
scalar x = 15

# --- simple if ----------------------------------
if x >= 10
   printf "%g is more than two digits long\n", x
endif

# --- if with else -------------------------------
if x >= 0
   printf "%g is non-negative\n", x
else
   printf "%g is negative\n", x
endif

# --- multiple branches --------------------------
if missing(x)
   printf "%g is missing\n", x
elif x < 0
   printf "%g is negative\n", x
elif floor(x) == x
   printf "%g is an integer\n", x
else
   printf "%g is positive integer\n", x
endif
\end{code}

Note to Stata users: hansl's \cmd{if} statement is fundamentally
different from Stata's \texttt{if} option: the latter selects a
subsample of observations for some action, while the former is used to
decide if a group of statements must be executed or not; Stata calls
this the \emph{branching \texttt{if}}.

\subsection{The ternary query operator}

Besides use of \texttt{if}, the ternary query operator, \texttt{?:},
can be used to perform conditional assignment on a more ``micro''
level. This has the form
\begin{code}
result = <condition> ? <t-value1> : <f-value>
\end{code}
If \texttt{<condition>} evaluates as true (non-zero) then
\texttt{<t-value>} is assigned to \texttt{result}, otherwise
\texttt{<f-value>} is assigned. This is obviously more compact than
\texttt{if} \dots{} \texttt{else} \dots{} \texttt{endif}. The
following example replicates the \cmd{abs} function by hand:
\begin{code}
scalar ax = x>=0 ? x : -x
\end{code}
Of course, in the above case it would have been much simpler to just
write \texttt{ax = abs(x)}. Consider, however, the following case,
which exploits the fact that the ternary operator can be nested:
\begin{code}
scalar days = (m==2) ? 28 : maxr(m.={4,6,9,11}) ? 30 : 31
\end{code}
This example deserves a few comments: we want to compute the number of
days in a month, codified in the variable \texttt{m}. So, what we put into
the scalar \texttt{days} comes from the following pathway:
\begin{enumerate}
\item first, we check if the month is February (\texttt{m==2}); if so,
  we set \texttt{days} to 28 and we're done;\footnote{Ok, we don't
    check for leap years here.}
\item otherwise, we compute a matrix of zeros and ones via the
  \verb|m.={4,6,9,11}| operation (note the ``dot'' operator); if
  \texttt{m} equals any of the elements in the vector, the
  corresponding element of the result will be one, and zero otherwise;
\item by the function \cmd{maxr}, we take its maximum, so we're
  checking whether \texttt{m} is one of the four values corresponding
  to 30-days months;
\item since the above evaluates to a scalar, we put the right value
  into \texttt{days}.
\end{enumerate}

It is also
more flexible, in this respect: in contrast to the usage of
\texttt{if}, \texttt{<condition>} need not evaluate to a scalar.  So
for example one can do this sort of thing with matrices:
\begin{code}
matrix A = mnormal(4,4)
matrix B = upper(A.<0) ? -A : A
\end{code}
which will switch the sign of the negative elements of \texttt{A} on
and above its diagonal. To achieve the same via a loop, one should
have had to write a considerably longer chunk of code, such as
\begin{code}
matrix A = mnormal(4,4)
matrix B = A

loop r = 1..rows(A)
  loop c = r .. cols(A)
     if A[r,c] < 0
       B[r,c] = -A[r,c]
     endif
  end loop
end loop
\end{code}

\section{The \cmd{catch} modifier}

Hansl offers a rudimentary form of exception handling via the
\cmd{catch} modifier.

This is not a command in its own right but can be used as a prefix to
most regular commands: the effect is to prevent termination of a
script if an error occurs in executing the command. If an error does
occur, this is registered in an internal error code which can be
accessed as \dollar{error} (a zero value indicates success). The value
of \dollar{error} should always be checked immediately after using
\texttt{catch}, and appropriate action taken if the command failed.

The catch keyword cannot be used before \cmd{if}, \cmd{elif} or
\cmd{endif}.

\emph{\ldots Add example? \ldots}

\section{The \cmd{quit} statement}

When the \cmd{quit} statement is encountered in a hansl script,
execution stops. If the command-line program \app{gretlcli} is running
in batch mode, control returns to the operating system; if gretl is
running in interactive mode, \app{gretl} will wait for interactive
input. The \texttt{quit} command is rarely used in scripts since
execution automatically stops when script input is exhausted, but it
could be used in conjunction with \cmd{catch}. A script author could
arrange matters so that on encountering a certain error condition an
appropriate message is printed and the script is halted.

%%% Local Variables: 
%%% mode: latex
%%% TeX-master: "hansl-primer"
%%% End: 
