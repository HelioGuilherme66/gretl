\chapter{Joining data sources}
\label{chap:join}

\section{Introduction}

As of version 1.9.10 gretl has a new command, \texttt{join}. The basic
function of this command is to pull data from file into an existing
in-memory dataset; it therefore overlaps to some degree with the
\texttt{append} command, but \texttt{join} is much more flexible and
sophisticated. This chapter gives an overview of the functionality of
the new command along with a detailed account of its syntax and
options. We provide several toy examples and discuss one real-world
case.

First, a point on terminology: in the following we use the terms 
``left-hand'' and ``inner'' to refer to the dataset that is already in
memory, and the terms ``right-hand'' and ``outer'' to refer to the
dataset in the file from which data are to be drawn. 

Two main things are new in \texttt{join}: 
\begin{itemize}
\item ``Key'' variables can be used to match specific observations
  (rows) in the inner and outer datasets, and this match need not be
  1 to 1.
\item A row filter may be applied to screen out unwanted observations
  in the outer dataset.
\end{itemize}

As will be explained below, these two features support rather complex
concatenation and manipulation of data from different sources. The
potential complexity of a \texttt{join} operation, when several
options are activated, means that it is not possible to support batch
importation of several series with a single command. In contrast with
the \texttt{append} command, which by default imports the entire
content of a data file, \texttt{join} imports one series at a time. A
command loop can be used to apply \texttt{join} to multiple series.

A related aspect of \texttt{join} should be noted---one that makes
this command particularly useful when dealing with very large
datafiles.  That is, when gretl executes a join it does not, in
general, read into memory the entire content of the right-hand side
dataset.  Only those columns that are actually needed for the
operation are read in full. This makes the join operation faster and
less demanding of RAM than the methods available in most other
software. On the other hand, gretl's asymmetrical treatment of the
``inner'' and ``outer'' datasets in \texttt{join} may require some
getting used to, for users of other packages.

\section{Basic syntax}
\label{sec:join-syntax}

The minimal invocation of \texttt{join} is

\qquad \texttt{join} \textsl{filename} \textsl{varname}

where \textsl{filename} is the name of a delimited text data file and
\textsl{varname} is the name of the series to be imported. Only
column-delimited text files are supported at present; the delimiter
may be comma, space, tab or semicolon. A series named \textsl{varname}
may already be present in the left-hand dataset, but that is not
required. The series to be imported may be numerical or string-valued.

The effect of the minimal version of \texttt{join} is this: gretl
looks for a data column labeled \textsl{varname} in the specified
file; if such a column is found and the number of observations on the
right matches the number of observations in the current sample range
on the left, then the values from the right are copied into the
relevant range of observations on the left. If \textsl{varname} does
not already exist on the left, any observations outside of the current
sample are set to \texttt{NA}; if it exists already then observations
outside of the current sample are left unchanged.

The case where you want to rename a series on import is handled by the
\verb|--data| option. This option has one required argument, the name
by which the series is known on the right. While the \textsl{varname}
argument must be a valid gretl variable name, the right-hand
\texttt{data} name need not be;\footnote{But see
  section~\ref{sec:join-colnames} below.} for example it can contain
embedded spaces (in which case it must be enclosed in double
quotes). For example, the command
%
\begin{code}
join foo.csv x --data="12. Some name"
\end{code}
%
will look for a column headed \verb|"12. Some name"| on the right and
import it as \texttt{x}.

\section{Filtering}

Rows from the outer dataset can be filtered using the \verb|--filter|
option, which has a required argument of the form

\qquad \textsl{term1} \textsl{op} \textsl{term2}

(with or without spaces). At least one of \textsl{term1} and
\textsl{term2} must be the name of a series (column) present in the
right-hand side dataset. The other may also be the name of a
right-hand series, or it may be a numerical or string constant (with
the type matching the specified series). For numerical data
\textsl{op} can be one of

\qquad \verb|==|, \verb|>|, \verb|<|, \verb|>=|, \verb|<=|, or
\verb|!=|

where these symbols have their usual meanings. If the series term in
the filter is string-valued, the following subset of operators is
supported: \verb|==| and \verb|!=| (the strings compare equal and
unequal respectively), \verb|>=| (the first string compares equal to
the second up to the length of the latter, but may be longer than the
second) and \verb|<=| (the second string compares equal with the first
up to the length of the first, but may be longer).

The filter specifies a Boolean condition in terms of either two
right-hand side variables or one right-hand variable and a constant,
and only those rows on which the condition is satisfied are
imported. If the filter string contains spaces or an equals sign it
must be double-quoted, otherwise double-quoting is optional. As with
the \texttt{data} option, the names of series in the outer dataset
need not be valid gretl identifiers.

Note that only a single filter, of binary form, is supported---this
may be generalized in the future. However, in some cases the effect of
a more elaborate filter may be achieved through subsampling the
left-hand dataset (see section~\ref{sec:join-SHIW} below).

Examples of potentially valid filter options (assuming that the
specified right-hand side columns are found):
\begin{code}
--filter="x15<=x17"

--filter="SEX==F"

--filter=nkids>2
\end{code}

\section{Matching with keys}
\label{sec:join-keys}

Things begin to get interesting when we come to key-matching. The
purpose of this facility is perhaps best introduced by example.
Suppose that (as with many survey and census-based datasets) we have a
dataset that is composed of two or more related files, each having a
different unit of observation; for example we have a ``persons'' data
file and a ``households'' datafile. Table~\ref{tab:csv} shows a
simple, artificial case. The file \texttt{people.csv} contains a
unique identifier for the individuals, \texttt{pid}. The households
file, \texttt{hholds.csv}, contains the unique household identifier
\texttt{hid}, which is also present in the persons file. 

As a first example of \texttt{join} with keys, let's add the
household-level variable \texttt{xh} to the persons dataset:
%
\begin{code}
open people.csv --quiet
join hholds.csv xh --ikey=hid
print --byobs
\end{code}

The basic key option is named \texttt{ikey}; this indicates ``inner
key'', that is, the key variable found in the left-hand or inner
dataset. By default it is assumed that the right-hand dataset contains
a column of the same name, though as we'll see below that assumption
can be overridden. The \texttt{join} command above says, find a series
named \texttt{xh} in the right-hand dataset and add it to the
left-hand one, using the values of \texttt{hid} to match rows.
Looking at the data in Table~\ref{tab:csv} we can see how this should
work. Persons 1 and 2 are both members of household 1, so they should
both get values of 1 for \texttt{xh}; persons 3 and 4 are members of
household 2, so that \texttt{xh} = 4; and so on. Note that the order
in which the key values occur on the right-hand side does not matter.
The gretl output from the \texttt{print} command is shown in the lower
panel of Table~\ref{tab:csv}.

\begin{table}[thbp]
\begin{center}
\setlength{\tabcolsep}{4em}
\begin{tabular}{ll}
\texttt{people.csv} & \texttt{hholds.csv} \\[6pt]
\texttt{pid,hid,gender,age,xp} & \texttt{hid,country,xh} \\
\texttt{1,1,M,50,1} & \texttt{1,US,1} \\
\texttt{2,1,F,40,2} & \texttt{6,IT,12} \\
\texttt{3,2,M,30,3} & \texttt{3,UK,6} \\
\texttt{4,2,F,25,2} & \texttt{4,IT,8} \\
\texttt{5,3,M,40,3} & \texttt{2,US,4} \\
\texttt{6,4,F,35,4} & \texttt{5,IT,10} \\
\texttt{7,4,M,70,3} \\
\texttt{8,4,F,60,3} \\
\texttt{9,5,F,20,4} \\
\texttt{10,6,M,40,4}
\end{tabular}

\vspace{1em}

\begin{tabular}{rrr}
  \texttt{pid}  & \texttt{hid}   &  \texttt{xh} \\[6pt]
    \texttt{1}  &  \texttt{1}    &   \texttt{1} \\
    \texttt{2}  &  \texttt{1}    &   \texttt{1} \\
    \texttt{3}  &  \texttt{2}    &   \texttt{4} \\
    \texttt{4}  &  \texttt{2}    &   \texttt{4} \\
    \texttt{5}  &  \texttt{3}    &   \texttt{6} \\
    \texttt{6}  &  \texttt{4}    &   \texttt{8} \\
    \texttt{7}  &  \texttt{4}    &   \texttt{8} \\
    \texttt{8}  &  \texttt{4}    &   \texttt{8} \\
    \texttt{9}  &  \texttt{5}    &  \texttt{10} \\
   \texttt{10}  &  \texttt{6}    &  \texttt{12}
\end{tabular}
\caption{Two linked CSV data files, and the effect of a \texttt{join}}
\label{tab:csv}
\end{center}
\end{table}

Note that key variables are treated conceptually as integers. If a
specified key contains fractional values these are truncated.

Three extensions of the basic key mechanism are available.
\begin{itemize}
\item If the outer dataset contains a relevant key variable but it
  goes under a different name from the inner key, you can use the
  \verb|--okey| option to specify the outer key. (As with other
  right-hand names, this does not have to be a valid gretl
  identifier.) So, for example, if \texttt{hholds.csv} contained the
  \texttt{hid} information, but under the name \texttt{HHOLD}, the
  \texttt{join} command above could be modified as
  \begin{code}
  join hholds.csv xh --ikey=hid --okey=HHOLD
  \end{code}

\item If a single key is not sufficient to generate the matches you
  want, you can specify a double key in the form of two series names
  separated by a comma; in this case the importation of data is
  restricted to those rows on which both keys match. The syntax here
  is, for example
  \begin{code}
  join foo.csv x --ikey=key1,key2
  \end{code}
  Again, the \verb|--okey| option may be used if the corresponding
  right-hand columns are named differently. The same number of keys 
  must be given on the left and the right, but when a double key is
  used and only one of the key names differs on the right, the name
  that is in common may be omitted (although the comma separator must
  be retained). For example, the second of the following lines is
  acceptable shorthand for the first:
  \begin{code}
  join foo.csv x --ikey=key1,Lkey2 --okey=key1,Rkey2
  join foo.csv x --ikey=key1,Lkey2 --okey=,Rkey2
  \end{code}

\item The example above is an instance of a 1 to 1 match, in the sense
  that applying the matching criterion produces exactly one value of
  the variable \texttt{xh} corresponding to each row of the inner
  dataset. Two other possibilities arise. 

  First, it may be that some rows in the inner dataset have no match
  on the right: by default such observations get \texttt{NA} for the
  imported data. In some cases, the appropriate value to record in the
  absence of a match may be zero rather than \texttt{NA}. For example,
  consider a query on ``number of hours worked'' when the inner
  dataset contains individuals and the outer file contains data on
  jobs: if an individual does not appear in the jobs file, the number
  of hours worked is implicitly zero. In such cases gretl's
  \texttt{misszero()} function can be used to adjust the imported
  data.

  Second, some left-hand rows may have more than one match on the
  right; we refer to this as 1 to $n$ matching. In that case it is
  necessary to specify an ``aggregation method''. This is the job
  of the \verb|--aggr| option, discussed below.

\end{itemize}

\section{Aggregation}
\label{sec:join-aggr}

The \verb|--aggr| option for 1 to $n$ matching of rows takes a single
argument, which must be one of the following:

\begin{center}
\begin{tabular}{ll}
\textit{Code} & \textit{Value returned} \\[6pt]
\texttt{count} & count of matches \\
\texttt{avg} & mean of matching values \\
\texttt{sum} & sum of matching values \\
\texttt{min} & minimum of matching values \\
\texttt{max} & maximum of matching values \\
\texttt{seq:}$i$ & the $i^{\rm th}$ matching value (e.g.\ \texttt{seq:2}) 
\end{tabular}
\end{center}

Note that the \texttt{count} aggregation method is special, in that
there is no need for a ``data series'' on the right; the imported
series is simply a function of the specified key(s). All the other
methods require that ``actual data'' are found on the right.  Also
note that when \texttt{count} is used, the value returned when no
match is found is (as one might expect) zero rather than \texttt{NA}.

With 1:$n$ matching, it may be that the number of matches differs
across the observations on the left. The \texttt{seq} method returns
\texttt{NA} when the (1-based) sequence parameter $i$ exceeds the
number of matches for a given observation.

Referring again to the data in Table~\ref{tab:csv}, suppose we want to
import data from the persons file into a dataset established at
household level.  Here's an example where we use the individual
\texttt{age} data from \texttt{people.csv} to add the average and
minimum age of household members.
\begin{code}
open hholds.csv --quiet
join people.csv avgage --ikey=hid --data=age --aggr=avg
join people.csv minage --ikey=hid --data=age --aggr=min
\end{code}

Here's a further example where we add to the household data the sum of
the personal data \texttt{xp}, with the twist that we apply filters to
get the sum specifically for household members under the age of 40,
and for women.
\begin{code}
open hholds.csv --quiet
join people.csv young_xp --ikey=hid --filter="age<40" --data=xp --aggr=sum
join people.csv female_xp --ikey=hid --filter="gender==F" --data=xp --aggr=sum
\end{code}

\section{String-valued key variables}
\label{sec:join-strings}

The examples above use numerical variables (household and individual
ID numbers) in the matching process. It is also possible to use
string-valued variables, in which case a match means that the string
values of the key variables compare equal (with case sensitivity). 
When using double keys, you can mix numerical and string keys, but
naturally you cannot mix a string variable on the left (via
\texttt{ikey}) with a numerical one on the right (via \texttt{okey}),
or vice versa.

Here's a simple example. Suppose that alongside \texttt{hholds.csv} we
have a file \texttt{countries.csv} with the following content:
\begin{code}
 country,GDP
 UK,100
 US,500
 IT,150
 FR,180
\end{code}

The variable \texttt{country}, which is also found in
\texttt{hholds.csv}, is string-valued. We can pull the GDP of the
country in which the household resides into our households dataset
with
\begin{code}
open hholds.csv -q
join countries.csv GDP --ikey=country
\end{code}
which gives
\begin{code}
           hid      country          GDP

1            1            1          500
2            6            2          150
3            3            3          100
4            4            2          150
5            2            1          500
6            5            2          150
\end{code}

\section{Right-hand column names}
\label{sec:join-colnames}

As noted above, the column names in the datafile from which data
are to be joined do not have to be valid gretl identifiers (that
is, composed of nothing but ASCII letters, numbers and the underscore
character, starting with a letter and with a maximum length of 15
characters). However, there are limits. 
\begin{itemize}
\item Names that include a comma will break the double-key
  mechanism, in which a comma is used to separate the two keys in
  the \texttt{okey} option.
\item Names that include any of the characters used in the operators
  for the \texttt{filter} option will break that option.
\item In reading a delimited text file, gretl truncates strings at
  71 bytes. 
\end{itemize}

\section{A real-world case}
\label{sec:join-SHIW}

For a real use-case for \texttt{join}, we turn to the Bank of Italy's
\textit{Survey on Household Income and Wealth}
(SHIW).\footnote{Details of the survey can be found at
  \url{http://www.bancaditalia.it/statistiche/indcamp/bilfait/dismicro}.
  The ASCII (CSV) data files for the 2010 survey are available at
  \url{http://www.bancaditalia.it/statistiche/indcamp/bilfait/dismicro/annuale/ascii/ind10_ascii.zip}.}
In ASCII form the 2010 survey results comprise 47\,MB of data in 29
files. In this exercise we will draw on five of the SHIW files to
construct a replica of the dataset used in Thomas Mroz's famous paper
(\textit{Econometrica}, 1987) on women's labor force participation,
which contains data on married women between the age of 30 and 60
along with certain characteristics of their households and husbands.

Our general strategy is as follows: we create a ``core'' dataset by
opening the file \texttt{carcom10.csv}, which contains basic data on
the individuals. After dropping unwanted individuals (all but married
women), we use the resulting dataset as a base for pulling in further
data via the \texttt{join} command.

The complete script to do the job is given in the Appendix to this
chapter; here we walk through the script with comments interspersed.
We assume that all the relevant files from the Bank of Italy survey
are contained in a subdirectory called \texttt{SHIW}.

Starting with \texttt{carcom10.csv}, we use the \verb|--cols| option
to the \texttt{open} command to import specific series,\footnote{This
  is a new usage of \texttt{cols}; see section~\ref{sec:join-cols} for
  details.} namely \texttt{NQUEST} (household ID number),
\texttt{NORD} (sequence number for individuals within each household),
\texttt{SEX} (male = 1, female = 2), \texttt{PARENT} (status in
household: 1 = head of household, 2 = spouse of head, etc.),
\texttt{STACIV} (marital status: married = 1), \texttt{STUDIO}
(educational level, coded from 1 to 8) and \texttt{ETA} (age in
years).
%
\begin{code}
open SHIW/carcom10.csv --cols=1,2,3,4,9,10,29
\end{code}
%
We then restrict the sample to married women from 30 to 60 years of
age, and additionally restrict the sample of women to those who are
either heads of households or spouses of the head.
%
\begin{code}
smpl SEX==2 && ETA>=30 && ETA<=60 && STACIV==1 --restrict
smpl PARENT<3  --restrict
\end{code}
%
For compatibility with the Mroz dataset as presented in the gretl
datafile \texttt{mroz87.gdt}, we rename the age and education
variables as \texttt{WA} and \texttt{WE} respectively, then we
store the reduced base dataset in gretl format.
%
\begin{code}
rename ETA WA
rename STUDIO WE
store mroz_rep.gdt
\end{code}

The next step will be to get data on working hours from the jobs file
\texttt{allb1.csv}. There's a complication here. We need the total
hours worked over the course of the year (for both the women and their
husbands). This is not available as such, but the variables
\texttt{ORETOT} and \texttt{MESILAV} give, respectively, average hours
worked per week and the number of months worked in 2010, each on a
per-job basis. If each person held at most one job over the year we
could compute his or her annual hours as
\begin{code}
HRS = ORETOT * 52 * MESILAV/12
\end{code}
However, some people had more than one job, and in this case what we
want is the sum of annual hours across their jobs.  We could use
\texttt{join} with the \texttt{seq} aggregation method to construct
this sum, but it is probably more straightforward to read the
\texttt{allb1} data, compute the \texttt{HRS} values per job as shown
above, and save the results to a temporary CSV file.
%
\begin{code}
open SHIW/allb1.csv --cols=1,2,8,11 --quiet
series HRS = misszero(ORETOT) * 52 * misszero(MESILAV)/12
store HRS.csv NQUEST NORD HRS
\end{code}

Now we can reopen the base dataset and join the hours variable from
\texttt{HRS.csv}.  Note that we need a double key here: the women are
uniquely identified by the combination of \texttt{NQUEST} and
\texttt{NORD}. We don't need an \texttt{okey} specification since
these keys go under the same names in the right-hand file. We define
labor force participation, \texttt{LFP}, based on hours.
%
\begin{code}
open mroz_rep.gdt
join HRS.csv WHRS --ikey=NQUEST,NORD --data=HRS --aggr=sum
WHRS = misszero(WHRS)
LFP = WHRS > 0
\end{code}
%
For reference, here's how we could have used \texttt{seq} to avoid
writing a temporary file:
%
\begin{code}
join SHIW/allb1.csv njobs --ikey=NQUEST,NORD --data=ORETOT --aggr=count
series WHRS = 0
loop i=1..max(njobs) -q
  join SHIW/allb1.csv htmp --ikey=NQUEST,NORD --data=ORETOT --aggr="seq:$i"
  join SHIW/allb1.csv mtmp --ikey=NQUEST,NORD --data=MESILAV --aggr="seq:$i"
  WHRS += misszero(htmp) * 52 * misszero(mtmp)/12 
endloop
\end{code}

To generate the work experience variable, \texttt{AX}, we use the file
\texttt{lavoro.csv}: this contains a variable named \texttt{ETALAV}
which records the age at which the person first started work.
%
\begin{code}
join SHIW/lavoro.csv ETALAV --ikey=NQUEST,NORD
series AX = misszero(WA - ETALAV)
\end{code}
%
We compute the woman's hourly wage, \texttt{WW}, as the ratio of total
employment income to annual working hours.  This requires drawing the
series \texttt{YL} (payroll income) and \texttt{YM} (net
self-employment income) from the persons file \texttt{rper10.csv}.
%
\begin{code}
join SHIW/rper10.csv YL --ikey=NQUEST,NORD --aggr=sum
join SHIW/rper10.csv YM --ikey=NQUEST,NORD --aggr=sum
series WW = LFP ? (YL + YM)/WHRS : 0
\end{code}
%
The family's net disposable income is available as \texttt{Y} in the file
\texttt{rfam10.csv}; we import this as \texttt{FAMINC}.
%
\begin{code}
join SHIW/rfam10.csv FAMINC --ikey=NQUEST --data=Y
\end{code}
%
Data on number of children are now obtained by applying the
\texttt{count} method. For the Mroz replication we want the number of
children under the age of 6, and also the number aged 6 to 18.
%
\begin{code}
join SHIW/carcom10.csv KIDS --ikey=NQUEST --aggr=count --filter="ETA<=18"
join SHIW/carcom10.csv KL6 --ikey=NQUEST --aggr=count --filter=ETA<6
series K618 = KIDS - KL6
\end{code}
%
We want to add data on the women's husbands, but how do we find them?
To do this we create an additional inner key which we'll call
\verb|H_ID| (husband ID), by sub-sampling in turn on the observations
falling into each of two classes: (a) those where the woman is
recorded as head of household and (b) those where the husband has that
status. In each case we want the individual ID (\texttt{NORD}) of the
household member whose status is complementary to that of the woman in
question. So for case (a) we subsample using \texttt{PARENT==1} (head
of household) and filter the join using \texttt{PARENT==2} (spouse of
head); in case (b) we do the converse. We thus construct \verb|H_ID|
piece-wise.\footnote{In principle the husband ID variable could be
  constructed without subsampling if more complex filtering were
  supported, but as noted above the \texttt{join} filter is at present
  limited to a binary expression involving only right-hand side
  variables and constants.}
%
\begin{code}
# for women who are household heads
smpl PARENT==1 --restrict --replace
join SHIW/carcom10.csv H_ID --ikey=NQUEST --data=NORD --filter="PARENT==2"
# for women who are not household heads
smpl PARENT==2 --restrict --replace
join SHIW/carcom10.csv H_ID --ikey=NQUEST --data=NORD --filter="PARENT==1"
smpl full
\end{code}
%
Now we can use our new inner key to retrieve the husbands' data,
matching \verb|H_ID| on the left with \texttt{NORD} on the right
within each household.
%
\begin{code}
join SHIW/carcom10.csv HA --ikey=NQUEST,H_ID --okey=NQUEST,NORD --data=ETA
join SHIW/carcom10.csv HE --ikey=NQUEST,H_ID --okey=NQUEST,NORD --data=STUDIO
join HRS.csv HHRS --ikey=NQUEST,H_ID --okey=NQUEST,NORD --data=HRS --aggr=sum
HHRS = misszero(HHRS)
\end{code}
%
The remainder of the script is straightforward and does not require
discussion here: we recode the education variables for compatibility;
delete some intermediate series that are not needed any more; add
informative labels; and save the final product. See the Appendix for
details.

To compare the results from this dataset with those from the earlier
US data used by Mroz, one can copy the input file \texttt{heckit.inp}
(supplied with the gretl package) and substitute \verb|mroz_rep.gdt|
for \texttt{mroz87.gdt}.\footnote{One other change is needed: in
  William Greene's specification of the heckit model an additional
  regressor named \texttt{CIT} is used. You can simply comment this
  out.} It turns out that the results are qualitatively very similar.

\section{A related change}
\label{sec:join-cols}

Alongside the introduction of the \texttt{join} command, some changes
have been made to the operation of the \texttt{open} command. The one
that concerns us here (and the only one that is backward-incompatible)
is a revised interpretation of the \verb|--cols| option. 

Up to gretl 1.9.9, this option was specific to the reading of
fixed-format datafiles. In that context the parameter to the
option---a list of comma-separated numbers---was interpreted as a set
of pairs of the form (start, width), telling gretl at which byte on
each line to start reading, and how many bytes to read, to get the
value for a given variable.

As of gretl 1.9.10, the facility just described is still available but
the option has been renamed as \verb|--fixed-cols|. The parameter to
the plain \verb|--cols| option is now interpreted as a (1-based) list
of the columns to be read from a delimited text datafile (e.g.\
CSV). Note that if the first column in a CSV file is an observations
column, it is not included in the count: the first actual data column
is treated as column 1.

\section*{Appendix: the full Mroz data script}

\begin{code}
# start with everybody; get gender, age and a few other variables 
# directly while we're at it
open SHIW/carcom10.csv --cols=1,2,3,4,9,10,29

# subsample on married women between the ages of 30 and 60
smpl SEX==2 && ETA>=30 && ETA<=60 && STACIV==1 --restrict
# for simplicity, restrict to heads of households and their spouses
smpl PARENT<3  --restrict

# rename the age and education variables for compatibility; then
# save the reduced base dataset
rename ETA WA
rename STUDIO WE
store mroz_rep.gdt

# make a temp file holding annual hours worked per job
open SHIW/allb1.csv --cols=1,2,8,11 --quiet
series HRS = misszero(ORETOT) * 52 * misszero(MESILAV)/12
store HRS.csv NQUEST NORD HRS

# reopen the base dataset and begin drawing assorted data in
open mroz_rep.gdt

# women's annual hours (summed across jobs) 
join HRS.csv WHRS --ikey=NQUEST,NORD --data=HRS --aggr=sum
WHRS = misszero(WHRS)

# labor force participation
LFP = WHRS > 0

# work experience: ETALAV = age when started first job
join SHIW/lavoro.csv ETALAV --ikey=NQUEST,NORD
series AX = misszero(WA - ETALAV)

# women's hourly wages
join SHIW/rper10.csv YL --ikey=NQUEST,NORD --aggr=sum
join SHIW/rper10.csv YM --ikey=NQUEST,NORD --aggr=sum
series WW = LFP ? (YL + YM)/WHRS : 0

# family income (Y = net disposable income)
join SHIW/rfam10.csv FAMINC --ikey=NQUEST --data=Y

# get data on children using the "count" method
join SHIW/carcom10.csv KIDS --ikey=NQUEST --aggr=count --filter="ETA<=18"
join SHIW/carcom10.csv KL6 --ikey=NQUEST --aggr=count --filter=ETA<6
series K618 = KIDS - KL6

# data on husbands: we first construct an auxiliary inner key for 
# husbands, using the little trick of subsampling the inner dataset
#
# for women who are household heads
smpl PARENT==1 --restrict --replace
join SHIW/carcom10.csv H_ID --ikey=NQUEST --data=NORD --filter="PARENT==2"
# for women who are not household heads
smpl PARENT==2 --restrict --replace
join SHIW/carcom10.csv H_ID --ikey=NQUEST --data=NORD --filter="PARENT==1"
smpl full

# add husbands' data via the newly-added secondary inner key
join SHIW/carcom10.csv HA --ikey=NQUEST,H_ID --okey=NQUEST,NORD --data=ETA
join SHIW/carcom10.csv HE --ikey=NQUEST,H_ID --okey=NQUEST,NORD --data=STUDIO
join HRS.csv HHRS --ikey=NQUEST,H_ID --okey=NQUEST,NORD --data=HRS --aggr=sum
HHRS = misszero(HHRS)

# final cleanup begins

# recode educational attainment as years of education
matrix eduyrs = {0, 5, 8, 11, 13, 16, 18, 21}
series WE = replace(WE, seq(1,8), eduyrs)
series HE = replace(HE, seq(1,8), eduyrs)

# cut some cruft
delete SEX STACIV KIDS YL YM PARENT H_ID ETALAV

# add some labels for the series
setinfo LFP -d "1 if woman worked in 2010"
setinfo WHRS -d "Wife's hours of work in 2010
setinfo KL6 -d "Number of children less than 6 years old in household"
setinfo K618 -d "Number of children between ages 6 and 18 in household"
setinfo WA -d "Wife's age"
setinfo WE -d "Wife's educational attainment, in years"
setinfo WW -d "Wife's average hourly earnings, in 2010 euros"
setinfo HHRS -d "Husband's hours worked in 2010"
setinfo HA -d "Husband's age"
setinfo HE -d "Husband's educational attainment, in years"
setinfo FAMINC -d "Family income, in 2010 euros"
setinfo AX -d "Actual years of wife's previous labor market experience"

# save the final product
store mroz_rep.gdt
\end{code}

