
\chapter{Going abroad}
% \section{The \texttt{set} command}
% \label{chap:settings}

% \section{Function packages}

Gretl is designed to interact as nicely as possible with other
programs which are likely to be of interest to the applied
econometrician. Most of the infrastructure that gretl provides for
this purpose carries over to hansl.

\section{Interaction with the underlying OS}

You can run external commands from a hansl script via the \cmd{launch}
and \cmd{!} commands.\footnote{Yes, that is an exclamation mark.} They
both pass the subsequent string to the OS, to be run asynchronously
or synchronously, respectively.

For example, on Unix-like systems (Linux and OS X), the code
\begin{code}
! echo "a(1) * 4" | bc -l && echo "Thats's pi!"
\end{code}

should yield

\begin{code}
3.14159265358979323844
Thats's pi!
\end{code}

If you feel playful, try \texttt{launch xeyes} on a system with X11
enabled.

\section{\LaTeX{} and \app{gnuplot}}

\LaTeX{} and \textsf{gnuplot} are both called by gretl in various GUI
contexts (e.g.\ displaying estimation results in nicely typeset form,
and displaying plots). It is also possible to exploit their
functionality via hansl scripting. 

As regards \LaTeX, the relevant commands are \cmd{tabprint} and
\cmd{modprint}. The first of these is used to generate a tabular
representation of the results from the last estimation command. As the 
\GCR\ details, this can be directed to a named file and the 
particulars of the \TeX\ output can be controlled (to an extent) via
the option flags \option{complete} and \option{format}.

FIXME: TODO

\begin{itemize}
\item For \LaTeX: briefly describe the \cmd{tabprint} and
  \cmd{modprint} commands.
\item For \app{gnuplot}: give a little example of the \cmd{gnuplot}
  command with the \option{input} and \option{output} options.
\end{itemize}

\section{The \cmd{foreign} block}

It is possible to embed ``foreign'' code in a hansl script and have it
passed to the appropriate program, so as to outsource certain
operations to other software and collect the results back.

While we have to refer the reader to the \GUG\ for a comprehensive
description of this feature, in a nutshell hansl recognizes as
possible counterparts the following external programs: \textsf{R},
\textsf{Octave}, \textsf{Ox}, \textsf{Python} and \textsf{Stata}.  For
example, the following piece of code
\begin{code}
foreign language=python
	import sys
	print(sys.version)
end foreign
\end{code}
should yield something like
\begin{code}
2.7.5+ (default, Sep 17 2013, 15:31:50) 
[GCC 4.8.1]
\end{code}
where the lines between \cmd{foreign} and \cmd{end foreign} were sent
literally to \textsf{python} and the result was displayed as if it
were gretl's own output. It goes without saying that, in order for the
above to work, \textsf{python} must be installed on your system and
gretl should be configured appropriately to find
it.\footnote{Configuration is done via the gretl GUI, under the
\textsf{Preferences} menu.}

Each of the possible targets of the \cmd{foreign} command has its own
peculiarities and quirks, so in order to make effective use of this
feature you should read the appropriate chapters of the \GUG. One word
of caution, however, applies generally: \emph{execution is
  synchronous}, so if the ``foreign'' programs hangs for some reason,
your hansl script will, too.

\chapter{Tricks}

\section{String substitution}
\label{sec:stringsub}

\textsl{This section comes with a health warning. ``String
  substitution'' affords great flexibility, and is part of the current
  hansl idiom, but carries a significant cost in terms of efficiency;
  it is no accident that this mechanism is not supported by
  ``serious'' programming languages. It is likely that in future we
  will provide alternative methods for achieving effects that at
  present can only be achieved by string substitution, and if so we
  may remove this facility altogether. That said, let's take a look at
  the current state of affairs\dots{}}

String variables can be used in two ways in hansl scripting: the name
of the variable can be typed ``as is'', or it may be preceded by the
``at'' sign, \verb|@|. In the first variant the named string is
treated as a variable in its own right, while the second calls for
``string substitution''. Which of these variants is appropriate
depends on the context.

In the following contexts the names of string variables should always
be given in plain form (without the ``at'' sign):

\begin{itemize}
\item When such a variable appears among the arguments to the
  commands \texttt{printf} or \texttt{sprintf}.
\item When such a variable is given as the argument to a function.
\item On the right-hand side of a \texttt{string} assignment.
\end{itemize}

Here is an illustration of (interactive) use of a named string
argument with \texttt{printf}:
%
\begin{code}
? string vstr = "variance"
Generated string vstr
? printf "vstr: %12s\n", vstr
vstr:     variance
\end{code}

String substitution, on the other hand, can be used in contexts where
a string variable is not acceptable as such. If gretl encounters
the symbol \verb|@| followed directly by the name of a string
variable, this notation is in effect treated as a ``macro'': the value
of the variable is sustituted literally into the command line before
the regular parsing of the command is carried out.

One common use of string substitution is when you want to construct
and use the name of a series programatically. For example, suppose you
want to create 10 random normal series named \texttt{norm1} to
\texttt{norm10}. This can be accomplished as follows.
%
\begin{code}
string sname = null
loop i=1..10
  sprintf sname "norm%d", i
  series @sname = normal()
endloop
\end{code}
%
Note that plain \texttt{sname} could not be used in the second line
within the loop: the effect would be to attempt to overwrite the
string variable named \texttt{sname} with a series of the same name,
hence generating an error. What we want is for the current
\textit{value} of \texttt{sname} to be dumped directly into the
command that defines a series, and the ``\verb|@|'' notation achieves
that.

Another typical use of string substitution is when you want the
options used with a particular command to vary depending on
some condition. For example,
%
\begin{code}
function void use_optstr (series y, list xlist, bool verbose)
   string optstr = verbose ? "" : "--simple-print"
   ols y xlist @optstr 
end function

open data4-1
list X = const sqft
use_optstr(price, X, 1)
use_optstr(price, X, 0)
\end{code}

In the first call to the function \texttt{use\_optstr} the option
string \verb|--simple-print| will be appended to the \cmd{ols}
command; in the second call it will be omitted.

When printing the value of a string variable using the \texttt{print}
command, the plain variable name should generally be used, as in
%
\begin{code}
string s = "Just testing"
print s
\end{code}
%
The following variant is equivalent, though clumsy and definitely not
recommended.
%
\begin{code}
string s = "Just testing"
print "@s"
\end{code}
%
But note that this next variant does something quite different.
%
\begin{code}
string s = "Just testing"
print @s
\end{code}
%
After string substitution, the command reads
%
\begin{code}
print Just testing
\end{code}
%
which would be taken as a request to print the values of two
variables, \texttt{Just} and \texttt{testing}.

\label{LastPage}

%%% Local Variables: 
%%% mode: latex
%%% TeX-master: "hansl-primer"
%%% End: 