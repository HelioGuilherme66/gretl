\chapter{The \texttt{set} command}
\label{chap:settings}

\chapter{Function packages}

\chapter{Going abroad}
``Foreign'' blocks.


Friendly relationship with \LaTeX{} and \app{gnuplot}


\chapter{Tricks}
\section{String substitution}
\label{sec:stringsub}

String variables can be used in two ways in hansl scripting: the name
of the variable can be typed ``as is'', or it may be preceded by the
``at'' sign, \verb|@|. In the first variant the named string is
treated as a variable in its own right, while the second calls for
``string substitution''. Which of these variants is appropriate
depends on the context.

In the following contexts the names of string variables should be
given in plain form (without the ``at'' sign):

\begin{itemize}
\item When such a variable appears among the arguments to the
  commands \texttt{printf} or \texttt{sprintf}.
\item When such a variable is given as the argument to a function.
\item On the right-hand side of a \texttt{string} assignment.
\end{itemize}

Here is an illustration of the use of a named string argument with
\texttt{printf}:
%
\begin{code}
? string vstr = "variance"
Generated string vstr
? printf "vstr: %12s\n", vstr
vstr:     variance
\end{code}

String substitution, on the other hand, can be used in contexts where
a string variable is not acceptable as such. If gretl encounters
the symbol \verb|@| followed directly by the name of a string
variable, this notation is in effect treated as a ``macro'': the value
of the variable is sustituted literally into the command line before
the regular parsing of the command is carried out.

One common use of string substitution is when you want to construct
and use the name of a series programatically. For example, suppose you
want to create 10 random normal series named \texttt{norm1} to
\texttt{norm10}. This can be accomplished as follows.
%
\begin{code}
string sname = null
loop i=1..10
  sprintf sname "norm%d", i
  series @sname = normal()
endloop
\end{code}
%
Note that plain \texttt{sname} could not be used in the second line
within the loop: the effect would be to attempt to overwrite the
string variable named \texttt{sname} with a series of the same name,
hence generating an error. What we want is for the current
\textit{value} of \texttt{sname} to be dumped directly into the
command that defines a series, and the ``\verb|@|'' notation achieves
that.

Another typical use of string substitution is when you want the
options used with a particular command to vary depending on
some condition. For example,
%
\begin{code}
function void use_optstr (series y, list xlist, int verbose)
   string optstr = verbose ? "" : "--simple-print"
   ols y xlist @optstr 
end function

open data4-1
list X = const sqft
use_optstr(price, X, 1)
use_optstr(price, X, 0)
\end{code}

In the first call to the function \texttt{use\_optstr} the option
string \verb|--simple-print| will be appended to the \cmd{ols}
command; in the second call it will be omitted.

When printing the value of a string variable using the \texttt{print}
command, the plain variable name should generally be used, as in
%
\begin{code}
string s = "Just testing"
print s
\end{code}
%
The following variant is equivalent, though clumsy and not
recommended.
%
\begin{code}
string s = "Just testing"
print "@s"
\end{code}
%
But note that this next variant does something quite different.
%
\begin{code}
string s = "Just testing"
print @s
\end{code}
%
After string substitution, the command reads
%
\begin{code}
print Just testing
\end{code}
%
which is taken as a request to print the values of two variables,
\texttt{Just} and \texttt{testing}.

\subsection{Performance hint}

Warning: gretl attempts to speed up execution of assignment
expressions within loops by ``pre-compiling'' the expression.
However, if such an expression contains string substitution it cannot
be pre-compiled. So be sure to use the \verb|@| operator only when
strictly necessary if you want to avoid a performance penalty.


\label{LastPage}

%%% Local Variables: 
%%% mode: latex
%%% TeX-master: "hansl-primer"
%%% End: 