
\chapter{Further topics}
% \section{The \texttt{set} command}
% \label{chap:settings}

% \section{Function packages}

\section{Going abroad}

Gretl is designed to interact as nicely as possible with other
programs which are likely to be of interest to the applied
econometrician. Most of the infrastructure that gretl provides for
this purpose carries over to hansl.

\subsection{Interaction with the underlying OS}

You can run external commands from a hansl script via the \cmd{launch}
and \cmd{!} commands\footnote{Yes, that is an exclamation mark}. They
both pass the subsequent string to the OS to be run asynchronously
and synchronously, respectively.

For example, on Unix-like systems (Linux and OSX), the code
\begin{code}
! echo "a(1) * 4" | bc -l && echo "Thats's pi!"
\end{code}

should yield

\begin{code}
3.14159265358979323844
Thats's pi!
\end{code}

If you feel playful, try \texttt{launch xeyes} on a system with X11
enabled.

\subsection{\LaTeX{} and \app{gnuplot}}

FIXME: TODO

\begin{itemize}
\item For \LaTeX: briefly describe the \cmd{tabprint} and
  \cmd{modprint} commands.
\item For \app{gnuplot}: give a little example of the \cmd{gnuplot}
  command with the \option{input} and \option{output} options.
\end{itemize}

\subsection{The \cmd{foreign} block}

It is possible to embed ``foreign'' code in a hansl script and pass it
to the appropriate program, so to outsource certain operations to
other software and collect the results back.

While we have to defer the reader to the \GUG\ for a comprehensive
description of this feature, an extremely concise description could be
the following: hansl recognizes as possible counterparts the following
list of external programs:
\begin{itemize}
\item R
\item Matlab and its dialects (\emph{e.g.}~ Octave)
\item Ox
\item Python
\item Stata
\end{itemize}

For example, the following piece of code
\begin{code}
foreign language=python
	import sys
	print(sys.version)
end foreign
\end{code}
should yield something like
\begin{code}
2.7.5+ (default, Sep 17 2013, 15:31:50) 
[GCC 4.8.1]
\end{code}
where the lines between \cmd{foreign} and \cmd{end foreign} were sent
literally to python and the result was displayed as if it were gretl's
own output. It goes without saying that, in order for the above to
work, python must be installed on your system and gretl should be
configured appropriately to find it.

Each of the possible partners to the \cmd{foreign} command has its own
peculiarities and quirks, so in order to make effective use of this
feature, you should read the appropriate chapters of the \GUG. One
word of caution, however, applies generally: \textbf{execution is
  synchronous}, so if the ``foreign'' programs hangs for some reason,
your hansl script will, too. 
\section{Tricks}
\subsection{String substitution}
\label{sec:stringsub}

String variables can be used in two ways in hansl scripting: the name
of the variable can be typed ``as is'', or it may be preceded by the
``at'' sign, \verb|@|. In the first variant the named string is
treated as a variable in its own right, while the second calls for
``string substitution''. Which of these variants is appropriate
depends on the context.

In the following contexts the names of string variables should be
given in plain form (without the ``at'' sign):

\begin{itemize}
\item When such a variable appears among the arguments to the
  commands \texttt{printf} or \texttt{sprintf}.
\item When such a variable is given as the argument to a function.
\item On the right-hand side of a \texttt{string} assignment.
\end{itemize}

Here is an illustration of the use of a named string argument with
\texttt{printf}:
%
\begin{code}
? string vstr = "variance"
Generated string vstr
? printf "vstr: %12s\n", vstr
vstr:     variance
\end{code}

String substitution, on the other hand, can be used in contexts where
a string variable is not acceptable as such. If gretl encounters
the symbol \verb|@| followed directly by the name of a string
variable, this notation is in effect treated as a ``macro'': the value
of the variable is sustituted literally into the command line before
the regular parsing of the command is carried out.

One common use of string substitution is when you want to construct
and use the name of a series programatically. For example, suppose you
want to create 10 random normal series named \texttt{norm1} to
\texttt{norm10}. This can be accomplished as follows.
%
\begin{code}
string sname = null
loop i=1..10
  sprintf sname "norm%d", i
  series @sname = normal()
endloop
\end{code}
%
Note that plain \texttt{sname} could not be used in the second line
within the loop: the effect would be to attempt to overwrite the
string variable named \texttt{sname} with a series of the same name,
hence generating an error. What we want is for the current
\textit{value} of \texttt{sname} to be dumped directly into the
command that defines a series, and the ``\verb|@|'' notation achieves
that.

Another typical use of string substitution is when you want the
options used with a particular command to vary depending on
some condition. For example,
%
\begin{code}
function void use_optstr (series y, list xlist, int verbose)
   string optstr = verbose ? "" : "--simple-print"
   ols y xlist @optstr 
end function

open data4-1
list X = const sqft
use_optstr(price, X, 1)
use_optstr(price, X, 0)
\end{code}

In the first call to the function \texttt{use\_optstr} the option
string \verb|--simple-print| will be appended to the \cmd{ols}
command; in the second call it will be omitted.

When printing the value of a string variable using the \texttt{print}
command, the plain variable name should generally be used, as in
%
\begin{code}
string s = "Just testing"
print s
\end{code}
%
The following variant is equivalent, though clumsy and not
recommended.
%
\begin{code}
string s = "Just testing"
print "@s"
\end{code}
%
But note that this next variant does something quite different.
%
\begin{code}
string s = "Just testing"
print @s
\end{code}
%
After string substitution, the command reads
%
\begin{code}
print Just testing
\end{code}
%
which is taken as a request to print the values of two variables,
\texttt{Just} and \texttt{testing}.

\subsection{Performance hint}

Warning: gretl attempts to speed up execution of assignment
expressions within loops by ``pre-compiling'' the expression.
However, if such an expression contains string substitution it cannot
be pre-compiled. So be sure to use the \verb|@| operator only when
strictly necessary if you want to avoid a performance penalty.


\label{LastPage}

%%% Local Variables: 
%%% mode: latex
%%% TeX-master: "hansl-primer"
%%% End: 