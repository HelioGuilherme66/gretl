\documentclass[a4paper]{article}
\usepackage[a4paper]{geometry}
\usepackage{smallhds,simplebib}
\usepackage[pdftex]{hyperref}

\usepackage{bm}
\usepackage{verbatim}

\newcommand{\obsvec}{\ensuremath\mathbf{y}}
\newcommand{\obsmat}{\ensuremath\mathbf{H}}
\newcommand{\obsx}{\ensuremath\mathbf{x}}
\newcommand{\obsxmat}{\ensuremath\mathbf{A}}
\newcommand{\obsdist}{\ensuremath\mathbf{w}}
\newcommand{\obsvar}{\ensuremath\mathbf{R}}

\newcommand{\statevec}{\ensuremath\mathbf{\bm \xi}}
\newcommand{\statemat}{\ensuremath\mathbf{F}}
\newcommand{\strdist}{\ensuremath\mathbf{v}}
\newcommand{\strvar}{\ensuremath\mathbf{Q}}
\newcommand{\statevar}{\ensuremath\mathbf{P}}
\newcommand{\gain}{\ensuremath\mathbf{K}}

\newcommand{\prederr}{\ensuremath\mathbf{e}}
\newcommand{\predvar}{\ensuremath{\bm\Sigma}}

\newcommand{\myvec}{\mbox{vec}}

%% \setcounter{secnumdepth}{2}

\title{The Kalman filter in gretl}
\author{Allin and Jack}

\begin{document}

\maketitle

\section{Preamble}
\label{sec:amble}

The Kalman filter has been used ``behind the scenes'' in gretl for
quite some time, in computing ARMA estimates.  But user access to the
Kalman filter is new and it has not yet been tested to any great
extent.  We have run some tests of relatively simple cases against the
benchmark of \textsf{SsfPack Basic}.  This is state-space software
written by Koopman, Shephard and Doornik and documented in Koopman
\textit{et al} (1999).  It requires Doornik's \textsf{ox} program.
Both \textsf{ox} and \textsf{SsfPack} are available as free downloads
for academic use but neither is open-source; see
\url{http://www.ssfpack.com}.  Since Koopman is one of the leading
researchers in this area, presumably the results from \textsf{SsfPack}
are generally reliable.  To date we have been able to replicate the
\textsf{SsfPack} results in gretl with a high degree of precision.

We welcome both success reports and bug reports.

\section{Notation}

It seems that in econometrics everyone is happy with $y = X \beta +
u$, but we can't, as a community, make up our minds on a standard
notation for state-space models. Harvey (1991), Hamilton (1994),
Harvey and Proietti (2005) and Pollock (1999) all use different
conventions. The notation used here is based on James Hamilton's, with
slight variations.

A state-space model can be written as
\begin{eqnarray}
  \label{eq:state}
  \statevec_{t} & = & \statemat_{t} \statevec_{t-1} + \strdist_t \\
  \label{eq:obs}
  \obsvec_t & = & \obsxmat_t'\obsx_t + \obsmat_t' \statevec_t + \obsdist_t
\end{eqnarray}
where (\ref{eq:state}) is the state transition equation and
(\ref{eq:obs}) is the observation or measurement equation.  The state
vector, $\statevec_{t}$, is ($r \times 1$) and the vector of
observables, $\obsvec_t$, is ($n \times 1$); $\mathbf{x}_t$ is a ($k
\times 1$) vector of exogenous variables.  The ($r \times 1$) vector
$\strdist_t$ and the ($n \times 1$) vector $\obsdist_t$ are assumed to
be vector white noise\footnote{At this stage, $\strdist_t$ and
  $\obsdist_t$ are assumed to be independent; there may be a case for
  a more general treatment of the innovations, by specifying
  $\strdist_t = \mathbf{M} {\bm \varepsilon_t}$ and $\obsdist_t =
  \mathbf{N} {\bm \varepsilon_t}$, as done in \textsf{SsfPack}, but
  this is not implemented so far.}:
\begin{eqnarray*}
E(\strdist_t \strdist_s') = \strvar_t \mbox{ for } t = s, 
    \mbox{ otherwise } \mathbf{0} \\
E(\obsdist_t \obsdist_s') = \obsvar_t \mbox{ for } t = s, 
    \mbox{ otherwise } \mathbf{0}
\end{eqnarray*}
%
The number of time-series observations is denoted by $T$.

In the special case when $\statemat_t = \statemat$, $\obsmat_t = \obsmat$,
$\obsxmat_t = \obsxmat$, $\strvar_t = \strvar$ and $\obsvar_t =
\obsvar$, the model is said to be \emph{time-invariant}.

\subsection{The Kalman recursions}

Using this notation the Kalman recursions can be written as follows.
(We omit time subscripts on the coefficient matrices for convenience.)

Initialization is via the unconditional mean and variance of
$\statevec_1$:
%
\begin{eqnarray*}
\hat{\statevec}_{1|0} &=& E(\statevec_1) \\
\statevar_{1|0} &=& E\left\{\left[\statevec_1 - E(\statevec_1)\right]
   \left[\statevec_1 - E(\statevec_1)\right]' \right\}
\end{eqnarray*}
Usually these are given by $\hat{\statevec}_{1|0} = \mathbf{0}$ and
\begin{equation}
\label{eq:inivar}
\myvec(\statevar_{1|0}) = \left[\mathbf{I}_{r^2} - \statemat \otimes
  \statemat\right]^{-1} \cdot \myvec(\strvar)
\end{equation}
but see below for further discussion of the initial variance.

Iteration then proceeds in two steps.  First we update the estimate of
the state\footnote{For a justification of these formulae see the
  classic book by Anderson and Moore (1979); for a more modern
  treatment see Pollock (1999) or Hamilton (1994).  A transcription of
  R. E. Kalman's original paper of 1960 is available at
  \url{http://www.cs.unc.edu/~welch/kalman/kalmanPaper.html}}
%
\begin{equation}
\hat{\statevec}_{t+1|t} = \statemat\hat{\statevec}_{t|t-1} + 
  \gain_t \prederr_t
\end{equation}
%
where $\gain_t$ is the gain matrix, given by
%
\[
\gain_t = \statemat\statevar_{t|t-1}\obsmat
  (\obsmat'\statevar_{t|t-1}\obsmat + \obsvar)^{-1}
\]
%
and $\prederr_t$ is the prediction error for the observable:
\[
\prederr_t = \obsvec_t - \obsxmat'\obsx_t - \obsmat' \hat{\statevec}_{t|t-1}
\]

The second step updates the estimate of the variance of the state,
$\statevar$.  Let us define $\predvar_t =
\obsmat'\statevar_{t|t-1}\obsmat + \obsvar$.  The update is then
\begin{equation}
\statevar_{t+1|t} = \statemat\left[\statevar_{t|t-1} -
 \statevar_{t|t-1}\obsmat \predvar_t^{-1}
 \obsmat' \statevar_{t|t-1} \right] \statemat' + \strvar
\end{equation}


\section{Intended usage}

The Kalman filter can be used in three ways: two of these are the
classic forward and backward pass, or filtering and smoothing
respectively; the third use is simulation.  In the
filtering/smoothing case you have the data $\obsvec_t$ and you want to
reconstruct the states $\statevec_t$ (and the forecast errors as a
by-product), but we may also have a computational apparatus that does
the reverse: given artificially-generated series $\obsdist_t$ and
$\strdist_t$, generate the states $\statevec_t$ (and the observables
$\obsvec_t$ as a by-product).

The usefulness of the classical filter is well known; the usefulness
of the Kalman filter as a simulation tool may be huge too. Think for
instance of Monte Carlo experiments, simulation-based inference---see
Gourieroux--Monfort (1996)---or Bayesian methods, especially in the
context of the estimation of DSGE models.

\section{Overview of syntax}

Using the Kalman filter in gretl is a two-step process.  First you set
up your filter, using a block of commands starting with
\texttt{kalman} and ending with \texttt{end kalman}---much like the
\texttt{gmm} command.  Then you invoke the functions \texttt{kfilter},
\texttt{ksmooth} or \texttt{ksimul} to do the actual work.  The next
two sections expand on these points.

\section{Defining the filter}

Each line within the \texttt{kalman} \dots{} \texttt{end kalman} block 
takes the form

\vspace{1ex}
\textsl{keyword} \textsl{value}
\vspace{1ex}

\noindent where \textsl{keyword} represents a matrix, as shown 
below.
% % in Table~\ref{tab:keywords}.

%% \begin{table}[htbp]
\begin{center}
\begin{tabular}{lcc}
Keyword & Symbol & Dimensions \\[6pt]
\texttt{obsy}     & $\obsvec$         & $T \times n$ \\
\texttt{obsymat}  & $\obsmat$         & $r \times n$ \\
\texttt{obsx}     & $\obsx$           & $T \times k$ \\
\texttt{obsxmat}  & $\obsxmat$        & $k \times n$ \\ 
\texttt{obsvar}   & $\obsvar$         & $n \times n$ \\
\texttt{statemat} & $\statemat$         & $r \times r$ \\
\texttt{statevar} & $\strvar$         & $r \times r$ \\
\texttt{inistate} & $\hat{\statevec}_{1|0}$  & $r \times 1$ \\
\texttt{inivar}   & $\statevar_{1|0}$ & $r \times r$ \\
\end{tabular}
%% \caption{Kalman block keywords}
%% \label{tab:keywords}
\end{center}
%% \end{table}

For the data matrices $\obsvec$ and $\obsx$ the corresponding
\textsl{value} may be the name of a predefined matrix, the name of a
data series, or the name of a list of series.\footnote{Note that the
  data matrices \texttt{obsy} and \texttt{obsx} have $T$ rows.  That
  is, the column vectors $\obsvec_t$ and $\obsx_t$ in (\ref{eq:state})
  and (\ref{eq:obs}) are in fact the transposes of the $t$-dated rows
  of the full matrices.}  For the other inputs, \textsl{value} may be
the name of a predefined matrix or, if the input in question happens
to be ($1 \times $1), the name of a scalar variable or a numerical
constant.  Details on the specification of time-varying matrices
are given below.

Not all of these inputs need be specified in every case; some are
optional. (In addition, you can specify the matrices in any order.)
The mandatory elements are $\obsvec$, $\obsmat$, $\statemat$ and
$\strvar$, so the minimal \texttt{kalman} block looks like this:

\begin{verbatim}
kalman 
  obsy y
  obsymat H
  statemat F
  statevar Q
end kalman
\end{verbatim} 

The optional matrices are listed below, along with the implication
of omitting the given matrix.

\begin{center}
\begin{tabular}{ll}
Keyword & If omitted\dots \\ [6pt]
\texttt{obsx} & no exogenous variables in observation equation\\
\texttt{obsxmat} & no exogenous variables in observation equation\\
\texttt{obsvar} & no disturbance term in observation equation\\
\texttt{inistate} & $\hat{\statevec}_{1|0}$ is set to a zero vector\\
\texttt{inivar} & $\statevar_{1|0}$ is set automatically\\
\end{tabular}
\end{center}

\subsection{Further details on the kalman block}

It might appear that the \texttt{obsx} ($\obsx$) and \texttt{obsxmat}
($\obsxmat$) matrices must go together---either both are given or
neither is given.  But an exception is granted for convenience.  If
the observation equation includes a constant but no additional
exogenous variables, you can give a ($1 \times n$) value for
$\obsxmat$ without having to specify \texttt{obsx}.  More generally,
if the row dimension of $\obsxmat$ is 1 greater than the column
dimension of $\obsx$, it is assumed that the first element of
$\obsxmat$ is associated with an implicit column of 1s.

Regarding the automatic initialization of $\statevar_{1|0}$ (in case
no \texttt{inivar} input is given): by default this is done as in
equation (\ref{eq:inivar}).  However, this method is applicable only
if all the eigenvalues of $\statemat$ lie inside the unit circle.  If
this condition is not satisfied we instead apply a diffuse prior,
setting $\statevar_{1|0} = \kappa \mathbf{I}_r$ with $\kappa = 10^7$.
If for some reason you wish to impose this diffuse prior regardless of
the characteristics of $\statemat$, append the option flag
\verb|--diffuse| to the \texttt{end kalman} statement.\footnote{
  Initialization of the Kalman filter outside of the case where
  equation (\ref{eq:inivar}) applies has been the subject of much
  discussion in the literature---see for example de Jong (1991),
  Koopman (1997).  At present gretl does not implement any of the
  ``fancier'' proposals that have been made.}

As we said above, it is possible to substitute series, lists or
scalars in place of matrix variables when building a \texttt{kalman}
block, provided the dimensions are OK.  However, one further point
should be noted.  When you give the name of a predefined matrix---and
only in this case---there is a gain in generality: the input matrix is
not ``hard-wired'' into the Kalman structure, rather a record is made
of the \textit{name} of the matrix, and on each run of a Kalman
function (as described below) the matrix is re-read.  It is therefore
possible to write one \texttt{kalman} block and then do several
filtering or smoothing passes using different sets of input values. An
example of this technique is provided later, in the example script
contained in Table \ref{tab:armaest}. The dimensions of the various
input matrices are, however, defined via the initial \texttt{kalman}
set-up and it is an error if any of the matrices are changed in size.

\subsection{Time-varying matrices}

Any or all of the matrices \texttt{obsymat}, \texttt{obsxmat},
\texttt{obsvar}, \texttt{statemat} and \texttt{statevar} may be
time-varying.  In that case there are two options: you may

\begin{itemize}
\item give a function which returns a matrix, or
\item give the name of a matrix and a function which modifies that
  matrix, separated by a semicolon.
\end{itemize}

Suppose the matrix $\obsmat$ is time-varying.  Then for an example of
the first option, we might write
%
\begin{verbatim}
  obsymat get_H(theta)
\end{verbatim}
%
where \texttt{get\_H} is a user-defined function which returns a
matrix and \texttt{theta} is a suitable argument to that function.
For an example of the second option, we might write
%
\begin{verbatim}
  obsymat H ; modify_H(&H, theta)
\end{verbatim}

In both cases the relevant function, \verb+get_H+ or \verb+modify_H+,
will be called at each time-step of the filter operation.  Here is the
difference. In the first case the matrix returned by the function is
anonymous and is ``owned'' by the Kalman structure; and at each step
the old matrix is destroyed and replaced by a new one.  In the second
case the matrix \texttt{H} must be predefined, and the function does
not return any value; rather it modifies the named matrix argument,
given in ``pointer'' form.  The advantage of the latter form is that
is does not involve creating and destroying a matrix at each step, and
so will be more efficient.

Such matrix-generating or matrix-modifying functions have access to
the current time-step of the Kalman filter, via the internal variable
\verb+$kalman_t+.  This has value 1 on the first iteration, 2 on the
second, and so on, up to iteration $T$.  They also have access to
the current $n$-vector of forecast errors, $\prederr_t$, under the
name \verb+$kalman_uhat+.


\subsection{Handling of missing values}

It is acceptable for the data matrices, \texttt{obsy} and
\texttt{obsx}, to contain missing values.  In this case the filtering
operation will work around the missing values, and the \texttt{ksmooth}
function can be used to obtain estimates of these values.  However,
there are two points to note.

First, the long-standing ``tradition'' in gretl is that while data
series can have missing values (\texttt{NA}s), matrices can contain
only valid numerical values.  To allow \texttt{NA}s in matrices you
must use the \texttt{set} command thus:
%
\begin{verbatim}
set force_finite off
\end{verbatim}
%
This point may be revisited in future.  

Second, the handling of missing values is not yet quite right for the
case where the observable vector $\obsvec_t$ contains more than one
element.  At present, if any of the elements of $\obsvec_t$ are
missing the entire observation is ignored.  Clearly it should be
possible to make use of any non-missing elements, and this is not
very difficult in principle, it's just awkward and is not
implemented yet.

\subsection{Persistence and identity of the filter}

At present there is no facility to create a ``named filter''.  Only
one filter can exist at any point in time, namely the one created by
the last \texttt{kalman} block.\footnote{This is not quite true: more
  precisely, there can be no more than one Kalman filter \textit{at
    each level of function execution}.  That is, if a gretl script
  creates a Kalman filter, a user-defined function called from that
  script may also create a filter, without interfering with the
  original one.}  If a filter is already defined, and you give a new
\texttt{kalman} block, the old filter is over-written.  Otherwise the
existing filter persists (and remains available for the
\texttt{kfilter}, \texttt{ksmooth} and \texttt{ksimul} functions)
until either (a) the gretl session is terminated or (b) the command
\texttt{delete kalman} is given.


\section{The \texttt{kfilter} function}

Once a filter is established, as discussed in the previous section,
\texttt{kfilter} can be used to run a forward, forecasting, pass.
This function returns a scalar code: 0 for successful completion, or 1
if numerical problems were encountered.  On successful completion, two
scalar accessor variables become available: \verb+$kalman_lnl+, which
gives the overall log-likelihood under the joint normality assumption,
%
\[
  \ell = -\frac{1}{2} \left[nT \log(2 \pi) + \sum_{t=1}^T\left|\predvar_t\right| + 
    \sum_{t=1}^T\prederr_t'\predvar_t^{-1} \prederr_t
  \right]
\]
%
and \verb+$kalman_s2+, which gives the estimated variance,
%
\[
\hat{\sigma}^2 = \frac{1}{nT} 
   \sum_{t=1}^T\prederr_t'\predvar_t^{-1} \prederr_t
\]
(but see below for modifications to these formulae for the case of a
diffuse prior).  In addition the accessor \verb+$kalman_llt+ gives a
($T \times 1$) vector, element $t$ of which is
%
\[
  \ell_t = -\frac{1}{2} \left[n \log(2 \pi) + \left|\predvar_t\right| + 
    \prederr_t'\predvar_t^{-1} \prederr_t
  \right]
\]
%

The \texttt{kfilter} function does not require any arguments, but up
to five matrix quantities may be retrieved via optional pointer
arguments.  Each of these matrices has $T$ rows, one for each
time-step; the number of columns is shown in the following listing.
%
\begin{enumerate}
\item Forecast errors for the observable variables: $n$ columns.
\item Variance matrix for the forecast errors: $n(n+1)/2$.
\item Estimate of the state vector: $r$.
\item Variance of estimate of the state vector: $r(r+1)/2$.
\item Kalman gain: $rn$.
\end{enumerate}

The variance matrices retrieved by arguments (2) and (4) are in vech
form, i.e., the unique elements of the matrix are stacked by column
and transposed to a row vector.  The Kalman gain matrix from each time
step is stored in transposed vec form.

Unwanted trailing arguments can be omitted, otherwise unwanted
arguments can be skipped by using the keyword \texttt{null}.  For
example, the following call retrieves the forecast errors in the
matrix \texttt{E} and the estimate of the state vector in \texttt{S}:
%
\begin{verbatim}
matrix E S
kfilter(&E, null, &S)
\end{verbatim}

Matrices given as pointer arguments do not have to be correctly
dimensioned in advance; they will be resized to receive the specified
content.

Further note: in the general case, the arguments to \texttt{kfilter}
should all be matrix-pointers, but under two conditions you can give a
pointer to a series variable instead.  The conditions are: (i) the
matrix in question has just one column in context (for example, the
first two matrices will have a single column if the length of the
observables vector, $n$, equals 1) and (ii) the time-series length of
the filter is equal to the current gretl sample size.

\subsection{Likelihood under the diffuse prior}

There seems to be general agreement in the literature that the
log-likelihood calculation should be modified in the case of a diffuse
prior for $\statevar_{1|0}$.  However, it is not clear to us that
there is a well-defined ``correct'' method for this.  At present we
emulate \textsf{SsfPack} (see Koopman \textit{et al} (1999) and
section~\ref{sec:amble}).  In case $\statevar_{1|0} = \kappa
\mathbf{I}_r$, we set $d = r$ and calculate
%
\[
  \ell = -\frac{1}{2} \left[(nT-d) \log(2 \pi) + 
    \sum_{t=1}^T\left|\predvar_t\right| + 
    \sum_{t=1}^T\prederr_t'\predvar_t^{-1} \prederr_t
    - d \log(\kappa)
  \right]
\]
%
and
%
\[
\hat{\sigma}^2 = \frac{1}{nT-d} 
   \sum_{t=1}^T\prederr_t'\predvar_t^{-1} \prederr_t
\]

\section{The \texttt{ksmooth} function}

This function returns the ($T \times r$) matrix of smoothed estimates
of the state vector---that is, estimates based on all $T$
observations: row $t$ of this matrix holds $\hat{\statevec}_{t|T}'$.  This
function has no required arguments but it offers one optional
matrix-pointer argument, which retrieves the variance of the smoothed
state estimate, $\statevar_{t|T}$.  The latter matrix is ($T \times
r(r+1)/2$); each row is in transposed vech form.  Examples:
%
\begin{verbatim}
matrix S = ksmooth()  # smoothed state only
matrix P
S = ksmooth(&P)       # the variance is wanted
\end{verbatim}

These values are computed via a backward pass of the filter.  The
final values of $\hat{\statevec}$ and $\statevar$ already incorporate
all the sample information.  The backward iteration, from $t=T-1$ to
1, is then
%
\begin{eqnarray*}
\hat{\statevec}_{t|T} &=& \hat{\statevec}_{t|t} + 
  \mathbf{J}_t\left(\hat{\statevec}_{t+1|T} - \hat{\statevec}_{t+1|t}\right) \\
\statevar_{t|T} &=& \statevar_{t|t} + 
  \mathbf{J}_t\left(\statevar_{t+1|T} - \statevar_{t+1|t}\right) \mathbf{J}_t'
\end{eqnarray*}
%
where
\[
\mathbf{J}_t = \statevar_{t|t} \statemat' \statevar_{t+1|t}^{-1}
\]
%
This iteration is preceded by a special forward pass in which the
matrices $\hat{\statevec}_{t|t}$ and $\statevar_{t|t}$ are stored
for all $t$.


\section{The \texttt{ksimul} function}

This simulation function takes up to three arguments.  The first,
mandatory, argument is a ($T \times r$) matrix containing artificial
disturbances for the state transition equation: row $t$ of this matrix
represents $\strdist_t'$.  If the current filter has a non-null
$\obsvar$ (\texttt{obsvar}) matrix, then the second argument should be
a ($T \times n$) matrix containing artificial disturbances for the
observation equation, on the same pattern.  Otherwise the second
argument should be given as \texttt{null}.  If $r=1$ you may give a
series for the first argument, and if $n=1$ a series is acceptable for
the second argument.

Provided that the current filter does not include exogenous variables
in the observation equation (\texttt{obsx}), the $T$ for simulation
need not equal that defined by the original \texttt{obsy} data matrix:
in effect $T$ is temporarily redefined by the row dimension of the
first argument to \texttt{ksimul}.  Once the simulation is completed,
the $T$ value associated with the original data is restored.

The value returned by \texttt{ksimul} is a ($T \times n$) matrix
holding simulated values for the observables at each time step.  A
third optional matrix-pointer argument allows you to retrieve a ($T
\times r$) matrix holding the simulated state.  Examples:
%
\begin{verbatim}
matrix Y = ksimul(V)    # obsvar is null
Y = ksimul(V, W)        # obsvar is non-null
matrix S
Y = ksimul(V, null, &S) # the simulated state is wanted
\end{verbatim}

\section{Example 1: ARMA estimation}
\label{sec:example_arma}

As is well known, the Kalman filter provides a very efficient way to
compute the likelihood of ARMA models; as an example, take an
ARMA(1,1) model
\[
  y_t = \phi y_{t-1} + \varepsilon_t + \theta \varepsilon_{t-1} .
\]
One of the ways the above equation can be cast in state-space form is
by defining a latent process $\xi_t = \frac{1}{1 - \phi L}
\varepsilon_t$ and noting that 
\[
y_t = \xi_t + \theta \xi_{t-1} ,
\]
which is the observation equation corresponding to (\ref{eq:obs});
the state transition equation corresponding to (\ref{eq:state}) is
\[
  \left[ \begin{array}{c} \xi_t \\ \xi_{t-1} \end{array} \right] =
  \left[ \begin{array}{cc} \phi & 0 \\ 1 & 0 \end{array} \right]
  \left[ \begin{array}{c} \xi_{t-1} \\ \xi_{t-2} \end{array} \right] +
  \left[ \begin{array}{c} \varepsilon_t \\ 0 \end{array} \right] 
\]

The gretl syntax for a corresponding \texttt{kalman} block would be
\begin{verbatim}
    matrix H = {1; theta}
    matrix F = {phi, 0; 1, 0}
    matrix Q = {s^2, 0; 0, 0}

    kalman
        obsy y
        obsymat H
        obsvar 0
        statemat F
        statevar Q
    end kalman
\end{verbatim}

Once the filter is set up, all it takes to compute the loglikelihood
for given values of $\phi$, $\theta$ and $\sigma^2$ is to execute the
\texttt{kfilter()} function and use the \verb+$kalman_lnl+ accessor
(which returns the total log-likelihood) or, more appropriately if the
likelihood has to be maximized through \texttt{mle}, the
\verb+$kalman_llt+ accessor, which returns the series of individual
contribution to the log-likelihood for each observation. An example
is shown in Table \ref{tab:armaest}.

\begin{table}[htbp]
  \caption{ARMA estimation}
  \label{tab:armaest}

\begin{small}
\begin{verbatim}
function arma11_via_kalman(series y)
    
    /* parameter initalization */
    phi = 0
    theta = 0
    sigma = 1

    /* Kalman filter setup */
    matrix H = {1; theta}
    matrix F = {phi, 0; 1, 0}
    matrix Q = {sigma^2, 0; 0, 0}

    kalman
        obsy y
        obsymat H
        obsvar 0
        statemat F
        statevar Q
    end kalman

    /* maximum likelihood estimation */
    mle logl = ERR ? NA : $kalman_llt
        H[2] = theta
        F[1,1] = phi
        Q[1,1] = sigma^2
        ERR = kfilter()
        params phi theta sigma
    end mle -h
   
end function

# --------------------------- main ---------------------------------

# open the "arma" example dataset
open arma.gdt
# estimate an arma(1,1) model via mle
arma11_via_kalman(y)
# check via native command
arma 1 1 ; y --nc
\end{verbatim}
\end{small}
\end{table}

\section{Example 2: local level model}
\label{sec:example_loclev}

Suppose we have a series $y_t = \mu_t + \varepsilon_t$, where $\mu_t$
is a random walk with normal increments of variance $\sigma^2_1$ and $
\varepsilon_t$ is a white noise with variance $\sigma^2_2$.  This is
known as the ``local level'' model in Harvey's (1991) terminology, and
it can be cast in state-space form as equations
(\ref{eq:state})-(\ref{eq:obs}) with
\begin{eqnarray*}
  \statemat & = & 1 \\
  \strdist_t & \sim & N(0, \sigma^2_1) \\
  \obsmat & = & 1 \\
  \obsdist_t & \sim & N(0, \sigma^2_2)
\end{eqnarray*}

Its translation as a \texttt{kalman} block is
\begin{verbatim}
    kalman
       obsy y
       obsymat 1
       statemat 1
       statevar s2
       obsvar s1
    end kalman --diffuse
\end{verbatim}

The two unknown parameters $\sigma^2_1$ and $\sigma^2_2$ can be
estimated via maximum likelihood.  The example script in
Table~\ref{tab:loclev} provides an example of simulation and
estimation of such a model. For the sake of brevity, simulation is
carried out via ordinary gretl commands, rather than the state-space
apparatus described above.

\begin{table}[htbp]
  \caption{Local level model}
  \label{tab:loclev}

\begin{small}
\begin{verbatim}

/* ----------- functions ----------- */

function loclev_ll (series y, scalar s1, scalar s2)
    /* log-likelihood function */
    kalman
       obsy y
       obsymat 1
       statemat 1
       statevar s2
       obsvar s1
    end kalman --diffuse
    kfilter()
    series ret = $kalman_llt
    return series ret
end function

function loclev_sm (series y, scalar s1, scalar s2)
    /* return the smoothed estimate of \mu_t */
    kalman
       obsy y
       obsymat 1
       statemat 1
       statevar s2
       obsvar s1
    end kalman --diffuse
    matrix state = ksmooth()
    series ret = state[,1]
    return series ret
end function

/* ---------- main script ---------- */

/* simulate the data */
nulldata 200
set seed 202020
setobs 1 1 --special
true_s1 = 0.25
true_s2 = 0.5
v = normal() * sqrt(true_s1)
w = normal() * sqrt(true_s2)
mu = 2 + cum(w)
y = mu + v

/* do the estimation */
s1 = 1
s2 = 1
mle ll = loclev_ll(y, s1, s2)
    params s1 s2
end mle

Uhat = loclev_sm(y, s1, s2) # compute the smoothed state
\end{verbatim}
\end{small}
\end{table}

By appending the following code snippet to the example in
Table~\ref{tab:loclev}, one may check the results via \textsf{R}.

\begin{verbatim}
foreign language=R --send-data
  y <- gretldata[,"y"]
  a <- StructTS(y, type="level")
  a
  StateFromR <- as.ts(tsSmooth(a))
  gretl.export(StateFromR)
end foreign

append @dotdir/StateFromR.csv

ols Uhat 0 StateFromR --simple
\end{verbatim}


\begin{thebibliography}

  Anderson B. and Moore J. (1979) \emph{Optimal Filtering}, Prentice-Hall.

  de Jong, P. (1991) ``The Diffuse Kalman Filter'', \emph{The Annals
    of Statistics}, 19, pp. 1073--1083.

  Gourieroux, Christian and Alain Monfort (1996),
  \emph{Simulation-Based Econometric Methods}, Oxford University
  Press.

  Hamilton, James D. (1994), \emph{Time Series Analysis}, Princeton
  University Press.

  Harvey, A. C. (1991), \emph{Forecasting, Structural Time Series
    Models and the Kalman Filter}, Cambridge University Press.

  Harvey, A. C. and T. Proietti (2005), \emph{Readings in Unobserved
    Component Models}, Oxford University Press.

  Kalman, R. E. (1960), ``A New Approach to Linear Filtering and
  Prediction Problems'', \emph{Transactions of the ASME--Journal of
    Basic Engineering}, 82, Series D, pp. 35--45

  Koopman, S. J. (1997), ``Exact Initial Kalman Filtering and
  Smoothing for Nonstationary Time Series Models'', \emph{Journal of
    the American Statistical Association}, 92, pp. 1630--1638

  Koopman, S. J., N. Shephard and J. A. Doornik (1999), 
  ``Statistical algorithms for models in state space using SsfPack
  2.2'', \emph{Econometrics Journal}, 2, pp. 113--166.
  
  Pollock, D.S.G. (1999), \emph{A Handbook of Time-Series Analysis,
    Signal Processing and Dynamics}, Academic Press.

\end{thebibliography}


\end{document}

%%% Local Variables: 
%%% mode: latex
%%% TeX-master: t
%%% End: 
