\documentclass[oneside]{book}
\usepackage{url,verbatim,fancyvrb}
\usepackage{pifont}
\usepackage[latin1]{inputenc}
\usepackage[pdftex]{graphicx}
%\usepackage[authoryear]{natbib}
\usepackage{color,gretl}
\usepackage[letterpaper,body={6.3in,9.15in},top=.8in,left=1.1in]{geometry}
\usepackage[pdftex,hyperfootnotes=false]{hyperref}
\usepackage{dcolumn,amsmath,bm,longtable}

%% \pdfimageresolution=120
\hypersetup{pdftitle={A Hansl Primer},
            pdfsubject={The scripting language of gretl},
            pdfauthor={Riccardo (Jack) Lucchetti},
            colorlinks=true,
            linkcolor=blue,
            urlcolor=red,
            citecolor=steel,
            bookmarks=true,
            bookmarksnumbered=true,
            plainpages=false
}

\begin{document}

\VerbatimFootnotes

\setlength{\parindent}{0pt}
\setlength{\parskip}{1ex}
\setcounter{tocdepth}{1}

%% titlepage

\thispagestyle{empty}

\begin{center}
\pdfbookmark[1]{A Hansl Primer}{titlepage}

\htitle{A Hansl Primer}

\gsubtitle{The scripting language of gretl in \pageref{LastPage} minutes}

{\large \sffamily
Allin Cottrell\\
Department of Economics\\
Wake Forest University\\

\vspace{20pt}
Riccardo (Jack) Lucchetti\\
Dipartimento di Scienze Economiche e Sociali\\
Universit� Politecnica delle Marche\\

\vspace{20pt}
\input date
}

\end{center}
\clearpage

%% end titlepage, start license page

\thispagestyle{empty}

\pdfbookmark[1]{License}{license}

\vspace*{2in}

Permission is granted to copy, distribute and/or modify this document
under the terms of the \emph{GNU Free Documentation License}, Version
1.1 or any later version published by the Free Software Foundation
(see \url{http://www.gnu.org/licenses/fdl.html}).

\cleardoublepage

%% end license page, start table of contents
\pdfbookmark[1]{Table of contents}{contents}

\pagenumbering{roman}
\pagestyle{headings}

\tableofcontents

\clearpage
\pagenumbering{arabic}
%\setcounter{chapter}{-1}


\chapter{Introduction}
%\addcontentsline{toc}{chapter}{Introduction}

\section*{What hansl is and what it is not}

Hansl is a recursive acronym: it stands for ``Hansl's A Neat Scripting
Language''. You might therefore expect something very general in
purpose. Not really.  Hansl was born as the scripting language for the
econometrics program gretl and its role is unlikely to change.  As a
consequence, hansl should not be viewed as a fully fledged programming
language such as C, Fortran, Perl or Python. Not because it lacks
anything to be considered as such,\footnote{Hansl is in fact
  Turing-complete.} but because its aim is different. Hansl should be
considered as a special-purpose or domain-specific language, designed
to make an econometrician's life easier. Hence it incorporates a
series of conventions and choices that may irritate purists and have
some marginal impact on raw performance, but that we, as professional
econometricians, consider ``nice to have''.  This makes hansl somewhat
different from plain matrix-oriented interpreted languages, such as
the Matlab/Octave family, Ox and so on.

On the other hand, hansl is by no means just a tool to automate rote
tasks. It has several features which support advanced work: structured
programming, recursion, complex data structures, and so on.  As for
style, the language which hansl most resembles is probably that of the
bash shell.

\section*{The intent and structure of this document}

The intended readers of this document are those who already know how
to write code, and are familiar with the associated do-s and don't-s.
Such people may wish to add hansl to their toolbox, alongside
languages like C or FORTRAN, or programs such as R, Ox, Matlab, Stata
or Gauss, some of which they are already confident with. Here,
therefore, the focus is not so much on ``How do I do this?'', but
rather on ``How do I do this \emph{in hansl}?''.

As a consequence, this document aims at making the reader a reasonably
proficient hansl user in a (relatively) short time; however, not all
the features of hansl are illustrated; for those, interested readers
should consult the \GCR{} and \GUG{}.

This guide comprises two main parts. Part~\ref{part:hp-nodata}
(``Without a dataset'') concentrates on hansl as a pure
matrix-oriented programming language. Part~\ref{part:hp-data} (``With
a dataset'') exploits the fact that hansl scripts are run through
gretl, which has very nice facilities for handling statistical
datasets (interactively if necessary). This provides hansl with a
series of extra constructs and features which make it extremely easy
to write hansl scripts to perform all sorts of statistical procedures.

% A short third part (which is likely to be expanded in future) gives a
% brief account of a few ``further topics''.

In order to use hansl, you will need a working installation of
gretl. We assume you have one. If you don't, please refer to chapter 1
of \GUG.

\section*{Other resources}

If you are serious about learning hansl then after working through
this primer---or in the process of doing so---you'll want to take a
look at the following additional resources.
\begin{itemize}
\item The \GCR. This contains a complete listing of the commands and
  built-in functions available in hansl, with a full account of their
  syntax and options. Examples of usage are provided in some
  instances. This is available in an ``online'' version for handy
  reference as well as in PDF, both accessible via the \textsf{Help} menu
  in the gretl GUI.
\item The \textit{Gretl User's Guide}. Chapters 10 to 16, in
  particular, go into more detail on some of the programming topics
  discussed here (data types, loops, the definition and use of
  functions). In addition Part II of the \textit{Guide}, on
  Econometric Methods, gives many examples of hansl usage. The
  \textit{Guide} is available via gretl's \textsf{Help} menu; the
  latest version can also be found online at
  \url{http://sourceforge.net/projects/gretl/files/manual/}.
\item Sample scripts. The gretl package comes with a large number of
  sample or practice scripts, which can be found under the menu item
  \textsf{/File/Script files/Example scripts}. Many of these are simple
  replication exercises for textbook problems but you will find some
  more interesting examples under the \textsf{Gretl} tab.
\item Function packages. Relatively ambitious examples of hansl coding
  can be found in the various contributed ``function packages''. You
  can download these packages via the gretl menu item
  \textsf{/Tools/Function packages/On server}. Once a package is
  downloaded it appears in the listing under \textsf{/Tools/Function
    packages/On local machine}; in that context you can right-click
  and select \textsf{View code} to examine the hansl functions.
\item The gretl-users mailing list. Most well-considered
  questions get answered quite quickly and fully. See
  \url{https://gretlml.univpm.it/postorius/lists/}.
\end{itemize}

\chapter{For the impatient}
\label{chap:impatient}

OK, so you're impatient. Then perhaps you're thinking ``Do I really
need to go through the whole thing? After all, I've been coding
econometric stuff for a while, and I'm pretty confident I can pick a
new scripting language if it's not too obscure.  I just need a few
tips to get me started''. If that's not what you're thinking at all,
we suggest you move along to chapter~\ref{chap:hello}; but if it is,
well then, we'll give you a hansl script which exemplifies a hefty
share of the topics discussed in the rest of this primer. We will use
for our example a Vector AutoRegressive model, or VAR for short.

As you probably know, a finite-order VAR can be estimated via
conditional maximum likelihood simply by applying OLS to each equation
individually. That amounts to solving a least-squares problem and its
solution can be easily written, in matrix notation, as $\hat{\Pi} =
(X'X)^{-1} X'Y$, where $Y$ contains your endogenous variables and $X$
contains their lags plus other exogenous terms (typically, a constant
term at least). But of course, you may choose to find the maximum of
the concentrated likelihood $\mathcal{L} = -(T/2) \ln|\hat{\Sigma}|$
numerically if you so wish.

The following example contains a hansl script which performs these
actions:
\begin{enumerate}
\item Reads data from a disk file.
\item Performs some data transformation and simple visualization.
\item Estimates the VAR via 
  \begin{enumerate}
  \item the native hansl \texttt{var} command
  \item sequential single-equation OLS
  \item matrix algebra (in 3 different ways)
  \item numerical maximization of the log-likelihood.
  \end{enumerate}
\item Prints out the results.
\end{enumerate}
The script also contains some concise comments.

\begin{scode}
open AWM.gdt --quiet                   # load data from disk

/* data transformations and visualisation */

series y = 100 * hpfilt(ln(YER))       # the "series" concept: operate on      
series u = 100 * URX                   # vectors on an element-by-element basis
series r = STN - 100*sdiff(ln(HICP))   # (but you also have special functions) 

scatters y r u --output=display        # command example with an option: graph data

/* in-house VAR */

scalar p = 2                           # strong typing: a scalar is not a
                                       # matrix nor a series

var p y r u                            # estimation command
A = $coeff                             # and corresponding accessor

/* by iterated OLS */

list X = y r u                         # the list is yet another variable type

matrix B = {}                          # initialize an empty matrix

loop foreach i X                       # loop over the 3 var equations
    ols $i const X(-1 to -p) --quiet   # using native OLS command
    B ~= $coeff                        # and store the estimated coefficients
endloop                                # as matrix columns

/* via matrices */

matrix mY = { y, r, u }                # construct a matrix from series
matrix mX = 1 ~ mlag(mY, {1,2})        # or from matrix operators/functions
mY = mY[p+1:,]                         # and select the appropriate rows
mX = mX[p+1:,]                         # via "range" syntax

C1 = mX\mY                             # matlab-style matrix inversion
C2 = mols(mY, mX)                      # or native function
C3 = inv(mX'mX) * (mX'mY)              # or algebraic primitives

/* or the hard, needlessly complicated way --- just to show off */

function scalar loglik(matrix param, const matrix X, const matrix Y)

    # this function computes the concentrated log-likelihood
    # for an unrestricted multivariate regression model

    scalar n = cols(Y)
    scalar k = cols(X)
    scalar T = rows(Y)
    matrix C = mshape(param, k, n)
    matrix E = Y - X*C
    matrix Sigma = E'E

    return -T/2 * ln(det(Sigma))
end function

matrix c = zeros(21,1)                  # initialize the parameters
mle ll = loglik(c, mX, mY)              # and maximize the log-likelihood
    params c                            # via BFGS, printing out the
end mle                                 # results when done

D = mshape(c, 7, 3)                     # reshape the results for conformability

/* print out the results */

# note: row ordering between alternatives is different 

print A B C1 C2 C3 D
\end{scode}
%$

If you were able to follow the script above in all its parts,
congratulations. You probably don't need to read the rest of this
document (though we don't mind if you do). But if you find the script
too scary, never fear: we'll take things step by step. Read on.

\part{Without a dataset}
\label{part:hp-nodata}
\chapter{Hello, world!}
\label{chap:hello}

We begin with the time-honored ``Hello, world'' program, the
obligatory first step in any programming language. It's actually very
simple in hansl:
\begin{code}
  # First example
  print "Hello, world!"
\end{code}

There are several ways to run the above example: you can put it in a
text file \dtkttt{first_ex.inp} and have gretl execute it from the
command line through the command
\begin{code}
  gretlcli -b first_ex.inp
\end{code}
or you could just copy its contents in the editor window of a GUI
gretl session and click on the ``gears'' icon. It's up to you; use
whatever you like best.

From a syntactical point of view, allow us to draw attention on
the following points:
\begin{enumerate}
\item The line that begins with a hash mark (\texttt{\#}) is a
  comment: if a hash mark is encountered, everything from that point
  to the end of the current line is treated as a comment, and ignored
  by the interpreter.
\item The next line contains a \emph{command} (\cmd{print}) followed
  by an \emph{argument}; this is fairly typical of hansl: many jobs
  are carried out by calling commands.
\item The quotation character is a straight double-quote.
\item Hansl does not have an explicit command terminator such as the
  ``\texttt{;}'' character in the C language family (C++, Java, C\#,
  \ldots) or GAUSS; instead it uses the newline character as an
  implicit terminator. So at the end of a command, you \emph{must}
  insert a newline. Conversely, you can't split a single command
  over more than one line unless (a) the line to be continued ends
  with a comma or (b) you insert a ``\textbackslash'' (backslash)
  character, which causes gretl to ignore the following line break.
\end{enumerate}

Note also that the \cmd{print} command automatically appends a line
break, and does not recognize ``escape'' sequences such as
``\verb|\n|''. Such sequences are just printed literally---with a
single exception, namely that a backslash immediately followed by
double-quote produces an embedded double-quote. The \cmd{printf}
command can be used for greater control over output; see chapter
\ref{chap:formatting}.

Let's now examine a simple variant of the above:
\begin{code}
  /*
    Second example
  */
  string foo = "Hello, world"
  print foo
\end{code}

In this example, the comment is written using the convention adopted
in the C programming language: everything between ``\verb|/*|'' and
``\verb|*/|'' is ignored.\footnote{Each type of comment can be masked
  by the other:
\begin{itemize}
\item If \texttt{/*} follows \texttt{\#} on a given line which does
  not already start in ignore mode, then there's nothing special about
  \texttt{/*}, it's just part of a \texttt{\#}-style comment.
\item If \texttt{\#} occurs when we're already in comment mode, it is
  just part of a comment.
\end{itemize}} Comments of this type cannot be nested.

Then we have the line
\begin{code}
  string foo = "Hello, world"
\end{code}
In this line, we assign the value ``\texttt{Hello, world}'' to the
variable named \texttt{foo}. Note that
\begin{enumerate}
\item The assignment operator is the equals sign (\texttt{=}).
\item The name of the variable (its \emph{identifier}) must follow the
  following convention: identifiers can be at most 31 characters long
  and must be plain ASCII. They must start with a letter, and can
  contain only letters, numbers and the underscore
  character.\footnote{Actually one exception to this rule is
    supported: identifiers taking the form of a single Greek
    letter. See chapter~\ref{chap:greeks} for details.} Identifiers in
  hansl are case-sensitive, so \texttt{foo}, \texttt{Foo} and
  \texttt{FOO} are three distinct names. Of course, some words are
  reserved and can't be used as identifiers (however, nearly all
  reserved words only contain lowercase characters).
\item The string delimiter is the double quote (\verb|"|). 
\end{enumerate}

In hansl, a variable has to be of one of these types: \texttt{scalar},
\texttt{series}, \texttt{matrix}, \texttt{list}, \texttt{string},
\texttt{bundle} or \texttt{array}. As we've just seen, string
variables are used to hold sequences of alphanumeric characters. We'll
introduce the other ones gradually; for example, the \texttt{matrix}
type will be the object of the next chapter.

The reader may have noticed that the line 
\begin{code}
  string foo = "Hello, world"
\end{code}
implicitly performs two tasks: it \emph{declares} \texttt{foo} as a
variable of type \texttt{string} and, at the same time, \emph{assigns}
a value to \texttt{foo}. The declaration component is not strictly
required. In most cases gretl is able to figure out by itself what
type a newly introduced variable should have, and the line
\verb|foo = "Hello, world"| (without a type specifier) would have
worked just fine.  However, it is more elegant (and leads to more
legible and maintainable code) to use a type specifier at least the
first time you introduce a variable.
  
In the next example, we will use a variable of the \texttt{scalar}
type:
\begin{code}
  scalar x = 42
  print x
\end{code}
A \texttt{scalar} is a double-precision floating point number, so
\texttt{42} is the same as \texttt{42.0} or \texttt{4.20000E+01}. Note
that hansl doesn't have a separate variable type for integers.

An important detail to note is that, contrary to most other
matrix-oriented languages in use in the econometrics community, hansl
is \emph{strongly typed}. That is, you cannot assign a value of one
type to a variable that has already been declared as having a
different type. For example, this will return an error:
\begin{code}
  string a = "zoo"
  a = 3.14 # no, no, no!
\end{code}
If you try running the example above, an error will be
flagged. However, it is acceptable to destroy the original variable,
via the \cmd{delete} command, and then re-declare it, as in
\begin{code}
  scalar X = 3.1415
  delete X
  string X = "apple pie"
\end{code}

There is no ``type-casting'' as in C, but some automatic type
conversions are possible (more on this later).

Many commands can take more than one argument, as in
\begin{code}
  set verbose off

  scalar x = 42
  string foo = "not bad"
  print x foo 
\end{code}
In this example, one \texttt{print} is used to print the values of two
variables; more generally, \texttt{print} can be followed by as many
arguments as desired. The other difference with respect to the
previous code examples is in the use of the \texttt{set}
command. Describing this command in detail would lead us to an overly
long diversion; suffice it to say that it is used to set the values of
various ``state variables'' that influence the behavior of the
program; here it is used as a way to silence unwanted output. See the
\GCR{} for more on \texttt{set}.

% There was a reference here to a {chap:settings}, but it has not
% been written at this point

The \cmd{eval} command is useful when you want to look at the result
of an expression without assigning it to a variable; for example
\begin{code}
  eval 2+3*4
\end{code}
will print the number 14. This is most useful when running gretl
interactively, like a calculator, but it is usable in a hansl script
for checking purposes, as in the following (rather silly) example:
\begin{code}
  scalar a = 1
  scalar b = -1
  # this ought to be 0
  eval a+b
\end{code}

\section{Manipulation of scalars}

Algebraic operations work in the obvious way, with the classic
algebraic operators having their traditional precedence rules. The
caret (\verb|^|) is used for exponentiation. For example,
\begin{code}
  scalar phi = exp(-0.5 * (x-m)^2 / s2) / sqrt(2 * $pi * s2)
\end{code}
%$
in which we assume that \texttt{x}, \texttt{m} and \texttt{s2} are
pre-existing scalars. The example above contains two noteworthy
points:
\begin{itemize}
\item The usage of the \cmd{exp} (exponential) and \cmd{sqrt} (square
  root) functions; it goes without saying that hansl possesses a
  reasonably wide repertoire of such functions. See the \GCR{} for the
  complete list.
\item The usage of \verb|$pi| for the constant $\pi$. While
  user-specified identifiers must begin with a letter, built-in
  identifiers for internal objects typically have a ``dollar'' prefix;
  these are known as \emph{accessors} (basically, read-only
  variables).  Most accessors are defined in the context of an open
  dataset (see part~\ref{part:hp-data}), but some represent
  pre-defined constants, such as $\pi$. Again, see the \GCR{} for a
  comprehensive list.
\end{itemize}

Hansl does not possess a specific Boolean type, but scalars can be
used for holding true/false values. It follows that you can also use
the logical operators \emph{and} (\verb|&&|), \emph{or} (\verb+||+),
and \emph{not} (\verb|!|) with scalars, as in the following example:
\begin{code}
  a = 1
  b = 0
  c = !(a && b) 
\end{code}
In the example above, \texttt{c} will equal 1 (true), since
\verb|(a && b)| is false, and the exclamation mark is the negation
operator.  Note that 0 evaluates to false, and anything else (not
necessarily 1) evaluates to true.

A few constructs are taken from the C language family: one is the
postfix increment operator:
\begin{code}
  a = 5
  b = a++
  print a b
\end{code}
the second line is equivalent to \texttt{b = a}, followed by
\texttt{a++}, which in turn is shorthand for \texttt{a = a+1}, so
running the code above will result in \texttt{b} containing 5 and
\texttt{a} containing 6. Postfix subtraction is also supported; prefix
operators, however, are not supported. Another C borrowing is
inflected assignment, as in \texttt{a += b}, which is equivalent to
\texttt{a = a + b}; several other similar operators are available,
such as \texttt{-=}, \texttt{*=} and more. See the \GCR{} for details.

The internal representation for a missing value is \texttt{NaN} (``not
a number''), as defined by the IEEE 754 floating point standard.  This
is what you get if you try to compute quantities like the square root
or the logarithm of a negative number. You can also set a value to
``missing'' directly using the keyword \texttt{NA}.  The complementary
functions \cmd{missing} and \cmd{ok} can be used to determine whether
a scalar is \texttt{NA}. In the following example a value of zero is
assigned to the variable named \texttt{test}:
\begin{code}
  scalar not_really = NA
  scalar test = ok(not_really)
\end{code}
Note that you cannot test for equality to \texttt{NA}, as in
\begin{code}
  if x == NA ... # wrong!
\end{code}
because a missing value is taken as indeterminate and hence not equal
to anything. This last example, despite being wrong, illustrates a
point worth noting: the test-for-equality operator in hansl is the
double equals sign, ``\texttt{==}'' (as opposed to plain
``\texttt{=}'' which indicates assignment).

\section{Manipulation of strings}

Most of the previous section applies, with obvious modifications, to
strings: you may manipulate strings via operators and/or
functions. Hansl's repertoire of functions for manipulating strings
offers all the standard capabilities one would expect, such as
\cmd{toupper}, \cmd{tolower}, \cmd{strlen}, etc., plus some more
specialized ones. Again, see the \GCR\ for a complete list.

In order to access part of a string, you may use the \cmd{substr}
function,\footnote{Actually, there is a cooler method, which uses the
  same syntax as matrix slicing (see chapter \ref{chap:matrices}):
  \cmd{substr(s, 3, 5)} is functionally equivalent to \cmd{s[3:5]}.} as in
\begin{code}
  string s = "endogenous"
  string pet = substr(s, 3, 5)
\end{code}
which would result to assigning the value \texttt{dog} to the variable
\texttt{pet}.

The following are useful operators for strings:
\begin{itemize}
\item the \verb|~| operator, to join two or more strings, as
  in\footnote{On some national keyboards, you don't have the tilde
    (\texttt{\~}) character. In gretl's script editor, this can be
    obtained via its Unicode representation: type Ctrl-Shift-U,
    followed by \texttt{7e}.}
  \begin{code}
    string s1 = "sweet"
    string s2 = "Home, " ~ s1 ~ " home."
  \end{code}
\item the closely related \verb|~=| operator, which acts as an
  inflected assignment operator (so \verb|a ~= "_ij"| is equivalent to
  \verb|a = a ~ "_ij"|);
\item the offset operator \texttt{+}, which yields a substring of the
  preceding element, starting at the given character offset.  An empty
  string is returned if the offset is greater than the length of the
  string in question.
\end{itemize}

A noteworthy point: strings may be (almost) arbitrarily long;
moreover, they can contain special characters such as line breaks and
tabs. It is therefore possible to use hansl for performing rather
complex operations on text files by loading them into memory as a very
long string and then operating on that; interested readers should take
a look at the \cmd{readfile}, \cmd{getline}, \cmd{strsub} and
\cmd{regsub} functions in the \GCR.\footnote{We are not claiming that
  hansl would be the tool of choice for text processing in
  general. Nonetheless the functions mentioned here can be very useful
  for tasks such as pre-processing plain text data files that do not
  meet the requirements for direct importation into gretl.}

For \emph{creating} complex strings, the most flexible tool is the
\cmd{sprintf} function. Its usage is illustrated in
Chapter~\ref{chap:formatting}.

% Finally, it is quite common to use \emph{string
%   substitution} in hansl sripts; however, this is another topic that
% deserves special treatment so we defer its description to section
% \ref{sec:stringsub}.

%%% Local Variables: 
%%% mode: latex
%%% TeX-master: "hansl-primer"
%%% End: 

\chapter{Matrices}
\label{chap:matrices}

Matrices are one- or two-dimensional arrays of double-precision
floating-point numbers. Hansl users who are accustomed to other matrix
languages should note that multi-index objects are not
supported. Matrices have rows and columns, and that's it.

\section{Matrix indexing}
\label{sec:mat-index}

Individual matrix elements are accessed through the \verb|[r,c]|
syntax, where indexing starts at 1. For example, \texttt{X[3,4]}
indicates the element of $X$ on the third row, fourth column. For
example,
\begin{code}
  matrix X = zeros(2,3)
  X[2,1] = 4
  print X
\end{code}
produces
\begin{code}
X (2 x 3)

  0   0   0 
  4   0   0 
\end{code}

Here are some more advanced ways to access matrix elements:
\begin{enumerate}
\item In case the matrix has only one row (column), the column (row)
  specification can be omitted, as in \texttt{x[3]}.
\item Including the comma but omitting the row or column specification
  means ``take them all'', as in \texttt{x[4,]} (fourth row, all columns).
\item For square matrices, the special syntax \texttt{x[diag]} can be
  used to access the diagonal.
\item Consecutive rows or columns can be specified via the colon
  (\texttt{:}) character, as in \texttt{x[,2:4]} (columns 2 to 4).
  But note that, unlike some other matrix languages, the syntax
  \texttt{[m:n]} is illegal if $m>n$.
\item It is possible to use a vector to hold indices to a matrix. E.g.\
  if $e = [2,3,6]$, then \texttt{X[,e]} contains the second, third and
  sixth columns of $X$.
\end{enumerate}
Moreover, matrices can be empty (zero rows and columns). 

In the example above, the matrix \texttt{X} was constructed using
the function \texttt{zeros()}, whose meaning should be obvious, but
matrix elements can also be specified directly, as in
\begin{code}
scalar a = 2*3
matrix A = { 1, 2, 3 ; 4, 5, a }
\end{code}
The matrix is defined by rows; the elements on each row are separated
by commas and rows are separated by semicolons.  The whole expression
must be wrapped in braces.  Spaces within the braces are not
significant. The above expression defines a $2\times3$ matrix.

Note that each element should be a numerical value, the name of a
scalar variable, or an expression that evaluates to a scalar. In the
example above the scalar \texttt{a} was first assigned a value and
then used in matrix construction. (Also note, in passing, that
\texttt{a} and \texttt{A} are two separate identifiers, due to
case-sensitivity.)

\section{Matrix operations}
\label{sec:mat-op}

Matrix sum, difference and product are obtained via \texttt{+},
\texttt{-} and \texttt{*}, respectively. The prime operator
(\texttt{'}) can act as a unary operator, in which case it transposes
the preceding matrix, or as a binary operator, in which case it acts
as in ordinary matrix algebra, multiplying the transpose of the first
matrix into the second one.\footnote{In fact, in this case an
  optimized algorithm is used; you should always use \texttt{a'a}
  instead of \texttt{a'*a} for maximal precision and performance.}
Errors are flagged if conformability is a problem. For example:
\begin{code}
  matrix a = {11, 22 ; 33, 44}  # a is square 2 x 2
  matrix b = {1,2,3; 3,2,1}     # b is 2 x 3

  matrix c = a'         # c is the transpose of a
  matrix d = a*b        # d is a 2x3 matrix equal to a times b

  matrix gina = b'd     # valid: gina is 3x3
  matrix lina = d + b   # valid: lina is 2x3

  /* -- these would generate errors if uncommented ----- */

  # pina = a + b  # sum non-conformability
  # rina = d * b  # product non-conformability
\end{code}

Other noteworthy matrix operators include \texttt{\^} (matrix power),
\texttt{**} (Kronecker product), and the ``concatenation'' operators,
\verb|~| (horizontal) and \texttt{|} (vertical). Readers are invited
to try them out by running the following code
\begin{code}
matrix A = {2,1;0,1}
matrix B = {1,1;1,0}

matrix KP = A ** B
matrix PWR = A^3 
matrix HC = A ~ B
matrix VC = A | B

print A B KP PWR HC VC
\end{code}
Note, in particular, that $A^3 = A \cdot A \cdot A$, which is different
from what you get by computing the cubes of each element of $A$
separately.

Hansl also supports matrix left- and right-``division'', via the
\verb'\' and \verb'/' operators, respectively. The expression
\verb|A\b| solves $Ax = b$ for the unknown $x$. $A$ is assumed to be
an $m \times n$ matrix with full column rank. If $A$ is square the
method is LU decomposition. If $m > n$ the QR decomposition is used to
find the least squares solution. In most cases, this is numerically
more robust and more efficient than inverting $A$ explicitly.

Element-by-element operations are supported by the so-called ``dot''
operators, which are obtained by putting a dot (``\texttt{.}'') before
the corresponding operator. For example, the code
\begin{code}
A = {1,2; 3,4}
B = {-1,0; 1,-1}
eval A * B
eval A .* B
\end{code}
produces
\begin{code}
   1   -2 
   1   -4 

  -1    0 
   3   -4 
\end{code}

It's easy to verify that the first operation performed is regular
matrix multiplication $A \cdot B$, whereas the second one is the
Hadamard (element-by-element) product $A \odot B$. In fact, dot
operators are more general and powerful than shown in the example
above; see the chapter on matrices in \GUG{} for details.

Dot and concatenation operators are less rigid than ordinary matrix
operations in terms of conformability requirements: in most cases
hansl will try to do ``the obvious thing''. For example, a common
idiom in hansl is \texttt{Y = X ./ w}, where $X$ is an $n \times k$
matrix and $w$ is an $n \times 1$ vector. The result $Y$ is an $n
\times k$ matrix in which each row of $X$ is divided by the
corresponding element of $w$. In proper matrix notation, this
operation should be written as
\[
  Y = \langle w \rangle^{-1} X,
\]
where the $\langle \cdot \rangle$ indicates a diagonal
matrix. Translating literally the above expression would imply
creating a diagonal matrix out of $w$ and then inverting it, which is
computationally much more expensive than using the dot operation. A
detailed discussion is provided in \GUG.

Hansl provides a reasonably comprehensive set of matrix functions,
that is, functions that produce and/or operate on matrices. For a
full list, see the \GCR, but a basic ``survival kit'' is provided
in Table~\ref{tab:essential-matfuncs}.  Moreover, most scalar
functions, such as \texttt{abs(), log()} etc., will operate on a
matrix element-by-element.

\begin{table}[htbp]
  \centering
  \begin{tabular}{rp{0.6\textwidth}}
    \textbf{Function(s)} & \textbf{Purpose} \\
    \hline
    \texttt{rows(X), cols(X)} & return the number of rows and columns
    of $X$, respectively \\
    \texttt{zeros(r,c), ones(r,c)} & produce matrices with $r$ rows
    and $c$ columns, filled with zeros and ones, respectively \\
    \texttt{mshape(X,r,c)} & rearrange the elements of $X$ into a
    matrix with $r$ rows and $c$ columns \\
    \texttt{I(n)} & identity matrix of size $n$ \\
    \texttt{seq(a,b)} & generate a row vector containing integers form
    $a$ to $b$ \\
    \texttt{inv(A)} & invert, if possible, the matrix $A$ \\
    \texttt{maxc(A), minc(A), meanc(A)} & return a row vector
    with the max, min, means of each column of $A$, respectively\\
    \texttt{maxr(A), minr(A), meanr(A)} & return a column vector
    with the max, min, means of each row of $A$, respectively\\
    \texttt{mnormal(r,c), muniform(r,c)} & generate $r \times c$
    matrices filled with standard Gaussian and uniform pseudo-random
    numbers, respectively \\
    \hline
  \end{tabular}
  \caption{Essential set of hansl matrix functions}
  \label{tab:essential-matfuncs}
\end{table}

The following piece of code is meant to provide a concise example of
all the features mentioned above.

\begin{code}
# example: OLS using matrices

# fix the sample size
scalar T = 256

# construct vector of coefficients by direct imputation
matrix beta = {1.5, 2.5, -0.5} # note: row vector

# construct the matrix of independent variables
matrix Z = mnormal(T, cols(beta)) # built-in functions

# now construct the dependent variable: note the
# usage of the "dot" and transpose operators

matrix y = {1.2} .+ Z*beta' + mnormal(T, 1)

# now do estimation
matrix X = 1 ~ Z  # concatenation operator
matrix beta_hat1 = inv(X'X) * (X'y) # OLS by hand
matrix beta_hat2 = mols(y, X)       # via the built-in function
matrix beta_hat3 = X\y              # via matrix division

print beta_hat1 beta_hat2 beta_hat3
\end{code}

\section{Matrix pointers}
\label{sec:mat-pointers}

Hansl uses the ``by value'' convention for passing parameters to
functions. That is, when a variable is passed to a function as an
argument, what the function actually gets is a \emph{copy} of the
variable, which means that the value of the variable at the caller
level is not modified by anything that goes on inside the function.
But the use of pointers allows a function and its caller to cooperate
such that an outer variable can be modified by the function.

This mechanism is used by some built-in matrix functions to provide
more than one ``return'' value. The primary result is always provided
by the return value proper but certain auxiliary values may be
retrieved via ``pointerized'' arguments; this usage is flagged by
prepending the ampersand symbol, ``\texttt{\&}'', to the name of the
argument variable.

The \texttt{eigensym} function, which performs the eigen-analysis of
symmetric matrices, is a case in point. In the example below the first
argument $A$ represents the input data, that is, the matrix
whose analysis is required. This variable will not be modified in any
way by the function call. The primary result is the vector of
eigenvalues of $A$, which is here assigned to the variable
\texttt{ev}. The (optional) second argument, \texttt{\&V} (which may
be read as ``the address of \texttt{V}''), is used to retrieve the
right eigenvectors of $A$. A variable named in this way must be
already declared, but it need not be of the right dimensions to
receive the result; it will be resized as needed.
\begin{code}
matrix A = {1,2 ; 2,5}
matrix V
matrix ev = eigensym(A, &V)
print A ev V
\end{code}

%%% Local Variables: 
%%% mode: latex
%%% TeX-master: "hansl-primer"
%%% End: 

\chapter{Nice-looking output}
\label{chap:formatting}

\section{Formatted output}
\label{sec:printf}

A common occurrence when you're writing a script---particularly when
you intend for the script to be used by others, and you'd like the
output to be reasonably self-explanatory---is that you want to output
something along the following lines:
\begin{code}
The coefficient on X is Y, with standard error Z
\end{code}
where \texttt{X}, \texttt{Y} and \texttt{Z} are placeholders for
values not known at the time of writing the script; they will be
filled out as the values of variables or expressions when the script
is run. Let's say that at run time the replacements in the sentence
above should come from variables named \texttt{vname} (a string),
\texttt{b} (a scalar value) and \texttt{se} (also a scalar value),
respectively.

Across the spectrum of programming languages there are basically two
ways of arranging for this. One way originates in the \textsf{C}
language and goes under the name \texttt{printf}. In this approach we
(a) replace the generic placeholders \texttt{X}, \texttt{Y} and
\texttt{Z} with more informative \textit{conversion specifiers}, and
(b) append the variables (or expressions) that are to be stuck into
the text, in order. Here's the hansl version:
\begin{code}
printf "The coefficient on %s is %g, with standard error %g\n", vname, b, se
\end{code}
The value of \texttt{vname} replaces the conversion specifier
``\texttt{\%s},'' and the values of \texttt{b} and \texttt{se} replace
the two ``\texttt{\%g}'' specifiers, left to right. In relation to
hansl, here are the basic points you need to know: ``\texttt{\%s}''
pairs with a string argument, and ``\texttt{\%g}'' pairs with a
numeric argument.

The \textsf{C}-derived \texttt{printf} (either in the form of a
function, or in the form of a command as shown above) is present in
most ``serious'' programming languages. It is extremely versatile, and
in its advanced forms affords the programmer fine control over
the output.

In some scripting languages, however, \texttt{printf} is reckoned
``too difficult'' for non-specialist users. In that case some sort of
substitute is typically offered. We're skeptical: ``simplified''
alternatives to \texttt{printf} can be quite confusing, and if at some
point you want fine control over the output, they either do not
support it, or support it only via some convoluted mechanism. A
typical alternative looks something like this (please note,
\texttt{display} is \textit{not} a hansl command, it's just
illustrative):
\begin{code}
display "The coefficient on ", vname, "is ", b, ", with standard error ", se, "\n"
\end{code}
That is, you break the string into pieces and intersperse the names of
the variables to be printed. The requirement to provide conversion
specifiers is replaced by a default automatic formatting of the
variables based on their type. By the same token, the command line
becomes peppered with mutiple commas and quotation marks. If this looks
preferable to you, you are welcome to join one of the gretl mailing
lists and argue for its provision!

Anyway, to be a bit more precise about \cmd{printf}, its syntax goes
like this:
\begin{flushleft}
  \texttt{printf \emph{format}, \emph{arguments}}
\end{flushleft}
The \emph{format} is used to specify the precise way in which you want
the \emph{arguments} to be printed.

\subsection{The format string}
\label{sec:fmtstring}

In the general case the \cmd{printf} format must be an expression that
evaluates to a string, but in most cases will just be a \textit{string
  literal} (an alphanumeric sequence surrounded by double
quotes). However, some character sequences in the format have a
special meaning. As illustrated above, those beginning with a
percent sign (\texttt{\%}) are interpreted as placeholders for the
items contained in the argument list. In addition, special characters
such as the newline character are represented via a combination
beginning with a backslash (\verb|\|).

For example,
\begin{code}
printf "The square root of %d is (roughly) %6.4f.\n", 5, sqrt(5)
\end{code}
will print 
\begin{code}
The square root of 5 is (roughly) 2.2361.
\end{code}

Let's see how:
\begin{itemize}
\item The first special sequence is \verb|%d|: this indicates that we
  want an integer at that place in the output; since it is the
  leftmost ``percent'' expression, it is matched to the first
  argument, that is 5.
\item The second special sequence is \verb|%6.4f|, which stands for a
  decimal value with 4 digits after the decimal separator\footnote{The
    decimal separator is the dot in English, but may be different in
    other locales.} and at least 6 digits wide; this will be matched
  to the second argument. Note that arguments are separated by
  commas. Also note that the second argument is neither a scalar
  constant nor a scalar variable, but an expression that evaluates to
  a scalar.
\item The format string ends with the sequence \verb|\n|, which
  inserts a newline.
\end{itemize}

The conversion specifiers in the square-root example are relatively
fancy, but as we noted earlier \texttt{\%g} will work fine for
almost all numerical values in hansl. So we could have used the
simpler form:
\begin{code}
printf "The square root of %g is (roughly) %g.\n", 5, sqrt(5)
\end{code}
The effect of \texttt{\%g} is to print a number using up to 6
significant digits (but dropping trailing zeros); it automatically
switches to scientific notation if the number is very large or very
small. So the result here is
\begin{code}
The square root of 5 is (roughly) 2.23607.
\end{code}

The escape sequences \verb|\n| (newline), \verb|\t| (tab), \verb|\v|
(vertical tab) and \verb|\\| (literal backslash) are recognized. To
print a literal percent sign, use \verb|%%|.

Apart from those shown in the above example, recognized numeric
formats are \verb|%e|, \verb|%E|, \verb|%f|, \verb|%g|, and \verb|%G|,
in each case with the various modifiers available in C. The format
\verb|%s| should be used for strings. As in C, numerical values that
form part of the format (width and or precision) may be given directly
as numbers, as in \verb|%10.4f|, or they may be given as variables. In
the latter case, one puts asterisks into the format string and
supplies corresponding arguments in order. For example,

\begin{code}
  scalar width = 12 
  scalar precision = 6 
  printf "x = %*.*f\n", width, precision, x
\end{code}

If a matrix argument is given in association with a numeric format,
the entire matrix is printed using the specified format for each
element. A few more examples are given in table \ref{tab:printf-ex}.
\begin{table}[htbp]
  \centering
   {\small
    \begin{tabular}{p{0.45\textwidth}p{0.3\textwidth}}
      \textbf{Command} & \textbf{effect} \\
      \hline
      \verb|printf "%12.3f", $pi| & 3.142 \\
      \verb|printf "%12.7f", $pi| & 3.1415927 \\
      \verb|printf "%6s%12.5f%12.5f %d\n", "alpha",| \\
      \verb|   3.5, 9.1, 3| &
      \verb| alpha     3.50000     9.10000 3| \\
      \verb|printf "%6s%12.5f%12.5f\t%d\n", "beta",| \\
      \verb|   1.2345, 1123.432, %11| &
      \verb|  beta     1.23450  1123.43200 11| \\
      \verb|printf "%d, %10d, %04d\n", 1,2,3| & 
      \verb|1,          2, 0003| \\
      \verb|printf "%6.0f (%5.2f%%)\n", 32, 11.232| & \verb|32 (11.23%)| \\
      \hline
    \end{tabular}
  }
  \caption{Print format examples}
  \label{tab:printf-ex}
\end{table}

\subsection{Output to a string}
\label{sec:sprintf}

A closely related effect can be achieved via the \cmd{sprintf}
function: instead of being printed directly the result is stored in a
named string variable, as in
\begin{code}
  string G = sprintf("x = %*.*f\n", width, precision, x)
\end{code}
after which the variable \texttt{G} can be the object of further
processing.

\subsection{Output to a file}
\label{sec:outfile}

Hansl does not have a file or ``stream'' type as such, but the
\cmd{outfile} command can be used to divert output to a named text
file. To start such redirection you must give the name of a file; by
default a new file is created or an existing one overwritten but the
\option{append} can be used to append material to an existing file.
Only one file can be opened in this way at any given time. The
redirection of output continues until the command \cmd{end outfile} is
given; then output reverts to the default stream.

Here's an example of usage:
\begin{code}
  printf "One!\n"
  outfile "myfile.txt"
    printf "Two!\n"
  end outfile
  printf "Three!\n"
  outfile "myfile.txt" --append
    printf "Four!\n"
  end outfile
  printf "Five!\n"
\end{code}
After execution of the above the file \texttt{myfile.txt} will contain
the lines
\begin{code}
Two!
Four!  
\end{code}

Three special variants on the above are available. If you give the
keyword \texttt{null} in place of a real filename along with the write
option, the effect is to suppress all printed output until redirection
is ended. If either of the keywords \texttt{stdout} or \texttt{stderr}
are given in place of a regular filename the effect is to redirect
output to standard output or standard error output, respectively.

This command also supports a \option{quiet} option: its effect is to
turn off the echoing of commands and the printing of auxiliary
messages while output is redirected. It is equivalent to doing
\begin{code}
  set echo off 
  set messages off
\end{code}
before invoking \cmd{outfile}, except that when redirection is ended
the prior values of the \texttt{echo} and \texttt{messages} state
variables are restored.

\section{Graphics}

The primary graphing command in hansl is \texttt{gnuplot} which, as
the name suggests, in fact provides an interface to the
\textsf{gnuplot} program. It is used for plotting series in a dataset
(see part~\ref{part:hp-data}) or columns in a matrix. For an account
of this command (and some other more specialized ones, such as
\texttt{boxplot} and \texttt{qqplot}), see the \GCR.

%%% Local Variables: 
%%% mode: latex
%%% TeX-master: "hansl-primer"
%%% End: 

\chapter{Structured data types}
\label{chap:structypes}

Hansl possesses two kinds of ``structured data type'': associative
arrays, called \emph{bundles} and arrays ind the proper sense of the
word. Loosely speaking, the main difference between the two is that in
a bundle you can pack together variables of different types, while
arrays can hold one type of variable only.

\section{Bundles}
\label{chap:bundles}

Bundles are \emph{associative arrays}, that is, generic containers for
any assortment of hansl types (including other bundles) in which each
element is identified by a string. Python users call these
\emph{dictionaries}; in C++ and Java, they are referred to as
\emph{maps}; they are known as \emph{hashes} in Perl. We call them
\emph{bundles}. Each item placed in the bundle is associated with a
key which can used to retrieve it subsequently.

To use a bundle you first either ``declare'' it, as in
%
\begin{code}
bundle foo
\end{code}
%
or define an empty bundle using the \texttt{null} keyword:
%
\begin{code}
bundle foo = null
\end{code}
%
These two formulations are basically equivalent, in that they both
create an empty bundle. The difference is that the second variant may
be reused---if a bundle named \texttt{foo} already exists the effect
is to empty it---while the first may only be used once in a given
\app{gretl} session; it is an error to declare a variable that already
exists.

To add an object to a bundle you assign to a compound left-hand value:
the name of the bundle followed by the key. Two forms of syntax are
acceptable in this context. The first form requires that the key be
given as a quoted string literal enclosed in square brackets.  For
example, the statement
\begin{code}
foo["matrix1"] = m
\end{code}
adds an object called \texttt{m} (presumably a matrix) to bundle
\texttt{foo} under the key \texttt{matrix1}. 

A simpler syntax is also available, \emph{provided} the key satisfies
the rules for a \app{gretl} variable name (31 characters maximum,
starting with a letter and composed of just letters, numbers or
underscore).\footnote{When using the more elaborate syntax, keys do
  not have to be valid as variable names---for example, they can
  include spaces---but they are still limited to 31 characters.}  In
that case you can simply join the key to the bundle name with a dot,
as in

\begin{code}
foo.matrix1 = m
\end{code}

To get an item out of a bundle, again use the name of the bundle
followed by the key, as in

\begin{code}
matrix bm = foo["matrix1"]
# or using the dot notation
matrix m = foo.matrix1
\end{code}

Note that the key identifying an object within a given bundle is
necessarily unique. If you reuse an existing key in a new assignment,
the effect is to replace the object which was previously stored under
the given key. It is not required that the type of the replacement
object is the same as that of the original.

Also note that when you add an object to a bundle, what in fact
happens is that the bundle acquires a copy of the object. The external
object retains its own identity and is unaffected if the bundled
object is replaced by another. Consider the following script fragment:

\begin{code}
bundle foo
matrix m = I(3)
foo.mykey = m
scalar x = 20
foo.mykey = x
\end{code}

After the above commands are completed bundle \texttt{foo} does not
contain a matrix under \texttt{mykey}, but the original matrix
\texttt{m} is still in good health.

To delete an object from a bundle use the \texttt{delete} command,
with the bundle/key combination, as in

\begin{code}
delete foo.mykey
delete foo["quoted key"]
\end{code}

This destroys the object associated with the key and removes the key
from the hash table.\footnote{Internally, gretl bundles in fact take
  the form of \textsf{GLib} hash tablrd.}

Besides adding, accessing, replacing and deleting individual items,
the other operations that are supported for bundles are union and
printing. As regards union, if bundles \texttt{b1} and \texttt{b2} are
defined you can say

\begin{code}
bundle b3 = b1 + b2
\end{code}

to create a new bundle that is the union of the two others. The
algorithm is: create a new bundle that is a copy of \texttt{b1}, then
add any items from \texttt{b2} whose keys are not already present in
the new bundle. (This means that bundle union is not necessarily
commutative if the bundles have one or more key strings in common.)

If \texttt{b} is a bundle and you say \texttt{print b}, you get a
listing of the bundle's keys along with the types of the corresponding
objects, as in

\begin{code}
? print b
bundle b:
 x (scalar)
 mat (matrix)
 inside (bundle)
\end{code}

\subsection{Bundle usage}
\label{sec:bundle-usage}

To illustrate the way a bundle can hold information, we will use the
Ordinary Least Squares (OLS) model as an example: the following code
estimates an OLS regression and stores all the results in a bundle.

\begin{code}
/* assume y and X are given T x 1 and T x k matrices */

bundle my_model = null               # initialization
my_model["T"] = rows(X)              # sample size
my_model["k"] = cols(X)              # number of regressors
matrix e                             # will hold the residuals
b = mols(y, X, &e)                   # perform OLS via native function
s2 = meanc(e.^2)                     # compute variance estimator
matrix V = s2 .* invpd(X'X)          # compute covariance matrix

/* now store estimated quantities into the bundle */

my_model["betahat"] = b
my_model["s2"] = s2
my_model["vcv"] = V
my_model["stderr"] = sqrt(diag(V))
\end{code}

The bundle so obtained is a container that can be used for all sort of
purposes. For example, the next code snippet illustrates how to use
a bundle with the same structure as the one created above to perform
an out-of sample forecast. Imagine that $k=4$ and the value of
$\mathbf{x}$ for which we vant to forecast $y$ is
\[
  \mathbf{x}' = [ 10 \quad 1  \quad -3 \quad 0.5 ]
\]
The formulae for the forecast would then be
\begin{eqnarray*}
  \hat{y}_f & = & \mathbf{x}'\hat{\beta} \\
  s_f & = & \sqrt{\hat{\sigma}^2 + \mathbf{x}'V(\hat{\beta})\mathbf{x}} \\
  CI & = & \hat{y}_f \pm 1.96 s_f 
\end{eqnarray*}
where $CI$ is the (approximate) 95 percent confidence interval. The
above formulae translate into
\begin{code}
  x = { 10, 1, -3, 0.5 }
  scalar ypred    = x * my_model["betahat"]
  scalar varpred  = my_model["s2"] + qform(x, my_model["vcv"])
  scalar sepred   = sqrt(varpred)
  matrix CI_95    = ypred + {-1, 1} .* (1.96*sepred)
  print ypred CI_95
\end{code}

\section{Arrays}
\label{sec:arrays}

A gretl array is a container which can hold zero or more objects of a
certain type, indexed by consecutive integers starting at 1. It is
one-dimensional. This type is implemented by a quite ``generic''
back-end. The types of object that can be put into arrays are strings,
matrices, bundles and lists; a given array can hold only one of these
types.

\subsection{Array operations}

The following is, we believe, rather self-explanatory:

\begin{code}
strings S1 = array(3)
matrices M = array(4)
strings S2 = defarray("fish", "chips")
S1[1] = ":)"
S1[3] = ":("
M[2] = mnormal(2,2)
print S1
eval inv(M[2])
S = S1 + S2
print S
\end{code}

The \texttt{array()} takes an integer argument for the array size; the
\texttt{defarray()} function takes a variable number of arguments (one
or more), each of which may be the name of a variable of the given
type or an expression which evaluates to an object of that type.  The
corresponding output is

\begin{code}
Array of strings, length 3
[1] ":)"
[2] null
[3] ":("

     0.52696      0.28883 
    -0.15332     -0.68140 

Array of strings, length 5
[1] ":)"
[2] null
[3] ":("
[4] "fish"
[5] "chips"
\end{code}

In order to find the number of elements in an array, you can use the
\texttt{nelem()} function.

%%% Local Variables: 
%%% mode: latex
%%% TeX-master: "hansl-primer"
%%% End: 


\chapter{Numerical methods}
\label{chap:numerical}

\section{Numerical optimization}
\label{sec:hp-numopt}

Many, if not most, cases in which an econometrician wants to use a
programming language such as hansl, rather than relying on pre-canned
routines, involve some form of numerical optimization. This could take
the form of maximization of a likelihood or similar methods of
inferential statistics. Alternatively, optimization could be used in a
more general and abstract way, for example to solve portfolio choice
or analogous resource allocation problems.

Since hansl is Turing-complete, in principle any numerical
optimization technique could be programmed in hansl itself. Some such
techniques, however, are included in hansl's set of native
functions, in the interest of both simplicity of use and
efficiency. These are geared towards the most common kind of problem
encountered in economics and econometrics, that is unconstrained
optimization of differentiable functions.

In this chapter, we will briefly review what hansl offers to solve
generic problems of the kind
\[
\hat{\mathbf{x}} \equiv \argmax_{\mathbf{x} \in \Re^k} f(\mathbf{x}; \mathbf{a}),
\]
where $f(\mathbf{x}; \mathbf{a})$ is a function of $\mathbf{x}$, whose
shape depends on a vector of parameters $\mathbf{a}$. The objective
function $f(\cdot)$ is assumed to return a scalar real value. In most
cases, it will be assumed it is also continuous and differentiable,
although this need not necessarily be the case. (Note that while
hansl's built-in functions maximize the given objective function,
minimization can be achieved simply by flipping the sign of
$f(\cdot)$.)

A special case of the above occurs when $\mathbf{x}$ is a vector of
parameters and $\mathbf{a}$ represents ``data''. In these cases, the
objective function is usually a (log-)likelihood and the problem is
one of estimation. For such cases hansl offers several special
constructs, reviewed in section~\ref{sec:est-blocks}. Here we deal
with more generic problems; nevertheless, the differences are only in
the hansl syntax involved: the mathematical algorithms that gretl
employs to solve the optimization problem are the same.

The reader is invited to read the ``Numerical methods'' chapter of
\GUG\ for a comprehensive treatment. Here, we will only give a small
example which should give an idea of how things are done.

\begin{code}
function scalar Himmelblau(matrix x)
    /* extrema:
    f(3.0, 2.0) = 0.0, 
    f(-2.805118, 3.131312) = 0.0,
    f(-3.779310, -3.283186) = 0.0
    f(3.584428, -1.848126) = 0.0
    */
    scalar ret = (x[1]^2 + x[2] - 11)^2
    return -(ret + (x[1] + x[2]^2 - 7)^2)
end function

# ----------------------------------------------------------------------

set max_verbose 1

matrix theta1 = { 0, 0 }
y1 = BFGSmax(theta1, "Himmelblau(theta1)")
matrix theta2 = { 0, -1 }
y2 = NRmax(theta2, "Himmelblau(theta2)")

print y1 y2 theta1 theta2
\end{code}

We use for illustration here a classic ``nasty'' function from the
numerical optimization literature, namely the Himmelblau function,
which has four different minima; $f(x, y) = (x^2+y-11)^2 +
(x+y^2-7)^2$. The example proceeds as follows.
\begin{enumerate}
\item First we define the function to optimize: it must return a
  scalar and have among its arguments the vector to optimize. In this
  particular case that is its only argument, but there could have been
  other ones if necessary.  Since in this case we are solving for a
  minimum our definition returns the negative of the Himmelblau
  function proper.
\item We next set \verb|max_verbose| to 1. This is another example of
  the usage of the \cmd{set} command; its meaning is ``let me see how
  the iterations go'' and it defaults to 0. By using the \cmd{set}
  command with appropriate parameters, you control several features of
  the optimization process, such as numerical tolerances,
  visualization of the iterations, and so forth.
\item Define $\theta_1 = [0, 0]$ as the starting point.
\item Invoke the \cmd{BFGSmax} function; this will seek the maximum
  via the BFGS technique. Its base syntax is \texttt{BFGSmax(arg1,
    arg2)}, where \texttt{arg1} is the vector contining the
  optimization variable and \texttt{arg2} is a string containing the
  invocation of the function to maximize. BFGS will try several values
  of $\theta_1$ until the maximum is reached. On successful
  completion, the vector \texttt{theta1} will contain the final
  point. (Note: there's \emph{much} more to this. For details, be sure
  to read \GUG\ and the \GCR.)
\item Then we tackle the same problem but with a different starting
  point and a different optimization technique. We start from
  $\theta_2 = [0, -1]$ and use Newton--Raphson instead of BFGS,
  calling the \cmd{NRmax()} function instead if \cmd{BFGSmax()}. The
  syntax, however, is the same.
\item Finally we print the results.
\end{enumerate}

Table~\ref{tab:optim-output} on page \pageref{tab:optim-output}
contains a selected portion of the output. Note that the second run
converges to a different local optimum than the first one. This is a
consequence of having initialized the algorithm with a different
starting point. In this example, numerical derivatives were used, but
you can supply analytically computed derivatives to both methods if
you have a hansl function for them; see \GUG\ for more detail.

\begin{table}[ht]
  \begin{footnotesize}
\begin{scode}
? matrix theta1 = { 0, 0 }
Replaced matrix theta1
? y1 = BFGSmax(theta1, "Himmelblau(11, theta1)")
Iteration 1: Criterion = -170.000000000
Parameters:       0.0000      0.0000
Gradients:        14.000      22.000 (norm 0.00e+00)

Iteration 2: Criterion = -128.264504038 (steplength = 0.04)
Parameters:      0.56000     0.88000
Gradients:        33.298      39.556 (norm 5.17e+00)

...

--- FINAL VALUES: 
Criterion = -1.83015730011e-28 (steplength = 0.0016)
Parameters:       3.0000      2.0000
Gradients:    1.7231e-13 -3.7481e-13 (norm 7.96e-07)

Function evaluations: 39
Evaluations of gradient: 16
Replaced scalar y1 = -1.83016e-28
? matrix theta2 = { 0, -1 }
Replaced matrix theta2
? y2 = NRmax(theta2, "Himmelblau(11, theta2)")
Iteration 1: Criterion = -179.999876556 (steplength = 1)
Parameters:   1.0287e-05     -1.0000
Gradients:        12.000  2.8422e-06 (norm 7.95e-03)

Iteration 2: Criterion = -175.440691085 (steplength = 1)
Parameters:      0.25534     -1.0000
Gradients:        12.000  4.5475e-05 (norm 1.24e+00)

...

--- FINAL VALUES: 
Criterion = -3.77420797114e-22 (steplength = 1)
Parameters:       3.5844     -1.8481
Gradients:   -2.6649e-10  2.9536e-11 (norm 2.25e-05)

Gradient within tolerance (1e-07)
Replaced scalar y2 = -1.05814e-07
? print y1 y2 theta1 theta2

             y1 = -1.8301573e-28

             y2 = -1.0581385e-07

theta1 (1 x 2)

  3   2 

theta2 (1 x 2)

      3.5844      -1.8481 
\end{scode}
    
  \end{footnotesize}
  \caption{Output from maximization}
  \label{tab:optim-output}
\end{table}

The optimization methods hansl puts at your disposal are:
\begin{itemize}
\item BFGS, via the \cmd{BFGSmax()} function. This is in most cases
  the best compromise between performance and robustness. It assumes
  that the function to maximize is differentiable and will try to
  approximate its curvature by clever use of the change in the
  gradient between iterations. You can supply it with an
  analytically-computed gradient for speed and accuracy, but if you
  don't, the first derivatives will be computed numerically.
\item Newton--Raphson, via the \cmd{NRmax()} function. Actually, the
  function is less specific than the name implies. This is a
  ``curvature-based'' method, relying on the iterations
  \[
    x_{i+1} = -\lambda_i C(x_i)^{-1} g(x_i)
  \]
  where $g(x)$ is the gradient and $C(x_i)$ is some measure of
  curvature of the function to optimize; if $C(x)$ is the Hessian
  matrix, you get Newton--Raphson proper. Again, you can code your own
  functions for $g(\cdot)$ and $C(\cdot)$, but if you don't then
  numerical approximations to the gradient and the Hessian will be
  used, respectively. Other popular optimization methods (such as BHHH
  and the scoring algorithm) can be implemented by supplying to
  \cmd{NRmax()} the appropriate curvature matrix $C(\cdot)$. This
  method is very efficient when it works, but is rather fragile: for
  example, if $C(x_i)$ happens to be non-negative definite at some
  iteration convergence may become problematic.
\item Derivative-free methods: hansl offers simulated annealing, via
  the \cmd{simann()} function, and the Nelder--Mead simplex algorithm
  (also known as the ``amoeba'' method) via the \cmd{NMmax} function.
  These methods work even when the function to be maximized has some
  form of disconinuity or is not everywhere differentiable; however,
  they may be very slow and CPU-intensive.
\end{itemize}

\section{Numerical differentiation}
\label{sec:hp-numdiff}

For numerical differentiation we have \texttt{fdjac}. For example:

\begin{code}
set echo off
set messages off

function scalar beta(scalar x, scalar a, scalar b)
    return x^(a-1) * (1-x)^(b-1)
end function

function scalar ad_beta(scalar x, scalar a, scalar b)
    scalar g = beta(x, a-1, b-1)
    f1 = (a-1) * (1-x)
    f2 = (b-1) * x
    return (f1 - f2) * g
end function

function scalar nd_beta(scalar x, scalar a, scalar b)
    matrix mx = {x}
    return fdjac(mx, beta(mx, a, b))
end function

a = 3.5
b = 2.5

loop for (x=0; x<=1; x+=0.1)
    printf "x = %3.1f; beta(x) = %7.5f, ", x, beta(x, a, b)
    A = ad_beta(x, a, b)
    N = nd_beta(x, a, b)
    printf "analytical der. = %8.5f, numerical der. = %8.5f\n", A, N
endloop
\end{code}

returns 

\begin{code}
x = 0.0; beta(x) = 0.00000, analytical der. =  0.00000, numerical der. =  0.00000
x = 0.1; beta(x) = 0.00270, analytical der. =  0.06300, numerical der. =  0.06300
x = 0.2; beta(x) = 0.01280, analytical der. =  0.13600, numerical der. =  0.13600
x = 0.3; beta(x) = 0.02887, analytical der. =  0.17872, numerical der. =  0.17872
x = 0.4; beta(x) = 0.04703, analytical der. =  0.17636, numerical der. =  0.17636
x = 0.5; beta(x) = 0.06250, analytical der. =  0.12500, numerical der. =  0.12500
x = 0.6; beta(x) = 0.07055, analytical der. =  0.02939, numerical der. =  0.02939
x = 0.7; beta(x) = 0.06736, analytical der. = -0.09623, numerical der. = -0.09623
x = 0.8; beta(x) = 0.05120, analytical der. = -0.22400, numerical der. = -0.22400
x = 0.9; beta(x) = 0.02430, analytical der. = -0.29700, numerical der. = -0.29700
x = 1.0; beta(x) = 0.00000, analytical der. = -0.00000, numerical der. =       NA
\end{code}

Details on the algorithm used can be found in the \GCR. Suffice it to
say here that you have a \texttt{fdjac\_quality} setting that goes
from 0 to 2. The default value is to 0, which gives you
forward-difference approximation: this is the fastest algorithm, but
sometimes may not be precise enough. The value of 1 gives you
bilateral difference, while 2 uses Richardson extrapolation. Higher
values give greater accuracy but the method becomes considerably more
CPU-intensive.

% \section{Random number generation}

% \begin{itemize}
% \item Mersenne Twister in its various incarnations
% \item Ziggurat vs Box--Muller
% \item Other distributions
% \end{itemize}

%%% Local Variables: 
%%% mode: latex
%%% TeX-master: "hansl-primer"
%%% End: 


\chapter{Control flow}
\label{chap:hp-ctrlflow}

The primary means for controlling the flow of execution in a hansl
script are the the \cmd{loop} statement (repeated execution), \cmd{if}
statement (conditional execution), the \cmd{catch} modifier (which
enables the trapping of errors that would otherwise halt execution)
and the \cmd{quit} command (which forces termination).

\section{Loops}
\label{sec:hr-loops}

The basic hansl command for looping is (surprise) \cmd{loop}, and
takes the form
\begin{code}
loop <control-expression> <options>
    ...
endloop
\end{code}
In other words, the pair of statements \cmd{loop} and \cmd{endloop}
enclose the statements to repeat. Of course, loops can be nested.
Several variants of the \texttt{<control-expression>} for a loop are
supported, as follows:
\begin{enumerate}
\item unconditional loop
\item while loop
\item index loop
\item foreach loop
\item for loop.
\end{enumerate}
These variants are briefly described below.

\subsection{Unconditional loop}

This is the simplest variant. It takes the form
\begin{code}
loop <times>
   ...
endloop
\end{code}
where \texttt{<times>} is any expression that evaluates to a scalar,
namely the required number of iterations. This is only evaluated at
the beginning of the loop, so the number of iterations cannot be
changed from within the loop itself.

\subsection{Index loop}

The syntax is
\begin{code}
loop <counter>=<min>..<max>
   ...
endloop
\end{code}
The limits \texttt{<min>} and \texttt{<max>} must evaluate to scalars;
they are automatically turned into integers if they have a fractional
part. The \texttt{<counter>} variable is started at \texttt{<min>} and
incremented by 1 on each iteration until it equals \texttt{<max>}.

The counter is ``read-only'' inside the loop. You can access either
its numerical value or (by prepending a dollar sign) its string
representation. For example:
\begin{code}
scalar Max = 4
loop i=1..Max --quiet
    printf "$i squared = %d\n", i*i
endloop
\end{code}
%$

\subsection{While loop}

Here you have
\begin{code}
loop while <condition>
   ...
endloop
\end{code}
where \texttt{<condition>} should evaluate to a scalar, which is
re-evaluated at each iteration. Looping stops as soon as
\texttt{<condition>} becomes false (0). If \texttt{<condition>}
becomes \texttt{NA}, an error is flagged and execution stops.  By
default, \texttt{while} loops cannot exceed 1000 iterations. This is
intended as a safeguard against potentially infinite loops. This
setting can be overridden if necessary by setting the
\texttt{loop\_maxiter} state variable to a different value (see
chapter \ref{chap:settings}).

\subsection{Foreach loop}
\label{sec:loop-foreach}

In this case the syntax is
\begin{code}
loop foreach <counter> <catalogue>
   ...
endloop
\end{code}
where \texttt{<catalogue>} can be either a variable of type
\texttt{list} (i.e.\ a list of series), or a collection of
space-separated strings. The counter variable automatically takes on
the numerical values 1, 2, 3, \dots{} as execution proceeds, but its
string value (accessed by prepending a dollar sign) shadows the names
of the series in the list or the space-separated strings; this sort of
loop is designed for string substitution.

Here is an example in which the \texttt{<catalogue>} is a collection
of function names: 
\begin{code}
loop foreach f sqrt exp ln
    printf "$f(1) = %g\n", $f(1)
endloop
\end{code}

\subsection{For loop}

The final form of loop control emulates the \cmd{for} statement in the
C programming language.  The syntax is \texttt{loop for}, followed by
three component expressions, separated by semicolons and surrounded by
parentheses, that is
\begin{code}
loop for ( <init> ; <cont> ; <modifier> )
   ...
endloop
\end{code}

The three components are as follows:
\begin{enumerate}
\item Initialization (\texttt{<init>}): This is evaluated once at the
  start of the loop.
\item Continuation condition (\texttt{<cont>}): this is evaluated at
  the top of each iteration (including the first).  If the expression
  evaluates as true (non-zero), iteration continues, otherwise it
  stops. 
\item Modifier (\texttt{<modifier>}): an expression which modifies the
  value of some variable.  This is evaluated prior to checking the
  continuation condition, on each iteration after the first.
\end{enumerate}

Here's a simple example:
%
\begin{code}
loop for (r=0.01; r<.991; r+=.01)
   ...
endloop
\end{code}

Be aware of the limited precision of floating-point arithmetic. For
example, the code snippet below will iterate forever on most platforms
because \texttt{x} will never equal \textit{exactly} 0.01, even though
it might seem that it should.
\begin{code}
loop for (x=1; x!=0.01; x=x*0.1)
    printf "x = .18g\n", x
endloop  
\end{code}
However, if you replace the condition \texttt{x!=0.01} with
\texttt{x>=0.01}, the code will run as (probably) intended.
 
\vspace{1ex}

\subsection{Loop options}

Two options can be given to the \cmd{loop} statement. One is
\option{quiet}. This has simply the effect of suppressing certain
lines of output that gretl prints by default to signal progress and
completion of the loop; it has no other effect and the semantics of
the loop contents remain unchanged. This option is commonly used in
complex scripts.

The \option{progressive} option is mostly used as a quick and
efficient way to set up simulation studies. When this option is given,
a few commands (notably \cmd{print} and \cmd{store}) are given a
special, \emph{ad hoc} meaning. Please refer to the \GUG\ for more
information.
 
\subsection{Breaking out of loops}
\label{sec:loop-break}

The \cmd{break} command makes it possible to break out of a loop if
necessary. Note that
\begin{itemize}
\item if you nest loops, \cmd{break} in the innermost loop will
  interrupt that loop only and not the outer ones; and
\item hansl does not provide any equivalent to the C \texttt{continue}
  statement.
\end{itemize}

\section{The \cmd{if} statement}

Conditional execution in hansl uses the \cmd{if} keyword. Its fullest
usage is as follows
\begin{code}
if <condition>
   ...
elif <condition>
   ...
else 
   ...
endif  
\end{code}

Points to note:
\begin{itemize}
\item The \texttt{<condition>} can be any expression that evaluates to a
  scalar: 0 is interpreted as ``false'', non-zero is interpreted as
  ``true''; \texttt{NA} generates an error.
\item The \cmd{elif} and \cmd{else} clauses are optional: the minimal
  form is just \texttt{if} \dots{} \texttt{endif}.
\item The \cmd{elif} keyword can be repeated, making hansl's \cmd{if}
  statement a multi-way branch statement. There is no separate
  \cmd{switch} or \cmd{case} statement in hansl.
\item Conditional blocks of this sort can be nested up to a maximum of
  1024.
\end{itemize}

Example:
\begin{code}
scalar x = 15

# --- simple if ----------------------------------
if x >= 10
   printf "%g is more than two digits long\n", x
endif

# --- if with else -------------------------------
if x >= 0
   printf "%g is non-negative\n", x
else
   printf "%g is negative\n", x
endif

# --- multiple branches --------------------------
if missing(x)
   printf "%g is missing\n", x
elif x < 0
   printf "%g is negative\n", x
elif floor(x) == x
   printf "%g is an integer\n", x
else
   printf "%g is positive integer\n", x
endif
\end{code}

Note to Stata users: hansl's \cmd{if} statement is fundamentally
different from Stata's \texttt{if} option: the latter selects a
subsample of observations for some action, while the former is used to
decide if a group of statements must be executed or not; Stata calls
this the \emph{branching \texttt{if}}.

\subsection{The ternary query operator}

Besides use of \texttt{if}, the ternary query operator, \texttt{?:},
can be used to perform conditional assignment on a more ``micro''
level. This has the form
\begin{code}
result = <condition> ? <t-value1> : <f-value>
\end{code}
If \texttt{<condition>} evaluates as true (non-zero) then
\texttt{<t-value>} is assigned to \texttt{result}, otherwise
\texttt{<f-value>} is assigned. This is obviously more compact than
\texttt{if} \dots{} \texttt{else} \dots{} \texttt{endif}. The
following example replicates the \cmd{abs} function by hand:
\begin{code}
scalar ax = x>=0 ? x : -x
\end{code}
Of course, in the above case it would have been much simpler to just
write \texttt{ax = abs(x)}. Consider, however, the following case,
which exploits the fact that the ternary operator can be nested:
\begin{code}
scalar days = (m==2) ? 28 : maxr(m.={4,6,9,11}) ? 30 : 31
\end{code}
This example deserves a few comments: we want to compute the number of
days in a month, codified in the variable \texttt{m}. So, what we put into
the scalar \texttt{days} comes from the following pathway:
\begin{enumerate}
\item first, we check if the month is February (\texttt{m==2}); if so,
  we set \texttt{days} to 28 and we're done;\footnote{Ok, we don't
    check for leap years here.}
\item otherwise, we compute a matrix of zeros and ones via the
  \verb|m.={4,6,9,11}| operation (note the ``dot'' operator); if
  \texttt{m} equals any of the elements in the vector, the
  corresponding element of the result will be one, and zero otherwise;
\item by the function \cmd{maxr}, we take its maximum, so we're
  checking whether \texttt{m} is one of the four values corresponding
  to 30-days months;
\item since the above evaluates to a scalar, we put the right value
  into \texttt{days}.
\end{enumerate}

It is also
more flexible, in this respect: in contrast to the usage of
\texttt{if}, \texttt{<condition>} need not evaluate to a scalar.  So
for example one can do this sort of thing with matrices:
\begin{code}
matrix A = mnormal(4,4)
matrix B = upper(A.<0) ? -A : A
\end{code}
which will switch the sign of the negative elements of \texttt{A} on
and above its diagonal. To achieve the same via a loop, one should
have had to write a considerably longer chunk of code, such as
\begin{code}
matrix A = mnormal(4,4)
matrix B = A

loop r = 1..rows(A)
  loop c = r .. cols(A)
     if A[r,c] < 0
       B[r,c] = -A[r,c]
     endif
  end loop
end loop
\end{code}

\section{The \cmd{catch} modifier}

Hansl offers a rudimentary form of exception handling via the
\cmd{catch} modifier.

This is not a command in its own right but can be used as a prefix to
most regular commands: the effect is to prevent termination of a
script if an error occurs in executing the command. If an error does
occur, this is registered in an internal error code which can be
accessed as \dollar{error} (a zero value indicates success). The value
of \dollar{error} should always be checked immediately after using
\texttt{catch}, and appropriate action taken if the command failed.

The catch keyword cannot be used before \cmd{if}, \cmd{elif} or
\cmd{endif}.

\emph{\ldots Add example? \ldots}

\section{The \cmd{quit} statement}

When the \cmd{quit} statement is encountered in a hansl script,
execution stops. If the command-line program \app{gretlcli} is running
in batch mode, control returns to the operating system; if gretl is
running in interactive mode, \app{gretl} will wait for interactive
input. The \texttt{quit} command is rarely used in scripts since
execution automatically stops when script input is exhausted, but it
could be used in conjunction with \cmd{catch}. A script author could
arrange matters so that on encountering a certain error condition an
appropriate message is printed and the script is halted.

%%% Local Variables: 
%%% mode: latex
%%% TeX-master: "hansl-primer"
%%% End: 

\chapter{User-written functions}

Hansl natively provides a reasonably wide array of pre-defined
functions for manipulating variables of all kinds; the previous
chapters contain several examples. However, it is also possible to
extend hansl's native capabilities by defining additional
functions.

Here's how a user-defined function looks like:
\begin{flushleft}
\texttt{function \emph{type} \emph{funcname}(\emph{parameters})}\\
   \quad \ldots\\
   \quad \texttt{\emph{function body}}\\
   \quad \ldots \\
\texttt{end function}
\end{flushleft}

The opening line of a function definition contains these elements, in
strict order:

\begin{enumerate}
\item The keyword \texttt{function}.
\item \texttt{\emph{type}}, which states the type of value returned by the
  function, if any.  This must be one of \texttt{void} (if the
  function does not return anything), \texttt{scalar},
  \texttt{series}, \texttt{matrix}, \texttt{list} or \texttt{string}.
\item \texttt{\emph{funcname}}, the unique identifier for the
  function.  Function names have a maximum length of 31 characters;
  they must start with a letter and can contain only letters, numerals
  and the underscore character. It cannot coincide with a native gretl
  command or function.
\item The functions's \texttt{\emph{parameters}}, in the form of a
  comma-separated list enclosed in parentheses.\footnote{If you have
    to pass many parameters, you might want to consider wrapping them
    into a bundle to avoid syntax cluttering.} Note: parameters are
  the only way hansl function can receive anything from ``the
  outside''. In hansl, there is no such thing as global variables.
\end{enumerate}

Function parameters can be of any of the types shown below.

\begin{center}
\begin{tabular}{ll}
  \multicolumn{1}{c}{Type} & 
  \multicolumn{1}{c}{Description} \\ [4pt]
  \texttt{bool}   & scalar variable acting as a Boolean switch \\
  \texttt{int}    & scalar variable acting as an integer  \\
  \texttt{scalar} & scalar variable \\
  \texttt{series} & data series (see section~\ref{sec:series})\\
  \texttt{list}   & named list of series  (see section~\ref{sec:lists})\\
  \texttt{matrix} & matrix or vector \\
  \texttt{string} & string variable or string literal \\
  \texttt{bundle} & all-purpose container
\end{tabular}
\end{center}

Each element in the listing of parameters must include two terms: a
type specifier, and the name by which the parameter shall be known
within the function.  

The \emph{function body} contains (almost) arbitrary hansl code, which
should compute the \emph{return value}, that is the value the function
is supposed to yield. Any variable declared inside the function is
\emph{local}, so it will cease to exist when the function ends.

The \cmd{return} command is used to stop
execution of the code inside the function and deliver its result to
the calling code (note that this typically happens at the end of the
function body, but doesn't have to).  The function definition must end
with the expression \verb|end function|, on a line of its own.

In order to get a feel for how functions work in practice, here's a
simple example:
\begin{code}
function scalar quasi_log(scalar x)
/* popular approximation to the natural logarithm
   via Pad� polynomials */

   if (x<0)
      scalar ret = NA
   else 
      scalar ret = 2*(x-1)/(x+1)
   endif

   return ret
end function

loop for (x=0.5; x<2; x+=0.1)
   printf "x = %4.2f; ln(x) = %g, approx = %g\n", x, ln(x), quasi_log(x)
end loop
\end{code}

The code above computes the rational function
\[
  f(x) = 2 \cdot \frac{x-1}{x+1} ,
\]
which provides a decent approximation to the natural logarithm in a
neighborhood of 1.

\begin{enumerate}
\item We begin by defining the function via the \cmd{function}
  keyword; the function definition will end with the \texttt{end
    function} marker below;
\item since the function is meant to return a scalar, we put the
  keyword \texttt{scalar} after \cmd{function};
\item between the round brackets, the arguments follow; in this case,
  we only have one, which we call \texttt{x} and is assumed to be a
  scalar;
\item on the next line, the function definition begins; in this case,
  it includes a comment and an \cmd{if} block;
\item the function ends with the \cmd{return} keyword, which exports
  the result;\footnote{Beware: unlike other languages (eg Matlab), you cannot
  return multiple outputs from a function. However, you can return a
  bundle and stuff it with as many objects as you want.}
\item the next lines provide a simple usage example; note that in the
  \cmd{printf} command, the two functions \cmd{ln()} and
  \cmd{quasi\_log()} are indistinguishable from a purely syntactic
  viewpoint, although the former is a native function and the
  latter is a user-defined one. 
\end{enumerate}

In many cases, you may end up writing several functions, which may be
quite long; in order to avoid cluttering your script with the function
definition, hansl provides the \cmd{include} command, so you can put all
your function definition in a separate file (or separate files if
necessary). For example, imagine you saved the \verb|quasi_log()|
function definition above in a separate file called
\verb|quasilog_def.inp|: the code above could be written more
compactly as
\begin{code}
include quasilog_def.inp

loop for (x=0.5; x<2; x+=0.1)
   printf "x = %4.2f; ln(x) = %g, approx = %g\n", x, ln(x), quasi_log(x)
end loop
\end{code}
Moreover, \cmd{include} commands can be nested.


\section{Parameter passing and return values}
\label{sec:params-returns}

In hansl, parameters are passed \emph{by value}, so what is used
inside the function is a copy of the original argument. You may modify
it, but you'll be just modifying the copy. The following example
should make this point clear:
\begin{code}
function void f(scalar x)
    x = x*2
    print x
end function

scalar x = 3
f(x)
print x
\end{code}
Running the above code yields
\begin{code}
              x =  6.0000000
              x =  3.0000000
\end{code}
The first \cmd{print} statement happens inside the function, and the
displayed value is 6 because \verb|x| is doubled; however, what really
gets doubled is simply the \emph{copy} of \texttt{x}: this is
demonstrated by the second \cmd{print} statement.  If you need a
function to modify its arguments, you ought to use pointers (see
below).

\subsection{Pointers}

Each of the type-specifiers, with the exception of \texttt{list} and
\texttt{string}, may be modified by prepending an asterisk to the
associated parameter name, as in
%    
\begin{code}
function scalar myfunc (series *y, scalar *b)
\end{code}
This indicates that \verb|*y|, isn't a series, but a \emph{pointer} to
a series.  In practice, a pointer to a variable contains the memory
address at which the variable is stored.

This is usually considered a bit mysterious by people unfamiliar with
the C programming language, so allow us to explain how pointers work
by means of a (quite silly) example: suppose you set up a barber
shop. Ideally, your customers would walk through your shop door, sit
on a chair and have their hair trimmed or their beard shaved. However,
local regulations forbid you to modify anything going through your
shop door; of course, you wouldn't do much business if people must
walk out of your barber shop with their hair untouched. Nevertheless,
you have a simple way to go around this limitation: your customers can
come to your shop, tell you their home address and walk out. Then,
nobody stops you from going to their place and excercise your fine
profession.

Unlike C, you have to take no special precaution inside the function
body to distinguish the variable from its address\footnote{In C, this
  would be called \emph{dereferencing} the pointer.}. You just use the
variable's name. In order to supply the address of a variable when you
invoke the function, you use the ampersand (\verb|&|) operator. An
example should, hopefully, make this clear; the following code
\begin{code}
function void swap(scalar *a, scalar *b)
    scalar tmp = a
    a = b
    b = tmp
end function

scalar x = 0
scalar y = 1000000
swap(&x, &y)
print x y
\end{code}
gives
\begin{code}
              x =  1000000.0
              y =  0.0000000
\end{code}
so \texttt{x} and \texttt{y} have in fact been swapped. How?

First you have the function definition, in which the arguments are
pointers to scalars. Inside the function body, the distinction is
moot, as \verb|a| is taken to mean ``the scalar that you'll find at
the address \verb|*a|, which you'll get as the first parameter'' (and
likewise for \verb|b|). The rest of the function simply swaps
\texttt{a} and \texttt{b} by means of a local temporary variable.

Outside the function, we first initialize the two scalars \texttt{x}
to 0 and \texttt{y} to a big number. Then we call the function.  When
the function is called, it's given as arguments \verb|&a| and
\verb|&b|, which hansl identifies as ``the memory address of'' the two
scalars \texttt{a} and \texttt{b}, respectively.

Writing a function with pointer arguments has two main consequences:
first, as we just saw, it makes it possible to modify the function
arguments. Second, it avoids the computational cost of having to
allocate memory for a copy of the arguments and performing the copy
operation; such cost is proportional to the size of the argument. 
Hence, for matrix arguments, this is a nice way to write faster
functions, as producing a copy of a large matrix can be quite
time-consuming.\footnote{However, the \cmd{const} qualifier achieves the same
effect. See \GUG\ for further details.}

\section{Recursion}

Hansl functions can be recursive; what follows is the obligatory
factorial example:
\begin{code}
function scalar factorial(scalar n)
    if (n<0) || (n>floor(n))
        # filter out everything that isn't a 
        # non-negative integer
        return NA
    elif n==0
        return 1
    else
        return n*factorial(n-1)
    endif
end function

loop i=0..6 --quiet
    printf "%d! = %d\n", i, factorial(i)
end loop
\end{code}

Note: this is fun, but in practice, you'll be much better off using
the pre-cooked gamma function (or, better still, its logarithm).

%%% Local Variables: 
%%% mode: latex
%%% TeX-master: "hansl-primer"
%%% End: 


\part{With a dataset}
\label{part:hp-data}
\chapter{Что такое набор данных?}
\label{chap:dataset}

Набор данных --- это область памяти, предназначенная для хранения
данных, с которыми вы хотите работать, если таковые имеются. Можно
представлять ее как большую глобальную переменную, содержащую
(возможно, гигантскую) матрицу данных и огромный набор метаданных.

Пользователи \app{R} могут подумать, что набор данных похож на то, что
вы получаете, когда присоединяете фрейм данных с помощью
\texttt{attach} в \app{R}. На самом деле в Hansl нельзя одновременно
открывать более одного набора данных. Вот почему мы говорим о наборе
данных.

Когда набор данных присутствует в памяти (то есть «открыт»), ряд
объектов становится доступным для вашего скрипта Hansl и вы можете с
ними работать прозрачным и удобным способом. Конечно, сами данные вот
какие: столбцы матрицы набора данных называются сериями, которые будут
описаны в разделе \ref{sec:series}; иногда вам может понадобиться
организовать одну или несколько серий в список (раздел
\ref{sec:lists}). Кроме того, у вас есть возможность использовать
некоторые скаляры или матрицы, такие как количество наблюдений,
количество переменных, характер вашего набора данных (поперечные
сечения, временные ряды или панель) и т. д., все это переменные,
доступные только для чтения. Они называются аксессорами (accessors) и
будут обсуждаться в разделе \ref{sec:accessors}.

Вы можете открыть набор данных из файла на диске, с помощью команды
\cmd{open} или создав его с нуля.

\section{Создание набора данных с нуля}

Основными командами в этом контексте являются \cmd{nulldata} и
\cmd{setobs}. Например:

\begin{code}
set echo off
set messages off

set seed 443322           # инициализировать генератор случайных чисел
nulldata 240              # определить длину серий
setobs 12 1995:1          # определить выборку как ежемесячный набор данных с 1.1995г   
\end{code}
Для получения дополнительной информации о командах \cmd{nulldata} и
\cmd{setobs} см. Руководство пользователя Gretl и Справочник по
командам Gretl. Однако на этом этапе важно сказать, что при
необходимости вы можете изменить размер своего набора данных и/или
некоторые его характеристики (такие как периодичность) практически в
любой точке внутри вашего скрипта.  Как только ваш набор данных будет
определен, вы можете начать заполнять его сериями, или открывая их из
файлов, или создавая их с помощью соответствующих команд и функций.

\section{Чтение набора данных из файла}

Основные команды здесь: \cmd{open}, \cmd{append} и \cmd{join}. Команда
\cmd{open} --- это то, что вы хотите использовать в большинстве
случаев. Открываются самые различные форматы (собственный, CSV,
электронные таблицы, файлы данных, созданные другими пакетами, такими
как \textsf{Stata}, \textsf{Eviews}, \textsf{SPSS} и \textsf{SAS})), а
также автоматически настраиваются наборы данных под ваши нужды.

\begin{code}
  open mydata.gdt    # native format
  open yourdata.dta  # Stata format
  open theirdata.xls # Excel format
\end{code}

Команду \cmd{open} также можно использовать для чтения материалов из
Интернета, используя URL-адрес вместо имени файла, как в
\begin{code}
  open http://someserver.com/somedata.csv
\end{code}

В Руководстве пользователя Gretl описаны требования к файлам данных с
открытым текстом типа «CSV» для прямого импорта в gretl. Там также
описаны собственные форматы данных gretl (двоичные и основанные на
XML).

Команды \cmd{append} и \cmd{join} могут использоваться для добавления
последующих серий из файла в ранее открытый набор данных. Команда
\cmd{join} очень гибкая, поэтому ей посвящена отдельная глава в
Руководстве пользователя Gretl

\section{Сохранение наборов данных}

Команда \cmd{store} используется для записи текущего набора данных
(или подмножества) в файл. Помимо записи в собственные форматы gretl,
\cmd{store} также можно использовать для экспорта данных в формате CSV
или в формате \textsf{R}. Серии могут быть записаны в виде матриц с
помощью функции \cmd{mwrite}. Если у вас есть особые требования,
которые не выполняются функциями \cmd{store} или \cmd{mwrite}, можно
использовать \cmd{outfile} и \cmd{printf} одновременно (см. главу
~\ref{chap:formatting}), чтобы получить полный контроль над способом
сохранения данных.

\section{Команда \cmd{smpl}}

После того, как вы открыли набор данных каким-либо из
вышеперечисленных способов, команда \cmd{smpl} позволяет выборочно
отбрасывать наблюдения, так что ваша серия будет содержать только те
наблюдения, которые вы хотите в ней видеть (в процессе произойдет
автоматическое изменение размера набора данных). Дополнительную
информацию см. в главе 4 Руководства пользователя
Gretl.\footnote{Пользователей, имеющих опыт работы со Stata, может
  сначала смутить способ делать аналогичные операции в Hansl. Здесь вы
  сначала ограничиваете свою выборку с помощью команды \cmd{smpl},
  которая применяется до дальнейших инструкций, затем вы делаете то,
  что нужно. В Hansl не существует эквивалента \texttt{if} команде в
  Stata. }  Существует три основных варианта команды \cmd{smpl}:

\begin{enumerate}
\item Выбор непрерывного подмножества наблюдений: это будет в основном
  полезно для наборов данных временных рядов. Например:

  \begin{code}
    smpl 4 122            # select observations for 4 to 122
    smpl 1984:1 2008:4    # the so-called "Great Moderation" period
    smpl 2008-01-01 ;     # form January 1st, 2008 onwards
  \end{code}
\item Выбор наблюдений на основе некоторого критерия: это обычно то,
  что вам нужно делать с наборами данных поперечного сечения. Пример:
  \begin{code}
    smpl male == 1 --restrict                # males only
    smpl male == 1 && age < 30 --restrict    # just the young guys
    smpl employed --dummy                    # via a dummy variable
  \end{code}
  Обратите внимание, что в этом контексте новые ограничения
  накладываются на предыдущие. Чтобы начать с нуля, вы либо полностью
  должны сбросить образец через \texttt{smpl full} или использовать
  параметр \option{replace} вместе с \option{restrict}.
\item Ограничение активного набора данных некоторыми наблюдениями,
  чтобы определенный эффект достигался автоматически: например,
  создание случайной подвыборки или автоматическое исключение всех
  строк с отсутствующими наблюдениями. Это достигается с помощью опций
  \option{no-missing}, \option{contiguous}, и \option{random}.
\end{enumerate}

Для работы с панельными данными необходимо сделать некоторые
дополнительные уточнения; см. руководство пользователя Gretl.

\section{Аксессоры набора данных}
\label{sec:accessors}

Некоторые характеристики текущего набора данных можно определить с
помощью встроенного аксессора переменных (знак «доллар»). Основные из
них, которые возвращают скалярные значения, перечислены в Таблице
~\ref{tab:dataset-accessors}.

\begin{table}[htbp]
  \centering
  \begin{tabular}{lp{0.7\textwidth}}
    \textbf{Аксессор} & \textbf{Возвр.значение} \\ \hline
    \verb|$datatype| & Кодирование для типа набора данных: 
    0 = нет данн.; 1 = попереч.сечен. (недатиров.); 2 = врем.ряды;
    3 = panel \\
    \verb|$nobs| & Количество наблюдений в текущем
    диапазоне выборки \\
    \verb|$nvars| & Количество серий (включая константу)\\
    \verb|$pd| & Частота данных (1 для попереч.сечен., 4 для квартальных, и т.д.) \\
    \verb|$t1| & Основан. на единице индекс первого наблюдения в текущ. выборке \\
    \verb|$t2| & Основан. на единице индекс последнего наблюдения в текущ. выборке \\
    \hline
  \end{tabular}
  \caption{Основные аксессоры набора данных}
  \label{tab:dataset-accessors}
\end{table}
Кроме того, есть еще несколько специализированных средств доступа:
\dollar{obsdate}, \dollar{obsmajor}, \dollar{obsminor},
\dollar{obsmicro} и \dollar{unit}. Они относятся к временным рядам
и/или панельным данным, и все они возвращают ряды, см. справочник по
командам Gretl.

%%% Local Variables: 
%%% mode: latex
%%% TeX-master: "hansl-primer"
%%% End: 

\chapter{Series and lists}
\label{chap:series-etc}

Scalars, matrices and strings can be used in a hansl script at any
point; series and lists, on the other hand, are inherently tied to a
dataset and therefore can be used only when a dataset is currently
open.

\section{The \texttt{series} type}
\label{sec:series}
 
Series are just what any applied economist would call ``variables'',
that is, repeated observations of a given quantity; a dataset is an
ordered array of series, complemented by additional information, such
as the nature of the data (time-series, cross-section or panel),
descriptive labels for the series and/or the observations, source
information and so on. Series are the basic data type on which gretl's
built-in estimation commands depend.

The series belonging to a dataset are named via standard hansl
identifiers (strings of maximum length 31 characters as described
above). In the context of commands that take series as arguments,
series may be referenced either by name or by \emph{ID number}, that
is, the index of the series within the dataset. Position 0 in a
dataset is always taken by the automatic ``variable'' known as
\texttt{const}, which is just a column of 1s. The IDs of the actual
data series can be displayed via the \cmd{varlist} command. (But note
that in \textit{function calls}, as opposed to commands, series must
be referred to by name.)  A detailed description of how a dataset
works can be found in chapter 4 of \GUG.

Some basic rules regarding series follow:
\begin{itemize}
\item If \texttt{lngdp} belongs to a time series or panel dataset,
  then the syntax \texttt{lngdp(-1)} yields its first lag, and
  \texttt{lngdp(+1)} its first lead.
\item To access individual elements of a series, you use square
  brackets enclosing
  \begin{itemize}
  \item the progressive (1-based) number of the observation you want,
    as in \verb|lngdp[15]|, or
  \item the corresponding date code in the case of time-series data,
    as in \verb|lngdp[2008:4]| (for the 4th quarter of 2008), or
  \item the corresponding observation marker string, if the dataset
    contains any, as in \verb|GDP["USA"]|.
  \end{itemize}
\end{itemize}

The rules for assigning values to series are just the same as for
other objects, so the following examples should be self-explanatory:
\begin{code}
  series k = 3         # implicit conversion from scalar; a constant series
  series x = normal()  # pseudo-rv via a built-in function
  series s = a/b       # element-by-element operation on existing series

  series movavg = 0.5*(x + x(-1)) # using lags
  series y[2012:4] = x[2011:2]    # using individual data points
  series x2000 = 100*x/x[2000:1]  # constructing an index
\end{code}

\tip{In hansl, you don't have separate commands for \emph{creating}
  series and \emph{modifying} them. Other popular packages make this
  distinction, but we still struggle to understand why this is
  supposed to be useful.}

\subsection{Converting series to or from matrices}

The reason why hansl provides a specific series type, distinct from
the matrix type, is historical. However, is also a very convenient
feature.  Operations that are typically performed on series in applied
work can be awkward to implement using ``raw'' matrices---for example,
the computation of leads and lags, or regular and seasonal
differences; the treatment of missing values; the addition of
descriptive labels, and so on.

Anyway, it is straightforward to convert data in either direction
between the series and matrix types.
\begin{itemize}
\item To turn series into matrices, you use the curly braces syntax,
  as in
  \begin{code}
    matrix MACRO = {outputgap, unemp, infl}
  \end{code}
  where you can also use lists; the number of rows of the resulting
  matrix will depend on your currently selected sample.
\item To turn matrices into series, you can just use matrix columns,
  as in
  \begin{code}
    series y = my_matrix[,4]
  \end{code}
  But note that this will work only if the number of rows in
  \texttt{my\_matrix} matches the length of the dataset (or the
  currently selected sample range).
\end{itemize}

Also note that the \cmd{lincomb} and \cmd{filter} functions are quite
useful for creating and manipulating series in complex ways without
having to convert the data to matrix form (which could be
computationally costly with large datasets).

\subsection{The ternary operator with series}

Consider this assignment:

\begin{code}
  worker_income = employed ? income : 0
\end{code}

Here we assume that \texttt{employed} is a dummy series coding for
employee status. Its value will be tested for each observation in the
current sample range and the value assigned to \texttt{worker\_income}
at that observation will be determined accordingly. It is therefore
equivalent to the following much more verbose formulation (where
\dollar{t1} and \dollar{t2} are accessors for the start and end of the
sample range):
\begin{code}
series worker_income
loop i=$t1..$t2
    if employed[i]
        worker_income[i] = income[i]
    else
        worker_income[i] = 0
    endif
endloop
\end{code}

\section{The \texttt{list} type}
\label{sec:lists}
 
In hansl parlance, a \textit{list} is an array of integers,
representing the ID numbers of a set (in a loose sense of the word) of
series.  For this reason, the most common operations you perform on
lists are set operations such as addition or deletion of members,
union, intersection and so on. Unlike sets, however, hansl lists are
ordered, so individual list members can be accessed via the
\texttt{[]} syntax, as in \texttt{X[3]}.

There are several ways to assign values to a list.  The most basic
sort of expression that works in this context is a space-separated
list of series, given either by name or by ID number.  For example,
\begin{code}
list xlist = 1 2 3 4
list reglist = income price 
\end{code}
An empty list is obtained by using the function \texttt{deflist}
without any arguments, as in
\begin{code}
list W = deflist()  
\end{code}
or simply by bare declaration. Some more special forms (for example,
using wildcards) are described in \GUG.

The main idea is to use lists to group, under one identifier, one or
more series that logically belong together somehow (for example, as
explanatory variables in a model). So, for example,
\begin{code}
list xlist = x1 x2 x3 x4
ols y 0 xlist
\end{code}
is an idiomatic way of specifying the OLS regression that could also
be written as
\begin{code}
ols y 0 x1 x2 x3 x4
\end{code}
Note that we used here the convention, mentioned in section
\ref{sec:series}, by which a series can be identified by its ID number
when used as an argument to a command, typing \texttt{0} instead
of \texttt{const}.

Lists can be concatenated, as in as in \texttt{list L3 = L1 L2} (where
\texttt{L1} and \texttt{L2} are names of existing lists). This will
not necessarily do what you want, however, since the resulting list
may contain duplicates. It's more common to use the following set
operations:

\begin{center}
  \begin{tabular}{rl}
    \textbf{Operator} & \textbf{Meaning} \\
    \hline
    \verb,||, & Union \\
    \verb|&&| & Intersection \\
    \verb|-|  & Set difference \\
    \hline
  \end{tabular}
\end{center}

So for example, if \texttt{L1} and \texttt{L2} are existing lists,
after running the following code snippet
\begin{code}
  list UL = L1 || L2 
  list IL = L1 && L2
  list DL = L1 - L2
\end{code}
the list \texttt{UL} will contain all the members of \texttt{L1}, plus
any members of \texttt{L2} that are not already in \texttt{L1};
\texttt{IL} will contain all the elements that are present in both
\texttt{L1} and \texttt{L2} and \texttt{DL} will contain all the
elements of \texttt{L1} that are not present in \texttt{L2}. 

To \textit{append} or \textit{prepend} variables to an existing list,
we can make use of the fact that a named list stands in for a
``longhand'' list.  For example, assuming that a list \texttt{xlist}
is already defined (possibly as \texttt{null}), we can do
\begin{code}
list xlist = xlist 5 6 7
xlist = 9 10 xlist 11 12
\end{code}
 
Another option for appending terms to, or dropping terms from, an
existing list is to use \texttt{+=} or \texttt{-=}, respectively, as
in
\begin{code}
xlist += cpi
zlist -= cpi
\end{code}
A nice example of the above is provided by a common idiom: you may
see in hansl scripts something like
\begin{code}
  list C -= const
  list C = const C
\end{code}
which ensures that the series \texttt{const} is included (exactly
once) in the list \texttt{C}, and comes first.

\subsection{Converting lists to or from matrices}

The idea of converting from a list, as defined above, to a matrix may
be taken in either of two ways. You may want to turn a list into a
matrix (vector) by filling the latter with the ID numbers contained in
the former, or rather to create a matrix whose columns contain the
series to which the ID numbers refer. Both interpretations are
legitimate (and potentially useful in different contexts) so hansl
lets you go either way.

If you assign a list to a matrix, as in
\begin{code}
  list L = moo foo boo zoo
  matrix A = L
\end{code}
the matrix \texttt{A} will contain the ID numbers of the four series
as a row vector. This operation goes both ways, so the statement
\begin{code}
  list C = seq(7,10)
\end{code}
is perfectly valid (provided, of course, that you have at least 10
series in the currently open dataset).

If instead you want to create a data matrix from the series which
belong to a given list, you have to enclose the list name in curly
brackets, as in
\begin{code}
  matrix X = {L}
\end{code}

\subsection{The \texttt{foreach} loop variant with lists}

Lists can be used as the ``catalogue'' in the \texttt{foreach} variant
of the \cmd{loop} construct (see section \ref{sec:loop-foreach}). This
is especially handy when you have to perform some operation on
multiple series. For example, the following syntax can be used to
calculate and print the mean of each of several series:
\begin{code}
list X = age income experience
loop foreach i X
    printf "mean($i) = %g\n", mean($i)
endloop
\end{code}

%%% Local Variables: 
%%% mode: latex
%%% TeX-master: "hansl-primer"
%%% End: 

\chapter{Estimation methods}
\label{chap:estimation}

You can, of course, estimate econometric models via hansl without
having a dataset (in the sense in which we're using that term here) in
place---just as you might in \textsf{Matlab}, for instance. You'll
need \textit{data}, but these can be loaded in matrix form (see the
\texttt{mread} function in the \GCR), or generated artificially via
functions such as \texttt{mnormal} or \texttt{muniform}. You can roll
your own estimator using hansl's linear algebra primitives, and you
also have access to more specialized functions such as \texttt{mols}
(see section \ref{sec:mat-op}) and \texttt{mrls} (restricted least
squares) if you need them.

However, unless you need to use an estimation method which is not
currently supported by gretl, or have a strong desire to reinvent the
wheel, you will probably want to make use of the built-in estimation
commands available in hansl. These commands are series-oriented and
therefore require a dataset. They fall into two main categories:
``canned'' procedures, and generic tools that can be used to estimate
a wide variety of models based on common principles.

\section{Canned estimation procedures}
\label{sec:canned}

``Canned'' doesn't sound very appetizing these days but it's the term
that's commonly used. Basically it means two things, neither of them
in fact unappetizing.
\begin{itemize}
\item The user is presented with a fairly simple interface. A few
  inputs must be specified, and perhaps a few options selected, then
  the heavy lifting is done within the gretl library.
\item The algorithm is written in C, by experienced coders. It is
  therefore faster (possibly \textit{much} faster) than an
  implementation in an interpreted language such as hansl.
\end{itemize}

Such procedures share, more or less, the syntax
\begin{flushleft}
\texttt{\emph{commandname parameters options}}
\end{flushleft}
with a few exceptions (e.g.\ systems).

A crude categorization: 
\begin{description}
\item[Linear, single equation] \cmd{ols}, \cmd{tsls}, \cmd{ar1},
  \cmd{mpols}
\item[Linear, multi-equation] \cmd{system}, \cmd{var}, \cmd{vecm} 
\item[Nonlinear, single equation] \cmd{logit}, \cmd{probit},
  \cmd{poisson}, \cmd{negbin}, \cmd{tobit}, \cmd{intreg},
  \cmd{logistic}
\item[Panel] \cmd{panel}, \cmd{dpanel}
\item[Assorted] \cmd{arima}, \cmd{garch}, \cmd{heckit},
  \cmd{quantreg}, \cmd{lad}, \cmd{biprobit}, \cmd{duration}
\end{description}

Don't let names deceive you: for example, the \cmd{probit} command can
estimate ordered models, random-effect panel probit models, \dots{}

See the \GCR{} for details.

\subsection{Simultaneous systems}

The \cmd{system} block.

\section{Post-estimation accessors}
\label{sec:postest-accessors}

After having estimated a model, you can access most of the relevant
quantities via accessors.

\begin{itemize}
\item Generic: \dollar{coeff}, \dollar{vcv}, \dollar{uhat}
\item Model-specific: for example, \dollar{jbeta}, \dollar{h},
  \dollar{mnlprobs}
\end{itemize}

\subsection{Named models}

\begin{code}
diff y x
ADL <- ols y const y(-1) x(0 to -1)
ECM <- ols d_y const d_x y(-1) x(-1)
ssr_a = ADL.$ess
ssr_e = ECM.$ess # should be equal
\end{code}

\section{Generic estimation tools}
\label{sec:est-blocks}

These are useful if you want to write your own estimator. Here we give
a generic overview. See the relevant chapters in \GUG{} for a full explanation.

\begin{itemize}
\item \cmd{nls}
\item \cmd{mle}
\item \cmd{gmm}
\end{itemize}

\subsection{Formatting your output}

The \cmd{modprint} command.

\label{LastPage}


%%% Local Variables: 
%%% mode: latex
%%% TeX-master: "hansl-primer"
%%% End: 


\part{Reference}
\label{part:hp-reference}
\chapter{Правила работы с пробелами}

Языки программирования различаются своими правилами использования
пробелов в программе. Здесь мы устанавливаем правила их использования
в Hansl. Правила несколько различаются между командами, с одной
стороны, и вызовами функций и присваиванием с другой.

\section{Пробелы в командах}
Команды Hansl структурированы следующим образом: сначала идет
командное слово (например, \texttt{ols}, \texttt{summary}); потом ноль
или более аргументов (часто это названия серий); затем следует ноль
или более вариантов (некоторые из которых могут заключать в себе
параметры). Правила следующие:

\begin{enumerate}
\item Упомянутые выше отдельные элементы всегда должны быть разделены
  хотя бы одним пробелом, а где требуется один пробел, вы можете
  вставить их столько, сколько захотите.
\item Каждый раз, когда параметр снабжен флагом опции, параметр должен
  быть присоединен к флагу со знаком равенства, без пробела:
\begin{code}
ols y 0 x --cluster=clustvar  # правильно
ols y 0 x --cluster =clustvar # неверно!
\end{code}
\end{enumerate}

\section{Пробелы в вызовах функций и присваивании}

По большей части пробелы в вызовах функций и назначении не имеют
значения; по желанию он может быть вставлен или нет. Например, в
следующих наборах операторов каждый член одинаково приемлем
синтаксически (хотя некоторые выглядят ужасно!):
\begin{code}
# set 1
y = sqrt(x)
y=sqrt(x)
# set 2
c = cov(y1, y2)
c=cov(y1,y2)
c  = cov(y1 , y2)
\end{code}

Пожалуйста, обратите особое внимание на следующие исключения:
\begin{enumerate}
\item Когда присвоение начинается с ключевого слова типа данных,
  такого как серия или матрица (\texttt{series} или \texttt{matrix}),
  они должны быть отделены от того, что следует за ними хотя бы одним
  пробелом, как в
\begin{code}
series y = normal() # или: series y=normal()
\end{code}
\item При вызове функции к ее имени без пробела должна быть добавлена
  открывающая скобка, обозначающая начало списка аргументов:
\begin{code}
c = cov(y1, y2)  # правильно
c = cov (y1, y2) # неверно!
\end{code}
\end{enumerate}

\label{LastPage}

%%% Local Variables: 
%%% mode: latex
%%% TeX-master: "hansl-primer"
%%% End: 
\chapter{Operators}
\label{chap:operators}

\section{Precedence}

Table~\ref{tab:ops} lists the operators available in \app{gretl} in
order of decreasing precedence: the operators on the first row have
the highest precedence, those on the second row have the second
highest, and so on. Operators on any given row have equal
precedence. Where successive operators have the same precedence the
order of evaluation is in general left to right. The exceptions are
exponentiation and matrix transpose-multiply. The expression
\verb|a^b^c| is equivalent to \verb|a^(b^c)|, not \verb|(a^b)^c|, 
and similarly \verb|A'B'C'| is equivalent to \verb|A'(B'(C'))|.

\begin{table}[htbp]
\caption{Operator precedence}
\label{tab:ops}
\begin{center}
\begin{tabular}{lllllllll}
\verb|()| & \verb|[]| & \texttt{.} & \verb|{}| \\
\texttt{!} & \texttt{++} & \verb|--| & \verb|^| & \verb|'| \\
\texttt{*} & \texttt{/} & \texttt{\%} & \verb+\+ & \texttt{**} \\
\texttt{+} & \texttt{-} & \verb|~| & \verb+|+ & \\
\verb|>| & \verb|<| & \verb|>=| & \verb|<=| & \texttt{..} \\
\texttt{==} & \texttt{!=} \\
\verb|&&| \\
\verb+||+ \\
\texttt{?:} \\
\end{tabular}
\end{center}
\end{table}

In addition to the basic forms shown in the Table, several operators
also have a ``dot form'' (as in ``\texttt{.+}'' which is read as ``dot
plus''). These are element-wise versions of the basic operators, for
use with matrices exclusively; they have the same precedence as their
basic counterparts. The available dot operators are as follows.

\begin{center}
\begin{tabular}{cccccccccc}
\verb|.^| & \texttt{.*} & \texttt{./} & \texttt{.+} &
 \texttt{.-} & \verb|.>| & \verb|.<| & \verb|.>=| &
 \verb|.<=| & \texttt{.=} \\
\end{tabular}
\end{center}

Each basic operator is shown once again in the following list along
with a brief account of its meaning. Apart from the first three sets
of grouping symbols, all operators are binary except where otherwise
noted.

\begin{longtable}{ll}
\verb|()| & Function call \\
\verb|[]|  & Subscripting \\
\texttt{.} & Bundle membership (see below) \\
\verb|{}|  & Matrix definition \\
\texttt{!} & Unary logical NOT \\
\texttt{++} & Increment (unary) \\
\verb|--| & Decrement (unary) \\
\verb|^|  & Exponentiation \\
\verb|'|  & Matrix transpose (unary) or transpose-multiply (binary) \\
\texttt{*} & Multiplication \\
\texttt{/} & Division, matrix ``right division'' \\
\texttt{\%} & Modulus \\
\verb+\+    & Matrix ``left division'' \\
\texttt{**} & Kronecker product \\
\texttt{+} & Addition \\
\texttt{-} & Subtraction \\
\verb|~| & Matrix horizontal concatenation \\
\verb+|+ & Matrix vertical concatenation \\
\verb|>| & Boolean greater than \\
\verb|<| & Boolean less than \\
\verb|>=| & Greater than or equal \\
\verb|<=| & Less than or equal \\
\texttt{..} & Range from--to (in constructing lists) \\
\texttt{==} & Boolean equality test \\
\texttt{!=} & Boolean inequality test \\
\verb|&&| & Logical AND \\
\verb+||+ & Logical OR \\
\texttt{?:} & Conditional expression \\
\end{longtable}

The interpretation of ``\texttt{.}'' as the bundle membership operator
is confined to the case where it is immediately preceded by the
identifier for a bundle, and immediately followed by a valid
identifier (key).

Details on the use of the matrix-related operators (including the dot
operators) can be found in the chapter on matrices in the
\textit{Gretl User's Guide}.

\section{Assignment}

The operators mentioned above are all intended for use on the
right-hand side of an expression which assigns a value to a variable
(or which just computes and displays a value---see the \texttt{eval}
command). In addition we have the assignment operator itself,
``\texttt{=}''. In effect this has the lowest precedence of all: the
entire right-hand side is evaluated before assignment takes place.

Besides plain ``\texttt{=}'' several ``inflected'' versions of
assignment are available. These may be used only when the left-hand
side variable is already defined. The inflected assignment yields a
value that is a function of the prior value on the left and the
computed value on the right. Such operators are formed by prepending a
regular operator symbol to the equals sign. For example,
%
\begin{code}
y += x
\end{code}
%
The new value assigned to \texttt{y} by the statement above is the
prior value of \texttt{y} plus \texttt{x}. The other available
inflected operators, which work in an exactly analogous fashion, are
as follows.

\begin{center}
\begin{tabular}{ccccccc}
\texttt{-=} & \texttt{*=} & \texttt{/=} & \verb|%=| & 
  \verb|^=| & \verb|~=| & \verb+|=+ \\
\end{tabular}
\end{center}

In addition, a special form of inflected assignment is provided for
matrices. Say matrix \texttt{M} is $2 \times 2$. If you execute
\texttt{M = 5} this has the effect of replacing \texttt{M} with a $1
\times 1$ matrix with single element 5. But if you do \texttt{M .= 5}
this assigns the value 5 to all elements of \texttt{M} without
changing its dimensions.

\section{Increment and decrement}

The unary operators \texttt{++} and \verb|--| follow their
operand,\footnote{The C programming language also supports prefix
  versions of \texttt{++} and \verb|--|, which increment or decrement
  their operand before yielding its value. Only the postfix form is
  supported by \app{gretl}.}  which must be a variable of scalar
type. Their simplest use is in stand-alone expressions, such as
%
\begin{code}
j++  # shorthand for j = j + 1
k--  # shorthand for k = k - 1
\end{code}
%
However, they can also be embedded in more complex expressions, in
which case they first yield the original value of the variable in
question, then have the side-effect of incrementing or decrementing
the variable's value. For example:
%
\begin{code}
scalar i = 3
k = i++
matrix M = zeros(10, 1)
M[i++] = 1
\end{code}
%
After the second line, \texttt{k} has the value 3 and \texttt{i} has
value 4. The last line assigns the value 1 to element 4 of
matrix \texttt{M} and sets \texttt{i} = 5.











\chapter{Greek-letter identifiers}
\label{chap:greeks}

As mentioned in chapter~\ref{chap:hello}, the sole exception to the
requirement that hansl identifiers must be plain ASCII is that they
may take the form of a single Greek letter. Here are the details.

\begin{itemize}
\item This exception applies only to names of variables \textit{other
    than series}; the names of series must always be ASCII.
\item The supported Greek characters are the 24 (unaccented) letters
  in the basic Greek alphabet, minus omicron, which is
  indistinguishable from the Latin `o'.
\item These letters may be used in lower or upper case (constituting
  distinct identifiers, as usual in hansl), except for the several
  upper-case letters which are indistinguishable in the Latin and
  Greek alphabets (`A', `B', `E', `K', `M', `N', \dots).
\item The letters must be encoded in UTF-8.
\end{itemize}

In gretl's graphical interface (script editor and GUI ``console''),
the acceptable Greek letters can be entered by typing a Latin letter
while the \texttt{Alt} key is depressed. The mappings from Latin to
Greek are shown below: lower case first, then upper case.

\begin{center}
  \begin{tabular}{ccl@{\hskip 4em}ccl}
    Latin & Greek & & Latin & Greek \\[4pt]
    a & $\alpha$ & alpha & n & $\nu$ & nu \\
    b & $\beta$ & beta & p & $\pi$ & pi \\
    c & $\chi$ & chi & q & $\theta$ & theta \\
    d & $\delta$ & delta & r & $\rho$ & rho \\
    e & $\epsilon$ & epsilon & s & $\sigma$ & sigma \\
    f & $\phi$ & phi & t & $\tau$ & tau \\
    g & $\gamma$ & gamma & u & $\upsilon$ & upsilon \\
    h & $\eta$ & eta & v & $\nu$ & nu \\
    i & $\iota$ & iota & w & $\omega$ & omega \\
    j & $\psi$ & psi & x & $\xi$ & xi \\
    k & $\kappa$ & kappa & y & $\upsilon$ & upsilon \\
    l & $\lambda$ & lambda & z & $\zeta$ & zeta \\
    m & $\mu$ & mu \\
  \end{tabular}
\end{center}

\begin{center}
  \begin{tabular}{ccl}
    Latin & Greek \\[4pt]
    D & $\Delta$ & Delta \\
    F & $\Phi$ & Phi \\
    G & $\Gamma$ & Gamma \\
    J & $\Psi$ & Psi\\
    L & $\Lambda$ & Lambda \\
    P & $\Pi$ & Pi \\
    Q & $\Theta$ & Theta \\
    S & $\Sigma$ & Sigma \\
    U & $\Upsilon$ & Upsilon \\
    W & $\Omega$ & Omega \\
    X & $\Xi$  & Xi \\
    Y & $\Upsilon$ & Upsilon \\
  \end{tabular}
\end{center}



% \part{Further topics}
% \part{Further topics}

\chapter{The \texttt{set} command}
\label{chap:settings}

\chapter{Nice-looking output}
\label{chap:formatting}

\cmd{printf}, \cmd{sprintf} and the ``commands vs functions'' thing.

The \cmd{modprint} command.

Friendly relationship with \LaTeX{} and \app{gnuplot}

\chapter{Function packages}

\chapter{Going abroad}
``Foreign'' blocks.

\chapter{Tricks}
\section{String substitution}
\label{sec:stringsub}

String variables can be used in two ways in hansl scripting: the name
of the variable can be typed ``as is'', or it may be preceded by the
``at'' sign, \verb|@|. In the first variant the named string is
treated as a variable in its own right, while the second calls for
``string substitution''. Which of these variants is appropriate
depends on the context.

In the following contexts the names of string variables should be
given in plain form (without the ``at'' sign):

\begin{itemize}
\item When such a variable appears among the arguments to the
  commands \texttt{printf} or \texttt{sprintf}.
\item When such a variable is given as the argument to a function.
\item On the right-hand side of a \texttt{string} assignment.
\end{itemize}

Here is an illustration of the use of a named string argument with
\texttt{printf}:
%
\begin{code}
? string vstr = "variance"
Generated string vstr
? printf "vstr: %12s\n", vstr
vstr:     variance
\end{code}

String substitution, on the other hand, can be used in contexts where
a string variable is not acceptable as such. If gretl encounters
the symbol \verb|@| followed directly by the name of a string
variable, this notation is in effect treated as a ``macro'': the value
of the variable is sustituted literally into the command line before
the regular parsing of the command is carried out.

One common use of string substitution is when you want to construct
and use the name of a series programatically. For example, suppose you
want to create 10 random normal series named \texttt{norm1} to
\texttt{norm10}. This can be accomplished as follows.
%
\begin{code}
string sname = null
loop i=1..10
  sprintf sname "norm%d", i
  series @sname = normal()
endloop
\end{code}
%
Note that plain \texttt{sname} could not be used in the second line
within the loop: the effect would be to attempt to overwrite the
string variable named \texttt{sname} with a series of the same name,
hence generating an error. What we want is for the current
\textit{value} of \texttt{sname} to be dumped directly into the
command that defines a series, and the ``\verb|@|'' notation achieves
that.

Another typical use of string substitution is when you want the
options used with a particular command to vary depending on
some condition. For example,
%
\begin{code}
function void use_optstr (series y, list xlist, int verbose)
   string optstr = verbose ? "" : "--simple-print"
   ols y xlist @optstr 
end function

open data4-1
list X = const sqft
use_optstr(price, X, 1)
use_optstr(price, X, 0)
\end{code}

In the first call to the function \texttt{use\_optstr} the option
string \verb|--simple-print| will be appended to the \cmd{ols}
command; in the second call it will be omitted.

When printing the value of a string variable using the \texttt{print}
command, the plain variable name should generally be used, as in
%
\begin{code}
string s = "Just testing"
print s
\end{code}
%
The following variant is equivalent, though clumsy and not
recommended.
%
\begin{code}
string s = "Just testing"
print "@s"
\end{code}
%
But note that this next variant does something quite different.
%
\begin{code}
string s = "Just testing"
print @s
\end{code}
%
After string substitution, the command reads
%
\begin{code}
print Just testing
\end{code}
%
which is taken as a request to print the values of two variables,
\texttt{Just} and \texttt{testing}.

\subsection{Performance hint}

Warning: gretl attempts to speed up execution of assignment
expressions within loops by ``pre-compiling'' the expression.
However, if such an expression contains string substitution it cannot
be pre-compiled. So be sure to use the \verb|@| operator only when
strictly necessary if you want to avoid a performance penalty.

%%% Local Variables: 
%%% mode: latex
%%% TeX-master: "hansl-primer"
%%% End: 

% \clearpage
% \begin{thebibliography}

  Akaike, H. (1974) ``A New Look at the Statistical Model
  Identification'', \emph{IEEE Transactions on Automatic Control},
  AC-19, pp.  716--23.

  Anderson, T. W. and Hsiao, C. (1981) ``Estimation of Dynamic Models
  with Error Components'', \emph{Journal of the American Statistical
    Association}, 76, pp. 598--606.

  Arellano, M. and Bond, S. (1991) ``Some Tests of Specification for
  Panel Data: Monte Carlo Evidence and an Application to Employment
  Equations'', \emph{The Review of Economic Studies}, 58, pp.
  277--297.

  Baiocchi, G. and Distaso, W. (2003) ``GRETL: Econometric software
  for the GNU generation'', \emph{Journal of Applied Econometrics},
  18, pp. 105--10.
  
  Baltagi, B. H. (1995) \emph{Econometric Analysis of Panel Data}, New
  York: Wiley.

  Baxter, M. and King, R. G. (1995) ``Measuring Business Cycles:
  Approximate Band-Pass Filters for Economic Time Series'', National
  Bureau of Economic Research, Working Paper No. 5022.

  Belsley, D., Kuh, E. and Welsch, R. (1980) \emph{Regression
    Diagnostics}, New York: Wiley.

  Berndt, E., Hall, B., Hall, R. and Hausman, J. (1974) ``Estimation
  and Inference in Nonlinear Structural Models'', \emph{Annals of
    Economic and Social Measurement}, 3/4, pp. 653--665.

  Blundell, R. and Bond S. (1998) ``Initial Conditions and Moment
  Restrictions in Dynamic Panel Data Models'', \emph{Journal of
    Econometrics}, 87, pp. 115--143.

  Box, G. E. P. and Jenkins, G. (1976) \textit{Time Series Analysis:
  Forecasting and Control}, San Franciso: Holden-Day.

  Box, G. E. P. and Muller, M. E.  (1958) ``A Note on the Generation
  of Random Normal Deviates'', \emph{Annals of Mathematical
    Statistics}, 29, pp.  610--11.

  Davidson, R. and MacKinnon, J. G. (1993) \emph{Estimation and
    Inference in Econometrics}, New York: Oxford University Press.

  Davidson, R. and MacKinnon, J. G. (2004) \emph{Econometric Theory
    and Methods}, New York: Oxford University Press.

  Doornik, J. A. and Hansen, H.  (1994) ``An Omnibus Test for
  Univariate and Multivariate Normality'', working paper, Nuffield
  College, Oxford.

  Doornik, J. A. (1998) ``Approximations to the Asymptotic
  Distribution of Cointegration Tests'', \emph{Journal of Economic
    Surveys}, 12, pp. 573--93.  Reprinted with corrections in M.
  McAleer and L. Oxley \emph{Practical Issues in Cointegration
    Analysis}, Oxford: Blackwell, 1999.

  Fiorentini, G., Calzolari, G. and Panattoni, L. (1996) ``Analytic
  Derivatives and the Computation of GARCH Estimates'', \emph{Journal
    of Applied Econometrics}, 11, pp. 399--417.

  Goossens, M., Mittelbach, F., and Samarin, A. (1993) \emph{The
    \LaTeX\ Companion}, Boston: Addison-Wesley.

  Greene, William H. (2000) \emph{Econometric Analysis}, 4th edition,
  Upper Saddle River, NJ: Prentice-Hall.

  Gujarati, Damodar N. (2003) \emph{Basic Econometrics}, 4th edition,
  Boston, MA: McGraw-Hill.

  Hamilton, James D. (1994) \emph{Time Series Analysis}, Princeton,
  NJ: Princeton University Press.

  Hannan, E. J. and Quinn, B. G. (1979) ``The Determination of the
    Order of an Autoregression'', \emph{Journal of the Royal
    Statistical Society}, B, 41, pp. 190--195.

  Hausman, J. A. (1978) ``Specification Tests in Econometrics'',
  \emph{Econometrica}, 46, pp.\ 1251--1271.

  Hodrick, Robert and Prescott, Edward C. (1997) ``Postwar U.S. Business
  Cycles: An Empirical Investigation'', \emph{Journal of Money, Credit and
  Banking}, 29, pp. 1--16.

  Johansen, S�ren (1995) \emph{Likelihood-Based Inference in
    Cointegrated Vector Autoregressive Models}, Oxford: Oxford
  University Press.

  Kiviet, J. F. (1986) ``On the Rigour of Some Misspecification Tests
  for Modelling Dynamic Relationships'', \emph{Review of Economic
    Studies}, 53, pp. 241--261.

  Kwiatkowski, D., Phillips, P. C. B., Schmidt, P. and Shin, Y. (1992)
  ``Testing the Null of Stationarity Against the Alternative of a Unit
  Root: How Sure Are We That Economic Time Series Have a Unit Root?'',
  \emph{ Journal of Econometrics}, 54, pp. 159--178.

  Locke, C. (1976) ``A Test for the Composite Hypothesis that a
  Population has a Gamma Distribution'', \emph{Communications in
    Statistics --- Theory and Methods}, A5(4), pp. 351--364.
  
  Lucchetti, R., Papi, L., and Zazzaro, A. (2001) ``Banks'
  Inefficiency and Economic Growth: A Micro Macro Approach'',
  \emph{Scottish Journal of Political Economy}, 48, pp. 400--424.

  McCullough, B. D. and Renfro, Charles G. (1998) ``Benchmarks and
  software standards: A case study of GARCH procedures'',
  \emph{Journal of Economic and Social Measurement}, 25, pp. 59--71.

  MacKinnon, J. G. (1996) ``Numerical Distribution Functions for Unit
  Root and Cointegration Tests'', \emph{Journal of Applied
    Econometrics}, 11, pp. 601--618.
  
  MacKinnon, J. G. and White, H.  (1985) ``Some
  Heteroskedasticity-Consistent Covariance Matrix Estimators with
  Improved Finite Sample Properties'', \emph{Journal of Econometrics},
  29, pp. 305--25.
  
  Maddala, G. S. (1992) \emph{Introduction to Econometrics}, 2nd
  edition, Englewood Cliffs, NJ: Prentice-Hall.
  
  Matsumoto, M. and Nishimura, T.  (1998) ``Mersenne twister: a
  623-dimensionally equidistributed uniform pseudo-random number
  generator'', \emph{ACM Transactions on Modeling and Computer
    Simulation}, 8, pp. 3--30.

  Nerlove, M, (1999) ``Properties of Alternative Estimators of Dynamic
  Panel Models: An Empirical Analysis of Cross-Country Data for the
  Study of Economic Growth'', in Hsiao, C., Lahiri, K., Lee, L.-F. and
  Pesaran, M. H. (eds) \emph{Analysis of Panels and Limited Dependent
    Variable Models}, Cambridge: Cambridge University Press.
    
  Neter, J. Wasserman, W. and Kutner, M. H. (1990) \emph{Applied
    Linear Statistical Models}, 3rd edition, Boston, MA: Irwin.
  
  R Core Development Team (2000) \emph{An Introduction to R}, version
  1.1.1.

  Ramanathan, Ramu (2002) \emph{Introductory Econometrics with
    Applications}, 5th edition, Fort Worth: Harcourt.
  
  Schwarz, G. (1978) ``Estimating the dimension of a model'',
  \textit{Annals of Statistics}, 6, pp. 461--64.
  
  Shapiro, S. and Chen, L. (2001) ``Composite Tests for the Gamma
  Distribution'', \emph{Journal of Quality Technology}, 33, pp.
  47--59.
    
  Silverman, B. W. (1986) \emph{Density Estimation for Statistics and
    Data Analysis}, London: Chapman and Hall.
  
  Stock, James H. and Watson, Mark W. (2003) \emph{Introduction to
    Econometrics}, Boston, MA: Addison-Wesley.

  Swamy, P. A. V. B. and Arora, S. S. (1972) ``The exact finite sample
  properties of the estimators of coefficients in the error components
  regression models'', \emph{Econometrica}, 40, pp. 261--75.  
  
  Wooldridge, Jeffrey M. (2002) \emph{Introductory Econometrics, A
    Modern Approach}, 2nd edition, Mason, Ohio: South-Western.

\end{thebibliography}

%%% Local Variables: 
%%% mode: latex
%%% TeX-master: "gretl-guide"
%%% End: 



\end{document}

%%% Local Variables:
%%% mode: latex
%%% TeX-master: t
%%% End:
