\documentclass{article}
\usepackage{wasysym,ifthen}
\usepackage{gretl-lite}
\usepackage[letterpaper,body={6.3in,9.15in},top=.8in,left=1.1in]{geometry}

\newcommand{\maccmd}{%
\ifthenelse{\isundefined{wasycmd}}{\texttt{Cmd}}{\wasycmd}}

\begin{document}

\setlength{\parindent}{0pt}
\setlength{\parskip}{1ex}
\thispagestyle{empty}

\begin{center}
{\large \textbf{Gretl keyboard shortcuts}}
\end{center}

Gretl provides keyboard shortcuts for various commands. Some of these
are standard and are indicated in the menus; for example
\texttt{Ctrl-s} (on the Mac, \maccmd\texttt{-s}) in the
main window saves the current dataset, as indicated under the
\textsf{File} menu. Others are specific to gretl and do not appear in
the menus; these are documented below.

\begin{center}
\begin{tabular}{p{.2\textwidth}lp{.6\textwidth}}
\textit{Context} & \textit{Key} & \textit{Effect} \\[6pt]

Most windows & \texttt{Alt-w} & 
  popup menu listing open \textsf{gretl} windows \\
 & \texttt{Alt-PgUp} & previous gretl window (Mac: \maccmd\texttt{-\textasciitilde})  \\
 & \texttt{Alt-PgDn} & next gretl window (Mac: \maccmd\texttt{-\`{}}) \\[6pt]

Main window & \texttt{g} & 
  open dialog for defining a new variable \\
 & \texttt{h} & open command reference \\
 & \texttt{c} & open the console \\
 & \texttt{e} & edit properties of the selected series \\
 & \texttt{t} & time-series plot of selected series (if applicable) \\
 & \texttt{Alt-x} & open dialog for executing a single command \\
 & \texttt{Ctrl-v} & paste data from clipboard (CSV or similar) \\
 & \texttt{Delete} & delete the selected series \\[6pt]

Graph window & \texttt{q} & close \\
 & \texttt{s} & save to session \\
 & \texttt{z} & start zooming \\[6pt]

Model window & \texttt{q} & close \\
 & \texttt{s} & save to session \\[6pt]

Script output & \texttt{Alt-c} & cascade windows \\[6pt]

Data files window & \texttt{Enter} & open the selected file \\[6pt]

Console window & \texttt{Up} & navigate the command history, back \\
 & \texttt{Down} & navigate command history, forward \\
 & \texttt{Tab} & command completion \\

\end{tabular}
\end{center}

\end{document}
