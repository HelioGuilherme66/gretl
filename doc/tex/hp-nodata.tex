\part{Without a dataset}

\chapter{Hello, world!}

We begin with the time-honored ``Hello, world'' program, the
obligatory first step in any programming language. It's actually very
simple in hansl:
\begin{code}
  # First example
  print "Hello, world!"
\end{code}

There are several ways to run the above example: you can put it in a
text file \texttt{first\_ex.inp} and have gretl execute it from the
command line through the command
\begin{code}
  gretlcli -b first_ex.inp
\end{code}
or you could just copy its contents in the editor window of a GUI
gretl session and click on the ``gears'' icon. It's up to you; use
whatever you like best.

From a syntactical point of view, allow us to draw attention on
the following points:
\begin{enumerate}
\item The line that begins with a hash mark (\texttt{\#}) is a
  comment: if a hash mark is encountered, everything from that point
  to the end of the current line is treated as a comment, and ignored
  by the interpreter.
\item The next line contains a \emph{command} (\cmd{print}) followed
  by an \emph{argument}; this is fairly typical of hansl: most jobs
  are carried out by calling commands.
\item Hansl does not have an explicit command terminator such as the
  ``\texttt{;}'' character in the C language family (C++, Java, C\#,
  \ldots) or GAUSS; instead it uses the newline character as an
  implicit terminator. So at the end of a command, you \emph{must}
  insert a newline; conversely, you \emph{can't} put a newline in the
  middle of a command---or not without taking special measures. If you
  need to break a command over more than one line for the sake of
  legibility you can use the ``\textbackslash'' (backslash) character,
  which causes gretl to ignore the next line break.
\end{enumerate}

Note also that the \cmd{print} command automatically appends a line
break, and does not recognize ``escape'' sequences such as
``\verb|\n|''; such sequences are just printed literally. The
\cmd{printf} command can be used for greater control over output; see
chapter \ref{chap:formatting}.

Let's now examine a simple variant of the above:
\begin{code}
  /*
    Second example
  */
  string foo = "Hello, world"
  print foo
\end{code}

In this example, the comment is written using the convention adopted
in the C programming language: everything between ``\verb|/*|'' and
``\verb|*/|'' is ignored.\footnote{Each type of comment can be masked
  by the other:
\begin{itemize}
\item If \texttt{/*} follows \texttt{\#} on a given line which does
  not already start in ignore mode, then there's nothing special about
  \texttt{/*}, it's just part of a \texttt{\#}-style comment.
\item If \texttt{\#} occurs when we're already in comment mode, it is
  just part of a comment.
\end{itemize}} Comments of this type cannot be nested.

Then we have the line
\begin{code}
  string foo = "Hello, world"
\end{code}
In this line, we assign the value ``\texttt{Hello, world}'' to the
variable named \texttt{foo}. Note that
\begin{enumerate}
\item The assignment operator is the equals sign (\texttt{=}).
\item The name of the variable (its \emph{identifier}) must follow
  the following convention: identifiers can be at most 31 characters
  long. They must start with a letter, and can contain only letters,
  numbers and the underscore character. Identifiers in hansl are
  case-sensitive, so \texttt{foo}, \texttt{Foo} and \texttt{FOO} are
  three distinct names. Of course, some words are reserved and can't
  be used as identifiers (however, nearly all reserved words only
  contain lowercase characters).
\end{enumerate}

In hansl, a variable has to be of one of these types: \texttt{scalar},
\texttt{series}, \texttt{matrix}, \texttt{list}, \texttt{string} or
\texttt{bundle}. As we've already seen, string variables are used to
hold sequences of alphanumeric characters. We'll introduce the other
ones gradually; for example, the \texttt{matrix} type will be the
object of the next chapter.  

The reader may have noticed that the line 
\begin{code}
  string foo = "Hello, world"
\end{code}
implicitly performs two tasks: it \emph{declares} \texttt{foo} as a
variable of type \texttt{string} and, at the same time, \emph{assigns}
a value to \texttt{foo}. This is, in fact, not strictly required.  In
most cases gretl is able to figure out by itself what type the
variable should have, so the line \verb|foo = "Hello, world"| (without
a type specifier) would have worked just fine.  However, it is more
elegant (and leads to more legible and maintainable code) to use a
type specifier at least the first time you introduce a variable.
  
In the next example, we will use a variable of the \texttt{scalar}
type:
\begin{code}
  scalar x = 42
  print x
\end{code}
A \texttt{scalar} is a double-precision floating point number, so
\texttt{42} is the same as \texttt{42.0} or \texttt{4.20000E+01}. Note
that hansl doesn't have a separate variable type for integers or
complex numbers.

An important detail to note is that, contrary to most other
matrix-oriented languages in use in the econometrics community, hansl
is \emph{strongly typed}. That is, you cannot assign a value of one
type to a variable that has already been declared as another type. For
example, this will return an error:
\begin{code}
  string a = "zoo"
  a = 3.14 # no, no, no!
\end{code}
If you try running the example above, an error will be
flagged. However, it is acceptable to destroy the original variable,
via the \cmd{delete} command, and then re-declare it, as in
\begin{code}
  scalar X = 3.1415
  delete X
  string X = "apple pie"
\end{code}

There is no ``type-casting'' as in C, but some automatic type
conversions are possible (more on this later).

Many commands can take more than one argument
\begin{code}
  set echo off
  set messages off

  scalar x = 42
  string foo = "not bad"
  print x foo 
\end{code}
In this example, one \texttt{print} is used to print the values of two
variables; more generally, \texttt{print} can be followed by as many
arguments as desired. The other difference with respect to the
previous code examples is in the use of the two \texttt{set}
commands. Describing the \texttt{set} command in detail would lead us
to an overly long diversion; suffice it to say that here it is used as
a way to silence unwanted output. See chapter \ref{chap:settings} or
the \GCR{}.

The \cmd{eval} command is useful when you want to look at the result
of an expression without assigning it to a variable; for example
\begin{code}
  eval 2+3*4
\end{code}
will print the number 14. This is most useful when running gretl
interactively, like a calculator, but it is usable in a hansl script
for checking purposes, as in the following (rather silly) example:
\begin{code}
  scalar a = 1
  scalar b = -1
  # this ought to be 0
  eval a+b
\end{code}

\section{Manipulation of scalars}

Algebraic operations work in the obvious way, with the classic
algebraic operators having their traditional precedence rules: the
caret (\verb|^|) is used for exponentiation. For example,
\begin{code}
  scalar phi = exp(-0.5 * (x-m)^2 / s2) / sqrt(2 * $pi * s2)
\end{code}
%$
In which we assume that \texttt{x}, \texttt{m} and \texttt{s2} are
pre-existing scalars. The example above contains two noteworthy
points:
\begin{itemize}
\item The usage of the \cmd{exp} and \cmd{sqrt} function; it goes
  without saying that hansl possesses a reasonably wide repertoire of
  functions. See the \GCR{} for the complete list.
\item The usage of \verb|$pi| for the constant $\pi$. In hansl, some
  identifiers have a ``dollar'' prefix; they work as read-only
  variables. Most of them are used in the context of an open dataset
  (see part \ref{part:hp-data}), but some are just pre-defined
  constants, such as $\pi$. Again, see the \GCR{} for a comprehensive
  list.
\end{itemize}

Hansl does not possess a boolean type, so scalars are commonly used
for holding true/false values. For this reason, you can also use
the logical operators \emph{and} (\verb|&&|), \emph{or} (\verb+||+),
and \emph{not} (\verb|!|) with scalars, as in the following example:
\begin{code}
  a = 1
  b = 0
  c = !(a && b)
\end{code}
Note that 0 evaluates to false, and anything else evaluates to
true.

A few constructs are taken from the C language family: one is the
postfix increment operator:
\begin{code}
  a = 5
  b = a++
  print a b
\end{code}
the second line is equivalent to \texttt{b = a}, followed by
\texttt{a++}, which in turn is shorthand for \texttt{a = a+1}, so
running the code above will result in \texttt{b} containing 5 and
\texttt{a} containing 6. Postfix subtraction is also supported; prefix
operators, however, are not. Another C borrowing is the inflected
assignment, as in \texttt{a += b}, which is equivalent to \texttt{a =
  a + b}; several other similar operators are available, such as
\texttt{-=}, \texttt{*=} and more. See the \GCR{} for details.

The internal representations for a missing value varies from sytem to
system, but is an inordinately large number (on most Unix flavors,
it's somewhere around 1.7E+308). This is what you get if you try to
compute quantities like the square root or the logarithm of a negative
number, or directly via the \texttt{NA} constant, as in
\begin{code}
  scalar not_really = NA
  scalar is_there = ok(not_really)
\end{code}
As shown in the example above, in hansl you have the \cmd{ok} function
which tells you whether a scalar is \texttt{NA} or not.

\section{Manipulation of strings}

A string may take any of the following forms:
\begin{itemize}
\item the keyord \texttt{null}, or \texttt{""}, for an empty string
\item a string literal (enclosed in double quotes); or
\item the name of an existing string variable; or
\item a function that returns a string (see below); or
\item any of the above followed by \texttt{+} and an integer offset.
\end{itemize}

The role of the integer offset is to use a substring of the preceding
element, starting at the given character offset.  An empty string is
returned if the offset is greater than the length of the string in
question.

You can use the \verb|~| operator to join two or more strings, as
in\footnote{Note on some national keyboards, you don't have the tilde
  (\texttt{\~}) character. In gretl's script editor, this can be
  obtained via its Unicode representation: type Ctrl-Shift-U, followed
  by \texttt{7e}.}
\begin{code}
  a = "now"
  b = "here"
  c = a ~ c
  print c
\end{code}
or
\begin{code}
string s1 = "sweet"
string s2 = "Home, " ~ s1 ~ " home."
\end{code}

To add to the end of an existing string you can use the operator
\verb|~=|, as in
\begin{code}
  a = "abcde"
  eval a+3
  a ~= "fgh"
\end{code}

Functions and stuff borrowed from C

A pretty nifty thing is \emph{string substitution}, but later (see
\ref{sec:stringsub}).

\chapter{Matrices}

Matrices are one- or two-dimensional arrays of double-precision
floating-point numbers. Hansl users who are accustomed to other matrix
languages should note that multi-index objects are not
supported. Matrices have rows and columns, that's it.

\section{Matrix indexing}
\label{sec:mat-index}

Individual matrix elements are accessed through the \verb|[r,c]|
syntax (indexing starts at 1). For example, \texttt{X[3,4]} indicates
the element of $X$ on the third row, fourth column. For example,
\begin{code}
  matrix X = zeros(2,3)
  X[2,1] = 4
  print X
\end{code}
produces
\begin{code}
X (2 x 3)

  0   0   0 
  4   0   0 
\end{code}

Here are some more advanced ways to access matrix elements:
\begin{enumerate}
\item In case the matrix has only one row (column), the column (row)
  specification can be omitted, as in \texttt{x[3]}.
\item Omitting the row or column specification means ``take them
  all'', as in \texttt{x[4,]} (fourth matrix row).
\item For square matrices, the special syntax \texttt{x[diag]} can be
  used, which will return the diagonal.
\item Consecutive rows (columns) can be specified via the colon
  (\texttt{:}) character, as in \texttt{x[,2:4]} (columns 2 to 4).
  (Note that, unlike some other matrix languages, the syntax
  \texttt{[m:n]} is illegal if $m>n$.)
\item It is possible to use a vector to hold indices to a matrix. E.g.\
  if $e = [2,3,6]$, then \texttt{X[,e]} contains the second, third and
  sixth column of $X$.
\end{enumerate}
Moreover, matrices can be empty (zero rows/columns). 

In the example above, the matrix \texttt{X} was constructed using
the function \texttt{zeros()}, whose meaning should be obvious, but
matrix elements can also be specified directly, as in
\begin{code}
scalar a = 2*3
matrix A = { 1, 2, 3 ; 4, 5, a }
\end{code}
The matrix is defined by rows; the elements on each row are separated
by commas and rows are separated by semicolons.  The whole expression
must be wrapped in braces.  Spaces within the braces are not
significant. The above expression defines a $2\times3$ matrix.

Note that each element should be a numerical value, the name of a
scalar variable, or an expression that evaluates to a scalar. In the
example above the scalar \texttt{a} was first assigned a value and
then used in matrix construction. (Also note, in passing, that
\texttt{a} and \texttt{A} are two separate identifiers, due to
case-sensitivity.)

\section{Matrix operations}
\label{sec:mat-op}

Matrix sum, difference and product are obtained via \texttt{+},
\texttt{-} and \texttt{*}, respectively. The prime operator
(\texttt{'}) can act as a unary operator, in which case it transposes
the preceeding matrix, or as a binary operator, in which case it acts
as in ordinary matrix algebra, that is transposes the first matrix and
multiplies it into the second done. Errors are flagged if
conformability is a problem. For example:
\begin{code}
  matrix a = {11, 22 ; 33, 44}  # a is square 2 x 2
  matrix b = {1,2,3; 3,2,1}     # b is 2 x 3

  matrix c = a'         # c is the trasnpose of a
  matrix d = b*a        # d is a 2x3 equal to b times a

  matrix gina = b'd     # valid: gina is 3x3
  matrix lina = d + b   # valid: lina is 2x3

  /* -- these would generate errors if uncommented ----- */

  # pina = c + a  # sum non-conformability
  # rina = d * b  # product non-conformability
\end{code}

Other noteworthy matrix operators include \texttt{\^} (matrix power),
\texttt{**} (Kronecker product), and the ``concatenation'' operators
\verb|~| (horizontal) and \texttt{|} (vertical). Readers are invited
to try them out by running the following code
\begin{code}
matrix A = {2,1;0,1}
matrix B = {1,1;1,0}

matrix KP = A ** B
matrix PWR = A^3 
matrix HC = A ~ B
matrix VC = A | B

print A B KP PWR HC VC
\end{code}
Note, in particular, that $A^3 = A \cdot A \cdot A$, which is different
from what you get by computing the cubes of each element of $A$
separately.

Element-by-element operations are supported by the so-called ``dot''
operators, which are obtained by putting a dot (``\texttt{.}'') before
the corresponding operator. For example, the code
\begin{code}
A = {1,2; 3,4}
B = {-1,0; 1,-1}
eval A * B
eval A .* B
\end{code}
produces
\begin{code}
   1   -2 
   1   -4 

  -1    0 
   3   -4 
\end{code}

It's easy to verify that the first operation performed is regular
matrix multiplication $A \cdot B$, whereas the second one is the
Hadamard (element-by-element) product $A \odot B$. In fact, dot
operators are more general and powerful than shown in the example
above; see the chapter on matrices in \GUG{} for details.

Dot and concatenation operators are less rigid than ordinary matrix
operations in terms of conformability requirements: (\ldots)


Hansl provides a reasonably comprehensive set of matrix functions,
that is, functions that produce and/or operate on matrices. For a
full list, see the \GCR, but a basic ``survival kit'' is provided
in Table~\ref{tab:essential-matfuncs}.  Moreover, most scalar
functions, such as \texttt{abs(), log()} etc., will operate on a
matrix element-by-element.

\begin{table}[htbp]
  \centering
  \begin{tabular}{rp{0.5\textwidth}}
    \textbf{Function(s)} & \textbf{Purpose} \\
    \hline
    \texttt{zeros(r,c), ones(r,c)} & produce matrices with $r$ rows
    and $c$ columns, filled with zeros and ones, respectively \\
    \texttt{I(n)} & identity matrix of size $n$ \\
    \texttt{rows(X), cols(X)} & return the number of rows and columns
    of $X$, respectively \\
    \texttt{inv(A)} & invert, if possible, the matrix $A$ \\
    \texttt{maxc(A), minc(A), sumc(A), meanc(A)} & return a row vector
    with the max, min, sum, means of each column of $A$, respectively\\
    \texttt{maxr(A), minr(A), sumr(A), meanr(A)} & return a column vector
    with the max, min, sum, means of each row of $A$, respectively\\
    \texttt{mnormal(r,c), muniform(r,c)} & generate $r \times c$
    matrices filled with standard Gaussian and uniform pseudo-random
    numbers, respectively \\
    \hline
  \end{tabular}
  \caption{Essential set of hansl matrix functions}
  \label{tab:essential-matfuncs}
\end{table}


\begin{code}
# example: OLS using matrices

# fix the sample size
scalar T = 256

# construct vector of coefficients by direct imputation
matrix beta = {1.5, 2.5, -0.5} # note: row vector

# construct the matrix of independent variables
matrix Z = mnormal(T, cols(beta)) # built-in functions

# now construct the dependent variable: note the
# usage of the "dot" and transpose operators

matrix y = {1.2} .+ Z*beta' + mnormal(T, 1)

# now do estimation
matrix X = 1 ~ Z  # concatenation operator
matrix beta_hat1 = inv(X'X) * (X'y) # by hand
matrix beta_hat2 = mols(y, X)       # via the built-in function

print beta_hat1 beta_hat2
\end{code}

\section{Matrix pointers}
\label{sec:mat-pointers}

Hansl uses the ``by value'' convention for passing parameters to
functions. That is, when a variable is passed to a function as an
argument, what the function actually ``gets'' is a \emph{copy} of the
variable, which means that the value of the variable at the caller
level is not modified by anything that goes on inside the function.
But the use of pointers allows a function and its caller to cooperate
such that an outer variable can be modified by the function.

This mechanism is used by some built-in matrix functions to provide
more than one ``return'' value. The primary result is always provided
by the return value proper but certain auxiliary values may be
retrieved via ``pointerized'' arguments; this usage is flagged by
prepending the ampersand symbol, \texttt{\&}, to the name of the
argument variable.

The \texttt{eigensym} function, which performs the eigen-analysis of
symmetric matrices, is a case in point. In the example below the first
argument, \texttt{A}, represents the input data, that is, the matrix
whose analysis is required. This variable will not be modified in any
way by the function call. The primary result is the vector of
eigenvalues of $A$, which is here assigned to the variable
\texttt{ev}. The (optional) second argument, \texttt{\&V} (which may
be read as ``the address of \texttt{V}''), is used to retrieve the
right eigenvectors of $A$. A variable named in this way must be
already declared, but it need not be of the right dimensions to
receive the result; it will be resized as needed.
\begin{code}
matrix A = {1,2 ; 2,5}
matrix V
matrix ev = eigensym(A, &V)
print A ev V
\end{code}

\chapter{Bundles}

Bundles are \emph{associative arrays}, that is, generic containers for
any assortment of hansl types (including other bundles) in which each
element is identified by a string. Python users call these
\emph{dictionaries}; in C++ and Java, they are referred to as
\emph{maps}; they are known as \emph{hashes} in Perl. We call them
\emph{bundles}. Each item placed in the bundle is associated with a
key which can used to retrieve it subsequently.

To use a bundle you first either ``declare'' it, as in
%
\begin{code}
bundle foo
\end{code}
%
or define an empty bundle using the \texttt{null} keyword:
%
\begin{code}
bundle foo = null
\end{code}
%
These two formulations are basically equivalent, in that they both
create an empty bundle. The difference is that the second variant
may be reused --- if a bundle named \texttt{foo} already exists the
effect is to empty it --- while the first may only be used once in
a given \app{gretl} session; it is an error to declare a variable that
already exists. 

To add an object to a bundle you assign to a compound left-hand value:
the name of the bundle followed by the key. Two forms of syntax are
acceptable in this context. The original syntax (and the only form
supported prior to version 1.9.12 of \app{gretl}) requires that the
key be given as a quoted string literal enclosed in square brackets.
For example, the statement

\begin{code}
foo["matrix1"] = m
\end{code}

adds an object called \texttt{m} (presumably a matrix) to bundle
\texttt{foo} under the key \texttt{matrix1}. From \app{gretl} 1.9.12,
however, a simpler syntax is available, \emph{provided} the key
satisfies the rules for a \app{gretl} variable name (31 characters
maximum, starting with a letter and composed of just letters, numbers
or underscore).\footnote{When using the original syntax, keys do not
  have to be valid as variable names---for example, they can include
  spaces---but they are limited to 31 characters.}  In that case you
can simply join the key to the bundle name with a dot, as in

\begin{code}
foo.matrix1 = m
\end{code}

To get an item out of a bundle, again use the name of the bundle
followed by the key, as in

\begin{code}
matrix bm = foo["matrix1"]
# or using the dot notation
matrix m = foo.matrix1
\end{code}

Note that the key identifying an object within a given bundle is
necessarily unique. If you reuse an existing key in a new assignment,
the effect is to replace the object which was previously stored under
the given key. It is not required that the type of the replacement
object is the same as that of the original.

Also note that when you add an object to a bundle, what in fact
happens is that the bundle acquires a copy of the object. The external
object retains its own identity and is unaffected if the bundled
object is replaced by another. Consider the following script fragment:

\begin{code}
bundle foo
matrix m = I(3)
foo.mykey = m
scalar x = 20
foo.mykey = x
\end{code}

After the above commands are completed bundle \texttt{foo} does not
contain a matrix under \texttt{mykey}, but the original matrix
\texttt{m} is still in good health.

To delete an object from a bundle use the \texttt{delete} command,
with the bundle/key combination, as in

\begin{code}
delete foo.mykey
delete foo["quoted key"]
\end{code}

This destroys the object associated with the key and removes the key
from the hash table.

Besides adding, accessing, replacing and deleting individual items,
the other operations that are supported for bundles are union,
printing and deletion. As regards union, if bundles \texttt{b1} and
\texttt{b2} are defined you can say

\begin{code}
bundle b3 = b1 + b2
\end{code}

to create a new bundle that is the union of the two others. The
algorithm is: create a new bundle that is a copy of \texttt{b1}, then
add any items from \texttt{b2} whose keys are not already present in
the new bundle. (This means that bundle union is not commutative if
the bundles have one or more key strings in common.)

If \texttt{b} is a bundle and you say \texttt{print b}, you get a
listing of the bundle's keys along with the types of the corresponding
objects, as in

\begin{code}
? print b
bundle b:
 x (scalar)
 mat (matrix)
 inside (bundle)
\end{code}

\section{Bundle usage}
\label{sec:bundle-usage}

To illustrate the way a bundle can hold information, we will use the
Ordinary Least Squares (OLS) model as an example: the following code
estimates an OLS regression and stores all the results in a bundle.

\begin{code}
/* assume y and X are given T x 1 and T x k matrices */

bundle my_model = null               # initialization
my_model["T"] = rows(X)              # sample size
my_model["k"] = cols(X)              # number of regressors
matrix e                             # will hold the residuals
b = mols(y, X, &e)                   # perform OLS via native function
s2 = meanc(e.^2)                     # compute variance estimator
matrix V = s2 .* invpd(X'X)          # compute covariance matrix

/* now store estimated quantities into the bundle */

my_model["betahat"] = b
my_model["s2"] = s2
my_model["vcv"] = V
my_model["stderr"] = sqrt(diag(V))
\end{code}

The bundle so obtained is a container that can be used for all sort of
purposes. For example, the next code snippet illustrates how to use
a bundle with the same structure as the one created above to perform
an out-of sample forecast. Imagine that $k=4$ and the value of
$\mathbf{x}$ for which we vant to forecast $y$ is
\[
  \mathbf{x}' = [ 10 \quad 1  \quad -3 \quad 0.5 ]
\]
The formulae for the forecast would then be
\begin{eqnarray*}
  \hat{y}_f & = & \mathbf{x}'\hat{\beta} \\
  s_f & = & \sqrt{\hat{\sigma}^2 + \mathbf{x}'V(\hat{\beta})\mathbf{x}} \\
  CI & = & \hat{y}_f \pm 1.96 s_f 
\end{eqnarray*}
where $CI$ is the (approximate) 95 percent confidence interval. The
above formulae translate into
\begin{code}
  x = { 10, 1, -3, 0.5 }
  scalar ypred    = x * my_model["betahat"]
  scalar varpred  = my_model["s2"] + qform(x, my_model["vcv"])
  scalar sepred   = sqrt(varpred)
  matrix CI_95    = ypred + {-1, 1} .* (1.96*sepred)
  print ypred CI_95
\end{code}

\chapter{Control flow}
\label{chap:hp-ctrlflow}

The primary means for controlling the flow of execution in a hansl
script are the the \cmd{loop} statement (repeated execution), \cmd{if}
statement (conditional execution), the \cmd{catch} modifier (which
enables the trapping of errors that would otherwise halt execution)
and the \cmd{quit} command (which forces termination).

\section{Loops}
\label{sec:hr-loops}

The basic hansl command for looping is (surprise) \cmd{loop}, and
takes the form
\begin{code}
loop <control-expression> <options>
    ...
endloop
\end{code}
In other words, the pair of statements \cmd{loop} and \cmd{endloop}
enclose the statements to repeat. Of course, loops can be nested.
Several variants of the \texttt{<control-expression>} for a loop are
supported, as follows:
\begin{enumerate}
\item unconditional loop
\item while loop
\item index loop
\item foreach loop
\item for loop.
\end{enumerate}
These variants are briefly described below.

\subsection{Unconditional loop}

This is the simplest variant. It takes the form
\begin{code}
loop <times>
   ...
endloop
\end{code}
where \texttt{<times>} is any expression that evaluates to a scalar,
namely the required number of iterations. This is only evaluated at
the beginning of the loop, so the number of iterations cannot be
changed from within the loop itself.

\subsection{Index loop}

The syntax is
\begin{code}
loop <counter>=<min>..<max>
   ...
endloop
\end{code}
The limits \texttt{<min>} and \texttt{<max>} must evaluate to scalars;
they are automatically turned into integers if they have a fractional
part. The \texttt{<counter>} variable is started at \texttt{<min>} and
incremented by 1 on each iteration until it equals \texttt{<max>}.

The counter is ``read-only'' inside the loop. You can access either
its numerical value or (by prepending a dollar sign) its string
representation. For example:
\begin{code}
scalar Max = 4
loop i=1..Max --quiet
    printf "$i squared = %d\n", i*i
endloop
\end{code}
%$

\subsection{While loop}

Here you have
\begin{code}
loop while <condition>
   ...
endloop
\end{code}
where \texttt{<condition>} should evaluate to a scalar, which is
re-evaluated at each iteration. Looping stops as soon as
\texttt{<condition>} becomes false (0). If \texttt{<condition>}
becomes \texttt{NA}, an error is flagged and execution stops.  By
default, \texttt{while} loops cannot exceed 1000 iterations. This is
intended as a safeguard against potentially infinite loops. This
setting can be overridden if necessary by setting the
\texttt{loop\_maxiter} state variable to a different value (see
chapter \ref{chap:settings}).

\subsection{Foreach loop}
\label{sec:loop-foreach}

In this case the syntax is
\begin{code}
loop foreach <counter> <catalogue>
   ...
endloop
\end{code}
where \texttt{<catalogue>} can be either a variable of type
\texttt{list} (i.e.\ a list of series), or a collection of
space-separated strings. The counter variable automatically takes on
the numerical values 1, 2, 3, \dots{} as execution proceeds, but its
string value (accessed by prepending a dollar sign) shadows the names
of the series in the list or the space-separated strings; this sort of
loop is designed for string substitution.

Here is an example in which the \texttt{<catalogue>} is a collection
of function names: 
\begin{code}
loop foreach f sqrt exp ln
    printf "$f(1) = %g\n", $f(1)
endloop
\end{code}

\subsection{For loop}

The final form of loop control emulates the \cmd{for} statement in the
C programming language.  The syntax is \texttt{loop for}, followed by
three component expressions, separated by semicolons and surrounded by
parentheses, that is
\begin{code}
loop for ( <init> ; <cont> ; <modifier> )
   ...
endloop
\end{code}

The three components are as follows:
\begin{enumerate}
\item Initialization (\texttt{<init>}): This is evaluated once at the
  start of the loop.
\item Continuation condition (\texttt{<cont>}): this is evaluated at
  the top of each iteration (including the first).  If the expression
  evaluates as true (non-zero), iteration continues, otherwise it
  stops. 
\item Modifier (\texttt{<modifier>}): an expression which modifies the
  value of some variable.  This is evaluated prior to checking the
  continuation condition, on each iteration after the first.
\end{enumerate}

Here's a simple example:
%
\begin{code}
loop for (r=0.01; r<.991; r+=.01)
   ...
endloop
\end{code}

Be aware of the limited precision of floating-point arithmetic. For
example, the code snippet below will iterate forever on most platforms
because \texttt{x} will never equal \textit{exactly} 0.01, even though
it might seem that it should.
\begin{code}
loop for (x=1; x!=0.01; x=x*0.1)
    printf "x = .18g\n", x
endloop  
\end{code}
However, if you replace the condition \texttt{x!=0.01} with
\texttt{x>=0.01}, the code will run as (probably) intended.
 
\vspace{1ex}

\subsection{Loop options}

Two options can be given to the \cmd{loop} statement. One is
\option{quiet}. This has simply the effect of suppressing certain
lines of output that gretl prints by default to signal progress and
completion of the loop; it has no other effect and the semantics of
the loop contents remain unchanged. This option is commonly used in
complex scripts.

The \option{progressive} option is mostly used as a quick and
efficient way to set up simulation studies. When this option is given,
a few commands (notably \cmd{print} and \cmd{store}) are given a
special, \emph{ad hoc} meaning. Please refer to the \GUG\ for more
information.
 
\subsection{Breaking out of loops}
\label{sec:loop-break}

The \cmd{break} command makes it possible to break out of a loop if
necessary. Note that
\begin{itemize}
\item if you nest loops, \cmd{break} in the innermost loop will
  interrupt that loop only and not the outer ones; and
\item hansl does not provide any equivalent to the C \texttt{continue}
  statement.
\end{itemize}

\section{The \cmd{if} statement}

Conditional execution in hansl uses the \cmd{if} keyword. Its fullest
usage is as follows
\begin{code}
if <condition>
   ...
elif <condition>
   ...
else 
   ...
endif  
\end{code}

Points to note:
\begin{itemize}
\item The \texttt{<condition>} can be any expression that evaluates to a
  scalar: 0 is interpreted as ``false'', non-zero is interpreted as
  ``true''; \texttt{NA} generates an error.
\item The \cmd{elif} and \cmd{else} clauses are optional: the minimal
  form is just \texttt{if} \dots{} \texttt{endif}.
\item The \cmd{elif} keyword can be repeated, making hansl's \cmd{if}
  statement a multi-way branch statement. There is no separate
  \cmd{switch} or \cmd{case} statement in hansl.
\item Conditional blocks of this sort can be nested up to a maximum of
  1024.
\end{itemize}

Example:
\begin{code}
scalar x = 15

# --- simple if ----------------------------------
if x >= 10
   printf "%g is more than two digits long\n", x
endif

# --- if with else -------------------------------
if x >= 0
   printf "%g is non-negative\n", x
else
   printf "%g is negative\n", x
endif

# --- multiple branches --------------------------
if missing(x)
   printf "%g is missing\n", x
elif x < 0
   printf "%g is negative\n", x
elif floor(x) == x
   printf "%g is an integer\n", x
else
   printf "%g is positive integer\n", x
endif
\end{code}

Note to Stata users: hansl's \cmd{if} statement is fundamentally
different from Stata's \texttt{if} option: the latter selects a
subsample of observations for some action, while the former is used to
decide if a group of statements must be executed or not; Stata calls
this the \emph{branching \texttt{if}}.

\subsection{The ternary query operator}

Besides use of \texttt{if}, the ternary query operator, \texttt{?:},
can be used to perform conditional assignment on a more ``micro''
level. This has the form
\begin{code}
result = <condition> ? <t-value1> : <f-value>
\end{code}
If \texttt{<condition>} evaluates as true (non-zero) then
\texttt{<t-value>} is assigned to \texttt{result}, otherwise
\texttt{<f-value>} is assigned. This is obviously more compact than
\texttt{if} \dots{} \texttt{else} \dots{} \texttt{endif}. The
following example replicates the \cmd{abs} function by hand:
\begin{code}
scalar ax = x>=0 ? x : -x
\end{code}
Of course, in the above case it would have been much simpler to just
write \texttt{ax = abs(x)}. Consider, however, the following case,
which exploits the fact that the ternary operator can be nested:
\begin{code}
scalar days = (m==2) ? 28 : maxr(m.={4,6,9,11}) ? 30 : 31
\end{code}
This example deserves a few comments: we want to compute the number of
days in a month, codified in the variable \texttt{m}. So, what we put into
the scalar \texttt{days} comes from the following pathway:
\begin{enumerate}
\item first, we check if the month is February (\texttt{m==2}); if so,
  we set \texttt{days} to 28 and we're done;\footnote{Ok, we don't
    check for leap years here.}
\item otherwise, we compute a matrix of zeros and ones via the
  \verb|m.={4,6,9,11}| operation (note the ``dot'' operator); if
  \texttt{m} equals any of the elements in the vector, the
  corresponding element of the result will be one, and zero otherwise;
\item by the function \cmd{maxr}, we take its maximum, so we're
  checking whether \texttt{m} is one of the four values corresponding
  to 30-days months;
\item since the above evaluates to a scalar, we put the right value
  into \texttt{days}.
\end{enumerate}

It is also
more flexible, in this respect: in contrast to the usage of
\texttt{if}, \texttt{<condition>} need not evaluate to a scalar.  So
for example one can do this sort of thing with matrices:
\begin{code}
matrix A = mnormal(4,4)
matrix B = upper(A.<0) ? -A : A
\end{code}
which will switch the sign of the negative elements of \texttt{A} on
and above its diagonal. To achieve the same via a loop, one should
have had to write a considerably longer chunk of code, such as
\begin{code}
matrix A = mnormal(4,4)
matrix B = A

loop r = 1..rows(A)
  loop c = r .. cols(A)
     if A[r,c] < 0
       B[r,c] = -A[r,c]
     endif
  end loop
end loop
\end{code}

\section{The \cmd{catch} modifier}

Hansl offers a rudimentary form of exception handling via the
\cmd{catch} modifier.

This is not a command in its own right but can be used as a prefix to
most regular commands: the effect is to prevent termination of a
script if an error occurs in executing the command. If an error does
occur, this is registered in an internal error code which can be
accessed as \dollar{error} (a zero value indicates success). The value
of \dollar{error} should always be checked immediately after using
\texttt{catch}, and appropriate action taken if the command failed.

The catch keyword cannot be used before \cmd{if}, \cmd{elif} or
\cmd{endif}.

\emph{\ldots Add example? \ldots}

\section{The \cmd{quit} statement}

When the \cmd{quit} statement is encountered in a hansl script,
execution stops. If the command-line program \app{gretlcli} is running
in batch mode, control returns to the operating system; if gretl is
running in interactive mode, \app{gretl} will wait for interactive
input. The \texttt{quit} command is rarely used in scripts since
execution automatically stops when script input is exhausted, but it
could be used in conjunction with \cmd{catch}. A script author could
arrange matters so that on encountering a certain error condition an
appropriate message is printed and the script is halted.

%%% Local Variables: 
%%% mode: latex
%%% TeX-master: "hansl-primer"
%%% End: 


\chapter{User-written functions}

Hansl natively provides a reasonably wide array of pre-defined
functions for manipulating variables of all kinds; the previous
chapters contain several examples. However, it is also possible to
extend hansl's native capabilities by defining additional
functions.

Here's how a user-defined function looks like:
\begin{flushleft}
\texttt{function \emph{type} \emph{funcname}(\emph{parameters})}\\
   \quad \ldots\\
   \quad \texttt{\emph{function body}}\\
   \quad \ldots \\
\texttt{end function}
\end{flushleft}

The opening line of a function definition contains these elements, in
strict order:

\begin{enumerate}
\item The keyword \texttt{function}.
\item \texttt{\emph{type}}, which states the type of value returned by the
  function, if any.  This must be one of \texttt{void} (if the
  function does not return anything), \texttt{scalar},
  \texttt{series}, \texttt{matrix}, \texttt{list} or \texttt{string}.
\item \texttt{\emph{funcname}}, the unique identifier for the
  function.  Function names have a maximum length of 31 characters;
  they must start with a letter and can contain only letters, numerals
  and the underscore character. It cannot coincide with a native gretl
  command or function.
\item The functions's \texttt{\emph{parameters}}, in the form of a
  comma-separated list enclosed in parentheses.\footnote{If you have
    to pass many parameters, you might want to consider wrapping them
    into a bundle to avoid syntax cluttering.} Note: parameters are
  the only way hansl function can receive anything from ``the
  outside''. In hansl, there is no such thing as global variables.
\end{enumerate}

Function parameters can be of any of the types shown below.

\begin{center}
\begin{tabular}{ll}
  \multicolumn{1}{c}{Type} & 
  \multicolumn{1}{c}{Description} \\ [4pt]
  \texttt{bool}   & scalar variable acting as a Boolean switch \\
  \texttt{int}    & scalar variable acting as an integer  \\
  \texttt{scalar} & scalar variable \\
  \texttt{series} & data series (see section~\ref{sec:series})\\
  \texttt{list}   & named list of series  (see section~\ref{sec:lists})\\
  \texttt{matrix} & matrix or vector \\
  \texttt{string} & string variable or string literal \\
  \texttt{bundle} & all-purpose container
\end{tabular}
\end{center}

Each element in the listing of parameters must include two terms: a
type specifier, and the name by which the parameter shall be known
within the function.  

The \emph{function body} contains (almost) arbitrary hansl code, which
should compute the \emph{return value}, that is the value the function
is supposed to yield. Any variable declared inside the function is
\emph{local}, so it will cease to exist when the function ends.

The \cmd{return} command is used to stop
execution of the code inside the function and deliver its result to
the calling code (note that this typically happens at the end of the
function body, but doesn't have to).  The function definition must end
with the expression \verb|end function|, on a line of its own.

In order to get a feel for how functions work in practice, here's a
simple example:
\begin{code}
function scalar quasi_log(scalar x)
/* popular approximation to the natural logarithm
   via Pad� polynomials */

   if (x<0)
      scalar ret = NA
   else 
      scalar ret = 2*(x-1)/(x+1)
   endif

   return ret
end function

loop for (x=0.5; x<2; x+=0.1)
   printf "x = %4.2f; ln(x) = %g, approx = %g\n", x, ln(x), quasi_log(x)
end loop
\end{code}

The code above computes the rational function
\[
  f(x) = 2 \cdot \frac{x-1}{x+1} ,
\]
which provides a decent approximation to the natural logarithm in a
neighborhood of 1.

\begin{enumerate}
\item We begin by defining the function via the \cmd{function}
  keyword; the function definition will end with the \texttt{end
    function} marker below;
\item since the function is meant to return a scalar, we put the
  keyword \texttt{scalar} after \cmd{function};
\item between the round brackets, the arguments follow; in this case,
  we only have one, which we call \texttt{x} and is assumed to be a
  scalar;
\item on the next line, the function definition begins; in this case,
  it includes a comment and an \cmd{if} block;
\item the function ends with the \cmd{return} keyword, which exports
  the result;\footnote{Beware: unlike other languages (eg Matlab), you cannot
  return multiple outputs from a function. However, you can return a
  bundle and stuff it with as many objects as you want.}
\item the next lines provide a simple usage example; note that in the
  \cmd{printf} command, the two functions \cmd{ln()} and
  \cmd{quasi\_log()} are indistinguishable from a purely syntactic
  viewpoint, although the former is a native function and the
  latter is a user-defined one. 
\end{enumerate}

In many cases, you may end up writing several functions, which may be
quite long; in order to avoid cluttering your script with the function
definition, hansl provides the \cmd{include} command, so you can put all
your function definition in a separate file (or separate files if
necessary). For example, imagine you saved the \verb|quasi_log()|
function definition above in a separate file called
\verb|quasilog_def.inp|: the code above could be written more
compactly as
\begin{code}
include quasilog_def.inp

loop for (x=0.5; x<2; x+=0.1)
   printf "x = %4.2f; ln(x) = %g, approx = %g\n", x, ln(x), quasi_log(x)
end loop
\end{code}
Moreover, \cmd{include} commands can be nested.


\section{Parameter passing and return values}
\label{sec:params-returns}

In hansl, parameters are passed \emph{by value}, so what is used
inside the function is a copy of the original argument. You may modify
it, but you'll be just modifying the copy. The following example
should make this point clear:
\begin{code}
function void f(scalar x)
    x = x*2
    print x
end function

scalar x = 3
f(x)
print x
\end{code}
Running the above code yields
\begin{code}
              x =  6.0000000
              x =  3.0000000
\end{code}
The first \cmd{print} statement happens inside the function, and the
displayed value is 6 because \verb|x| is doubled; however, what really
gets doubled is simply the \emph{copy} of \texttt{x}: this is
demonstrated by the second \cmd{print} statement.  If you need a
function to modify its arguments, you ought to use pointers (see
below).

\subsection{Pointers}

Each of the type-specifiers, with the exception of \texttt{list} and
\texttt{string}, may be modified by prepending an asterisk to the
associated parameter name, as in
%    
\begin{code}
function scalar myfunc (series *y, scalar *b)
\end{code}
This indicates that \verb|*y|, isn't a series, but a \emph{pointer} to
a series.  In practice, a pointer to a variable contains the memory
address at which the variable is stored.

This is usually considered a bit mysterious by people unfamiliar with
the C programming language, so allow us to explain how pointers work
by means of a (quite silly) example: suppose you set up a barber
shop. Ideally, your customers would walk through your shop door, sit
on a chair and have their hair trimmed or their beard shaved. However,
local regulations forbid you to modify anything going through your
shop door; of course, you wouldn't do much business if people must
walk out of your barber shop with their hair untouched. Nevertheless,
you have a simple way to go around this limitation: your customers can
come to your shop, tell you their home address and walk out. Then,
nobody stops you from going to their place and excercise your fine
profession.

Unlike C, you have to take no special precaution inside the function
body to distinguish the variable from its address\footnote{In C, this
  would be called \emph{dereferencing} the pointer.}. You just use the
variable's name. In order to supply the address of a variable when you
invoke the function, you use the ampersand (\verb|&|) operator. An
example should, hopefully, make this clear; the following code
\begin{code}
function void swap(scalar *a, scalar *b)
    scalar tmp = a
    a = b
    b = tmp
end function

scalar x = 0
scalar y = 1000000
swap(&x, &y)
print x y
\end{code}
gives
\begin{code}
              x =  1000000.0
              y =  0.0000000
\end{code}
so \texttt{x} and \texttt{y} have in fact been swapped. How?

First you have the function definition, in which the arguments are
pointers to scalars. Inside the function body, the distinction is
moot, as \verb|a| is taken to mean ``the scalar that you'll find at
the address \verb|*a|, which you'll get as the first parameter'' (and
likewise for \verb|b|). The rest of the function simply swaps
\texttt{a} and \texttt{b} by means of a local temporary variable.

Outside the function, we first initialize the two scalars \texttt{x}
to 0 and \texttt{y} to a big number. Then we call the function.  When
the function is called, it's given as arguments \verb|&a| and
\verb|&b|, which hansl identifies as ``the memory address of'' the two
scalars \texttt{a} and \texttt{b}, respectively.

Writing a function with pointer arguments has two main consequences:
first, as we just saw, it makes it possible to modify the function
arguments. Second, it avoids the computational cost of having to
allocate memory for a copy of the arguments and performing the copy
operation; such cost is proportional to the size of the argument. 
Hence, for matrix arguments, this is a nice way to write faster
functions, as producing a copy of a large matrix can be quite
time-consuming.\footnote{However, the \cmd{const} qualifier achieves the same
effect. See \GUG\ for further details.}

\section{Recursion}

Hansl functions can be recursive; what follows is the obligatory
factorial example:
\begin{code}
function scalar factorial(scalar n)
    if (n<0) || (n>floor(n))
        # filter out everything that isn't a 
        # non-negative integer
        return NA
    elif n==0
        return 1
    else
        return n*factorial(n-1)
    endif
end function

loop i=0..6 --quiet
    printf "%d! = %d\n", i, factorial(i)
end loop
\end{code}

Note: this is fun, but in practice, you'll be much better off using
the pre-cooked gamma function (or, better still, its logarithm).

%%% Local Variables: 
%%% mode: latex
%%% TeX-master: "hansl-primer"
%%% End: 


\chapter{Numerical methods}

\section{Numerical differentiation and optimisation}

For optimization:
\begin{itemize}
\item BFGS (tried and true)
\item Newton--Raphson; actually, the name is misleading. It should be
  called something like ``curvature-based'', since it can be used for
  BHHH and scoring method if necessary.
\item Derivative-free methods: simulated annealing and (coming soon)
  Nelder-Mead
\end{itemize}

These can use numerical or analytical derivatives when needed.

\textbf{Note:} these functions are generic optimizers. In the context
of an estimation problem, in which the maximand is a likelihood or
something like that, we have specialized commands (\cmd{mle},
\cmd{gmm} etcetera).

For numerical differentiation we have
\texttt{fdjac}\footnote{\textbf{Note to self:} we \emph{must}
  implement bilateral numerical differentiation if we don't already
  (have to check) and parallelization.}


\section{Random number generation}

\begin{itemize}
\item Mersenne Twister in its various incarnations
\item Ziggurat vs Box--Muller
\item Other distributions
\end{itemize}

%%% Local Variables: 
%%% mode: latex
%%% TeX-master: "hansl-primer"
%%% End: 
