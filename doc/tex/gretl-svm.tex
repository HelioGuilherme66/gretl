\documentclass{article}
\usepackage[letterpaper,body={6.0in,9.1in},top=0.7in,left=1.25in,nohead]{geometry}
\usepackage{url,verbatim,fancyvrb}
\usepackage{color,gretl}
\usepackage[authoryear]{natbib}
\usepackage[pdftex]{graphicx}
\usepackage[pdftex,hyperfootnotes=false]{hyperref}

\definecolor{steel}{rgb}{0.03,0.20,0.45}

\hypersetup{pdftitle={gretl + SVM},
            pdfauthor={Allin Cottrell},
            colorlinks=true,
            linkcolor=blue,
            urlcolor=red,
            citecolor=steel,
            bookmarksnumbered=true,
            plainpages=false
}

\begin{document}

\setlength{\parindent}{0pt}
\setlength{\parskip}{1ex}
\setcounter{secnumdepth}{2}

\title{gretl + SVM}
\author{Allin Cottrell}
\maketitle

\section{Introduction}
\label{sec:intro}

This is documentation for a gretl function named \texttt{svm}, which
offers an interface to the machine-learning functionality provided by
\textsf{libsvm} (SVM = Support Vector Machine). We assume that the
reader knows at least a little about machine learning and how it
relates to econometrics. If this assumption does not hold please take
a look at \cite{mull-spiess17}, or google the topic if you prefer, and
come back when you're ready.\footnote{For a good account of the
  technical background on SVMs, see \cite{smola04}.}

\textsf{Libsvm} is an open-source library (see
\url{https://www.csie.ntu.edu.tw/~cjlin/libsvm/}) and gretl (as we
suppose you know) is an open-source, cross-platform econometrics
package (\url{http://gretl.sourceforge.net/}). Support for
\textsf{libsvm} in gretl was added in the \texttt{2017d} release
(2017-11-07). Since then, however, some improvements have been made
and at present you'll be better off using gretl \texttt{2019a} or
higher to work with this.

In section~\ref{sec:function} we outline the gretl \texttt{svm}
function and illustrate its usage via a few example scenarios;
section~\ref{sec:options} goes into more detail on the supported
options. Section~\ref{sec:xvalid} discusses cross-validation;
section~\ref{sec:SVC} gives a classification example;
section~\ref{sec:SV-probs} discusses probability estimation; and
section~\ref{sec:implement} provides some details on the
implementation of \textsf{libsvm} in gretl.

\section{The \texttt{svm} function}
\label{sec:function}

The signature of \texttt{svm} is as follows:
\begin{code}
series svm(list L, bundle bparms, bundle *bmod[null], bundle *bprob[null])
\end{code}
That is, this function returns a series (predictions) and it has two
required arguments, a list (of series) and a bundle containing zero
or more parameter specifications. The third and fourth arguments take
the form of ``pointers to bundle''; they are optional, and may be set
to \texttt{null} or omitted altogether; we illustrate their use below
but for now we just note that they offer means of ferrying additional
information from \texttt{svm} to the user or vice versa.

The list argument to \texttt{svm} works like the list argument to
regression commands in gretl: it should contain the dependent variable
first, followed by the independent variables. Note that there's no
point in including an intercept (the series \texttt{const} or
\texttt{0}) in \texttt{L}; it will just be dropped by \textsf{libsvm}.

The \texttt{bparms} bundle may contain quite a number of parameter
specifications and other options but most of these have default
values which may be acceptable, in which case you can get by with a
minimal set.  A listing is given in Table~\ref{tab:options} and usage
is discussed in section~\ref{sec:options}.

The workflow for supervised machine learning goes like this:
\begin{enumerate}
\item Gather suitable data and divide them into a training set and a
  testing or ``holdout'' set.
\item Build a model using the training data and assess its fit.
\item Using this model, generate predictions for the testing data and
  assess the fit.
\end{enumerate}

Then if the out-of-sample performance is deemed satisfactory, the
model may be used to generate ``live'' predictions for additional
data.

Gretl's \texttt{svm} is set up to perform steps 2 and 3 above with
ease (though it can also do other things). We'll walk through this
sort of basic usage in the first scenario below.

\subsection{First scenario}
\label{sec:s1}

Suppose you have an ``integrated'' dataset, comprising $n$ training
observations followed by $m$ testing observations, and this dataset is
loaded in gretl. You construct a list \texttt{L} as described above to
pass as the first argument to \texttt{svm}. In the simplest case, the
only parameter you need include in the bundle to pass as the second
argument is \texttt{n\_train}, which sets the number of observations
to use for training (assumed to be at the start of the dataset); for
example, if you have 5000 training observations,
\begin{code}
bundle b = defbundle("n_train", 5000)
series yhat = svm(L, b)
\end{code}

Here's what happens by default.
\begin{enumerate}
\item Gretl examines your dependent variable (the first member of
  \texttt{L}): if this is a binary dummy, or its ``coded'' attribute
  is set, the $C$-SVC SVM type is selected (classification), otherwise
  $\epsilon$-SVR (regression) is selected. In both cases the default
  kernel is RBF (radial basis function).
\item Gretl finds the maxima and minima of the independent variables
  in the sample range 1 to \texttt{n\_train} and scales them onto
  [$-1,1$].
\item The \textsf{libsvm} training function is called, with all
  parameters at their default values as shown in
  Table~\ref{tab:options}. The resulting model is stored in memory.
\item The \textsf{libsvm} prediction function is called on the
  training range, and the degree of fit is shown (for classification,
  the count and percentage of correct predictions; for regression the
  MSE and $R^2$).
\item If all has gone OK, gretl checks if you have enough observations
  for a testing phase---that is, somewhat arbitrarily, there are at
  least 10 observations between \texttt{n\_train} and the end of the
  current sample range. (One would generally expect there to be a good
  deal more testing observations.) If so, the independent variables in
  the test set are scaled using the ranges saved from the training
  data, a second prediction call is made, and the out-of-sample fit is
  shown.
\end{enumerate}

We'll give a real example, replication of the Mullainathan and Spiess
study referenced in the Introduction. It's not an exact replication
since the authors used different machine-learning software (a
collection of \textsf{R} tools), but we can replicate their baseline
OLS results and our \textsf{libsvm} results are comparable to those
shown in their article for machine learning.

A hansl script for this purpose is shown in Listing~\ref{listing:MS}.
We discuss it below.

\begin{script}[htbp]
  \caption{Mullainathan and Spiess replication script}
  \label{listing:MS}
  \begin{center}
    {\small
    \url{http://gretl.sourceforge.net/svm/MS_simple.inp}}
  \end{center}
\begin{scode}
set verbose off

# helper function: get R^2 and MSE
function matrix get_stats (const series y, const series yhat)
  scalar SSR = sum((y - yhat)^2)
  return {1 - SSR/sst(y), SSR / $nobs}
end function

# BLOCK 1: open and arrange data
open ahs_jep.gdtb -q
rename LOGVALUE y
dataset sortby holdout
list X = dataset
list drop = y folds8 holdout
X -= drop
X = cdummify(X)
ntrain = 10000
test1 = ntrain + 1

# BLOCK 2: OLS baseline
smpl 1 ntrain
X = dropcoll(X)
ols y 0 X --quiet
printf "OLS: training R^2 = %.3f, MSE = %.3f\n", $rsq, $ess/$nobs
smpl test1 $tmax
yhat = lincomb($xlist, $coeff)
m = get_stats(y, yhat)
printf "OLS: holdout R^2 = %.3f, MSE = %.3f\n", m[1], m[2]

# BLOCK 3: SVM training and testing
smpl full
bundle parms = defbundle("n_train", ntrain, "quiet", 1)
list L = y X
series svmpred = svm(L, parms)
smpl 1 ntrain
m = get_stats(y, svmpred)
printf "SVM: training R^2 = %.3f, MSE = %.3f\n", m[1], m[2]
smpl test1 $tmax
m = get_stats(y, svmpred)
printf "SVM: holdout R^2 = %.3f, MSE = %.3f\n", m[1], m[2]
\end{scode}
\end{script}

In \texttt{BLOCK 1} of the Mullainathan--Spiess script we open a data
file named \texttt{ahs\_jep.gdtb}. This is a native gretl version of
the dataset used by the authors, which they created from ``raw''
American Housing Survey data with the help of \textsf{R}. Our version
is a translation of the \textsf{R} data; it is available for download
as

\url{http://gretl.sourceforge.net/svm/ahs_jep.gdtb}

The dataset contains the log-values of 51,808 housing units along with
162 covariates, most of which are encodings of qualitative
characteristics. 10,000 of the observations are designated for training,
leaving 41,808 for testing.

We begin by renaming, for the sake of brevity, the dependent variable
\texttt{LOGVALUE} as \texttt{y}. The dummy variable \texttt{holdout}
has value 1 for observations in the testing set, 0 for observations in
the training set. In the original data these are interspersed, but for
gretl's \texttt{svm} we want all the training observations to come
first, hence the line
\begin{code}
dataset sortby holdout
\end{code}
Then, since many of the independent variables are encodings
(``factors,'' in \textsf{R} parlance), we turn them into sets of 0/1
dummies using the \texttt{cdummify} function: the list \texttt{X} then
contains 338 series.

In \texttt{BLOCK 2} of Listing~\ref{listing:MS} we first set the
sample to the training data only and purge the list \texttt{X} of
perfectly collinear terms, leaving 274 series.\footnote{These terms
  would have been dropped automatically by the \texttt{ols} command,
  but we also want to remove them from the list passed to
  \texttt{svm}.}  We then estimate a linear model via OLS. After
switching to the holdout sample, we use the OLS parameter
estimates (\texttt{\$coeff}) to generate fitted values using the
\texttt{lincomb} function, and calculate the measures of fit employed
by Mullainathan--Spiess (hereinafter M--S).

In \texttt{BLOCK 3} we come to the SVM part. This is straightforward
(though it takes a while to run). We tell \texttt{svm} how many
observations to use for training, and (optionally) tell it to be
quiet. The remainder of this block just uses the original \texttt{y}
data and the predictions from \texttt{svm} (which we name
\texttt{svmpred}) to calculate the measures of fit in the training and
testing sub-samples.

The output obtained from running this script should closely resemble
the following (which was obtained in 70 seconds on a quad-core desktop
with Intel i7 Haswell processors running Arch Linux):
\begin{code}
OLS: training R^2 = 0.473, MSE = 0.589
OLS: holdout  R^2 = 0.417, MSE = 0.674
SVM: training R^2 = 0.494, MSE = 0.565
SVM: holdout  R^2 = 0.448, MSE = 0.639
\end{code}
The OLS $R^2$ values agree with those shown in Table 1 of M--S, which
inspires confidence that we're really working with the same data. M--S
show results from four machine-learning algorithms: (1) Regression
tree tuned by depth, (2) LASSO, (3) Random forest (a linear
combination of trees) and (4) Ensemble (a weighted combination of the
first three methods). Our ``out of the box'' \textsf{libsvm} results
are in the middle in terms of out-of-sample (or ``holdout'') fit:
better than (1) and (2) but not quite as good as (3) or (4), which
have $R^2$ values of 0.455 and 0.459 respectively as against our
0.448.

In section~\ref{sec:ms-xvalid} below we follow up on this in light of
cross validation.

\subsection{Second scenario}
\label{sec:s2}

Say you have separate data files for training and testing and you
don't wish to combine them. You'd like to do training on the first
dataset and testing on the second.

In the first variant of this scenario we assume you're nonetheless
willing to do the two operations in the context of a single hansl
script. In that case your solution (or a bare-bones version of it) is
shown in Listing~\ref{listing:two-in-one}. Note that in this script
(and others to follow) we construct the list to pass to \texttt{svm}
as simply
\begin{code}
list L = dataset
\end{code}
This works if (a) your dependent variable occupies position 1 in the
dataset and (b) you wish to use all the other series in the dataset as
supports. Otherwise you need to compose \texttt{L} yourself.

\begin{script}[htbp]
  \caption{Scenario 2: using two data files in one script}
  \label{listing:two-in-one}
\begin{scode}
open train.gdt
list L = dataset
bundle b = defbundle("ranges_outfile", "train.ranges")
bundle savemod = defbundle()
svm(L, b, &savemod)

open test.gdt --preserve
list L = dataset
bundle b = defbundle("ranges_infile", "train.ranges", "loadmod", 1)
svm(L, b, &savemod)
\end{scode}
\end{script}

In Listing~\ref{listing:two-in-one} we save the ranges (for scaling)
to file in the training run, and also use the third argument to
\texttt{svm} to save the model as a bundle (namely
\texttt{savemod}). After opening the test data file (with the
\verb|--preserve| option to avoid destroying \texttt{savemod}) we then
load the ranges from file, and also load the saved model bundle via
\texttt{svm}'s third argument.\footnote{See section~\ref{sec:options}
  for explanation of usage of this argument as well as other options
  illustrated in these examples.}

OK, but what if you want two separate scripts, to be run on distinct
occasions? Then try Listing~\ref{listing:s2-separate}. In this case we
save both the ranges and the model to file in the first script, then
reload them in the second.

\begin{script}[htbp]
  \caption{Scenario 2: using two separate scripts}
  \label{listing:s2-separate}
\begin{scode}
# script 1
open train.gdt
list L = dataset
bundle b = defbundle("ranges_outfile", "my.ranges", "model_outfile", "my.model")
svm(L, b)

# script 2
open test.gdt
list L = dataset
bundle b = defbundle("ranges_infile", "my.ranges", "model_infile", "my.model")
svm(L, b)
\end{scode}
\end{script}

%%% combined parameter/options table

\begin{table}[p]
  (a) Values for \texttt{svm\_type} (default automatic):
  \begin{center}
  \begin{tabular}{clll}
\textit{code} & \textit{string} & \textit{SVM} & \textit{comment} \\[2pt]
0 & \verb|C-SVC| & $C$-SVC & multi-class classification \\
1 & \verb|nu-SVC| & $\nu$-SVC & multi-class classification \\
2 & \verb|one-class| & one-class SVM & \\
3 & \verb|eps-SVR| & $\epsilon$-SVR & regression \\
4 & \verb|nu-SVR| & $\nu$-SVR & regression
  \end{tabular}
  \end{center}

  (b) Values for \texttt{kernel\_type} (default 2, RBF):
  \begin{center}
  \begin{tabular}{clll}
\textit{code} & \texttt{string} & \textit{kernel} & \textit{comment} \\[2pt]
0 & \verb|linear| & linear & $u'v$ \\
1 & \verb|polynomial| & polynomial & $(\gamma\, u'v + c_0)^d$ \\
2 & \verb|RBF| & Radial Basis Function & $\exp(-\gamma\, \lVert u-v \rVert^2)$ \\
3 & \verb|sigmoid| & sigmoid & $\mbox{tanh}(\gamma\, u'v + c_0)$ \\
4 & TBA & \texttt{TBA}
  \end{tabular}
  \end{center}

  (c) Additional \textsf{libsvm} parameters:
  \begin{center}
  \begin{tabular}{ll}
    \textit{key} & \textit{parameter controlled} \\[2pt]
\texttt{degree} & degree of polynomial ($d$) in kernel function (default 3) \\
\texttt{gamma} & $\gamma$ in kernel function (default 1/\# of covariates) \\
\texttt{coef0} & $c_0$ in kernel function (default 0) \\
\texttt{C} & the cost parameter $C$ of $C$-SVC, $\epsilon$-SVR, and $\nu$-SVR (default 1) \\
\texttt{nu} & the parameter $\nu$ of $\nu$-SVC, one-class SVM, and $\nu$-SVR (default
     0.5) \\
\texttt{toler} & tolerance of termination criterion (default 0.001) \\
\texttt{epsilon} & the $\epsilon$ in loss function of $\epsilon$-SVR (default 0.1) \\
\texttt{cachesize} & cache memory size in MB (default 1024) \\
\texttt{shrinking} & use shrinking heuristics, 0 or 1 (default 1) \\
\texttt{probability} & produce probability estimates, 0 or 1 (default 0)
  \end{tabular}
  \end{center}

  (d) Additional gretl-specific parameters:
  \begin{center}
  \begin{tabular}{ll}
    \textit{key} & \textit{comment} \\[2pt]
    \texttt{n\_train} & integer: the number of training observations \\
    \texttt{scaling} & 0 = none, 1 = [$-1,1$], 2 = [$0,1$] \\
    \texttt{loadmod} & 0/1: load model from bundle-pointer \\
    \texttt{predict} & 0, 1 or 2: do prediction if relevant (see text) \\
    \texttt{quiet} & 0/1: suppress printed output from \texttt{svm} \\
    \texttt{model\_outfile} & filename for writing model \\
    \texttt{model\_infile} & filename for reading model \\
    \texttt{ranges\_outfile} & filename for writing data ranges \\
    \texttt{ranges\_infile} & filename for reading data ranges \\
    \texttt{data\_outfile} & filename for writing data values \\
    \texttt{range\_format} & string: \texttt{libsvm} or \texttt{gretl}
  \end{tabular}
  \end{center}
  \caption{Parameters and options that may be set using the second
    (bundle) argument to the \texttt{svm} function}
  \label{tab:options}
\end{table}

\section{Options}
\label{sec:options}

Table~\ref{tab:options} shows the options that can be set via the
parameter bundle passed to \texttt{svm}---other than those pertaining
to cross-validation, for which see section~\ref{sec:xvalid}.

As mentioned above, gretl selects a default SVM type based on the
character of the dependent variable. However, this can be overridden
by including an appropriate integer code under the key
\texttt{svm\_type}---see panel (a) of the table---as in
\begin{code}
bundle parms = defbundle()
parms.svm_type = 4 # select nu-SVR
\end{code}
We also mentioned that gretl defaults to the RBF kernel; this can be
overridden by use of the \texttt{kernel\_type} key:
\begin{code}
parms.kernel_type = 3 # sigmoid, unlikely to be superior!
\end{code}

\subsection{Libsvm options}
\label{sec:libsvm-options}

Panel (c) of the options Table shows additional options for
controlling the behavior of \textsf{libsvm}. We discuss the
\texttt{epsilon} option in section~\ref{sec:svr} and the
\texttt{probability} option in section~\ref{sec:SV-probs}. For further
details please see the \textsf{libsvm} documentation, \cite{HCL16}.

All but one of the default values shown are as per \textsf{libsvm}
itself, the exception being \texttt{cachesize}, the size of the memory
cache for use in training, in megabytes. Training can take quite a
while on a big dataset, and a large cache is likely to speed things
up. The \textsf{libsvm} default value is 100\,MB, but many computers
today can support a gigabyte of cache so we raised this to 1024. If
your machine's memory is limited you may need to reduce this, for
example
\begin{code}
parms.cachesize = 200
\end{code}

\subsection{Other options}
\label{sec:gretl-options}

That leaves the gretl-specific settings in panel (d) of the options
Table, some of which may require more explanation.

\texttt{n\_train}: has already been discussed: it simply tells gretl
how many training observations can be found at the start of the
dataset. But in this context ``the start'' means the first usable
observation. If your dataset is complete (no missing observations)
then specifying \texttt{n\_train} = 5000 will result in the use of
observations 1 to 5000 for training. But if there are, say, 10 missing
observations at the beginning of the data, the training range will
then be 11 to 5010.

\texttt{scaling}: is the data-scaling option: 0 means that no scaling
should be done (not advisable unless the input data are already
suitably scaled); 1 means to scale to a range from $-1$ to $+1$ (the
default); and 2 means to use a range of 0 to 1. At present we do not
support user-specified lower and upper limits, and neither do we
support scaling of the dependent variable (the above applying only to
the independent variables).

\texttt{loadmod}: this binary variable governs the interpretation of
the optional third (pointer-to-bundle) argument to \texttt{svm}. By
default it is assumed that this should be used to save an SVM model in
the form of a bundle (an auxiliary ``return'' value, if you
will). Note that this overwrites any prior content the bundle may have
had. By giving \texttt{loadmod} a non-zero value you are telling gretl
to expect a model bundle on input instead (and not to overwrite it).

\texttt{predict}: this governs how much prediction \texttt{svm} does.
The default is 2, meaning that if a model is built then prediction
should be done on both the training data used to build the model (to
get a measure of fit) and also on any testing data present. By setting
\texttt{predict} = 1 you are requesting that prediction be done for
(at most) only the training. By setting it to 0 you are telling gretl
not to make any calls to the \textsf{libsvm} prediction function (or
none beyond those required by cross validation---see
section~\ref{sec:xvalid}).

The several \texttt{outfile} and \texttt{infile} keys can be used to
supply filenames for writing or reading, respectively, certain sorts
of data. Only if such names are given does \texttt{svm} write or read
any files; by default everything goes on in memory.

\texttt{model\_infile} can be used to get gretl to read a previously
trained \textsf{libsvm} model file (which implies that \texttt{svm}
will skip the training step), and \texttt{model\_outfile} can be used
to produce a file readable by the \texttt{svm-predict} executable.
The \texttt{ranges} keys work in a similar manner, for files
containing \textsf{libsvm} data-range information. However, gretl's
own range files contain more information than standard \textsf{libsvm}
ones and cannot be read by the \texttt{svm-} executables, so if you
wish to output a ranges file readable by them you must in addition
specify
\begin{code}
parms.range_format = "libsvm"
\end{code}
(substituting the name of your parameter bundle). There is at present
no facility for the converse---that is, getting gretl's \texttt{svm()}
to read a ranges file written by \texttt{svm-scale}. Attempting to do
so will generate an error, regardless of the setting of
\texttt{range\_format}.

Finally, \texttt{data\_outfile} can be used to write data in the
format required by \texttt{svm-train} and \texttt{svm-predict}. Note
that there is no \texttt{data\_infile} key: \texttt{svm} only accepts
data passed from a gretl dataset via its first (list) argument.

\section{Cross validation}
\label{sec:xvalid}

\textsf{Libsvm} provides a cross-validation function that works as
follows. Given a number, $v \geq 2$, of ``folds,'' the data are
divided randomly into $v$ equally sized subsets,\footnote{That is by
  default, but see also section~\ref{sec:user-folds} below.} then
for each subset $i$ a sub-model is trained on the complementary subset
of the data (the other $v-1$ folds) and predictions are generated for
subset $i$. Once this process is complete we have ``pseudo out of
sample'' predictions for all the data supplied (which should be either
the full training dataset or perhaps a subset thereof), and suitable
figures of merit can be calculated for these predictions.

Here's the context in which this can be particularly useful: when
\textsf{libsvm} trains a model it automatically adjusts a (possibly
large) number of coefficients, but it takes certain kernel parameters
as given. For the recommended RBF kernel this includes $C$ and
$\gamma$; for the SVM type $\epsilon$-SVR it also includes $\epsilon$;
and for $\nu$-SVC, $\nu$-SVR and one-class classification it includes
$\nu$. It may be that prediction can be improved by tuning these
parameters. However, simply tuning on the full training dataset is
likely to result in over-fitting, to the detriment of prediction on
the testing data. So the recommended procedure is:
\begin{enumerate}
\item Run a line or grid search over ranges of values for one or more
  of the above-mentioned parameters, on each iteration calculating a
  suitable figure of merit via cross validation.
\item Thereby identify the parameter value(s) that produce the best
  cross-validation results (the maximum percentage of correct
  predictions for classification and, by default, minimum MSE for
  regression---but see section~\ref{sec:regcrit} below).
\item Use the optimized parameters to train a model on the full
  training dataset, and then use this model to generate predictions
  for the testing set.
\end{enumerate}

It is still possible that this will result in over-fitting to the
training data, but that risk should at least be lessened.\footnote{For
  further discussion see section 3 of \cite{HCL16}.}

The cross-validation options supported by gretl's \texttt{svm} are
shown in Table~\ref{tab:xvalid}.

\begin{table}[htbp]
  \centering
  \begin{tabular}{lll}
  & \textit{key} & \textit{type and effect} \\[4pt]
   & \texttt{folds} & integer $\geq 2$: number of folds (default 5) \\
  \textit{or} & \texttt{foldvar} & 0/1: flags inclusion of a folds variable
    (see text) \\[4pt]
   & \texttt{search} & 0/1: parameter search using default grid \\
 \textit{or} & \texttt{grid} & matrix: user-specified grid \\[4pt]
    & \texttt{search\_only} & 0/1: see text \\
    & \texttt{contiguous} & 0/1: see section~\ref{sec:user-folds} \\
    & \texttt{seed} & integer: see section~\ref{sec:random-folds} \\
    & \texttt{refold} & 0/1: see section~\ref{sec:random-folds} \\
    & \texttt{regcrit} & integer: see section~\ref{sec:regcrit}
  \end{tabular}
  \caption{Cross-validation options in \texttt{svm} parameter bundle}
  \label{tab:xvalid}
\end{table}

We have more to say about specification of the folds in
section~\ref{sec:user-folds} below.

Note that only one of \texttt{search} and \texttt{grid} should be
given. And if either of these is given, the \texttt{folds} option
need not be supplied: in its absence the number of folds defaults to
5.

By default grid ranges are specified in terms of the base-2 logs of
the parameters.\footnote{But see section~\ref{sec:local-search}
  below.}  The \texttt{svm} default grid is based on that used by the
\texttt{grid.py} tool supplied with the \textsf{libsvm} distribution:
\begin{center}
\begin{tabular}{lrrr}
 & \textit{start} & \textit{stop} & \textit{step} \\[2pt]
$C$ & $-5$ & 9 & 2 \\
$\gamma$ & 3 & $-15$ & $-2$
\end{tabular}
\end{center}

This means that $C$ will range from $2^{-5}$ to $2^9$, quadrupling at
each step, while $\gamma$ will range from $2^3$ down to $2^{-15}$,
shrinking by a factor of 4 at each step. Note that implicitly
$\epsilon$ or $\nu$ (where applicable) are clamped at their initial
values (whether their defaults or user-specified).

A user-defined matrix under the \texttt{grid} key should conform to
this general pattern, but with some flexibility: it should have from 1
to 3 rows and either 3 or 4 columns.\footnote{The use of a 4th column
  is explained in section~\ref{sec:local-search}.} The first row is
assumed to pertain to $C$; the second row, if present, to $\gamma$;
and the third, if present, to $\epsilon$ or $\nu$. If you just want a
line-search for $C$ a one-row matrix will suffice.  If you want to
search for $C$ and $\epsilon$, with $\gamma$ clamped, you need a
three-row matrix: a row of zeros (in this example, the second or
$\gamma$ row) indicates a clamped value.

The \texttt{search\_only} option has the effect of stopping
\texttt{svm} from doing any training or prediction following parameter
tuning.

To be clear, the number of observations used in tuning---call this
$T_0$---is either the size of the current sample range, as set by the
\texttt{smpl} command prior to calling \texttt{svm}, or
\texttt{n\_train} if that is specified in the parameter bundle. Once
tuning is completed, by default \texttt{svm} trains a model on all
$T_0$ observations using the optimized parameters, then performs
prediction for any additional data (i.e., from $T_0 + 1$ to the end of
the sample range).

You should define \texttt{search\_only} if you just want to find and
save the ``best'' parameter values. As an aid, setting this flag has
an additional effect: if a pointer-to-bundle is supplied as the third
argument to \texttt{svm}, a matrix named \texttt{xvalid\_results} is
written into it. (Any other content of the bundle is not touched.)
This matrix has either three or four columns and as many rows as there
are parameter combinations. Columns 1 and 2 hold, respectively, the
values of $C$ and $\gamma$.  If the model uses the parameter
$\epsilon$ or $\nu$ this occupies the third column; the last column
holds the associated value of the search criterion.  If
\texttt{search\_only} is the only cross-validation option given, this
implies use of the default grid and 5 folds.

Note that grid search with cross validation can be very expensive on a
big dataset. The default grid has $8 \times 10$ pairs of $(C, \gamma)$
values; with 5 folds this means 400 training runs. Whatever timing you
find for a single training run, you can expect a default grid search
to take longer by two orders of magnitude. Moreover, training involves
more computation the greater the value of the cost factor, $C$, so if
your search space includes large values of $C$ the time taken by cross
validation may be three orders of magnitude more than just training on
a set of parameters in the neighborhood of the default values.

Listing~\ref{listing:xvalid} illustrates usage of cross validation in
\texttt{svm}.

\begin{script}[htbp]
  \caption{Grid search with cross validation}
  \label{listing:xvalid}
\begin{scode}
# script 1: just get optimized parameters using 1000
# training observations with custom grid
open data.gdt
list L = dataset
matrix gmat = {-3,7,1; 1,-13,-2}
smpl 1 1000
bundle b = defbundle("grid", gmat, "search_only", 1)
bundle bestparms = defbundle()
svm(L, b, &bestparms)
print bestparms

# script 2: integrated search, training, prediction, with
# default grid and 5 folds
open data.gdt
list L = dataset
bundle b = defbundle("n_train", 2000, "search", 1)
series yhat = svm(L, b)
\end{scode}
\end{script}

\subsection{Local search, linear search}
\label{sec:local-search}

After doing a coarse-grained parameter search you may wish to do an
additional search in the neighborhood of some values that look
promising.  The $\log_2$ apparatus may not seem very convenient for
that purpose, but a little thought reveals how it may be
used. Listing~\ref{listing:log2} presents a function that takes as
input a specific parameter value, \texttt{p0}, a specification of the
lower limit of the search range expressed as a fraction
(\texttt{frac}) of \texttt{p0}, and a number of steps. The return
value is a suitable row for the grid matrix.

\begin{script}[htbp]
  \caption{Determining a local search specification}
  \label{listing:log2}
\begin{scode}
# helper function
function matrix make_grid_row (scalar p0, scalar frac, int nsteps)
  if p0 <= 0 || frac <= 0 || frac >= 1
    funcerr "Invalid input"
  endif
  if nsteps % 2 == 0
    nsteps++
  endif
  k = nsteps - 1
  start = log2(frac * p0)
  step = (log2(p0) - start)/(k/2)
  stop = start + k * step
  return {start, stop, step}
end function

# sample usage of the above
matrix r = make_grid_row(1, 0.8, 5)
\end{scode}
\end{script}

The sample usage shown in Listing~\ref{listing:log2} produces
\begin{code}
r = {-0.32193, 0.32193, 0.16096}
\end{code}
which translates into the following values for the parameter itself:
(0.8, 0.8944, 1, 1.1180, 1.25). This vector includes the \texttt{p0}
value of 1.0, around which it is ``exponentially symmetrical.''

Alternatively, you may specify that one or more rows of the grid
matrix are to be taken as linear rather than $\log_2$-based. In that
case you must append a fourth column to the grid, with 1 on linear
rows and 0 on $\log_2$-based rows. Here's an example:
\begin{code}
params.svm_type = 4 # nu-SVR
params.grid = {0, 3, 1, 0; 0, 0, 0, 0; 0.5, 0.7, 0.05, 1}
\end{code}
This specifies an exponential search for $C$ (from 1 to 8, doubling at
each step), a clamped value for $\gamma$ and a linear search for
$\nu$ (0.5, 0.55,\dots,0.7).

\subsection{Cross validation criteria}
\label{sec:regcrit}

As mentioned above, cross validation results in the selection of a set
of parameter values that are deemed ``best'' on some criterion. In the
case of classification the optimality criterion is always the
proportion of correctly classified cases but in SVM regression a
choice is offered, via the optional \texttt{regcrit} parameter. This
must be an integer from 1 to 4, with 1 being the default, as follows:
%
\begin{center}
  \begin{tabular}{ll}
    1 & Minimum MSE (Mean Squared Error) \\
    2 & Minimum MAD (Mean Absolute Deviation) \\
    3 & Minimum rounded MAD \\
    4 & Minimum rounded misses\\
  \end{tabular}
\end{center}
%
By ``rounded MAD'' we mean the MAD calculated using the absolute
difference between the actual value of the dependent variable and the
model's prediction rounded to the nearest integer. And by ``rounded
misses'' we mean cases where the rounded prediction does not equal the
actual dependent variable. Options 3 and 4 are applicable only when
the dependent variable is integer-valued; an error is flagged if this
condition is not met.

\subsection{User-specified folds}
\label{sec:user-folds}

In some contexts there may be reason to divide the data into subsets
for cross validation on some chosen criterion rather than via random
selection. The \texttt{svm} function supports two (mutually
incompatible) options to this effect.
\begin{itemize}
\item If the parameter bundle contains the key \texttt{foldvar} with a
  non-zero value, \texttt{svm} will take the \textit{last} series in
  the incoming list as a ``folding index'' rather than a
  regressor. Such a series must contain only integer values from 1 to
  the desired number of folds, which form a consecutive sequence when
  sorted. Each observation is then assigned to fold $i$ if its value
  for the folding index is $i$.
\item If the parameter bundle contains the key \texttt{consecutive}
  with a non-zero value this indicates that the folds should simply be
  blocks of consecutive observations, of equal size except that any
  remainder is assigned to the last fold. The number of folds defaults
  to 5 but can be adjusted via the \texttt{folds} key. This option may
  be useful if the order of observations in the dataset is already
  randomized.
\end{itemize}

\subsection{More on random cross-validation folds}
\label{sec:random-folds}

Unmodified \textsf{libsvm} calls the C-library function \texttt{rand}
to produce its permutations---and it doesn't set a seed, so
\texttt{rand} uses its default seed of 1 on each initialization. This
means that each time a given \textsf{libsvm} program is executed (on a
given platform) it will generate exactly the same set of ``random''
folds. However, it also means that if a given program calls the
\textsf{libsvm} cross-validation function several times, as happens in
the course of parameter search, it will get a different set of folds
each time since automatic initialization of \texttt{rand} occurs only
once per program invocation.

This behavior seems debatable on both counts. First, if the folds are
advertised as random, one might expect to get non-identical results
when running a given cross-validation program multiple times (unless
one takes steps to ensure they are the same). Second, in the context
of parameter search one might wish to suppress variation in outcome
stemming from data-subsetting in order to focus on variation stemming
from the differing parameter vectors. Moreover, the \texttt{rand}
function is not guaranteed to produce the same sequence of values for
a given seed on different operating systems.

Gretl's \texttt{svm} function addresses these points as
follows. First, we use the \textsf{SFMT} implementation of the
Mersenne Twister from the gretl library in place of \texttt{rand};
this ensures consistency across platforms. By default we obtain a seed
based on the time at which \texttt{svm} execution starts, and we
(re-)initialize \textsf{SFMT} with this seed immediately prior to each
call to the \textsf{libsvm} cross-validation function. This means that
each invocation of a gretl \texttt{svm} cross-validation program is
likely to produce slightly different results, while in the context of
a given search each parameter combination will employ the same set of
folds. This default behavior can be modified in two ways.
\begin{itemize}
\item To obtain strictly reproducible results you can override the
  execution-time based \textsf{SFMT} seed with an integer
  \texttt{seed} value specified in the parameter bundle passed to
  \texttt{svm}.
\item To emulate \textsf{libsvm}'s behavior, whereby each call to
  randomized cross validation generates a new set of folds, specify a
  non-zero value under the key \texttt{refold} in the parameter
  bundle.
\end{itemize}

When random folding with the default time-based seed is employed, it
may be useful to know what seed was actually used: you can retrieve
this value from the parameter bundle under the key
\texttt{autoseed}. (This item is added to the bundle only if random
folding is performed and no \texttt{seed} value is specified on
input.)

One further note: when pseudo-random numbers are required in the
context of \texttt{svm} they are drawn from an independent instance of
\textsf{SFMT}, distinct from the one governed by the \texttt{set seed}
command and employed by functions such as \texttt{normal()} and
\texttt{randgen()}.

\subsection{More on SV regression}
\label{sec:svr}

We noted above that libsvm supports two types of SV regression:
$\epsilon$-SVR and $\nu$-SVR. These are alternative parameterizations
of the regression problem. To understand this point one must know what
$\epsilon$ and $\nu$ represent.
\begin{itemize}
\item $\epsilon \geq 0$ sets a margin within which prediction errors
  are ignored (costed at zero). To enforce a tight fit---at the risk
  of over-fitting---therefore, one would specify a small value of
  $\epsilon$.
\item $0 < \nu \leq 1$ controls the number of support vectors used by
  the SVM, expressed as a fraction of the maximum (which equals the
  number of observations in the training data). Use of more support
  vectors will generally give a better fit---again at the risk of
  over-fitting.
\end{itemize}
These values cannot be set independently of each other. In
$\epsilon$-SVR $\epsilon$ is parametric and $\nu$ is adjusted
endogenously; the converse holds for $\nu$-SVR.

The literature on this topic suggests that if you want a quick fit at
moderate computational cost you might choose $\nu$-SVR with a
relatively low value of $\nu$. On the other hand, if you want to
obtain the best possible prediction and are not too concerned about
computational cost it would be common to use $\epsilon$-SVR.

One point to note here is that $\epsilon$ is not scale-free: the
implied ``tightness'' of setting, say, $\epsilon = 0.05$ will depend
on the scale of the regressand. Experimentation may be useful.

\subsection{Use of MPI in cross validation}
\label{sec:mpi}

As mentioned above, cross validation can be very expensive
computationally if you have a big training dataset. Fortunately, it's
also ``embarrassingly parallel'' (as the computer scientists say);
that is, trivial to parallelize. The (potentially quite large) set of
parameter combinations can be parceled out to separate processes,
since each set of training runs (across the folds, for a given
combination of parameters) is independent of the others. This is an
obvious use-case for MPI.

MPI is ``Message Passing Interface,'' a standard for running several
distinct processes that execute the same program using different
data. Support for MPI in gretl is documented in
\cite{gretl-mpi}. Usually we leave it up to users of gretl to exploit
MPI via scripting (which demands a certain proficiency in coding), but
in the case of cross validation under \texttt{svm} we have implemented
MPI support internally. All the user has to do is execute a call to
\texttt{svm} that calls for cross validation, either via the
command-line program \texttt{gretlmpi} or within an \texttt{mpi} block
in a script that's run in the gretl GUI. For details of these options,
please see \cite{gretl-mpi}.

To make a serious dent in the time taken by a big cross validation
problem using MPI one would want to exploit a high-performance cluster
if possible. However, to illustrate the potential of MPI, even just on
a single machine, we constructed a task that is big enough to produce
meaningful differences in execution times but not so big as to waste a
lot of time. We used one of the \textsf{libsvm} sample regression
datasets,\footnote{See
  \url{https://www.csie.ntu.edu.tw/~cjlin/libsvmtools/datasets/}.}
\texttt{cadata}, a native gretl version of which is available as

\url{http://gretl.sourceforge.net/svm/cadata.gdtb}

Like the Mullainathan--Spiess dataset this concerns housing values,
but it is smaller both in number of observations (20640) and number of
covariates (8).  Listing~\ref{listing:cadata} shows a script which
performs cross-validation using 3000 observations, with a grid for $C$
and $\gamma$ which gives 32 parameter combinations. The times taken to
produce the results matrix under different configurations on a Dell
desktop with 4 Intel Haswell cores and up to 8 hyperthreads are shown
in Table~\ref{tab:cadata}. In this example it turns out that
multi-threading via \textsf{OpenMP} is not greatly advantageous, but
MPI works quite nicely---for up to six MPI processes.

\begin{script}[htbp]
  \caption{Parameter search, script \texttt{ca\_xvalid.inp}}
  \label{listing:cadata}
  \begin{center}
    {\small
    \url{http://gretl.sourceforge.net/svm/ca_xvalid.inp}}
  \end{center}
\begin{scode}
set verbose off
open cadata.gdtb -q
list X = dataset - median_value
series lny = log(median_value)
list L = lny X
bundle b = defbundle("n_train", 3000, "shrinking", 0, "search_only", 1)
b.grid = {-1, 5, 2; 3, -12, -2}
b.seed = 1611771
bundle res = defbundle()
svm(L, b, &res)
if $mpirank < 1 # just print once
  matrix R = res.xvalid_results
  print R
endif
\end{scode}
\end{script}

\begin{table}[htbp]
\centering
\begin{tabular}{rcr}
\textit{configuration} & \textit{wall time} & \textit{\%CPU} \\
single-threaded & 1:17 & 99 \\
OMP 4 & 1:00 & 274 \\
OMP 8 & 1:04 & 515 \\
MPI 4, OMP 2 & 0:27 & 535 \\
MPI 4, OMP 1 & 0:27 & 357 \\
MPI 6, OMP 1 & 0:23 & 478
\end{tabular}
\caption{Execution times (m:s) for \texttt{ca\_xvalid.inp} (OMP =
  number of OpenMP threads, MPI = number of MPI processes)}
\label{tab:cadata}
\end{table}

For reference, the command line for the fastest run represented in
Table~\ref{tab:cadata} (MPI 6, OMP 1) was
\begin{code}
OMP_NUM_THREADS=1 mpirun -N 6 gretlmpi ca_xvalid.inp
\end{code}
The optimized result (the same on all runs) was a cross-validation MSE
of 0.0581 at $C=2$ and $\gamma=2$.

% \clearpage

\subsection{Mullainathan and Spiess revisited}
\label{sec:ms-xvalid}

Having discussed cross validation we return to the replication of
\cite{mull-spiess17} discussed in section~\ref{sec:s1}. How much
can we improve on the results of $\epsilon$-SVR ``out of the
box'' by means of parameter search?

We began with some ``sighting shots''---small, coarse-grained
cross-validation runs, performed on a standard desktop machine and
using the \texttt{libsvm} default of 5 random folds---which gave us
the idea that $C$ values between 2.5 and 3.0 could be helpful (as
against the default of 1.0). It was also apparent that moderate
variations in $\gamma$ and $\epsilon$ were worth exploring.

On that basis we set up a finer (linear) grid with 80 parameter
combinations and used \texttt{gretlmpi} on an HPC node with 32 Xeon
cores, running 16 MPI processes with two threads each. In addition,
rather than random folds we used the \texttt{folds8} variable in the
M--S dataset, which specifies 8 random subsets of the training sample
based on clusters in the data.\footnote{See \cite{mull-spiess-17a} for
  details.}  Cross-validation took almost an hour and yielded minimum
MSE at 0.606208 for $C=2.8$, $\gamma=0.00365$, $\epsilon=0.09$. The
selected $C$ was in the range we expected and $\epsilon$ was somewhat
smaller than its default of 0.1; the selected $\gamma$ was equal to
the default (the reciprocal of the number of covariates).

We then trained a model on all 10,000 training observations using the
selected parameters and produced predictions for all the data, giving
\begin{code}
SVM: training R^2 = 0.532, MSE = 0.522
SVM: holdout  R^2 = 0.454, MSE = 0.633
\end{code}
as opposed to the results shown earlier
\begin{code}
SVM: training R^2 = 0.494, MSE = 0.565
SVM: holdout  R^2 = 0.448, MSE = 0.639
\end{code}
The new ``holdout'' performance is practically indistinguishable from
that obtained by M--S using the ``Random forest'' method; it still falls
a little short of the ``Ensemble'' method but we're talking about the
third digit of $R^2$. Perhaps we could do a little better with a
fuller exploration of the parameter space but these results seem
quite satisfactory.

\subsection{Parameter tuning plot}
\label{sec:plot}

It may be helpful to get a visual fix on what's going on in the tuning
of the RBF kernel parameters via cross validation. There's an
experimental (and so far, little tested) option for this: you can
include in the parameter bundle under the \texttt{plot} key either
``\texttt{display}'' or an output filename, as in
\begin{code}
parms.plot = "display"
# or
parms.plot = "tuning.pdf"
\end{code}
This has an effect only if a two-dimensional ($C, \gamma$) parameter
search is performed. If \texttt{svm} catches on we have in mind to
generalize this and make it more convenient.

\section{Classification via SVM}
\label{sec:SVC}

The focus of this document has been on regression, but gretl's
\texttt{svm} function also supports classification. As mentioned
above, the \texttt{svm} function automatically switches to
classification mode---by default using $C$-SVC---if the dependent
variable is marked as ``coded'' using the \texttt{setinfo} command.
However, the \texttt{svm\_type} switch in the parameter bundle passed
to \texttt{svm} can be used to force the issue in favor of
classfication via $C$-SVC (\texttt{svm\_type = 0}) or via $\nu$-SVC
(\texttt{svm\_type = 1}).

\subsection{Classification example}
\label{sec:SVC-example}

Listing~\ref{listing:keane} shows an example of classification based
on the script \texttt{keane.inp}, which is supplied with the gretl
distribution. The data file is \texttt{keane.gdt} (also supplied), a
subset of the data used in \cite{keane97}. The original script
exemplifies multinomial logit regression. The dependent variable,
\texttt{status}, encodes the jointly exhaustive states ``in school''
(1), ``at home'' (2) and ``in work'' (3); the explanatory variables
are years of education (\texttt{educ}), years of work experience
(\texttt{exper}) and its square (\texttt{expersq}), and a
\texttt{black} dummy variable.

Listing~\ref{listing:keane} plays $C$-SVC against standard Maximum
Likelihood estimation. With only 4 covariates and 3 classes
\texttt{svm} runs quite quickly so the script is integrated: parameter
search using the default grid is followed by training and
prediction. The dataset comprises several years of data; we estimate
the respective models on the data from 1982 and test using the 1984
data. The results (obtained in about half a minute on the desktop
machine described earlier) are:
\begin{code}
MLE: training percent correct = 61.16 (n=1864)
MLE: testing percent correct  = 69.76 (n=1802)
SVM: training percent correct = 61.70 (n=1864)
SVM: testing percent correct  = 70.20 (n=1802)
\end{code}

\begin{script}[p]
  \caption{Parameter search and estimation: $C$-SVC vs multinomial logit}
  \label{listing:keane}
    \begin{center}
    {\small
    \url{http://gretl.sourceforge.net/svm/multinomial.inp}}
  \end{center}
\begin{scode}
set verbose off

# helper function
function scalar mnlogit_pc_correct (const series y,
                                    const matrix X,
                                    matrix b)
  matrix yvals = values(y)
  scalar n = rows(X)
  b = mshape(b, cols(X), nelem(yvals) - 1)
  matrix P = ones(n, 1) ~ exp(X * b)
  return (100/n) * sumc({y} .= yvals[imaxr(P)])
end function

# open and organize data
open keane.gdt -q
dataset sortby year
setinfo status --coded
ytrain = 82 # train on data from 1982
ytest = 84  # test using data from 1984
list All = status educ exper expersq black

# (1) MLE: multinomial logit
# restrict to complete observations in training year
smpl year == ytrain && ok(All) --restrict
logit status 0 educ exper expersq black --multinomial --quiet
pcc = 100 * sum(status == $yhat) / $T
printf "MLE: training percent correct = %.2f (n=%d)\n", pcc, $nobs
# restrict to complete observations in testing year
smpl year == ytest && ok(All) --restrict --replace
matrix X = {const, educ, exper, expersq, black}
pcc = mnlogit_pc_correct(status, X, $coeff)
printf "MLE: testing percent correct  = %.2f (n=%d)\n", pcc, $nobs

# (2) SVM: C-SVC
# sample: all complete observations in the two selected years
smpl (year==ytrain || year==ytest) && ok(All) --restrict --replace
scalar ntrain = sum(year == ytrain)
# start with cross validation search using default grid
bundle parms = defbundle("n_train", ntrain, "search", 1)
parms.seed = 3211765 # for reproducibility
parms.quiet = 1
yhat_svm = svm(All, parms)
# assess fit in training year
smpl year == ytrain && ok(All) --restrict --replace
pcc = 100 * sum(status == yhat_svm) / $nobs
printf "SVM: training percent correct = %.2f (n=%d)\n", pcc, $nobs
# assess fit in testing year
smpl year == ytest && ok(All) --restrict --replace
pcc = 100 * sum(status == yhat_svm) / $nobs
printf "SVM: testing percent correct  = %.2f (n=%d)\n", pcc, $nobs
\end{scode}
\end{script}

So we see a slim predictive advantage for $C$-SVC over multinomial
logit estimation. Given the paucity of the explanatory data it is
perhaps surprising that the SVM even equals the MLE results.

\section{Probability estimation}
\label{sec:SV-probs}

By default the output of \texttt{svm} is just a series of
point-predictions of the dependent variable conditional on the
covariates or ``features''. It is possible, however, to obtain a
measure of the uncertainty attaching to these predictions.  The
character of the probability information delivered by \textsf{libsvm}
differs between SV regression and SV classification, but the mechanism
for obtaining it via gretl is basically the same: you set
\texttt{probability} to 1 in the parameter bundle, and supply a
pointer-to-bundle as the fourth argument to the \texttt{svm} function,
as in
\begin{code}
...
params.probability = 1
bundle bprob = defbundle()
series yhat = svm(L, params, null, &bprob)
\end{code}
(where the \texttt{null} argument indicates that we're skipping the
optional third parameter). On successful completion \texttt{bprob}
will contain probability information.

\subsection{Regression error distribution}
\label{sec:SVR-probs}

In the case of SV regression, the extra information takes the
form of a scalar value under the key \texttt{svr\_sigma}: this is an
estimate of the scale parameter, $\sigma$, for the Laplace density of
the prediction errors, $\hat{y}_i - y_i$, based on the training data,
namely
\[
\frac{1}{2\sigma} \exp \left(-\frac{|\hat{y}_i - y_i|}{\sigma}\right)
\]
Using this information one can, for example, calculate confidence
intervals for the SVR predictions. The following code snippet shows
how one could obtain a 90 percent interval, using the \texttt{invcdf}
(inverse CDF) function with first argument \texttt{L} for the Laplace
distribution:
\begin{code}
sigma = brob.svr_sigma
maxerr90 = invcdf(L, 0, sigma, 0.95)
series lo = yhat - maxerr90
series hi = yhat + maxerr90
print yhat lo hi --byobs
\end{code}

\subsection{Discrete outcome probabilities}
\label{sec:SVC-probs}

In the classification case, the information supplied via the bundle
passed as the fourth argument to \texttt{svm} takes the form of the
matrices \texttt{Ptrain} (if prediction over the training range is
requested) and/or \texttt{Ptest} (if prediction over the testing range
is called for). Each of these matrices has as many columns as there
are outcomes, and as many rows as there are observations in the
relevant range. The column headings identify the outcome values.

In relation to the example in Listing~\ref{listing:keane} above, the
following modifications could be made to generate and inspect
probability estimates. First, after the line ``\texttt{parms.quiet =
  1}'' (9 lines from the foot of the script) and replacing the line
``\texttt{yhat\_svm = svm(All, parms)}'', insert:
%
\begin{code}
parms.probability = 1
bundle bprob = defbundle()
yhat_svm = svm(All, parms, null, &bprob)
\end{code}
Then, to inspect the probabilities for the testing data, one could
append to the script:
%
\begin{code}
matrix P = bprob.Ptest
print P
\end{code}
%
We show below the first few lines of output from this variant of
the script.
%
\begin{code}
           1            2            3
     0.17717      0.60506      0.21777
    0.063540      0.23416      0.70230
    0.098154      0.64375      0.25810
    0.094531      0.36033      0.54514
\end{code}
%
The columns are ordered by ascending value of numerical coding of the
dependent variable. So, for example, we see that the model estimates a
probability of 0.605 for an outcome of 2 (``at home'') for the first
observation in the training set, and a probability of 0.702 for
outcome 3 (``in work'') for the second observation.

\section{Implementation}
\label{sec:implement}

SVM support is implemented via a gretl ``plugin'' module. This module
is supplied in our packages for MS Windows and macOS; it should be
available in the gretl packages prepared by Linux distributions for
gretl 2019a and higher.

For anyone building gretl from the git sources, all the required files
are included. (That is, it is not necessary to have \textsf{libsvm}
installed in its own right.)  The main files are these:
\begin{verbatim}
plugin/svm.c
plugin/libsvm/svmlib.cpp
plugin/libsvm/svmlib.h
\end{verbatim}

\texttt{svm.c} contains ``driver'' code and implements the
gretl-specific options for our \texttt{svm} function. The other two
files are derived from the \textsf{libsvm} package (version 3.23,
dated 2018-07-15).

\texttt{svmlib.cpp} is a slightly modified version of the
\textsf{libsvm} C++ source file \texttt{svm.cpp}. It has been edited
to support parallelization via \textsf{OpenMP} when the symbol
\texttt{\_OPENMP} is defined. The changes are as described in answer
to ``How can I use OpenMP to parallelize LIBSVM on a
multicore/shared-memory computer?'' in the file \texttt{FAQ.html} in
the \textsf{libsvm} distribution. In addition, the pseudo-random
numbers used in cross validation are taken from libgretl's
\textsf{SFMT} (Mersenne Twister) instead of the C library's
\texttt{rand}.

\texttt{svmlib.h} is just a renamed version of the \textsf{libsvm}
header file \texttt{svm.h}.

\bibliographystyle{gretl}
\bibliography{gretl}

\end{document}
