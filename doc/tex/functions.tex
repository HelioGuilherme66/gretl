\chapter{User-defined functions}
\label{chap:functions}

\section{Defining a function}
\label{func-define}

Gretl offers a mechanism for defining functions, which may be
called via the command line, in the context of a script, or (if
packaged appropriately, see section~\ref{sec:func-packages}) via the
program's graphical interface.

The syntax for defining a function looks like this:

\begin{raggedright}
\texttt{function} \textsl{type} \textsl{funcname}
\texttt{(}\textsl{parameters}\texttt{)} \\
\qquad  \textsl{function body} \\
\texttt{end function}
\end{raggedright}

The opening line of a function definition contains these elements, in
strict order:

\begin{enumerate}
\item The keyword \texttt{function}.
\item \textsl{type}, which states the type of value returned by the
  function, if any.  This must be one of \texttt{void} (if the
  function does not return anything), \texttt{scalar},
  \texttt{series}, \texttt{matrix}, \texttt{list}, \texttt{string},
  \texttt{bundle} or one of gretl's array types, \texttt{matrices},
  \texttt{bundles}, \texttt{strings} (see section~\ref{sec:arrays}).
\item \textsl{funcname}, the unique identifier for the function.
  Function names have a maximum length of 31 characters; they must
  start with a letter and can contain only letters, numerals and the
  underscore character.  You will get an error if you try to define a
  function having the same name as an existing gretl command.
\item The function's \textsl{parameters}, in the form of a
  comma-separated list enclosed in parentheses.  This may be run into
  the function name, or separated by white space as shown.  In case
  the function takes no arguments (unusual, but acceptable) this
  should be indicated by placing the keyword \texttt{void} between
  the parameter-list parentheses.
\end{enumerate}

Function parameters can be of any of the types shown
below.\footnote{An additional parameter type is available for GUI use,
  namely \texttt{obs}; this is equivalent to \texttt{int} except for
  the way it is represented in the graphical interface for calling a
  function.}

\begin{center}
\begin{tabular}{ll}
\multicolumn{1}{c}{Type} & 
\multicolumn{1}{c}{Description} \\ [4pt]
\texttt{bool}   & scalar variable acting as a Boolean switch \\
\texttt{int}    & scalar variable acting as an integer  \\
\texttt{scalar} & scalar variable \\
\texttt{series} & data series \\
\texttt{list}   & named list of series \\
\texttt{matrix} & matrix or vector \\
\texttt{string} & string variable or string literal \\
\texttt{bundle} & all-purpose container (see
                  section~\ref{sec:Bundles}) \\
\texttt{matrices} & array of matrices (see
                    section~\ref{sec:arrays}) \\
\texttt{bundles}  & array of bundles \\
\texttt{strings}  & array of strings
\end{tabular}
\end{center}

Each element in the listing of parameters must include two terms: a
type specifier, and the name by which the parameter shall be known
within the function.  An example follows:
%    
\begin{code}
function scalar myfunc (series y, list xvars, bool verbose)
\end{code}

Each of the type-specifiers, with the exception of \texttt{list} and
\texttt{string}, may be modified by prepending an asterisk to the
associated parameter name, as in
%    
\begin{code}
function scalar myfunc (series *y, scalar *b)
\end{code}

The meaning of this modification is explained below (see section
\ref{sec:funscope}); it is related to the use of pointer arguments in
the C programming language.

\subsection{Function parameters: optional refinements}

Besides the required elements mentioned above, the specification of a
function parameter may include some additional fields, as follows:
\begin{itemize}
\item The \texttt{const} modifier.
\item For \texttt{scalar} or \texttt{int} parameters: minimum, maximum
  and/or default values; or for \texttt{bool} parameters, just a
  default value.
\item For optional arguments other than \texttt{scalar}, \texttt{int}
  and \texttt{bool}, the special default value \texttt{null}.
\item For all parameters, a descriptive string.
\item For \texttt{int} parameters with minimum and maximum values
  specified, a set of strings to associate with the allowed numerical
  values (value labels).
\end{itemize}

The first three of these options may be useful in many contexts; the
last two may be helpful if a function is to be packaged for use in
the gretl GUI (but probably not otherwise). We now expand on
each of the options.

\begin{itemize}

\item The \texttt{const} modifier: must be given as a prefix to the
  basic parameter specification, as in
%    
\begin{code}
const matrix M
\end{code} 
%
  This constitutes a promise that the corresponding argument will not
  be modified within the function; gretl will flag an error if the
  function attempts to modify the argument.

\item Minimum, maximum and default values for \texttt{scalar} or
  \texttt{int} types: These values should directly follow the name of
  the parameter, enclosed in square brackets and with the individual
  elements separated by colons.  For example, suppose we have an
  integer parameter \texttt{order} for which we wish to specify a
  minimum of 1, a maximum of 12, and a default of 4.  We can write
%    
\begin{code}
int order[1:12:4]
\end{code} 
%
  If you wish to omit any of the three specifiers, leave the
  corresponding field empty.  For example \texttt{[1::4]} would
  specify a minimum of 1 and a default of 4 while leaving the maximum
  unlimited. However, as a special case, it is acceptable to give just
  one value, with no colons, in which case the value is interpreted as
  a default. So for example
%    
\begin{code}
int k[0]
\end{code}
%
  designates a default value of 0 for the parameter \texttt{k}, with
  no minimum or maximum specified. If you wished to specify a minimum
  of zero with no maximum or default you would have to write
%    
\begin{code}
int k[0::]
\end{code}

  For a parameter of type \texttt{bool} (whose values are just zero or
  non-zero), you can specify a default of 1 (true) or 0 (false), as in
%    
\begin{code}
bool verbose[0]
\end{code} 

\item Descriptive string: This will show up as an aid to the user if
  the function is packaged (see section~\ref{sec:func-packages} below)
  and called via gretl's graphical interface.  The string should
  be enclosed in double quotes and separated from the preceding
  elements of the parameter specification with a space, as in
%
\begin{code}
series y "dependent variable"
\end{code} 

\item Value labels: These may be used only with \texttt{int}
  parameters for which minimum and maximum values have been specified
  (so that there is a fixed number of admissible values) and the
  number of labels must match the number of values. They will show up
  in the graphical interface in the form of a drop-down list, making
  the function writer's intent clearer when an integer argument
  represents a categorical selection. A set of value labels must be
  enclosed in braces, and the individual labels must be enclosed in
  double quotes and separated by commas or spaces.  For example:
%
\begin{code}
int case[1:3:1] {"Fixed effects", "Between model", "Random effects"}
\end{code} 

\end{itemize}

If two or more of the trailing optional fields are given in a
parameter specification, they must be given in the order shown above:
min/max/default, description, value labels. Note that there is no
facility for ``escaping'' characters within descriptive strings or
value labels; these may contain spaces but they cannot contain the
double-quote character.

Here is an example of a well-formed function specification using all
the elements mentioned above:
%
\begin{code}
function matrix myfunc (series y "dependent variable",
                        list X "regressors",
                        int p[0::1] "lag order",
                        int c[1:2:1] "criterion" {"AIC", "BIC"},
                        bool quiet[0])
\end{code} 

One advantage of specifying default values for parameters, where
applicable, is that in script or command-line mode users may omit
trailing arguments that have defaults. For example, \texttt{myfunc}
above could be invoked with just two arguments, corresponding to
\texttt{y} and \texttt{X}; implicitly \texttt{p} = 1, \texttt{c} = 1
and \texttt{quiet} is false.

\subsection{Functions taking no parameters}

You may define a function that has no parameters (these are called
``routines'' in some programming languages).  In this case,  
use the keyword \texttt{void} in place of the listing of parameters:
%    
\begin{code}
function matrix myfunc2 (void)
\end{code}


\subsection{The function body}
   
The \textsl{function body} is composed of gretl commands, or
calls to user-defined functions (that is, function calls may be
nested).  A function may call itself (that is, functions may be
recursive). While the function body may contain function calls, it may
not contain function definitions.  That is, you cannot define a
function inside another function.  For further details, see
section~\ref{func-details}.


\section{Calling a function}
\label{func-call}

A user function is called by typing its name followed by zero or more
arguments enclosed in parentheses.  If there are two or more arguments
they must be separated by commas.

There are automatic checks in place to ensure that the number of
arguments given in a function call matches the number of parameters,
and that the types of the given arguments match the types specified in
the definition of the function.  An error is flagged if either of
these conditions is violated.  One qualification: allowance is made
for omitting arguments at the end of the list, provided that default
values are specified in the function definition.  To be precise, the
check is that the number of arguments is at least equal to the number
of \textit{required} parameters, and is no greater than the total
number of parameters.

In general, an argument to a function may be given either as the name
of a pre-existing variable or as an expression which evaluates to a
variable of the appropriate type.

The following trivial example illustrates a function call that
correctly matches the corresponding function definition.
    
\begin{code}
# function definition
function scalar ols_ess (series y, list xvars)
  ols y 0 xvars --quiet
  printf "ESS = %g\n", $ess
  return $ess
end function

# main script
open data4-1
list xlist = 2 3 4
# function call (the return value is ignored here)
ols_ess(price, xlist)
\end{code}
%$
The function call gives two arguments: the first is a data series
specified by name and the second is a named list of regressors.  Note
that while the function offers the Error Sum of Squares as a return
value, it is ignored by the caller in this instance.  (As a side note
here, if you want a function to calculate some value having to do with
a regression, but are not interested in the full results of the
regression, you may wish to use the \option{quiet} flag with the
estimation command as shown above.)
    
A second example shows how to write a function call that assigns
a return value to a variable in the caller:
    
\begin{code}
# function definition
function series get_uhat (series y, list xvars)
  ols y 0 xvars --quiet
  return $uhat
end function

# main script
open data4-1
list xlist = 2 3 4
# function call
series resid = get_uhat(price, xlist)
\end{code}
%$
\section{Deleting a function}
\label{func-del}

If you have defined a function and subsequently wish to clear it out
of memory, you can do so using the keywords \texttt{delete} or
\texttt{clear}, as in

\begin{code}
function myfunc delete
function get_uhat clear
\end{code}

Note, however, that if \texttt{myfunc} is already a defined function,
providing a new definition automatically overwrites the previous one,
so it should rarely be necessary to delete functions explicitly.

\section{Function programming details}
\label{func-details}

\subsection{Variables versus pointers}
\label{sec:funscope}

Most arguments to functions can be passed in two ways: ``as they
are'', or via pointers (the exception is the list type, which cannot
be passed as a pointer). First consider the following rather
artificial example:
\begin{code}
function series triple1 (series x)
  return 3*x
end function
  
function void triple2 (series *x)
  x *= 3
end function

nulldata 10
series y = normal()
series y3 = triple1(y)
print y3
triple2(&y)
print y
\end{code}

These two functions produce essentially the same result---the two
\texttt{print} statements in the caller will show the same
values---but in quite different ways. The first explicitly returns a
modified version of its input (which must be a plain series): after
the call to \texttt{triple1}, \texttt{y} is unaltered; it would have
been altered only if the return value were assigned back to \texttt{y}
rather than \texttt{y3}. The second function modifies its input (given
as a pointer to a series) in place without actually \textit{returning}
anything.

It's worth noting that \texttt{triple2} as it stands would
\textit{not} be considered idiomatic as a gretl function (although
it's formally OK). The point here is just to illustrate the
distinction between passing an argument in the default way and in
pointer form.

Why make this distinction? There are two main reasons for doing so:
modularity and performance.

By modularity we mean the insulation of a function from the rest of
the script which calls it.  One of the many benefits of this approach
is that your functions are easily reusable in other contexts.  To
achieve modularity, \emph{variables created within a function are
  local to that function, and are destroyed when the function exits},
unless they are made available as return values and these values are
``picked up'' or assigned by the caller.  In addition, functions do
not have access to variables in ``outer scope'' (that is, variables
that exist in the script from which the function is called) except
insofar as these are explicitly passed to the function as arguments.

By default, when a variable is passed to a function as an argument,
what the function actually ``gets'' is a \emph{copy} of the outer
variable, which means that the value of the outer variable is not
modified by anything that goes on inside the function. This means that
you can pass arguments to a function without worrying about possible
side effects; at the same time the function writer can use argument
variables as workspace without fear of disruptive effects at the level
of the caller.

The use of pointers, however, allows a function and its caller to
cooperate such that an outer variable \textit{can} be modified by the
function.  In effect, this allows a function to ``return'' more than
one value (although only one variable can be returned directly---see
below).  To indicate that a particular object is to be passed as a
pointer, the parameter in question is marked with a prefix of
\texttt{*} in the function definition, and the corresponding argument
is marked with the complementary prefix \verb+&+ in the caller.  For
example,
%
\begin{code}
function series get_uhat_and_ess(series y, list xvars, scalar *ess)
  ols y 0 xvars --quiet
  ess = $ess
  series uh = $uhat
  return uh
end function

open data4-1
list xlist = 2 3 4
scalar SSR
series resid = get_uhat_and_ess(price, xlist, &SSR)
\end{code}
%
In the above, we may say that the function is given the \emph{address}
of the scalar variable \texttt{SSR}, and it assigns a value to that
variable (under the local name \texttt{ess}).  (For anyone used to
programming in \textsf{C}: note that it is not necessary, or even
possible, to ``dereference'' the variable in question within the
function using the \texttt{*} operator.  Unadorned use of the name of
the variable is sufficient to access the variable in outer scope.)

An ``address'' parameter of this sort can be used as a means of
offering optional information to the caller.  (That is, the
corresponding argument is not strictly needed, but will be used if
present).  In that case the parameter should be given a default value
of \texttt{null} and the the function should test to see if the caller
supplied a corresponding argument or not, using the built-in function
\texttt{exists()}.  For example, here is the simple function shown
above, modified to make the filling out of the \texttt{ess} value
optional.
%
\begin{code}
function series get_uhat_and_ess(series y, list xvars, scalar *ess[null])
  ols y 0 xvars --quiet
  if exists(ess) 
     ess = $ess
  endif
  return $uhat
end function
\end{code}
%
If the caller does not care to get the \texttt{ess} value, it can
use \texttt{null} in place of a real argument:
%
\begin{code}
series resid = get_uhat_and_ess(price, xlist, null)
\end{code}
%
Alternatively, trailing function arguments that have default values 
may be omitted, so the following would also be a valid call:
%
\begin{code}
series resid = get_uhat_and_ess(price, xlist)
\end{code}

One limitation on the use of pointer-type arguments should be noted:
you cannot supply a given variable as a pointer argument more than
once in any given function call. For example, suppose we have a
function that takes two matrix-pointer arguments,
\begin{code}
function scalar pointfunc (matrix *a, matrix *b)
\end{code}
And suppose we have two matrices, \texttt{x} and \texttt{y}, at the
caller level.  The call
\begin{code}
pointfunc(&x, &y)
\end{code}
is OK, but the call
\begin{code}
pointfunc(&x, &x) # will not work
\end{code}
will generate an error. That's because the situation inside the
function would become too confusing, with what is really the same
object existing under two names.

\subsection{Const parameters}
\label{sec:const-args}

Pointer-type arguments may also be useful for optimizing performance.
Even if a variable is not modified inside the function, it may be a
good idea to pass it as a pointer if it occupies a lot of
memory. Otherwise, the time gretl spends transcribing the value of the
variable to the local copy may be non-negligible compared to the time
the function spends doing the job it was written for.

Listing \ref{ex:perf-pointers} takes this to the extreme.  We define
two functions which return the number of rows of a matrix (a pretty
fast operation).  The first gets a matrix as argument while the second
gets a pointer to a matrix.  The functions are evaluated 500 times on
a matrix with 2000 rows and 2000 columns; on a typical system
floating-point numbers take 8 bytes of memory, so the total size of
the matrix is roughly 32 megabytes.

Running the code in example \ref{ex:perf-pointers} will produce output
similar to the following (the actual numbers of course depend on the
machine you're using):
\begin{code}
Elapsed time:
without pointers (copy) = 2.47197 seconds,
with pointers (no copy) = 0.00378627 seconds
\end{code}

\begin{script}[htbp]
  \caption{Performance comparison: values versus pointer}
  \label{ex:perf-pointers}
  \begin{scode}
function scalar rowcount1 (matrix X)
  return rows(X)
end function

function scalar rowcount2 (const matrix *X)
  return rows(X)
end function

set verbose off
X = zeros(2000,2000)
scalar r

set stopwatch
loop 500
  r = rowcount1(X)
endloop
e1 = $stopwatch

set stopwatch
loop 500
  r = rowcount2(&X)
endloop
e2 = $stopwatch

printf "Elapsed time:\n\
without pointers (copy) = %g seconds,\n \
with pointers (no copy) = %g seconds\n", e1, e2
\end{scode}
%$
\end{script}

If a pointer argument is used for this sort of purpose---and the
object to which the pointer points is not modified (is treated as
read-only) by the function---one can signal this to the user by adding
the \texttt{const} qualifier, as shown for function \texttt{rowcount2}
in Listing~\ref{ex:perf-pointers}.  When a pointer argument is
qualified in this way, any attempt to modify the object within the
function will generate an error.

However, combining the \texttt{const} flag with the pointer mechanism
is technically redundant for the following reason: if you mark a
matrix argument as \texttt{const} then gretl will in fact pass it in
pointer mode internally (since it can't be modified within the
function there's no downside to simply making it available to the
function rather than copying it). So in the example above we could
revise the signature of the second function as
\begin{code}
function scalar rowcount2a (const matrix X)
\end{code}
and call it with \texttt{r = rowcount2a(X)}, for the same speed-up
relative to \texttt{rowcount1}.

From the caller's point of view the second option---using the
\texttt{const} modifier \textit{without} pointer notation---is
preferable, as it allows the caller to pass an object created ``on the
fly''. Suppose the caller has two matrices, \texttt{A} and \texttt{B},
in scope, and wishes to pass their vertical concatenation as an
argument. The following call would work fine:
\begin{code}
r = rowcount2a(A|B)
\end{code}
To use \texttt{rowcount2}, on the other hand, the caller would have to
create a named variable first (since you cannot give the ``address''
of a anonymous object such as \verb+A|B+):
\begin{code}
matrix AB = A|B
r = rowcount2(&AB)
\end{code}
This requires an extra line of code, and leaves \texttt{AB} occupying
memory after the call.

We have illustrated using a matrix parameter, but the \texttt{const}
modifier may be used with the same effect---namely, the argument is
passed directly, without being copied, but is protected against
modification within the function---for all the types that support
the pointer apparatus.

\subsection{List arguments}

The use of a named list as an argument to a function gives a means of
supplying a function with a set of variables whose number is unknown
when the function is written---for example, sets of regressors or
instruments.  Within the function, the list can be passed on to
commands such as \texttt{ols}.

A list argument can also be ``unpacked'' using a \texttt{foreach} loop
construct, but this requires some care.  For example, suppose you have
a list \texttt{X} and want to calculate the standard deviation of each
variable in the list.  You can do:
%
\begin{code}
loop foreach i X
   scalar sd_$i = sd(X.$i)
endloop
\end{code}

\textit{Please note}: a special piece of syntax is needed in this
context.  If we wanted to perform the above task on a list in a
regular script (not inside a function), we could do
%
\begin{code}
loop foreach i X
   scalar sd_$i = sd($i)
endloop
\end{code}
%
where \dollar{i} gets the name of the variable at position $i$ in the
list, and \verb|sd($i)| gets its standard deviation.  But inside a
function, working on a list supplied as an argument, if we want to
reference an individual variable in the list we must use the syntax
\textsl{listname.varname}.  Hence in the example above we write
\verb|sd(X.$i)|.

This is necessary to avoid possible collisions between the name-space
of the function and the name-space of the caller script.  For example,
suppose we have a function that takes a list argument, and that
defines a local variable called \texttt{y}.  Now suppose that this
function is passed a list containing a variable named \texttt{y}.  If
the two name-spaces were not separated either we'd get an error, or
the external variable \texttt{y} would be silently over-written by the
local one.  It is important, therefore, that list-argument variables
should not be ``visible'' by name within functions.  To ``get hold
of'' such variables you need to use the form of identification just
mentioned: the name of the list, followed by a dot, followed by the
name of the variable.

\paragraph{Constancy of list arguments} When a named list of
variables is passed to a function, the function is actually provided
with a copy of the list.  The function may modify this copy (for
instance, adding or removing members), but the original list at the
level of the caller is not modified.

\paragraph{Optional list arguments} If a list argument to a function is
optional, this should be indicated by appending a default value of
\texttt{null}, as in
%
\begin{code}
function scalar myfunc (scalar y, list X[null])
\end{code}
%
In that case, if the caller gives \texttt{null} as the list argument
(or simply omits the last argument) the named list \texttt{X} inside the
function will be empty.  This possibility can be detected using the
\texttt{nelem()} function, which returns 0 for an empty list.

\subsection{String arguments}

String arguments can be used, for example, to provide flexibility in
the naming of variables created within a function.  In the following
example the function \texttt{mavg} returns a list containing two
moving averages constructed from an input series, with the names of
the newly created variables governed by the string argument.
%
\begin{code}
function list mavg (series y, string vname)
   list retlist = null
   string newname = sprintf("%s_2", vname)
   retlist += genseries(newname, (y+y(-1)) / 2)
   newname = sprintf("%s_4", vname)
   retlist += genseries(newname, (y+y(-1)+y(-2)+y(-3)) / 4)
   return retlist
end function

open data9-9
list malist = mavg(nocars, "nocars")
print malist --byobs
\end{code}
%
The last line of the script will print two variables named
\verb|nocars_2| and \verb|nocars_4|.  For details on the handling of
named strings, see chapter~\ref{chap:lists-strings}.

If a string argument is considered optional, it may be given a
\texttt{null} default value, as in
%
\begin{code}
function scalar foo (series y, string vname[null])
\end{code}

\subsection{Retrieving the names of arguments}

The variables given as arguments to a function are known inside the
function by the names of the corresponding parameters.  For example,
within the function whose signature is
%
\begin{code}
function void somefun (series y)
\end{code}
%
we have the series known as \texttt{y}.  It may be useful, however, to
be able to determine the names of the variables provided as arguments.
This can be done using the function \texttt{argname}, which takes the
name of a function parameter as its single argument and returns a
string.  Here is a simple illustration:
%
\begin{code}
function void namefun (series y)
  printf "the series given as 'y' was named %s\n", argname(y)
end function

open data9-7
namefun(QNC)
\end{code}
%
This produces the output
%
\begin{code}
the series given as 'y' was named QNC
\end{code}

Please note that this will not always work: the arguments given
to functions may be anonymous variables, created on the fly, as in
\texttt{somefun(log(QNC))} or \texttt{somefun(CPI/100)}.  In that case
the \textsf{argname} function returns an empty string.  Function
writers who wish to make use of this facility should check the return
from \texttt{argname} using the \texttt{strlen()} function: if this
returns 0, no name was found.

\subsection{Return values}

Functions can return nothing (just printing a result, perhaps), or
they can return a single variable.  The return value, if any, is
specified via a statement within the function body beginning with the
keyword \texttt{return}, followed by either the name of a variable
(which must be of the type announced on the first line of the function
definition) or an expression which produces a value of the correct
type.

Having a function return a list or bundle is a way of permitting the
``return'' of more than one variable.  For example, you can define
several series inside a function and package them as a list; in this
case they are not destroyed when the function exits.  Here is a simple
example, which also illustrates the possibility of setting the
descriptive labels for variables generated in a function.
%    
\begin{code}
function list make_cubes (list xlist)
   list cubes = null
   loop foreach i xlist
      series $i3 = (xlist.$i)^3
      setinfo $i3 -d "cube of $i"
      list cubes += $i3
   endloop
   return cubes
end function

open data4-1
list xlist = price sqft
list cubelist = make_cubes(xlist)
print xlist cubelist --byobs
labels
\end{code}
%$

A \texttt{return} statement causes the function to return (exit) at
the point where it appears within the body of the function.  A
function may also exit when (a) the end of the function code is
reached (in the case of a function with no return value), (b) a
gretl error occurs, or (c) a \verb+funcerr+ statement is
reached.

The \verb+funcerr+ keyword---which may be followed by a string
enclosed in double quotes, or the name of a string variable, or
nothing---causes a function to exit with an error flagged.  If a
string is provided (either literally or via a variable), this is
printed on exit, otherwise a generic error message is printed.  This
mechanism enables the author of a function to pre-empt an ordinary
execution error and/or offer a more specific and helpful error
message.  For example,
%
\begin{code}
if nelem(xlist) == 0
   funcerr "xlist must not be empty"
endif
\end{code}

A function may contain more than one \texttt{return} statement, as in
%
\begin{code}
function scalar multi (bool s)
   if s
      return 1000
   else
      return 10
   endif
end function
\end{code}
%
However, it is recommended programming practice to have a single
return point from a function unless this is very inconvenient.  The
simple example above would be better written as
%
\begin{code}
function scalar multi (bool s)
   return s ? 1000 : 10
end function
\end{code}
    

\subsection{Error checking}

When gretl first reads and ``compiles'' a function definition there is
minimal error-checking: the only checks are that the function name is
acceptable, and, so far as the body is concerned, that you are not
trying to define a function inside a function (see Section
\ref{func-define}). Otherwise, if the function body contains invalid
commands this will become apparent only when the function is called
and its commands are executed.

\subsection{Debugging}

The usual mechanism whereby gretl echoes commands and reports on
the creation of new variables is by default suppressed when a function
is being executed.  If you want more verbose output from a particular
function you can use either or both of the following commands within
the function:
%
\begin{code}
set echo on
set messages on
\end{code}

Alternatively, you can achieve this effect for all functions via
the command \texttt{set debug 1}.  Usually when you set the value of a
state variable using the \texttt{set} command, the effect applies only
to the current level of function execution.  For instance, if you do
\texttt{set messages on} within function \texttt{f1}, which in turn
calls function \texttt{f2}, then messages will be printed for
\texttt{f1} but not \texttt{f2}.  The debug variable, however, acts
globally; all functions become verbose regardless of their level.

Further, you can do \texttt{set debug 2}: in addition to command echo
and the printing of messages, this is equivalent to setting
\verb|max_verbose| (which produces verbose output from the BFGS
maximizer) at all levels of function execution.

\section{Function packages}
\label{sec:func-packages}

At various points above we have alluded to function packages, and the
use of these via the gretl GUI. This topic is covered in depth by the
\PKGBOOK. If you're running gretl you can find this under the
\textsf{Help} menu. Alternatively you may download it from

\url{https://sourceforge.net/projects/gretl/files/manual/}

%%% Local Variables: 
%%% mode: latex
%%% TeX-master: "gretl-guide"
%%% End: 

