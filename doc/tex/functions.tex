\chapter{User-defined functions}
\label{functions}

\section{Defining a function}
\label{func-define}

Since version 1.3.3, \app{gretl} has contained a mechanism for
defining functions in the context of a script.  This functionality has
been through some changes in search of a stable and extensible
framework.  We believe that the version present in \app{gretl} 1.6.1
should provide a good basis for future development.

Functions must be defined before they are called.  The syntax for
defining a function looks like this
    
\begin{code}
      function function-name(parameters)
         function body
      end function
\end{code}

\textsl{function-name} is the unique identifier for the function.
Names must start with a letter. They have a maximum length of 31
characters; if you type a longer name it will be truncated.  Function
names cannot contain spaces.  You will get an error if you try to
define a function having the same name as an existing \app{gretl}
command.

The \textsl{parameters} for a function are given in the form of a
comma-separated list.  Parameters can be of any of the types shown
below.

\begin{center}
\begin{tabular}{ll}
\multicolumn{1}{c}{Type} & 
\multicolumn{1}{c}{Description} \\ [4pt]
\texttt{bool} & scalar variable acting as a Boolean switch \\
\texttt{int}  & scalar variable acting as an integer  \\
\texttt{scalar} & scalar variable \\
\texttt{series} & data series \\
\texttt{list}   & named list of series \\
\texttt{matrix} & named matrix or vector 
\end{tabular}
\end{center}

Each element in the listing of parameters must include two terms: a
type specifier, and the name by which the parameter shall be known
within the function.  An example follows:
%    
\begin{code}
      function myfunc(series y, list xvars, bool verbose)
\end{code}

Besides these required elements, the specification of a
\texttt{scalar} parameter may include up to three pieces of additional
information: a minimum value, a maximum, and a default.  These
additional values should directly follow the name of the parameter,
enclosed in square brackets and with the individual elements separated
by colons.  For example, suppose we have an integer parameter
\texttt{order} for which we wish to specify a minimum of 1, a maximum
of 12, and a default of 4.  We can write
%    
\begin{code}
      int order[1:12:4]
\end{code} 
%
If you wish to omit any of the three specifiers, leave the
corresponding field empty.  For example \texttt{[1::4]} would specify
a minimum of 1 and a default of 4 while leaving the maximum
unlimited.  

For a parameter of type \texttt{bool}, you can specify a default of
1 (true) or 0 (false), as in
%    
\begin{code}
      bool verbose[0]
\end{code} 
%

You may define a function that has no parameters (these are called
``routines'' in some programming languages).  In this case,  
use the keyword \texttt{void} in place of the listing of parameters:
%    
\begin{code}
      function myfunc2(void)
\end{code}

When a function is called, the parameters are instantiated by
arguments given by the caller.  There are automatic checks in place to
ensure that the number of arguments given in a function call matches
the number of parameters, and that the types of the given arguments
match the types specified in the definition of the function.  An error
is flagged if either of these conditions is
violated.\footnote{Allowance is made for omitting arguments at the end
  of the list, provided that default values are specified in the
  function definition.  To be precise, the check is that the number of
  arguments is at least equal to the number of \textit{required}
  parameters, and is no greater than the total number of parameters.}
A scalar, series or matrix argument to a function may be given either
as the name of a pre-existing variable or (with one exception
discussed below) as an expression which evaluates to a variable of the
appropriate type.  Scalar arguments may also be given as numerical
values.  List arguments must be specified by name.
    
The \textsl{function body} is composed of \app{gretl} commands, or
calls to user-defined functions (that is, functions may be nested).  A
function may call itself (that is, functions may be recursive). There
is a maximum ``stacking depth'' for user functions: at present this is
set to 8.  While the function body may contain function calls, it may
not contain function definitions.  That is, you cannot define a
function inside another function.  


\section{Calling a function}
\label{func-call}

A user function is called or invoked by typing its name followed by
zero or more arguments enclosed in parentheses.  If there are two or
more arguments these should be separated by commas.  The following
trivial example illustrates a function call that correctly matches the
function definition.
    
\begin{code}
      # function definition
      function ols_ess(series y, list xvars)
        ols y 0 xvars --quiet
        scalar myess = $ess
        printf "ESS = %g\n", myess
        return scalar myess
      end function
      # main script
      open data4-1
      list xlist = 2 3 4
      # function call (the return value is ignored here)
      ols_ess(price, xlist)
\end{code}

The function call gives two arguments: the first is a data series
specified by name and the second is a named list of regressors.  Note
that while the function offers the variable \verb+myess+ as a return
value, it is ignored by the caller in this instance.  (As a side note
here, if you want a function to calculate some value having to do with
a regression, but are not interested in the full results of the
regression, you may wish to use the \verb+--quiet+ flag with the
estimation command as shown above.)
    
A second example shows how to write a function call that assigns
a return value to a variable in the caller:
    
\begin{code}
      # function definition
      function get_uhat(series y, list xvars)
        ols y 0 xvars --quiet
        series uh = $uhat
        return series uh
      end function
      # main script
      open data4-1
      list xlist = 2 3 4
      # function call
      series resid = get_uhat(price, xlist)
\end{code}

\section{Function programming details}
\label{func-details}

\subsection{Scope of variables}

All variables created within a function are local to that function,
and are destroyed when the function exits, unless they are made
available as return values and these values are ``picked up'' or
assigned by the caller.
    
Functions do not have access to variables in ``outer scope'' (that is,
variables that exist in the script from which the function is called)
except insofar as these are explicitly passed to the function as
arguments.  

In the standard case, when a variable is passed to a function as an
argument, what the function actually ``gets'' is a copy of the outer
variable, which means that the value of the outer variable is not
modified by whatever goes on inside the function.  There is, however,
a mechanism for allowing a function and its caller to ``cooperate''
such that an outer variable can be modified by the function.  In
effect, this allows a function to ``return'' more than one value
(although only one variable can be returned directly --- see below).
The method is (at least superficially) similar to the passing of the
address of a variable in the C programming language.  The parameter in
question is marked with a prefix of \texttt{*} in the function
definition, and the corresponding argument is marked with the
complementary prefix \verb+&+ in the caller.  For example,
%
\begin{code}
      function get_uhat_and_ess(series y, list xvars, scalar *ess)
        ols y 0 xvars --quiet
        ess = $ess
        series uh = $uhat
        return series uh
      end function
      # main script
      open data4-1
      list xlist = 2 3 4
      # function call
      scalar SSR
      series resid = get_uhat_and_ess(price, xlist, &SSR)
\end{code}
%
In the above, we may say that the function is given the address of
the scalar variable \texttt{SSR}, and it assigns a value to that
variable (under the name \texttt{ess}).  For anyone used to
programming in C, note that it is not necessary (or possible) to
``dereference'' the variable in question within the function using the 
\texttt{*} operator.  Unembellished use of the name of the variable is
sufficient to access the variable in outer scope.

The specification of one or more such special parameters in the
definition of a function may be used to offer optional information to
the caller.  If the ``address'' parameter is optional, this should be
marked by providing a default value of \texttt{null}.  In that case
the function should test to see if the caller supplied a corresponding
argument or not, using the built-in function \texttt{isnull()}.  For
example, here is the simple function shown above, modified to make the
filling out of the \texttt{ess} value optional.
%
\begin{code}
      function get_uhat_and_ess(series y, list xvars, scalar *ess[null])
        ols y 0 xvars --quiet
        if !isnull(ess) 
           ess = $ess
        endif
        series uh = $uhat
        return series uh
      end function
\end{code}
%
If the caller does not care to get the \texttt{ess} value, it should
use \texttt{null} in place of a real argument:
%
\begin{code}
      series resid = get_uhat_and_ess(price, xlist, null)
\end{code}


\subsection{Return values}

Functions can return nothing (just printing a result, perhaps), or
they can return a single variable --- a scalar, series or matrix.  The
return value, if any, is specified via a statement within the function
body beginning with the keyword \verb+return+, followed by type
specifier and the name of a variable (as in the listing of
parameters).  There can be only one such statement.  An example of a
valid return statement is shown below:
%    
\begin{code}
      return scalar SSR
\end{code}
%
Note that the \verb+return+ statement does \emph{not} cause the
function to return (exit) at the point where it appears within the
body of the function. Rather, it specifies which variable is
available for assignment when the function exits, and a function exits
only when (a) the end of the function code is reached, (b) a
\verb+funcerr+ statement is reached (see below), or (c) a gretl error
occurs.
    
The \verb+funcerr+ keyword, which may be followed by a string enclosed
in double quotes, causes a function to exit with an error flagged.  If
a string is provided, this is printed on exit otherwise a generic
error message is printed.
    

\subsection{Error checking}

When gretl first reads and ``compiles'' a function definition there is
minimal error-checking: the only checks are that the function name is
acceptable, and, so far as the body is concerned, that you are not
trying to define a function inside a function (see Section
\ref{func-define}). Otherwise, if the function body contains invalid
commands this will become apparent only when the function is called,
and its commands are executed.

\section{Function packages}
\label{func-packages}

As of \app{gretl} 1.6.0, there is a mechanism to package functions and
make them available to other users of \app{gretl}.  This is currently
experimental, but here is a walk-through of the process.

Start the GUI program and take a look at the ``File, Function files'' menu.
This menu contains four items: ``On local machine'', ``On server'', ``Edit
package'', ``New package''.

The ``New package'' command will return an error message, unless at least one
user-defined function is currently loaded in memory.

There are several ways to load a function:

\begin{itemize}
\item If you have a script file containing function definitions, open
  that file and run it.
\item Create a script file from scratch.  Include at least one
  function definition, and run the script.
\item Open the GUI console and type a function definition
  interactively.  This method is not particularly recommended; you are
  probably better composing a function non-interactively.
\end{itemize}

After loading a function, try again using the command ``File, Function
files, New package''. In the first dialog you get to select:

\begin{itemize}
\item A public function to package.
\item Zero or more ``private'' helper functions.
\end{itemize}

Public functions are directly available to users; private functions are
part of the ``behind the scenes'' mechanism in a function package.

On clicking ``OK'' a second dialog should appear, where you get to
enter the package information (currently, author, version, date, and a
short description).  You also get to enter help text for the public
interface.  You have a further chance to edit the code of the functions
to be packaged, by selecting them from the drop-down selector and
clicking on ``Edit function code''. Finally, you can choose to upload
the package on gretl's server as soon as it is saved, by checking the
relevant checkbox.

Clicking ``OK'' in this dialog leads you to a File Save dialog.  All
being well, this should be pointing towards a directory named
\texttt{functions}, either under the \app{gretl} system directory (if
you have write permission on that) or the \app{gretl} user directory.
This is the recommended place to save function package files, since
that is where the program will look in the special routine for opening
such files (see below).

Needless to say, the menu command ``File, Function files, Edit package''
allows you to edit again a local function package.

\vspace{6pt}

A word on the file you just saved.  By default, it will have a
\texttt{.gfn} extension.  This is a ``function package'' file: unlike
an ordinary \app{gretl} script file, it is an XML file containing both
the function code and the extra information entered in the packager.
Hackers might wish to write such a file from scratch rather than using
the GUI packager, but most people are likely to find it awkward.  Note
that XML-special characters in the function code have to be escaped,
e.g.\ \texttt{\&} must be represented as \texttt{\&amp;}.  Also, some
elements of the function syntax differ from the standard script
representation: the parameters and return values (if any) are
represented in XML.  Basically, the function is pre-parsed, and ready
for fast loading using \textsf{libxml}.

\vspace{6pt}

Why package functions in this way?  To see what's on offer so far, try
this second phase of the walk-through.

Close gretl, then re-open it.  Now go to ``File, Function files, On
local machine''. If the first stage above has gone OK, you should
see the file you packaged and saved, with its short description.  If
you click on ``Info'' you get a window with all the information gretl
has gleaned from the function package.  If you click on the ``View
code'' icon in the toolbar of this new window, you get a script view
window showing the actual function code. Now, back to the ``Function
packages'' window, if you click on the package's name, the functions
are loaded into gretl, ready to be called by clicking on the ``Call''
button, or by using the CLI.

After loading the function(s) from the package, open the GUI console.
Try typing \texttt{help foo}, replacing \texttt{foo} with the name of
a public interface from the loaded function package: if any help text
was provided for the function, it should be presented.

In a similar way, you can browse and load the functions packages
available on gretl's server, by selecting ``File, Function files, On
server''.

%%% Local Variables: 
%%% mode: latex
%%% TeX-master: "gretl-guide"
%%% End: 

