\chapter{Loop constructs}
\label{looping}

\section{Introduction}
\label{loop-intro}

The command \cmd{loop} opens a special mode in which \app{gretl}
accepts a block of commands to be repeated one or more times.  This
feature may be useful for, among other things, Monte Carlo simulations,
bootstrapping of test statistics and iterative estimation procedures.
The general form of a loop is:

\begin{code}
      loop control-expression [ --progressive | --verbose | --quiet ]
         loop body
      endloop
\end{code}

Five forms of control-expression are available, as explained in
section~\ref{loop-control}.

Not all \app{gretl} commands are available within loops.  The commands
that are accepted in this context are shown in
Table~\ref{tab:loopcmds}.

\begin{table}[htbp]
\caption{Commands usable in loops}
\label{tab:loopcmds}
\begin{center}
%% The following is generated automatically
\input tabloopcmds.tex
\end{center}
\end{table}

By default, the \cmd{genr} command operates quietly in the context of
a loop (without printing information on the variable generated).  To
force the printing of feedback from \cmd{genr} you may specify the
\verb+--verbose+ option to \cmd{loop}.  The \verb+--quiet+ option
suppresses the usual printout of the number of iterations performed,
which may be desirable when loops are nested.

The \verb+--progressive+ option to \cmd{loop} modifies the behavior of
the commands \cmd{ols}, \cmd{print} and \cmd{store} in a manner that
may be useful with Monte Carlo analyses (see Section
\ref{loop-progressive}).
    
The following sections explain the various forms of the loop control
expression and provide some examples of use of loops.  

\tip{If you are carrying out a substantial Monte Carlo analysis with
  many thousands of repetitions, memory capacity and processing time
  may be an issue.  To minimize the use of computer resources, run
  your script using the command-line program, \app{gretlcli}, with
  output redirected to a file.}

\section{Loop control variants}
\label{loop-control}

\subsection{Count loop}
\label{loop-count}

The simplest form of loop control is a direct specification of the
number of times the loop should be repeated.  We refer to this as a
``count loop''.  The number of repetitions may be a numerical
constant, as in \verb+loop 1000+, or may be read from a variable, as
in \verb+loop replics+.

In the case where the loop count is given by a variable, say
\verb+replics+, in concept \verb+replics+ is an integer scalar.  If it
is in fact a series, its first value is read.  If the value is not
integral, it is converted to an integer by truncation.  Note that
\verb+replics+ is evaluated only once, when the loop is initially
compiled.
      

\subsection{While loop}
\label{loop-while}

A second sort of control expression takes the form of the keyword
\cmd{while} followed by an inequality: the left-hand term should be
the name of a predefined variable; the right-hand side may be either a
numerical constant or the name of another predefined variable.  For
example, 

\cmd{loop while essdiff > .00001} 

Execution of the commands within the loop will continue so long as the
specified condition evaluates as true. If the right-hand term of the
inequality is a variable, it is evaluated at the top of the loop at
each iteration.

\subsection{Index loop}
\label{loop-index}

A third form of loop control uses the special internal index variable
\verb+i+.  In this case you specify starting and ending values for
\verb+i+, which is incremented by one each time round the loop.  The
syntax looks like this: \cmd{loop i=1..20}.

The index variable may be used within the loop body in one or both of
two ways: you can access the value of \verb+i+ (see Example
\ref{loop-panel-script}) or you can use its string representation,
\verb+$i+ (see Example \ref{loop-string-script}).

The starting and ending values for the index can be given in numerical
form, or by reference to predefined variables.  In the latter case the
variables are evaluated once, when the loop is set up.  In addition,
with time series data you can give the starting and ending values in
the form of dates, as in \cmd{loop i=1950:1..1999:4}.
      

\subsection{For each loop}
\label{loop-each}

The fourth form of loop control also uses the internal variable
\verb+i+, but in this case the variable ranges over a specified list
of strings.  The loop is executed once for each string in the list.
This can be useful for performing repetitive operations on a list of
variables.  Here is an example of the syntax:
      
\begin{code}
	loop foreach i peach pear plum
	   print "$i"
	endloop
\end{code}

This loop will execute three times, printing out ``peach'', ``pear''
and ``plum'' on the respective iterations.  

If you wish to loop across a list of variables that are contiguous in
the dataset, you can give the names of the first and last variables in
the list, separated by ``\verb+..+'', rather than having to type all
the names.  For example, say we have 50 variables \verb+AK+,
\verb+AL+, \dots{}, \verb+WY+, containing income levels for the states
of the US.  To run a regression of income on time for each of the
states we could do:

\begin{code}
	genr time
	loop foreach i AL..WY
	   ols $i const time
	endloop
\end{code}


\subsection{For loop}
\label{loop-for}

The final form of loop control uses a simplified version of the
\cmd{for} statement in the C programming language.  The expression is
composed of three parts, separated by semicolons.  The first part
specifies an initial condition, expressed in terms of a control
variable; the second part gives a continuation condition (in terms of
the same control variable); and the third part specifies an increment
(or decrement) for the control variable, to be applied each time round
the loop.  The entire expression is enclosed in parentheses.  For
example:

\cmd{loop for (r=0.01; r<.991; r+=.01)}

In this example the variable \verb+r+ will take on the values 0.01,
0.02, \dots{}, 0.99 across the 99 iterations.  Note that due to the
finite precision of floating point arithmetic on computers it may be
necessary to use a continuation condition such as the above,
\verb+r<.991+, rather than the more ``natural'' \verb+r<=.99+.  (Using
double-precision numbers on an x86 processor, at the point where you
would expect \verb+r+ to equal 0.99 it may in fact have value
0.990000000000001.)

To expand on the rules for the three components of the control
expression:

\begin{enumerate}
\item The initial condition must take the form LHS1 = RHS1.  RHS1 must
  be a numeric constant or a predefined variable.  If the LHS1
  variable does not exist already, it is automatically created.
\item The continuation condition must be of the form LHS1 op RHS2,
  where op can be \verb+<+, \verb+>+, \verb+<=+ or \verb+>=+ and RHS2
  must be a numeric constant or a predefined variable.  If RHS2 is a
  variable it is evaluated each time round the loop.
\item The increment or decrement expression must be of the form LHS1
  += DELTA or LHS1 -= DELTA, where DELTA is a numeric constant or a
  predefined variable.  If DELTA is a variable, it is evaluated only
  once, when the loop is set up.
\end{enumerate}
      

\section{Progressive mode}
\label{loop-progressive}

If the \verb+--progressive+ option is given for a command loop, the
effects of the commands \cmd{ols}, \cmd{print} and \cmd{store} are
modified as follows.

\cmd{ols}: The results from each individual iteration of the
regression are not printed.  Instead, after the loop is completed you
get a printout of (a) the mean value of each estimated coefficient
across all the repetitions, (b) the standard deviation of those
coefficient estimates, (c) the mean value of the estimated standard
error for each coefficient, and (d) the standard deviation of the
estimated standard errors.  This makes sense only if there is some
random input at each step.

\cmd{print}: When this command is used to print the value of a
variable, you do not get a print each time round the loop.  Instead,
when the loop is terminated you get a printout of the mean and
standard deviation of the variable, across the repetitions of the
loop.  This mode is intended for use with variables that have a single
value at each iteration, for example the error sum of squares from a
regression.

\cmd{store}: This command writes out the values of the specified
variables, from each time round the loop, to a specified file.  Thus
it keeps a complete record of the variables across the iterations.
For example, coefficient estimates could be saved in this way so as to
permit subsequent examination of their frequency distribution. Only
one such \cmd{store} can be used in a given loop.

\section{Loop examples}
\label{loop-examples}


\subsection{Monte Carlo example}
\label{loop-mc-example}

A simple example of a Monte Carlo loop in ``progressive'' mode is
shown in Example~\ref{monte-carlo-loop}.

\begin{script}[htbp]
  \caption{Simple Monte Carlo loop}
  \label{monte-carlo-loop}
\begin{code}
	  nulldata 50
	  seed 547
	  genr x = 100 * uniform()
	  # open a "progressive" loop, to be repeated 100 times
	  loop 100 --progressive
	     genr u = 10 * normal()
	     # construct the dependent variable
	     genr y = 10*x + u
	     # run OLS regression
	     ols y const x
	     # grab the coefficient estimates and R-squared
	     genr a = $coeff(const)
	     genr b = $coeff(x)
	     genr r2 = $rsq
	     # arrange for printing of stats on these
	     print a b r2
	     # and save the coefficients to file
	     store coeffs.gdt a b
	  endloop
\end{code}
\end{script}

This loop will print out summary statistics for the `a' and `b'
estimates and $R^2$ across the 100 repetitions.  After running the
loop, \verb+coeffs.gdt+, which contains the individual coefficient
estimates from all the runs, can be opened in \app{gretl} to examine
the frequency distribution of the estimates in detail.

The command \cmd{nulldata} is useful for Monte Carlo work.  Instead of
opening a ``real'' data set, \cmd{nulldata 50} (for instance) opens a
dummy data set, containing just a constant and an index variable, with
a series length of 50. Constructed variables can then be added using
the \cmd{genr} command.See the \cmd{set} command for information on
generating repeatable pseudo-random series.

\subsection{Iterated least squares}
\label{loop-ils-examples}

Example \ref{greene-ils-script} uses a ``while'' loop to replicate the
estimation of a nonlinear consumption function of the form
	
\[ C = \alpha + \beta Y^{\gamma} + \epsilon \]

as presented in Greene (2000, Example 11.3).  This script is included
in the \app{gretl} distribution under the name \verb+greene11_3.inp+;
you can find it in \app{gretl} under the menu item ``File, Open
command file, practice file, Greene...''.

The option \verb+--print-final+ for the \cmd{ols} command arranges
matters so that the regression results will not be printed each time
round the loop, but the results from the regression on the last
iteration will be printed when the loop terminates.

\begin{script}[htbp]
  \caption{Nonlinear consumption function}
  \label{greene-ils-script}
\begin{code}
	  open greene11_3.gdt
	  # run initial OLS
	  ols C 0 Y
	  genr essbak = $ess
	  genr essdiff = 1
	  genr beta = $coeff(Y)
	  genr gamma = 1
	  # iterate OLS till the error sum of squares converges
	  loop while essdiff > .00001
	     # form the linearized variables
	     genr C0 = C + gamma * beta * Y^gamma * log(Y)
	     genr x1 = Y^gamma
	     genr x2 = beta * Y^gamma * log(Y)
	     # run OLS 
	     ols C0 0 x1 x2 --print-final --no-df-corr --vcv
	     genr beta = $coeff(x1)
	     genr gamma = $coeff(x2)
	     genr ess = $ess
	     genr essdiff = abs(ess - essbak)/essbak
	     genr essbak = ess
	  endloop 
	  # print parameter estimates using their "proper names"
	  noecho
	  printf "alpha = %g\n", $coeff(0)
	  printf "beta  = %g\n", beta
	  printf "gamma = %g\n", gamma
\end{code}
\end{script}

Example~\ref{jack-arma} shows how a loop can be used to
estimate an ARMA model, exploiting the ``outer product of the
gradient'' (OPG) regression discussed by Davidson and MacKinnon in
their \emph{Estimation and Inference in Econometrics}.

\begin{script}[htbp]
  \caption{ARMA 1, 1}
  \label{jack-arma}

\begin{code}
	open armaloop.gdt

	genr c = 0
	genr a = 0.1
	genr m = 0.1

	series e = 1.0
	genr de_c = e
	genr de_a = e
	genr de_m = e

	genr crit = 1
	loop while crit > 1.0e-9

	   # one-step forecast errors
	   genr e = y - c - a*y(-1) - m*e(-1)  

	   # log-likelihood 
	   genr loglik = -0.5 * sum(e^2)
	   print loglik

	   # partials of forecast errors wrt c, a, and m
	   genr de_c = -1 - m * de_c(-1) 
	   genr de_a = -y(-1) -m * de_a(-1)
	   genr de_m = -e(-1) -m * de_m(-1)
   
	   # partials of l wrt c, a and m
	   genr sc_c = -de_c * e
	   genr sc_a = -de_a * e
	   genr sc_m = -de_m * e
   
	   # OPG regression
	   ols const sc_c sc_a sc_m --print-final --no-df-corr --vcv

	   # Update the parameters
	   genr dc = $coeff(sc_c) 
	   genr c = c + dc
	   genr da = $coeff(sc_a) 
	   genr a = a + da
	   genr dm = $coeff(sc_m) 
	   genr m = m + dm

	   printf "  constant        = %.8g (gradient = %#.6g)\n", c, dc
	   printf "  ar1 coefficient = %.8g (gradient = %#.6g)\n", a, da
	   printf "  ma1 coefficient = %.8g (gradient = %#.6g)\n", m, dm

	   genr crit = $T - $ess
	   print crit
	endloop

	genr se_c = $stderr(sc_c)
	genr se_a = $stderr(sc_a)
	genr se_m = $stderr(sc_m)

	noecho
	print "
	printf "constant = %.8g (se = %#.6g, t = %.4f)\n", c, se_c, c/se_c
	printf "ar1 term = %.8g (se = %#.6g, t = %.4f)\n", a, se_a, a/se_a
	printf "ma1 term = %.8g (se = %#.6g, t = %.4f)\n", m, se_m, m/se_m
\end{code}
\end{script}



\subsection{Indexed loop examples}


Example \ref{loop-panel-script} shows an indexed loop in which the
\cmd{smpl} is keyed to the index variable \verb+i+.  Suppose we have a
panel dataset with observations on a number of hospitals for the years
1991 to 2000 (where the year of the observation is indicated by a
variable named \verb+year+).  We restrict the sample to each of these
years in turn and print cross-sectional summary statistics for
variables 1 through 4.

\begin{script}[htbp]
  \caption{Panel statistics}
  \label{loop-panel-script}
\begin{code}
	  open hospitals.gdt
	  loop i=1991..2000
	    smpl (year=i) --restrict --replace
	    summary 1 2 3 4
	  endloop
\end{code}
\end{script}


Example \ref{loop-string-script} illustrates string substitution in an
indexed loop.

\begin{script}[htbp]
  \caption{String substitution}
  \label{loop-string-script}
\begin{code}
	  open bea.dat
	  loop i=1987..2001
	    genr V = COMP$i
	    genr TC = GOC$i - PBT$i
	    genr C = TC - V
	    ols PBT$i const TC V
	  endloop
\end{code}
\end{script}

The first time round this loop the variable \verb+V+ will be set to
equal \verb+COMP1987+ and the dependent variable for the \cmd{ols}
will be \verb+PBT1987+. The next time round \verb+V+ will be redefined
as equal to \verb+COMP1988+ and the dependent variable in the
regression will be \verb+PBT1988+.  And so on.

%%% Local Variables: 
%%% mode: latex
%%% TeX-master: "gretl-guide"
%%% End: 

