\chapter{Maximum likelihood estimation}

\newcommand{\LogLik}{\ensuremath\ell}
\newcommand{\stackunder}[2]{\ensuremath\mathrel{\mathop{#2}\limits_{#1}}}
\newcommand{\pder}[2]{\frac{\ensuremath\partial #1}{\partial #2}}

\section{Generic ML estimation with gretl}

Maximum likelihood estimation is a cornerstone of modern inferential
procedures. Gretl provides a way to implement this method for a wide
range of estimation problems, by use of the \texttt{mle} command. We
give here a few examples.

To give a fiundation for the examples that follow, we start from a
brief reminder on the basics of ML estimation.  Given a sample of size
$T$, it is possible to define the density function\footnote{We are
  imagining here that our data are a realisation of continuous random
  variables. For discrete random variable, everything continues to
  apply by referring to the probability function instead of the
  density. In both cases, the distribution may be conditional on some
  exogenous variables.} for the whole sample, namely the joint
distribution of all the observations $f(\mathbf{Y} ; \theta)$, where
$\mathbf{Y} = \left\{ y_1, \ldots, y_T \right\}$.  Its shape is
determined by a vector of unknown parameters $\theta$, which we assume
is contained in a set $\Theta$, and which can be used to evaluate the
probability of observing a sample with any given characteristic.

After observing the data, the values $\mathbf{Y}$ are given, and this
function can be evaluated for any legitimate value of $\theta$. In
this case, we prefer to call it the \emph{likelihood} function; the
need for another name stems from the fact that this function works as
a density when we use the $y_t$'s as arguments and $\theta$ as
parameters, whereas in this context $\theta$ is taken as the
function's argument, and the data $\mathbf{Y}$ only have the role of
determining its shape.

In standard cases, this function has a unique maximum.  The location
of the maximum is unaffected if we consider the logarithm of the
likelihood (or log-likelihood for short): this function will be
denoted as
\[
  \LogLik(\theta) = \log  f(\mathbf{Y}; \theta) .
\] 
The log-likelihood functions that gretl can handle are those when
$\LogLik(\theta)$ can be written as
\[
  \LogLik(\theta) = \sum_{t=1}^T \ell_t(\theta) ,
\] 
which is true in most cases of interest. The functions $\ell_t(\theta)$
are called the log-likelihood contributions.

Moreover, the location of the maximum is obviously determined by the
data $\mathbf{Y}$. This means that the value
\begin{equation}
  \label{eq:maxlik}
  \hat{\theta}(\mathbf{Y}) = \stackunder{\theta \in \Theta}{\mathrm{Argmax}} \LogLik(\theta)
\end{equation}
is some function of the observed data (a statistic), which has the
property, under mild conditions, of being a consistent, asymptotically
normal and asymptotically efficient estimator of $\theta$.

Sometimes it is possible to write down explicitly the function
$\hat{\theta}(\mathbf{Y})$; in general, it need not be so. In these
circumstances, the maximum can be found by means of numerical
techniques. These often rely on the fact that the log-likelihood is a
smooth function of $\theta$, and therefore on the maximum its partial
derivatives should all be 0. The gradient of the log-likelihood is
called the \emph{score} vector and is a function that enjoys many
interesting statistical properties in its own right; it will be
denoted here as $\mathbf{s}(\theta)$.

Gradient-based methods can be shortly illustrated as follows:

\begin{enumerate}
\item pick a point $\theta_0 \in \Theta$ ;
\item evaluate $\mathbf{s}(\theta_0)$;
\item if $\mathbf{s}(\theta_0)$ is ``small'', stop. Otherwise, compute
  a direction vector $d(\mathbf{s}(\theta_0))$;
\item evaluate $\theta_1 = \theta_0 + d(\mathbf{s}(\theta_0))$ ;
\item substitute $\theta_0$ with $\theta_1$ ;
\item restart from 2.
\end{enumerate}

Many algorithms of this kind exist; they basically differ from one
another in the way they compute the direction vector
$d(\mathbf{s}(\theta_0))$, to ensure that $\LogLik(\theta_1) >
\LogLik(\theta_0)$ (so that we eventually end up on the maximum).

The method \app{gretl} uses to maximize the log-likelihood is a
gradient-based algorithm known as the \textbf{BFGS} (Broyden,
Fletcher, Goldfarb and Shanno) method. This technique is used in most
econometric and statistical packages, as is well-established and
remarkably powerful. Clearly, in order to make this technique
operational, it must be possible to compute the vector
$\mathbf{s}(\theta)$ for any value of $\theta$. In some cases, the
function $\mathbf{s}(\theta)$ can be written explicitly in term of
$\mathbf{Y}$. At times, this is not possible or too difficult;
therefore, the function $\mathbf{s}(\theta)$ is evaluated numerically.

The choice of the starting value $\theta_0$ is crucial in some contexts
and inconsequential in others. In general, however, it is
advisable to start the algorithm from ``sensible'' values whenever
possible. If a consistent estimator is available, this is usually a
safe and efficient choice: this ensures that in large samples the
starting point will be likely close to $\hat{\theta}$ and convergence
can be achieved in few iterations. 

\section{Gamma estimation}
\label{sec:gamma}

Suppose we have a sample of $T$ independent and identically
distributed observations from a Gamma distribution. The density
function for each observation $x_t$ is
\begin{equation}
  \label{eq:gammadens}
  f(x_t) = \frac{\alpha^p}{\Gamma(p)} x_t^{p-1} \exp\left({-\alpha
      x_t}\right) .
\end{equation}
The log-likelihood for the entire sample can be written as the
logarithm of the joint density of all the observations. Since these
are independent and identical, the joint density is the product of the
individual densities, and hence its log is
\begin{equation}
  \label{eq:gammaloglik}
  \LogLik(\alpha, p) = \sum_{t=1}^T \log \left[ \frac{\alpha^p}{\Gamma(p)} x_t^{p-1} \exp\left({-\alpha
      x_t}\right) \right] = 
      \sum_{t=1}^T \ell_t ,
\end{equation}
where 
\[
  \ell_t = p \cdot \log (\alpha x_t) - \gamma(p) - \log x_t - \alpha x_t
\]
and $\gamma(\cdot)$ is the log of the gamma function.  In order to
estimate the parameters $\alpha$ and $p$ via ML, we need to maximize
(\ref{eq:gammaloglik}) with respect to them. The corresponding
\app{gretl} code snippet is

\begin{code}
    scalar alpha = 1
    scalar p = 1

    mle logl =  p*ln(alpha * x) - lngamma(p) - ln(x) - alpha * x 
    end mle 
\end{code}

The two statements

\begin{code}
    alpha = 1
    p = 1
\end{code}

are necessary to ensure that the variables \texttt{p} and
\texttt{alpha} exist before the computation of \texttt{logl} is
attempted. The values of these variables will be changed by the
execution of the \texttt{mle} command; upon successful completion,
they will be replaced by the ML estimates. The starting value is 1 for
both; this is arbitrary and does not matter much in this example (more
on this later).

The above code can be made more readable, and marginally more
efficient, by defining a variable to hold $\alpha \cdot x_t$. This
command can be embedded into the \texttt{mle} block as follows:
\begin{code}
    scalar alpha = 1
    scalar p = 1

    mle logl =  p*ln(ax) - lngamma(p) - ln(x) - ax 
    series ax = alpha*x
    params alpha p
    end mle 
\end{code}
In this case, it is necessary to include the line \texttt{params alpha
  p} to set the symbols \texttt{p} and \texttt{alpha} apart from
\texttt{ax}, which is a temporarily generated variable and not a
parameter to be estimated.

In a simple example like this, the choice of the starting values is
almost inconsequential; the algorithm is likely to converge no
matter what the starting values are. However, consistent
method-of-moments estimators of $p$ and $\alpha$ can be simply
recovered from the sample mean $m$ and variance $V$: since it can be
shown that
\[
  E(x_t) = p/\alpha \qquad  V(x_t) = p/\alpha^2 ,
\]
it follows that the following estimators 
\begin{eqnarray*}
  \bar{\alpha} & = &  m/V \\
  \bar{p} & = & m \cdot \bar{\alpha} 
\end{eqnarray*}
are consistent, and therefore suitable to be used as starting point
for the algorithm.  The gretl script code then becomes
\begin{code}
    scalar m = mean(x)
    scalar alpha = var(x)/m
    scalar p = m*alpha

    mle logl =  p*ln(ax) - lngamma(p) - ln(x) - ax 
    series ax = alpha*x
    params alpha p
    end mle 
\end{code}

\section{Stochastic frontier cost function}
\label{sec:frontier}

When modeling a cost function, it is sometimes worthwhile to
incorporate explicitly into the statistical model the notion that
firms may be inefficient, so that the observed cost deviates from the
theoretical figure not only because of unobserved heterogeneity
between firms, but also because two firms could be operating at a
different efficiency level, despite being identical under all other
respects. In this case we may write
\[
  C_i = C^*_i + u_i + v_i ,
\]
where $C_i$ is some variable cost indicator, $C_i^*$ is its
``theoretical'' value, $u_i$ is a zero-mean disturbance term and
$v_i$ is the inefficiency term, which is supposed to be nonnegative
by its very nature.

A linear specification for $C_i^*$ is often chosen. For example, the
Cobb--Douglas cost function arises when $C_i^*$ is a linear function
of the logarithms of the input prices and the output quantities.

The \emph{stochastic frontier} model is a linear model of the form
$y_i = x_i \beta + \varepsilon_i$ in which the error term
$\varepsilon_i$ is the sum of $u_i$ and $v_i$.  A common postulate is
that $u_i \sim N(0,\sigma_u^2)$ and $v_i \sim
\left|N(0,\sigma_v^2)\right|$. If independence between $u_i$ and $v_i$
is also assumed, then it is possible to show that the density function
of $\varepsilon_i$ has the form:
\begin{equation}
  \label{eq:frontdens}
  f(\varepsilon_i) = 
   \sqrt{\frac{2}{\pi}} 
   \Phi\left(\frac{\lambda \varepsilon_i}{\sigma}\right)
   \frac{1}{\sigma} \phi\left(\frac{\varepsilon_i}{\sigma}\right) ,
\end{equation}
where $\Phi(\cdot)$ and $\phi(\cdot)$ are, respectively, the distribution and density
function of the standard normal, $\sigma =
\sqrt{\sigma^2_u + \sigma^2_v}$ and $\lambda = \frac{\sigma_u}{\sigma_v}$.

As a consequence, the log-likelihood for one observation takes the
form (apart form an irrelevant constant)
\[
  \ell_t = 
  \log\Phi\left(\frac{\lambda \varepsilon_i}{\sigma}\right) -
  \left[ \log(\sigma) + \frac{\varepsilon_i^2}{2 \sigma^2} \right];
\]
therefore, a Cobb--Douglas cost function with stochastic frontier is the
model described by the following equations: 
\begin{eqnarray*}
  \log C_i & = & \log C^*_i + \varepsilon_i \\
  \log C^*_i & = & c + \sum_{j=1}^m \beta_j \log y_{ij} + \sum_{j=1}^n \alpha_j \log p_{ij} \\
  \varepsilon_i & = & u_i + v_i \\
  u_i & \sim & N(0,\sigma_u^2) \\
  v_i & \sim & \left|N(0,\sigma_v^2)\right|  .
\end{eqnarray*}

In most cases, one wants to ensure that the homogeneity of the cost
function with respect to the prices holds by construction. Since this
requirement is equivalent to $\sum_{j=1}^n \alpha_j = 1$, the above
equation for $C^*_i$ can be rewritten as

\begin{equation}
  \label{eq:CobbDouglasFrontier}
  \log C_i - \log p_{in}  = c + \sum_{j=1}^m \beta_j \log y_{ij} +
  \sum_{j=2}^n \alpha_j (\log p_{ij} - \log p_{in})  + \varepsilon_i.
\end{equation}

The above equation could be estimated by OLS, but it would suffer from
two drawbacks: first, the OLS estimator for the intercept $c$ is
inconsistent because the disturbance term has a non-zero expected
value; second, the OLS estimators for the other parameters are
consistent, but inefficient in view of the non-normality of
$\varepsilon_i$. Both issues can be addressed by estimating
(\ref{eq:CobbDouglasFrontier}) by maximum likelihood. Nevertheless,
OLS estimation is a quick and convenient way to provide starting
values for the MLE algorithm.

Example \ref{cost-estimation} shows how to implement the model
described so far. The \texttt{banks91} file contains part of the data
used in Lucchetti, Papi and Zazzaro (2001).

\begin{script}[htbp]
  \caption{Estimation of stochastic frontier cost function}
  \label{cost-estimation}
\begin{code}
open banks91

# Cobb-Douglas cost function

ols cost const y p1 p2 p3

# Cobb-Douglas cost function with homogeneity restrictions

genr rcost = cost - p3
genr rp1 = p1 - p3
genr rp2 = p2 - p3

ols rcost const y rp1 rp2

# Cobb-Douglas cost function with homogeneity restrictions 
# and inefficiency 

scalar b0 = $coeff(const)
scalar b1 = $coeff(y)
scalar b2 = $coeff(rp1)
scalar b3 = $coeff(rp2)

scalar su = 0.1
scalar sv = 0.1

mle logl = ln(cnorm(e*lambda/ss)) - (ln(ss) + 0.5*(e/ss)^2)
  scalar ss = sqrt(su^2 + sv^2)
  scalar lambda = su/sv
  series e = rcost - b0*const - b1*y - b2*rp1 - b3*rp2
  params b0 b1 b2 b3 su sv
end mle
\end{code}
\end{script}

\section{GARCH models}
\label{sec:garch}

GARCH models are handled by gretl via a native function. However, it is
instructive to see how they can be estimated through the \texttt{mle}
command.

The following equations provide the simplest example of a GARCH(1,1)
model:
\begin{eqnarray*}
  y_t & = & \mu + \varepsilon_t \\
  \varepsilon_t & = & u_t \cdot \sigma_t \\
  u_t & \sim & N(0,1) \\
  h_t & = & \omega + \alpha \varepsilon^2_{t-1} + \beta h_{t-1}.
\end{eqnarray*}
Since the variance of $y_t$ depends on past values, writing down the
log-likelihood function is not simply a matter of summing the log
densities for individual observations. As is common in time series
models, $y_t$ cannot be considered independent of the other
observations in our sample, and consequently the density function for
the whole sample (the joint density for all observations) is not just
the product of the marginal densities.

Maximum likelihood estimation, in these cases, is achieved by
considering \emph{conditional} densities, so what we maximize is a
conditional likelihood function. If we define the information set at
time $t$ as
\[
  F_t = \left\{ y_t, y_{t-1}, \ldots \right\} ,
\]
then the density of $y_t$ conditional on $F_{t-1}$ is normal:
\[
  y_t | F_{t-1} \sim N\left[ \mu, h_{t} \right].
\]

By means of the properties of conditional distributions, the joint
density can be factorized as follows
\[
  f(y_t, y_{t-1}, \ldots) = \left[ \prod_{t=1}^T f(y_t |F_{t-1})
  \right] \cdot f(y_0) ;
\]
if we treat $y_0$ as fixed, then the term $f(y_0)$ does not depend on
the unknown parameters, and therefore the conditional log-likelihood
can then be written as the sum of the individual contributions as
\begin{equation}
  \label{eq:garchloglik}
  \LogLik(\mu,\omega,\alpha,\beta) = \sum_{t=1}^T \ell_t ,
\end{equation}
where 
\[
  \ell_t = \log \left[ \frac{1}{\sqrt{h_t}} \phi\left( \frac{y_t - \mu}{\sqrt{h_t}}
    \right) \right] = 
    - \frac{1}{2} \left[ \log(h_t) + \frac{(y_t - \mu)^2}{h_t} \right] .
\]

The following script shows a simple application of this technique,
which uses the data file \texttt{djclose}; it is one of the example
dataset supplied with gretl and contains daily data from the Dow Jones
stock index.

\begin{code}
open djclose

series y = 100*ldiff(djclose)

scalar mu = 0.0
scalar omega = 1
scalar alpha = 0.4
scalar beta = 0.0

mle ll = -0.5*(log(h) + (e^2)/h)
  series e = y - mu
  series h = var(y)
  series h = omega + alpha*(e(-1))^2 + beta*h(-1)
  params mu omega alpha beta
end mle
\end{code}

\section{Analytical derivatives}
\label{sec:anal-der}

Computation of the score vector is essential for the working of the
BFGS method. In all the previous examples, no explicit formula for the
computation of the score was given, so the algorithm was fed
numerically evaluated gradients. Numerical computation of the score for
the $i$-th parameter is performed via a finite approximation of the
derivative, namely
\[
  \pder{\LogLik(\theta_1, \ldots, \theta_n)}{\theta_i} \simeq 
  \frac{\LogLik(\theta_1, \ldots, \theta_i + h, \ldots, \theta_n) -
    \LogLik(\theta_1, \ldots, \theta_i - h, \ldots, \theta_n)}{2h} ,
\]
where $h$ is a small number. 

In many situations, this is rather efficient and accurate. However,
one might want to avoid the approximation and specify an exact
function for the derivatives. As an example, consider the following
script:
\begin{code}
nulldata 1000

genr x1 = normal()
genr x2 = normal()
genr x3 = normal()

genr ystar = x1 + x2 + x3 + normal()
genr y = (ystar > 0)

scalar b0 = 0
scalar b1 = 0
scalar b2 = 0
scalar b3 = 0

mle logl = y*ln(P) + (1-y)*ln(1-P)
  series ndx = b0 + b1*x1 + b2*x2 + b3*x3
  series P = cnorm(ndx)
  params b0 b1 b2 b3
end mle --verbose
\end{code}

Here, 1000 data points are artificially generated for an ordinary
probit model\footnote{Again, gretl does provide a native
  \texttt{probit} command, but a probit model makes for a nice
  example here.}: $y_t$ is a binary variable, which takes the value 1 if
$y_t^* = \beta_1 x_{1t} + \beta_2 x_{2t} + \beta_3 x_{3t} +
\varepsilon_t > 0$ and 0 otherwise. Therefore, $y_t = 1$ with
probability $\Phi(\beta_1 x_{1t} + \beta_2 x_{2t} + \beta_3 x_{3t}) =
\pi_t$.  The probability function for one observation can be written
as
\[
  P(y_t) = \pi_t^{y_t} ( 1 -\pi_t )^{1-y_t} ;
\]
since the observations are independent and identically distributed,
the log-likelihood is simply the sum of the individual
contributions. Hence
\[
  \LogLik = \sum_{t=1}^T y_t \log(\pi_t) + (1 - y_t) \log(1 - \pi_t) .
\]
The \texttt{--verbose} switch at the end of the \texttt{end mle}
statement produces a detailed account of the iterations done by the
BFGS algorithm.

In this case, numerical differentiation works rather well;
nevertheless, computation of the analytical score is straightforward,
since the derivative $\pder{\LogLik}{\beta_i}$ can be written as
\[
  \pder{\LogLik}{\beta_i} = \pder{\LogLik}{\pi_t} \cdot \pder{\pi_t}{\beta_i}
\]
via the chain rule, and it is easy to see that
\begin{eqnarray*}
  \pder{\LogLik}{\pi_t} & = & \frac{y_t}{\pi_t} - \frac{1 - y_t}{1 -
    \pi_t} \\
  \pder{\pi_t}{\beta_i} & = & \phi(\beta_1 x_{1t} + \beta_2 x_{2t} +
  \beta_3 x_{3t}) \cdot x_{it} .
\end{eqnarray*}

The \texttt{mle} block in the above script can therefore be modified
as follows:

\begin{code}
mle logl = y*ln(P) + (1-y)*ln(1-P)
  series ndx = b0 + b1*x1 + b2*x2 + b3*x3
  series P = cnorm(ndx)
  series tmp = dnorm(ndx)*(y/P - (1-y)/(1-P))
  deriv b0 = tmp
  deriv b1 = tmp*x1
  deriv b2 = tmp*x2
  deriv b3 = tmp*x3
end mle --verbose
\end{code}

Note that the \texttt{params} statement has been replaced by a series
of \texttt{deriv} statements; these have the double function of
identifying the parameters over which to optimize and providing an
analytical expression for their respective score elements.

%%% Local Variables: 
%%% mode: latex
%%% TeX-master: "gretl-guide"
%%% End: 


