\chapter{Gretl and Python}
\label{chap:gretlPython}

\section{Introduction}
\label{Python-intro}

According to \url{www.python.org}, \app{Python} is ``an easy to learn,
powerful programming language. It has efficient high-level data
structures and a simple but effective approach to object-oriented
programming. Python's elegant syntax and dynamic typing, together with
its interpreted nature, make it an ideal language for scripting and
rapid application development in many areas on most platforms.''

Indeed, \app{Python} is widely used in a great variety of
contexts. Numerous add-on modules are available; the ones likely to be
of greatest interest to econometricians include \app{NumPy} (``the
fundamental package for scientific computing with Python'' --- see
\url{www.numpy.org}); \app{SciPy} (which builds on \app{NumPy} --- see
\url{www.scipy.org}); and \app{Statsmodels}
(\url{http://statsmodels.sourceforge.net/}).

\section{Python support in gretl}
\label{sec:Python-support}

The support offered for \app{Python} in \app{gretl} is similar to that
offered for \app{Octave} (chapter~\ref{chap:gretlOctave}). You can
open and edit \app{Python} scripts in the \app{gretl} GUI.  Clicking
the ``execute'' icon in the editor window will send your code to
\app{Python} for execution. In addition you can embed \app{Python}
code within a \app{gretl} script using a \texttt{foreign} block, as
described in connection with \app{R}.

When you launch \app{Python} from within \app{gretl} three convenience
functions are pre-defined, as follows.
\begin{code}
gretl_dotdir()
gretl_loadmat(filename, autodot=0)
gretl_export(M, filename, autodot=0)
\end{code}
The first of these returns the path to the user's ``dot directory.''
The second loads a matrix of the given \texttt{filename} as written by
\app{gretl}'s \texttt{mwrite} function. If you want the traffic to be
handled via the dot directory, you can (a) give a non-zero value for
the optional \textsl{export} parameter to \texttt{mwrite}, and (b)
give non-zero for \texttt{autodot} when calling \verb|gretl_loadmat|.
The third function writes matrix \texttt{M}, under the given
\texttt{filename}, in the format wanted by \app{gretl}. If
\texttt{autodot} is supplied and non-zero, writing is directed to the
dot directory; this complements the optional \textsl{import} parameter
for \app{gretl}'s \texttt{mread} function.

Note that \verb|gretl_loadmat| and \verb|gretl_export| depend on
\app{NumPy}; they make use of the functions \texttt{loadtxt} and
\texttt{savetxt} respectively. Nonetheless, the presence of
\app{NumPy} is not an absolute requirement if you don't need
to use these two functions.

\section{Illustration: linear regression with multicollinearity}
\label{sec:Python-longley}

Example~\ref{Python-longley} compares the numerical accuracy of
\app{gretl}'s \texttt{ols} and \texttt{mpols} commands with that of
\app{NumPy}'s \texttt{linalg.lstsq}, using the notorious Longley test
data which exhibit extreme multicollinearity.  Unlike some
econometrics packages, \app{NumPy} does a good job on these data. The
script computes and prints the log-relative error in estimation of the
regression coefficients, using the NIST-certified values as a
benchmark;\footnote{See
  \url{http://www.itl.nist.gov/div898/strd/lls/data/Longley.shtml}.}
the error values correspond to the number of correct digits (with a
maximum of 15). The results will differ somewhat by computer
architecture; the output shown was obtained on a 32-bit Linux Intel i5
system.

\begin{script}[htbp]
  \caption{Comparing regression results with \app{Python}}
\begin{scode}
set echo off
set messages off

function matrix logrel_error (matrix est, matrix true)
  return -log10(abs(est - true) ./ abs(true))
end function

open longley.gdt -q
list LX = prdefl .. year
# gretl's regular OLS
ols employ 0 LX -q
matrix b = $coeff

mwrite({employ}, "y.mat", 1)
mwrite({const} ~ {LX}, "X.mat", 1)

foreign language=python
   import numpy as np
   y = gretl_loadmat('y.mat', 1)
   X = gretl_loadmat('X.mat', 1)
   # NumPy's OLS   
   b, u, rnk, s = np.linalg.lstsq(X, y)
   gretl_export(np.transpose(np.matrix(b)), 'py_b.mat', 1)
end foreign

# NIST's certified coefficient values
matrix nist_b = {-3482258.63459582, 15.0618722713733,
    -0.358191792925910E-01, -2.02022980381683,
    -1.03322686717359, -0.511041056535807E-01,
     1829.15146461355}'

matrix py_b = mread("py_b.mat", 1)
matrix errs = logrel_error(b, nist_b) ~ logrel_error(py_b, nist_b)
colnames(errs, "gretl python")
printf "Log-relative errors, Longley coefficients:\n\n%12.5g\n", errs
printf "Column means\n%12.5g\n", meanc(errs)
\end{scode}
Output:
\begin{code}
Log-relative errors, Longley coefficients:

       gretl      python
      12.844       12.85
      11.528      11.414
      12.393      12.401
      13.135      13.121
      13.738      13.318
      12.587      12.363
      12.848      12.852

Column means
      12.725      12.617
\end{code}
\label{Python-longley}
\end{script}

%%% Local Variables: 
%%% mode: latex
%%% TeX-master: "gretl-guide"
%%% End: 

