\chapter{Data file details}
\label{app-datafile}

\section{Basic native format}
\label{native}

In \app{gretl}'s native data format, a data set is stored in XML
(extensible mark-up language). Data files correspond to the simple DTD
(document type definition) given in \verb+gretldata.dtd+, which is
supplied with the \app{gretl} distribution and is installed in the
system data directory (e.g. \url{/usr/share/gretl/data} on Linux.)
Data files may be plain text or gzipped.  They contain the actual data
values plus additional information such as the names and descriptions
of variables, the frequency of the data, and so on.Most users will
probably not have need to read or write such files other than via
\app{gretl} itself, but if you want to manipulate them using other
software tools you should examine the DTD and also take a look at a
few of the supplied practice data files: \verb+data4-1.gdt+ gives a
simple example; \verb+data4-10.gdt+ is an example where observation
labels are included.

\section{Traditional ESL format}
\label{traddata}

For backward compatibility, \app{gretl} can also handle data files in
the ``traditional'' format inherited from Ramanathan's \app{ESL}
program.  In this format (which was the default in \app{gretl} prior
to version 0.98) a data set is represented by two files.  One contains
the actual data and the other information on how the data should be
read.  To be more specific:
\begin{enumerate}
\item \emph{Actual data}: A rectangular matrix of white-space
  separated numbers.  Each column represents a variable, each row an
  observation on each of the variables (spreadsheet style). Data
  columns can be separated by spaces or tabs. The filename should have
  the suffix \verb+.gdt+.  By default the data file is ASCII (plain
  text).  Optionally it can be gzip-compressed to save disk space. You
  can insert comments into a data file: if a line begins with the hash
  mark (\verb+#+) the entire line is ignored. This is consistent with
  gnuplot and octave data files.
\item \emph{Header}: The data file must be accompanied by a header
  file which has the same basename as the data file plus the suffix
  \verb+.hdr+.  This file contains, in order:
  \begin{itemize}
  \item (Optional) \emph{comments} on the data, set off by the opening
    string \verb+(*+ and the closing string \verb+*)+, each of these
    strings to occur on lines by themselves.
  \item (Required) list of white-space separated \emph{names of the
      variables} in the data file. Names are limited to 8 characters,
    must start with a letter, and are limited to alphanumeric
    characters plus the underscore.  The list may continue over more
    than one line; it is terminated with a semicolon, \verb+;+.
  \item (Required) \emph{observations} line of the form \verb+1 1 85+.
    The first element gives the data frequency (1 for undated or
    annual data, 4 for quarterly, 12 for monthly).  The second and
    third elements give the starting and ending observations.
    Generally these will be 1 and the number of observations
    respectively, for undated data.  For time-series data one can use
    dates of the form \cmd{1959.1} (quarterly, one digit after the
    point) or \cmd{1967.03} (monthly, two digits after the point). See
    [?] for special use of this line in the case of panel data.
  \item The keyword \verb+BYOBS+.
  \end{itemize}


\end{enumerate}

Here is an example of a well-formed data header file.

	
\begin{code} 
	  (* 
	  DATA9-6: 
	  Data on log(money), log(income) and interest rate from US. 
	  Source: Stock and Watson (1993) Econometrica 
	  (unsmoothed data) Period is 1900-1989 (annual data). 
	  Data compiled by Graham Elliott. 
	  *) 
	  lmoney lincome intrate ; 
	  1 1900 1989 BYOBS
\end{code}

The corresponding data file contains three columns of data, each
having 90 entries.  Three further features of the ``traditional'' data
format may be noted.
    
\begin{enumerate}
\item If the \verb+BYOBS+ keyword is replaced by \verb+BYVAR+, and
  followed by the keyword \verb+BINARY+, this indicates that the
  corresponding data file is in binary format.  Such data files can be
  written from \app{gretlcli} using the \cmd{store} command with the
  \cmd{-s} flag (single precision) or the \cmd{-o} flag (double
  precision).
\item If \verb+BYOBS+ is followed by the keyword \verb+MARKERS+,
  \app{gretl} expects a data file in which the \emph{first column}
  contains strings (8 characters maximum) used to identify the
  observations.  This may be handy in the case of cross-sectional data
  where the units of observation are identifiable: countries, states,
  cities or whatever.  It can also be useful for irregular time series
  data, such as daily stock price data where some days are not trading
  days --- in this case the observations can be marked with a date
  string such as \cmd{10/01/98}.  (Remember the 8-character maximum.)
  Note that \cmd{BINARY} and \cmd{MARKERS} are mutually exclusive
  flags.  Also note that the ``markers'' are not considered to be a
  variable: this column does not have a corresponding entry in the
  list of variable names in the header file.
\item If a file with the same base name as the data file and header
  files, but with the suffix \verb+.lbl+, is found, it is read to fill
  out the descriptive labels for the data series. The format of the
  label file is simple: each line contains the name of one variable
  (as found in the header file), followed by one or more spaces,
  followed by the descriptive label. Here is an example:
  \verb+price New car price index, 1982 base year+

	
\end{enumerate}

If you want to save data in traditional format, use the \cmd{-t} flag
with the \cmd{store} command, either in the command-line program or in
the console window of the GUI program.

\section{Binary database details}
\label{dbdetails}

A \app{gretl} database consists of two parts: an ASCII index file
(with filename suffix \verb+.idx+) containing information on the
series, and a binary file (suffix \verb+.bin+) containing the actual
data.  Two examples of the format for an entry in the \verb+idx+ file
are shown below:
\begin{code}
      G0M910  Composite index of 11 leading indicators (1987=100) 
      M 1948.01 - 1995.11  n = 575
      currbal Balance of Payments: Balance on Current Account; SA 
      Q 1960.1 - 1999.4 n = 160
\end{code}
The first field is the series name.  The second is a description of
the series (maximum 128 characters).  On the second line the first
field is a frequency code: \verb+M+ for monthly, \verb+Q+ for
quarterly, \verb+A+ for annual, \verb+B+ for business-daily (daily
with five days per week) and \verb+D+ for daily (seven days per week).
No other frequencies are accepted at present.  Then comes the starting
date (N.B. with two digits following the point for monthly data, one
for quarterly data, none for annual), a space, a hyphen, another
space, the ending date, the string ``\verb+n = +'' and the integer
number of observations. In the case of daily data the starting and
ending dates should be given in the form \verb+YYYY/MM/DD+. This
format must be respected exactly.Optionally, the first line of the
index file may contain a short comment (up to 64 characters) on the
source and nature of the data, following a hash mark.  For example:
\begin{code}
      # Federal Reserve Board (interest rates)
\end{code}
The corresponding binary database file holds the data values,
represented as ``floats'', that is, single-precision floating-point
numbers, typically taking four bytes apiece.  The numbers are packed
``by variable'', so that the first \emph{n} numbers are the
observations of variable 1, the next \emph{m} the observations on
variable 2, and so on.

\chapter{Technical notes}
\label{app-technote}

\app{Gretl} is written in the C programming language.  I have abided
as far as possible by the ISO/ANSI C Standard (C89), although the
graphical user interface and some other components necessarily make
use of platform-specific extensions.
  
\app{gretl} was developed under Linux. The shared library and
command-line client should compile and run on any platform that (a)
supports ISO/ANSI C, (b) has the following libraries installed: zlib
(data compression), libxml (XML manipulation), and LAPACK (linear
algebra support). The homepage for zlib can be found at
\href{http://www.info-zip.org/pub/infozip/zlib/}{info-zip.org}; libxml
is at \href{http://xmlsoft.org/}{xmlsoft.org}; LAPACK is at
\href{http://www.netlib.org/lapack/}{netlib.org}. If the GNU readline
library is found on the host system this will be used for
\app{gretcli}, providing a much enhanced editable command line.  See
the
\href{http://cnswww.cns.cwru.edu/~chet/readline/rltop.html}{readline
  homepage}.  The graphical client program should compile and run on
any system that, in addition to the above requirements, offers GTK
version 1.2.3 or higher (see \href{http://www.gtk.org/}{gtk.org}).  As
of this writing there are two main variants of the GTK libraries: the
1.2 series and the 2.0 series which was launched in summer 2002.
These variants are mutually incompatible.  \app{gretl} can be built
using either one --- the source code package includes two
sub-directories, \verb+gui+ for GTK 1.2 and \verb+gui2+ for GTK 2.0.
I recommend use of GTK 2.0 if it is available, since it offers many
enhancements over GTK 1.2.
  
\app{gretl} calls gnuplot for graphing. You can find gnuplot at
\href{http://www.gnuplot.info/}{gnuplot.info}.  As of this writing the
most recent official release is 4.0 (of April, 2004).  The MS Windows
version of \app{gretl} comes with a Windows version gnuplot 4.0; the
gretl website also offers an rpm of gnuplot 3.8j0 for x86 Linux
systems.
  
Some features of \app{gretl} make use of Adrian Feguin's
\app{gtkextra} library. You can find gtkextra at
\href{http://gtkextra.sourceforge.net/}{gtkextra.sourceforge.net}.
The relevant parts of this package are included (in slightly modified
form) with the \app{gretl} source distribution.
  
A binary version of the program is available for the Microsoft Windows
platform (32-bit version, i.e. Windows 95 or higher). This version was
cross-compiled under Linux using mingw (the GNU C compiler, \app{gcc},
ported for use with win32) and linked against the Microsoft C library,
\verb+msvcrt.dll+.  It uses Tor Lillqvist's port of GTK 2.0 to win32.
The (free, open-source) Windows installer program is courtesy of
Jordan Russell (\href{http://www.jrsoftware.org/}{jrsoftware.org}).
I'm hopeful that some users with coding skills may consider
\app{gretl} sufficiently interesting to be worth improving and
extending.  The documentation of the libgretl API is by no means
complete, but you can find some details by following the link
``Libgretl API docs'' on the \app{gretl} homepage.

\chapter{Numerical accuracy}
\label{app-accuracy}

\app{gretl} uses double-precision arithmetic throughout --- except for
the multiple-precision plugin invoked by the menu item ``Model/High
precision OLS'' which represents floating-point values using a number
of bits given by the environment variable \verb+GRETL_MP_BITS+
(default value 256).  The normal equations of Least Squares are by
default solved via Cholesky decomposition, which is accurate enough
for most purposes (with the option of using QR decomposition instead).
The program has been tested rather thoroughly on the statistical
reference datasets provided by NIST (the U.S. National Institute of
Standards and Technology) and a full account of the results may be
found on the gretl website (follow the link ``Numerical accuracy'').In
October 2002 I had a useful exchange with Giovanni Baiocchi and Walter
Distaso, who were writing a review of \app{gretl} for the
\emph{Journal of Applied Econometrics}, and James MacKinnon, software
review editor for the journal.\footnote{This review has since been
  published; see Baiocchi and Distaso (2003).} and I am grateful to
Baiocchi and Distaso for their careful examination of the program,
which prompted the following modifications.

\begin{enumerate}
\item The reviewers pointed out that there was a bug in \app{gretl}'s
  ``p-value finder'', whereby the program printed the complement of
  the correct probability for negative values of \emph{z}.  This was
  fixed in version 0.998 of the program (released July 9, 2002).
\item They also noted that the p-value finder produced inaccurate
  results for extreme values of \emph{x} (e.g. values of around 8 to
  10 in the \emph{t} distribution with 100 degrees of freedom).  This
  too was fixed in \app{gretl} version 0.998, with a switch to more
  accurate probability distribution code.
\item The reviewers noted a flaw in the presentation of regression
  coefficients in \app{gretl}, whereby some coefficients could be
  printed to an unacceptably small number of significant figures.
  This was fixed in version 0.999 (released August 25, 2002): now all
  the statistics associated with a regression are printed to 6
  significant figures.
\item It transpired from the reviewer's tests that the numerical
  accuracy of \app{gretl} on MS Windows was less than on Linux, where
  I had done my testing.  For example, on the Longley data --- a
  well-known ``ill-conditioned'' dataset often used for testing
  econometrics programs --- the Windows version of gretl was getting
  some coefficients wrong at the 7th digit while the same coefficients
  were correct on Linux.  This anomaly was fixed in \app{gretl}
  version 1.0pre3 (released October 10, 2002).
\end{enumerate}

The current version of \app{gretl} includes a ``plugin'' that runs the
NIST linear regression test suite.  You can find this under the
``Utilities'' menu in the main window.  When you run this test, the
introductory text explains the expected result.  If you run this test
and see anything other than the expected result, please send a bug
report to \verb+cottrell@wfu.edu+.  As mentioned above, all regression
statistics are printed to 6 significant figures in the current version
of \app{gretl} (except when the multiple-precision plugin is used,
then results are given to 12 figures).  If you want to examine a
particular value more closely, first save it (for example, using the
\cmd{genr} command) then print it using \cmd{print --ten} (see the
\emph{Gretl Command Reference}).  This will show the value to 10
digits.

\chapter{Advanced econometric analysis with free software}
\label{app-advanced}

As mentioned in the main text, \app{gretl} offers a reasonably full
selection of least-squares based estimators, plus a few additional
estimators such as (binomial) logit and probit and Least Absolute
Deviations. Advanced users may, however, find \app{gretl}'s menu of
statistical routines restrictive.No doubt some advanced users will
prefer to write their own statistical code in a fundamental computer
language such as C, C++ or Fortran.  Another option is to use a
relatively high-level language that offers easy matrix manipulation
and that already has numerous statistical routines built in, or
available as add-on packages. If the latter option sounds attractive,
and you are interested in using free, open source software, I would
recommend taking a look at either GNU R
(\href{http://www.r-project.org/}{r-project.org}) or GNU Octave
(\href{http://www.octave.org/}{www.octave.org}).  These programs are
very close to the commercial programs S and Matlab respectively.
  
Also as mentioned above, \app{gretl} offers the facility of exporting
data in the formats of both Octave and R.  In the case of Octave, the
\app{gretl} data set is saved as a single matrix, \verb+X+. You can
pull the \verb+X+ matrix apart if you wish, once the data are loaded
in Octave; see the Octave manual for details.  As for R, the exported
data file preserves any time series structure that is apparent to
\app{gretl}.  The series are saved as individual structures. The data
should be brought into R using the \cmd{source()} command.
  
Of these two programs, R is perhaps more likely to be of immediate
interest to econometricians since it offers more in the way of
statistical routines (e.g. generalized linear models, maximum
likelihood estimation, time series methods). I have therefore supplied
\app{gretl} with a convenience function for moving data quickly into
R. Under \app{gretl}'s Session menu, you will find the entry ``Start
GNU R''.  This writes out an R version of the current \app{gretl} data
set (\verb+Rdata.tmp+, in the user's gretl directory), and sources it
into a new R session. A few details on this follow.
  
First, the data are brought into R by writing a temporary version of
\verb+.Rprofile+ in the current working directory.  (If such a file
exists it is referenced by R at startup.)  In case you already have a
personal \verb+.Rprofile+ in place, the original file is temporarily
moved to \verb+.Rprofile.gretltmp+, and on exit from \app{gretl} it is
restored.  (If anyone can suggest a cleaner way of doing this I'd be
happy to hear of it.)
  
Second, the particular way R is invoked depends on the internal
\app{gretl} variable \verb+Rcommand+, whose value may be set under the
File, Preferences menu.  The default command is \cmd{RGui.exe} under
MS Windows. Under X it is either \cmd{R --gui=gnome} if an
installation of the Gnome desktop
(\href{http://www.gnome.org/}{gnome.org}) was detected at compile
time, or \cmd{xterm -e R} if Gnome was not found. Please note that at
most three space-separated elements in this command string will be
processed; any extra elements are ignored.

\chapter{Listing of URLs}
\label{app-urls}

Below is a listing of the full URLs of websites mentioned in the text.

\begin{description}

\item[Census Bureau, Data Extraction Service]
  \href{http://www.census.gov/ftp/pub/DES/www/welcome.html}{
    http://www.census.gov/ftp/pub/DES/www/welcome.html}
\item[Estima (RATS)] \href{http://www.estima.com/}{
    http://www.estima.com/}
\item[Gnome desktop homepage] \href{http://www.gnome.org/}{
    http://www.gnome.org/}
\item[GNU Multiple Precision (GMP) library]
  \href{http://swox.com/gmp/}{ http://swox.com/gmp/}
\item[GNU Octave homepage] \href{http://www.octave.org/}{
    http://www.octave.org/}
\item[GNU R homepage] \href{http://www.r-project.org/}{
    http://www.r-project.org/}
\item[GNU R manual]
  \href{http://cran.r-project.org/doc/manuals/R-intro.pdf}{
    http://cran.r-project.org/doc/manuals/R-intro.pdf}
\item[Gnuplot homepage] \href{http://www.gnuplot.info/}{
    http://www.gnuplot.info/}
\item[Gnuplot manual] \href{http://ricardo.ecn.wfu.edu/gnuplot.html}{
    http://ricardo.ecn.wfu.edu/gnuplot.html}
\item[Gretl data page]
  \url{http://gretl.sourceforge.net/gretl_data.html}
\item[Gretl homepage] \href{http://gretl.sourceforge.net/}{
    http://gretl.sourceforge.net/}
\item[GTK+ homepage] \href{http://www.gtk.org/}{ http://www.gtk.org/}
\item[GTK+ port for win32]
  \href{http://user.sgic.fi/~tml/gimp/win32/}{
    http://www.gimp.org/~tml/gimp/win32/}
\item[Gtkextra homepage] \href{http://gtkextra.sourceforge.net/}{
    http://gtkextra.sourceforge.net/}
\item[InfoZip homepage]
  \href{http://www.info-zip.org/pub/infozip/zlib/}{
    http://www.info-zip.org/pub/infozip/zlib/}
\item[JRSoftware] \href{http://www.jrsoftware.org/}{
    http://www.jrsoftware.org/}
\item[Mingw (gcc for win32) homepage] \href{http://www.mingw.org/}{
    http://www.mingw.org/}
\item[Minpack] \href{http://www.netlib.org/minpack/}{
    http://www.netlib.org/minpack/}
\item[Penn World Table] \href{http://pwt.econ.upenn.edu/}{
    http://pwt.econ.upenn.edu/}
\item[Readline homepage]
  \href{http://cnswww.cns.cwru.edu/~chet/readline/rltop.html}{
    http://cnswww.cns.cwru.edu/~chet/readline/rltop.html}
\item[Readline manual]
  \href{http://cnswww.cns.cwru.edu/~chet/readline/readline.html}{
    http://cnswww.cns.cwru.edu/~chet/readline/readline.html}
\item[Xmlsoft homepage] \href{http://xmlsoft.org/}{
    http://xmlsoft.org/}

\end{description}


%%% Local Variables: 
%%% mode: latex
%%% TeX-master: "gretl-guide"
%%% End: 

