\chapter{What is a dataset?}
\label{chap:dataset}

A dataset is a memory area designed to hold the data you want to work
on, if any. It may be thought of a big global variable, containing a
(possibly huge) matrix of data and a hefty collection of metadata.

\app{R} users may think that a dataset is similar to what you get when
you \texttt{attach} a data frame in \app{R}. Not really: in hansl, you
cannot have more than one dataset open at the same time. That's why we
talk about \emph{the} dataset.

When a dataset is present in memory (that is, ``open''), a number of
objects become available for your hansl script in a transparent and
convenient way. Of course, the data themselves: the columns of the
dataset matrix are called \emph{series}, which will be described in
section \ref{sec:series}; sometimes, you will want to organize one or
more series in a \emph{list} (section \ref{sec:lists}). Additionally,
you have the possibility of using, as read-only global variables, some
scalars or matrices, such as the number of observations, the number of
variables, the nature of your dataset (cross-sectional, time series or
panel), and so on. These are called \emph{accessors}, and will be
discussed in section \ref{sec:accessors}.

You can open a dataset by reading data from a disk file, via the
\cmd{open} command, or by creating one from scratch.

\section{Creating a dataset from scratch}

The primary commands in this context are \cmd{nulldata} and
\cmd{setobs}.  For example:
\begin{code}
set echo off
set messages off

set seed 443322           # initialize the random number generator
nulldata 240              # stipulate how long your series will be
setobs 12 1995:1          # define as monthly dataset, starting Jan 1995   
\end{code}

For more details, see \GUG\ and the \GCR\ for the \cmd{nulldata} and
\cmd{setobs} commands. The only important thing to say at this point,
however, is that you can resize your dataset and/or change some of its
characteristics, such as its periodicity, at nearly any point inside
your script if necessary.

Once your dataset is in place, you can start populating it with
series, either by reading them from files or by generating them via
appropriate commands and functions.

\section{Reading a dataset from a file}

The primary commands here are \cmd{open}, \cmd{append} and \cmd{join}.

The \cmd{open} command is what you'll want to use in most cases. It
handles transparently a wide variety of formats (native, CSV,
spreadsheet, data files produced by other packages such as
\textsf{Stata}, \textsf{Eviews}, \textsf{SPSS} and \textsf{SAS}) and
it also takes care of setting up the dataset for you automatically.
\begin{code}
  open mydata.gdt    # native format
  open yourdata.dta  # Stata format
  open theirdata.xls # Excel format
\end{code}

\begin{itemize}
\item native format
\item import capabilities (from URLs as well)
\item nifty tricks with csv data (eg \cmd{join}).
\end{itemize}

You can also use \cmd{open} to read stuff off the Internet, by using a URL
instead of a filename, as in
\begin{code}
  open http://someserver.com/somedata.csv
\end{code}

\section{Saving datasets}

\begin{itemize}
\item Storing with \cmd{store} (binary format vs XML)
\item Using \cmd{store} to export to CSV
\item (maybe) using \cmd{printf} and \cmd{outfile} if you need to do
  something unusual; real-life cases?
\end{itemize}

\section{The \cmd{smpl} command}

Having opened a dataset somehow, the \cmd{smpl} command allows you to
discard observations selectively, so that your series will contain
only the observations you want (automatically changing the dimension
of the dataset in the process). See chapter 4 in \GUG\ for further
information.\footnote{Users with a Stata background may find the hansl
  way of doing things a little disconcerting at first. In hansl, you
  first restrict your sample through the \cmd{smpl} command, which
  applies until further notice, then you do what you have to. There is
  no equivalent to Stata's \texttt{if} clause to commands.}

There are basically three variants to the \cmd{smpl} command:
\begin{enumerate}
\item Selecting a contiguous subset of observations: this will be
  mostly useful with time-series datasets. For example:
  \begin{code}
    smpl 4 122            # select observations for 4 to 122
    smpl 1984:1 2008:4    # the so-called "Great Moderation" period
    smpl 2008-01-01 ;     # form January 1st, 2008 onwards
  \end{code}
\item Selecting observations on the basis of some criterion: this is
  typically what you want with cross-sectional datasets. Example:
  \begin{code}
    smpl male == 1 --restrict                # males only
    smpl male == 1 && age < 30 --restrict    # just the young guys
    smpl employed --dummy                    # via a dummy variable
  \end{code}
  Note that, in this context, restrictions go ``on top'' of previous
  ones. In order to start from scratch, you either reset the full
  sample via \texttt{smpl full} or use the \option{replace} option
  along with \option{restrict}.
\item Restricting the active dataset to some observations so that a
  certain effect is achieved automatically: for example, drawing a
  random subsample, or ensuring that all rows that have missing
  observations are automatically excluded. This is achieved via the
  \option{no-missing}, \option{contiguous}, and \option{random}
  options.
\end{enumerate}

In the context of panel datasets, some extra qualifications have to be
made; see \GUG.

\section{Dataset accessors}
\label{sec:accessors}

TO BE WRITTEN.

%%% Local Variables: 
%%% mode: latex
%%% TeX-master: "hansl-primer"
%%% End: 
