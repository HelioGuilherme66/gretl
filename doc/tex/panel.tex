\chapter{Panel data}
\label{chap-panel}

\section{Panel structure}
\label{panel-structure}

Panel data are inherently three dimensional --- the dimensions being
variable, cross-sectional unit, and time-period.  For representation
in a textual computer file (and also for gretl's internal
calculations) these three dimensions must somehow be flattened into
two.  This ``flattening'' involves taking layers of the data that
would naturally stack in a third dimension, and stacking them in the
vertical dimension.\app{Gretl} always expects data to be arranged ``by
observation'', that is, such that each row represents an observation
(and each variable occupies one and only one column).  In this context
the flattening of a panel data set can be done in either of two ways:
\begin{itemize}
\item Stacked cross-sections: the successive vertical blocks each
  comprise a cross-section for a given period.
\item Stacked time-series: the successive vertical blocks each
  comprise a time series for a given cross-sectional unit.
\end{itemize}

You may use whichever arrangement is more convenient. Under
\app{gretl}'s \textsf{Sample} menu you will find an item ``Restructure
panel'' which allows you to convert from stacked cross section form to
stacked time series or vice versa.When you import panel data into
\app{gretl} from a spreadsheet or comma separated format, the panel
nature of the data will not be recognized automatically (most likely
the data will be treated as ``undated'').  A panel interpretation can
be imposed on the data in either of two ways.

\begin{enumerate}
\item Use the GUI menu item ``Sample, Dataset structure''.  In the
  first dialog box that appears, select ``Panel''.  In the next
  dialog, make a selection between stacked time series or stacked
  cross sections depending on how your data are organized.  In the
  next, supply the number of cross-sectional units in the dataset.
  Finally, check the specification that is shown to you, and confirm
  the change if it looks OK.
\item Use the script command \cmd{setobs}.  For panel data this
  command takes the form \verb+setobs+ freq \verb+1:1+ structure,
  where freq is replaced by the ``block size'' of the data (that is,
  the number of periods in the case of stacked time series, or the
  number of cross-sectional units in the case of stacked
  cross-sections) and structure is either \verb+--stacked-time-series+
  or \verb+--stacked-cross-section+.  Two examples are given below:
  the first is suitable for a panel in the form of stacked time series
  with observations from 20 periods; the second for stacked cross
  sections with 5 cross-sectional units.
        
\begin{code}
            setobs 20 1:1 --stacked-time-series
            setobs 5 1:1 --stacked-cross-section
\end{code}

\end{enumerate}

\section{Dummy variables}
\label{dummies}

In a panel study you may wish to construct dummy variables of one or
both of the following sorts: (a) dummies as unique identifiers for the
cross-sectional units, and (b) dummies as unique identifiers for the
time periods.  The former may be used to allow the intercept of the
regression to differ across the units, the latter to allow the
intercept to differ across periods.Three special functions are
available to create such dummies.  These are found under the ``Data,
Add variables'' menu in the GUI, or under the \cmd{genr} command in
script mode or \app{gretlcli}.

\begin{enumerate}
\item ``periodic dummies'' (script command \cmd{genr dummy}).  This
  command creates a set of dummy variables identifying the periods.
  The variable \verb+dummy_1+ will have value 1 in each row
  corresponding to a period 1 observation, 0 otherwise; \verb+dummy_2+
  will have value 1 in each row corresponding to a period 2
  observation, 0 otherwise; and so on.
\item ``unit dummies'' (script command \cmd{genr unitdum}).  This
  command creates a set of dummy variables identifying the
  cross-sectional units.  The variable \verb+du_1+ will have value 1
  in each row corresponding to a unit 1 observation, 0 otherwise;
  \verb+du_2+ will have value 1 in each row corresponding to a unit 2
  observation, 0 otherwise; and so on.
\item ``panel dummies'' (script command \cmd{genr paneldum}).  This
  creates both period and unit dummy variables. The unit dummies are
  named \verb+du_1+, \verb+du_2+ and so on, while the period dummies
  are named \verb+dt_1+, \verb+dt_2+, etc.
\end{enumerate}

If a panel data set has the \verb+YEAR+ of the observation entered as
one of the variables you can create a periodic dummy to pick out a
particular year, e.g. \cmd{genr dum = (YEAR=1960)}.  You can also
create periodic dummy variables using the modulus operator,
\verb+%+.  For instance, to create a dummy with
value 1 for the first observation and every thirtieth observation
thereafter, 0 otherwise, do
\begin{code}
      genr index 
      genr dum = ((index-1)%30) = 0
\end{code}

\section{Lags and differences with panel data}
\label{panel-lagged}

If the time periods are evenly spaced you may want to use lagged
values of variables in a panel regression; you may also with to
construct first differences of variables of interest.Once a dataset is
properly identified as a panel (as described in the previous section),
\app{gretl} will handle the generation of such variables correctly.
For example the command \verb+genr x1_1 = x1(-1)+ will create a
variable that contains the first lag of \verb+x1+ where available, and
the missing value code where the lag is not available.  When you run a
regression using such variables, the program will automatically skip
the missing observations.

\section{Pooled estimation}
\label{pooled-est}

There is a special purpose estimation command for use with panel data,
the ``Pooled OLS'' option under the \textsf{Model} menu. This command
is available only if the data set is recognized as a panel.  To take
advantage of it, you should specify a model without any dummy
variables representing cross-sectional units.  The routine presents
estimates for straightforward pooled OLS, which treats cross-sectional
and time-series variation at par.  This model may or may not be
appropriate.  Under the \textsf{Tests} menu in the model window, you
will find an item ``panel diagnostics'', which tests pooled OLS
against the principal alternatives, the fixed effects and random
effects models.The fixed effects model adds a dummy variable for all
but one of the cross-sectional units, allowing the intercept of the
regression to vary across the units.  An \emph{F}-test for the joint
significance of these dummies is presented: if the p-value for this
test is small, that counts against the null hypothesis (that the
simple pooled model is adequate) and in favor of the fixed effects
model.The random effects model, on the other hand, decomposes the
residual variance into two parts, one part specific to the
cross-sectional unit or ``group'' and the other specific to the
particular observation.  (This estimator can be computed only if the
panel is ``wide'' enough, that is, if the number of cross-sectional
units in the data set exceeds the number of parameters to be
estimated.)  The Breusch--Pagan LM statistic tests the null hypothesis
(again, that the pooled OLS estimator is adequate) against the random
effects alternative.It is quite possible that the pooled OLS model is
rejected against both of the alternatives, fixed effects and random
effects. How, then, to assess the relative merits of the two
alternative estimators?  The Hausman test (also reported, provided the
random effects model can be estimated) addresses this issue.  Provided
the unit- or group-specific error is uncorrelated with the independent
variables, the random effects estimator is more efficient than the
fixed effects estimator; otherwise the random effects estimator is
inconsistent, in which case the fixed effects estimator is to be
preferred.  The null hypothesis for the Hausman test is that the
group-specific error is not so correlated (and therefore the random
effects model is preferable).  Thus a low p-value for this tests
counts against the random effects model and in favor of fixed
effects.For a rigorous discussion of this topic, see Greene (2000),
chapter 14.

\section{Illustration: the Penn World Table}
\label{PWT}

The Penn World Table (homepage at
\href{http://pwt.econ.upenn.edu/}{pwt.econ.upenn.edu}) is a rich
macroeconomic panel dataset, spanning 152 countries over the years
1950--1992.  The data are available in \app{gretl} format; please see
the \app{gretl}
\href{http://gretl.sourceforge.net/gretl_data.html}{data site} (this
is a free download, although it is not included in the main
\app{gretl} package).Example \ref{examp-pwt} below opens
\verb+pwt56_60_89.gdt+, a subset of the pwt containing data on 120
countries, 1960--89, for 20 variables, with no missing observations
(the full data set, which is also supplied in the pwt package for
\app{gretl}, has many missing observations). Total growth of real GDP,
1960--89, is calculated for each country and regressed against the
1960 level of real GDP, to see if there is evidence for
``convergence'' (i.e. faster growth on the part of countries starting
from a low base).

\begin{script}[htbp]
  \caption{Use of the Penn World Table}
  \label{examp-pwt}

\begin{code}
          open pwt56_60_89.gdt 
          # for 1989 (the last obs), lag 29 gives 1960, the first obs 
          genr gdp60 = RGDPL(-29) 
          # find total growth of real GDP over 30 years
          genr gdpgro = (RGDPL - gdp60)/gdp60
          # restrict the sample to a 1989 cross-section 
          smpl --restrict YEAR=1989 
          # convergence: did countries with a lower base grow faster?  
          ols gdpgro const gdp60 
          # result: No! Try an inverse relationship?
          genr gdp60inv = 1/gdp60 
          ols gdpgro const gdp60inv 
          # no again.  Try treating Africa as special? 
          genr afdum = (CCODE = 1)
          genr afslope = afdum * gdp60 
          ols gdpgro const afdum gdp60 afslope 
\end{code}
\end{script}


%%% Local Variables: 
%%% mode: latex
%%% TeX-master: "gretl-guide"
%%% End: 

