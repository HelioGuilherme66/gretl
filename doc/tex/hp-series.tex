\chapter{Series and lists}
\label{chap:series-etc}

Scalars, matrices and strings can be used in a hansl script at any
point; series and lists, on the other hand, are inherently tied to a
dataset and therefore can be used only when a dataset is currently
open.

\section{The \texttt{series} type}
\label{sec:series}
 
Series are just what any applied economist would call ``variables'',
that is, repeated observations of a given quantity; a dataset is an
ordered array of series, complemented by additional information, such
as the nature of the data (time-series, cross-section or panel),
descriptive labels for the series and/or the observations, source
information and so on. Series are the basic data type on which gretl's
built-in estimation commands depend.

The series belonging to a dataset are named via standard hansl
identifiers (strings of maximum length 31 characters as described
above). In the context of commands that take series as arguments,
series may be referenced either by name or by \emph{ID number}, that
is, the index of the series within the dataset. Position 0 in a
dataset is always taken by the automatic ``variable'' known as
\texttt{const}, which is just a column of 1s. The IDs of the actual
data series can be displayed via the \cmd{varlist} command. (But note
that in \textit{function calls}, as opposed to commands, series must
be referred to by name.)  A detailed description of how a dataset
works can be found in chapter 4 of \GUG.

Some basic rules regarding series follow:
\begin{itemize}
\item If \texttt{lngdp} belongs to a time series or panel dataset,
  then the syntax \texttt{lngdp(-1)} yields its first lag, and
  \texttt{lngdp(+1)} its first lead.
\item To access individual elements of a series, you use square
  brackets enclosing
  \begin{itemize}
  \item the progressive (1-based) number of the observation you want,
    as in \verb|lngdp[15]|, or
  \item the corresponding date code in the case of time-series data,
    as in \verb|lngdp[2008:4]| (for the 4th quarter of 2008), or
  \item the corresponding observation marker string, if the dataset
    contains any, as in \verb|GDP["USA"]|.
  \end{itemize}
\end{itemize}

The rules for assigning values to series are just the same as for
other objects, so the following examples should be self-explanatory:
\begin{code}
  series k = 3         # implicit conversion from scalar; a constant series
  series x = normal()  # pseudo-rv via a built-in function
  series s = a/b       # element-by-element operation on existing series

  series movavg = 0.5*(x + x(-1)) # using lags
  series y[2012:4] = x[2011:2]    # using individual data points
  series x2000 = 100*x/x[2000:1]  # constructing an index
\end{code}

\tip{In hansl, you don't have separate commands for \emph{creating}
  series and \emph{modifying} them. Other popular packages make this
  distinction, but we still struggle to understand why this is
  supposed to be useful.}

\subsection{Converting series to or from matrices}

The reason why hansl provides a specific series type, distinct from
the matrix type, is historical. However, is also a very convenient
feature.  Operations that are typically performed on series in applied
work can be awkward to implement using ``raw'' matrices---for example,
the computation of leads and lags, or regular and seasonal
differences; the treatment of missing values; the addition of
descriptive labels, and so on.

Anyway, it is straightforward to convert data in either direction
between the series and matrix types.
\begin{itemize}
\item To turn series into matrices, you use the curly braces syntax,
  as in
  \begin{code}
    matrix MACRO = {outputgap, unemp, infl}
  \end{code}
  where you can also use lists; the number of rows of the resulting
  matrix will depend on your currently selected sample.
\item To turn matrices into series, you can just use matrix columns,
  as in
  \begin{code}
    series y = my_matrix[,4]
  \end{code}
  But note that this will work only if the number of rows in
  \texttt{my\_matrix} matches the length of the dataset (or the
  currently selected sample range).
\end{itemize}

Also note that the \cmd{lincomb} and \cmd{filter} functions are quite
useful for creating and manipulating series in complex ways without
having to convert the data to matrix form (which could be
computationally costly with large datasets).

\subsection{The ternary operator with series}

Consider this assignment:

\begin{code}
  worker_income = employed ? income : 0
\end{code}

Here we assume that \texttt{employed} is a dummy series coding for
employee status. Its value will be tested for each observation in the
current sample range and the value assigned to \texttt{worker\_income}
at that observation will be determined accordingly. It is therefore
equivalent to the following much more verbose formulation (where
\dollar{t1} and \dollar{t2} are accessors for the start and end of the
sample range):
\begin{code}
series worker_income
loop i=$t1..$t2
    if employed[i]
        worker_income[i] = income[i]
    else
        worker_income[i] = 0
    endif
endloop
\end{code}

\section{The \texttt{list} type}
\label{sec:lists}
 
In hansl parlance, a \textit{list} is an array of integers,
representing the ID numbers of a set (in a loose sense of the word) of
series.  For this reason, the most common operations you perform on
lists are set operations such as addition or deletion of members,
union, intersection and so on. Unlike sets, however, hansl lists are
ordered, so individual list members can be accessed via the
\texttt{[]} syntax, as in \texttt{X[3]}.

There are several ways to assign values to a list.  The most basic
sort of expression that works in this context is a space-separated
list of series, given either by name or by ID number.  For example,
\begin{code}
list xlist = 1 2 3 4
list reglist = income price 
\end{code}
An empty list is obtained by using the keyword \texttt{null}, as in
\begin{code}
list W = null  
\end{code}
or simply by bare declaration. Some more special forms (for example,
using wildcards) are described in \GUG.

The main idea is to use lists to group, under one identifier, one or
more series that logically belong together somehow (for example, as
explanatory variables in a model). So, for example,
\begin{code}
list xlist = x1 x2 x3 x4
ols y 0 xlist
\end{code}
is an idiomatic way of specifying the OLS regression that could also
be written as
\begin{code}
ols y 0 x1 x2 x3 x4
\end{code}
Note that we used here the convention, mentioned in section
\ref{sec:series}, by which a series can be identified by its ID number
when used as an argument to a command, typing \texttt{0} instead
of \texttt{const}.

Lists can be concatenated, as in as in \texttt{list L3 = L1 L2} (where
\texttt{L1} and \texttt{L2} are names of existing lists). This will
not necessarily do what you want, however, since the resulting list
may contain duplicates. It's more common to use the following set
operations:

\begin{center}
  \begin{tabular}{rl}
    \textbf{Operator} & \textbf{Meaning} \\
    \hline
    \verb,||, & Union \\
    \verb|&&| & Intersection \\
    \verb|-|  & Set difference \\
    \hline
  \end{tabular}
\end{center}

So for example, if \texttt{L1} and \texttt{L2} are existing lists,
after running the following code snippet
\begin{code}
  list UL = L1 || L2 
  list IL = L1 && L2
  list DL = L1 - L2
\end{code}
the list \texttt{UL} will contain all the members of \texttt{L1}, plus
any members of \texttt{L2} that are not already in \texttt{L1};
\texttt{IL} will contain all the elements that are present in both
\texttt{L1} and \texttt{L2} and \texttt{DL} will contain all the
elements of \texttt{L1} that are not present in \texttt{L2}. 

To \textit{append} or \textit{prepend} variables to an existing list,
we can make use of the fact that a named list stands in for a
``longhand'' list.  For example, assuming that a list \texttt{xlist}
is already defined (possibly as \texttt{null}), we can do
\begin{code}
list xlist = xlist 5 6 7
xlist = 9 10 xlist 11 12
\end{code}
 
Another option for appending terms to, or dropping terms from, an
existing list is to use \texttt{+=} or \texttt{-=}, respectively, as
in
\begin{code}
xlist += cpi
zlist -= cpi
\end{code}
A nice example of the above is provided by a common idiom: you may
see in hansl scripts something like
\begin{code}
  list C -= const
  list C = const C
\end{code}
which ensures that the series \texttt{const} is included (exactly
once) in the list \texttt{C}, and comes first.

\subsection{Converting lists to or from matrices}

The idea of converting from a list, as defined above, to a matrix may
be taken in either of two ways. You may want to turn a list into a
matrix (vector) by filling the latter with the ID numbers contained in
the former, or rather to create a matrix whose columns contain the
series to which the ID numbers refer. Both interpretations are
legitimate (and potentially useful in different contexts) so hansl
lets you go either way.

If you assign a list to a matrix, as in
\begin{code}
  list L = moo foo boo zoo
  matrix A = L
\end{code}
the matrix \texttt{A} will contain the ID numbers of the four series
as a row vector. This operation goes both ways, so the statement
\begin{code}
  list C = seq(7,10)
\end{code}
is perfectly valid (provided, of course, that you have at least 10
series in the currently open dataset).

If instead you want to create a data matrix from the series which
belong to a given list, you have to enclose the list name in curly
brackets, as in
\begin{code}
  matrix X = {L}
\end{code}

\subsection{The \texttt{foreach} loop variant with lists}

Lists can be used as the ``catalogue'' in the \texttt{foreach} variant
of the \cmd{loop} construct (see section \ref{sec:loop-foreach}). This
is especially handy when you have to perform some operation on
multiple series. For example, the following syntax can be used to
calculate and print the mean of each of several series:
\begin{code}
list X = age income experience
loop foreach i X
    printf "mean($i) = %g\n", mean($i)
endloop
\end{code}

%%% Local Variables: 
%%% mode: latex
%%% TeX-master: "hansl-primer"
%%% End: 
