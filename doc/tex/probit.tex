\chapter{Discrete and censored dependent variables}
\label{chap:discr-models}

\section{Logit and probit models}
\label{sec:logit-probit}

It often happens that one wants to specify and estimate a model in
which the dependent variable is not continuous, but discrete. A
typical example is a model in which the dependent variable is the
occupational status of an individual (1 = employed, 0 = unemployed). A
convenient way of formalizing this situation is to consider the
variable $y_i$ as a Bernoulli random variable and analyze its
distribution conditional on the explanatory variables $x_i$.  That is,
%
\begin{equation}
  \label{eq:qr-Bernoulli}
  y_i \left\{ 
    \begin{array}{ll} 
      1 & P_i \\ 0 & 1 - P_i 
    \end{array}
    \right.
\end{equation}
%
where $P_i = P(y_i = 1 | x_i) $ is a given function of the explanatory
variables $x_i$.

In most cases, the function $P_i$ is a cumulative distribution
function $F$, applied to a linear combination of the $x_i$s. In the
probit model, the normal cdf is used, while the logit model employs
the logistic function $\Lambda()$. Therefore, we have
%
\begin{eqnarray}
  \label{eq:qr-link}
  \textrm{probit} & \qquad & P_i = F(z_i) = \Phi(z_i)  \\
  \textrm{logit}  & \qquad & P_i = F(z_i) = \Lambda(z_i) = \frac{1}{1 + e^{-z_i}} \\
  & &z_i = \sum_{j=1}^k x_{ij} \beta_j
\end{eqnarray}
%
where $z_i$ is commonly known as the \emph{index} function. Note that
in this case the coefficients $\beta_j$ cannot be interpreted as the
partial derivatives of $E(y_i | x_i)$ with respect to
$x_{ij}$.  However, for a given value of $x_i$ it is possible to
compute the vector of ``slopes'', that is
\[
  \mathrm{slope}_j(\bar{x}) = \left. \pder{F(z)}{x_j} \right|_{z =
    \bar{z}}
\]
\app{Gretl} automatically computes the slopes, setting each
explanatory variable at its sample mean.

Another, equivalent way of thinking about this model is in terms of
an unobserved variable $y^*_i$ which can be described thus:
%
\begin{equation}
  \label{eq:qr-latent}
  y^*_i = \sum_{j=1}^k x_{ij} \beta_j + \varepsilon_i = z_i  +
  \varepsilon_i 
\end{equation}
%
We observe $y_i = 1$ whenever $y^*_i > 0$ and $y_i = 0$ otherwise. If
$\varepsilon_i$ is assumed to be normal, then we have the probit
model. The logit model arises if we assume that the density function
of $\varepsilon_i$ is
%
\[
  \lambda(\varepsilon_i) =
  \pder{\Lambda(\varepsilon_i)}{\varepsilon_i} =
  \frac{e^{-\varepsilon_i}}{(1 + e^{-\varepsilon_i})^2}
\]

Both the probit and logit model are estimated in \app{gretl} via
maximum likelihood; since the score equations do not have a closed
form solution, numerical optimization is used. However, in most cases
this is totally transparent to the user, since usually only a few
iterations are needed to ensure convergence. The \texttt{--verbose}
switch can be used to track the maximization algorithm.

\begin{script}[htbp]
  \caption{Estimation of simple logit and probit models}
  \label{simple-QR}
\begin{scode}
open greene19_1

logit GRADE const GPA TUCE PSI
probit GRADE const GPA TUCE PSI
\end{scode}
\end{script}

As an example, we reproduce the results given in Greene (2000),
chapter 21, where the effectiveness of a program for teaching
economics is evaluated by the improvements of students' grades.
Running the code in example \ref{simple-QR} gives the following output:
\begin{code}

Model 1: Logit estimates using the 32 observations 1-32
Dependent variable: GRADE

      VARIABLE       COEFFICIENT        STDERROR      T STAT       SLOPE
                                                                  (at mean)
  const               -13.0213           4.93132      -2.641
  GPA                   2.82611          1.26294       2.238      0.533859   
  TUCE                  0.0951577        0.141554      0.672      0.0179755  
  PSI                   2.37869          1.06456       2.234      0.449339   

  Mean of GRADE = 0.344
  Number of cases 'correctly predicted' = 26 (81.2%)
  f(beta'x) at mean of independent vars = 0.189
  McFadden's pseudo-R-squared = 0.374038
  Log-likelihood = -12.8896
  Likelihood ratio test: Chi-square(3) = 15.4042 (p-value 0.001502)
  Akaike information criterion (AIC) = 33.7793
  Schwarz Bayesian criterion (BIC) = 39.6422
  Hannan-Quinn criterion (HQC) = 35.7227

           Predicted
             0    1
  Actual 0  18    3
         1   3    8

Model 2: Probit estimates using the 32 observations 1-32
Dependent variable: GRADE

      VARIABLE       COEFFICIENT        STDERROR      T STAT       SLOPE
                                                                  (at mean)
  const                -7.45232          2.54247      -2.931
  GPA                   1.62581          0.693883      2.343      0.533347   
  TUCE                  0.0517288        0.0838903     0.617      0.0169697  
  PSI                   1.42633          0.595038      2.397      0.467908   

  Mean of GRADE = 0.344
  Number of cases 'correctly predicted' = 26 (81.2%)
  f(beta'x) at mean of independent vars = 0.328
  McFadden's pseudo-R-squared = 0.377478
  Log-likelihood = -12.8188
  Likelihood ratio test: Chi-square(3) = 15.5459 (p-value 0.001405)
  Akaike information criterion (AIC) = 33.6376
  Schwarz Bayesian criterion (BIC) = 39.5006
  Hannan-Quinn criterion (HQC) = 35.581

           Predicted
             0    1
  Actual 0  18    3
         1   3    8

\end{code}

In this context, the \verb+$uhat+ accessor function takes a
special meaning: it returns generalized residuals as defined in
Gourieroux \textit{et al} (1987), which can be interpreted as unbiased
estimators of the latent disturbances $\varepsilon_t$. These are
defined as
%
\begin{equation}
  \label{eq:QR-genres}
  u_i = \left\{
    \begin{array}{ll}
      y_i - \hat{P}_i & \textrm{for the logit model} \\
      y_i\cdot \frac{\phi(\hat{z}_i)}{\Phi(\hat{z}_i)} - 
      ( 1 - y_i ) \cdot \frac{\phi(\hat{z}_i)}{1 - \Phi(\hat{z}_i)}
      & \textrm{for the probit model} \\
    \end{array}
    \right.
\end{equation}

Among other uses, generalized residuals are often used for diagnostic
purposes.  For example, it is very easy to set up an omitted variables
test equivalent to the familiar LM test in the context of a linear
regression; example \ref{QR-add} shows how to perform a variable
addition test.

\begin{script}[htbp]
  \caption{Variable addition test in a probit model}
  \label{QR-add}
\begin{scode}
open greene19_1

probit GRADE const GPA PSI
series u = $uhat 
%$
ols u const GPA PSI TUCE -q
printf "Variable addition test for TUCE:\n"
printf "Rsq * T = %g (p. val. = %g)\n", $trsq, pvalue(X,1,$trsq) 
\end{scode}
\end{script}

\subsection{Ordered models}
\label{sec:ordered}

These models are simple variations of ordinary logit/probit models,
and are usually applied in case the dependent variable is a discrete
and ordered measurement, not necessarily quantitative. For example,
this sort of model can be applied when the dependent variable is a
qualitative assessment like ``Good'', ``Average'' and ``Bad''.
Assuming we have $p$ categories, the probability that individual $i$
falls in the $j$-th category is given by
%
\begin{equation}
  \label{eq:QR-ordered}
  P(y_i = j | x_i) = \left\{
    \begin{array}{ll}
      F(z_i + \mu_0) & \textrm{for } j = 0 \\
      F(z_i + \mu_j) -  F(z_i + \mu_{j-1}) & \textrm{for } 0 < j < p \\
      1 -  F(z_i + \mu_{p-1}) & \textrm{for } j = p 
    \end{array}
    \right.
\end{equation}
%
The unknown parameters $\mu_j$ are called the ``cutoff
points'' and are estimated together with the $\beta$s. For
identification purposes, $\mu_0$ is assumed to be 0. In terms of the
unobserved variable $y^*_i$, the model can be equivalently cast as
$P(y_i = j | x_i) = P(\mu_{j-1} \le y^*_i < \mu_j)$. 

\begin{script}[htbp]
  \caption{Ordered probit model}
  \label{ex:oprobit}
\begin{scode}
open pension.gdt
series pctstck = pctstck/50
discrete pctstck
probit pctstck const choice age educ female black married finc25 finc35 \
  finc50 finc75 finc100 finc101 wealth89 prftshr
\end{scode}
\end{script}

In order to apply these models, the dependent variable must be marked
as discrete and its lowest value must be 0. Example \ref{ex:oprobit}
reproduces the estimation given in chap. 15 of Wooldridge (2002a). Note
that \app{gretl} does not provide a separate command for ordered
models: the \texttt{logit} and \texttt{probit} commands automatically
estimate the ordered version if the dependent variable is not binary
(provided it has already been marked as discrete).

After estimating ordered models, the \verb+$uhat+ accessor yields
generalized residuals as in binary models; additionally, the
\verb+$yhat+ accessor function returns $\hat{z}_i$, so it is
possible to compute an unbiased estimator of the latent variable
$y^*_i$ simply by adding the two together.

\subsection{Multinomial logit}
\label{sec:mlogit}

When the dependent variable is not binary and does not have a natural
ordering, \emph{multinomial} models are used. \app{Gretl} does not
provide a native implementation of these yet, but simple models can be
handled via the \texttt{mle} command (see chapter \ref{chap:mle}). We
give here an example of a multinomial logit model.  Let the dependent
variable, $y_i$, take on integer values $0,1,\dots p$.  The
probability that $y_i = k$ is given by
\[
  P(y_i = k |  x_i) = \frac{\exp(x_i \beta_k)}{\sum_{j=0}^p \exp(x_i \beta_j)}
\]
For the purpose of identification one of the outcomes must be taken as
the ``baseline''; it is usually assumed that $\beta_0 = 0$, in which case
\[
  P(y_i = k |  x_i) = \frac{\exp(x_i \beta_k)}{1 + \sum_{j=1}^p \exp(x_i \beta_j)} 
\]
and
\[
  P(y_i = 0 |  x_i) = \frac{1}{1 + \sum_{j=1}^p \exp(x_i \beta_j)} .
\]

Example~\ref{ex:mlogit} reproduces Table 15.2 in Wooldridge (2002a),
based on data on career choice from Keane and Wolpin (1997).  The
dependent variable is the occupational status of an individual (0 = in
school; 1 = not in school and not working; 2 = working), and the
explanatory variables are education and work experience (linear and
square) plus a ``black'' binary variable.  The full data set is a
panel; here the analysis is confined to a cross-section for 1987.  For
explanations of the matrix methods employed in the script, see
chapter~\ref{chap:matrices}.

\begin{script}[htbp]
  \caption{Multinomial logit}
  \label{ex:mlogit}
\begin{scode}
function mlogitlogprobs(series y, matrix X, matrix theta)

  scalar n = max(y)
  scalar k = cols(X)
  matrix b = mshape(theta,k,n)

  matrix tmp = X*b
  series ret = -ln(1 + sumr(exp(tmp)))

  loop for i=1..n --quiet
    series x = tmp[,i]
    ret += (y=$i) ? x : 0 
  end loop

  return series ret

end function

open Keane.gdt
status = status-1 # dep. var. must be 0-based
smpl (year=87 & ok(status)) --restrict

matrix X = { educ exper expersq black const }
scalar k = cols(X)
matrix theta = zeros(2*k, 1)

mle loglik = lp
  series lp = mlogitlogprobs(status,X,theta)
  params theta
end mle --verbose --hessian
\end{scode}
%$
\end{script}


\section{The Tobit model}
\label{sec:tobit}

The Tobit model is used when the dependent variable of a model is
\emph{censored}.\footnote{We assume here that censoring occurs from
  below at 0. Censoring from above, or at a point different from zero,
  can be rather easily handled by re-defining the dependent variable
  appropriately. The more general case of two-sided censoring is not
  handled by \app{gretl} via a native command yet, but it is possible
  to estimate such models using the \texttt{mle} command (see chapter
  \ref{chap:mle}).}  
Assume a latent variable $y^*_i$ can be described
as
%
\[
  y^*_i = \sum_{j=1}^k x_{ij} \beta_j + \varepsilon_i ,
\]
%
where $\varepsilon_i \sim N(0,\sigma^2)$. If $y^*_i$ were observable,
the model's parameters could be estimated via ordinary least squares.
On the contrary, suppose that we observe $y_i$, defined as
%
\begin{equation}
  \label{eq:tobit}
  y_i \left\{ 
    \begin{array}{ll} 
      y^*_i & \mathrm{for} \quad y^*_i > 0 \\ 
      0 & \mathrm{for} \quad y^*_i \le 0 
    \end{array}
    \right. 
\end{equation}
%
In this case, regressing $y_i$ on the $x_i$s does not yield
consistent estimates of the parameters $\beta$, because the
conditional mean $E(y_i|x_i)$ is not equal to $\sum_{j=1}^k x_{ij}
\beta_j$.  It can be shown that restricting the sample to non-zero
observations would not yield consistent estimates either. The solution
is to estimate the parameters via maximum likelihood. The syntax is
simply
%
\begin{code}
tobit depvar indvars
\end{code}

As usual, progress of the maximization algorithm can be tracked via
the \texttt{--verbose} switch, while \verb+$uhat+ returns the
generalized residuals.

An important difference between the Tobit estimator and OLS is that
the consequences of non-normality of the disturbance term are much
more severe: non-normality implies inconsistency for the Tobit
estimator. For this reason, the output for the tobit model includes
the Chesher--Irish (1987) test for normality by default.

\subsection{Generalized Tobit model}
\label{sec:heckit}

In the so-called ``Tobit II'' model, there are two latent variables:
%
\begin{eqnarray}
  \label{eq:heckit1}
  y^*_i & = & \sum_{j=1}^k x_{ij} \beta_j + \varepsilon_i \\
  \label{eq:heckit2}
  s^*_i & = & \sum_{j=1}^p z_{ij} \gamma_j + \eta_i 
\end{eqnarray}
%
and the observation rule is given by
%
\begin{equation}
  \label{eq:tobitII}
  y_i \left\{ 
    \begin{array}{ll} 
      y^*_i & \mathrm{for} \quad s^*_i > 0 \\ 
      0 & \mathrm{for} \quad s^*_i \le 0 
    \end{array}
    \right. 
\end{equation}

One of the most popular applications of this model in econometrics is
a wage equation coupled with a labor force participation equation: we
only observe the wage for the employed. If $y^*_i$ and $s^*_i$ were
(conditionally) independent, there would be no reason not to use OLS
for estimating equation (\ref{eq:heckit1}); otherwise, OLS does not
yield consistent estimates of the parameters $\beta_j$.

A widely used estimator is the so-called \emph{Heckit} estimator,
named after Heckman (1979). The procedure can be briefly outlined as
follows: first, a probit model is fit on equation (\ref{eq:heckit2});
next, the generalized residuals are inserted in equation
(\ref{eq:heckit1}) to correct for the effect of sample selection.

Example \ref{ex:heckit} shows two estimates from the dataset used in
Mroz (1987): the first one replicates Table 22.7 in Greene (2003),
while the second one replicates table 17.1 in Wooldridge (2002a). Note
that the \texttt{heckit.inp} script (provided with \app{gretl} as an
example script) is invoked.

\begin{script}[htbp]
  \caption{Heckit model}
  \label{ex:heckit}
\begin{scode}
open mroz.gdt
include heckit.inp

genr EXP2 = AX^2
genr WA2 = WA^2
genr KIDS = (KL6+K618)>0

# Greene's specification

list X = const AX EXP2 WE CIT
list Z = const WA WA2 FAMINC KIDS WE

heckit(WW,X,LFP,Z)

# Wooldridge's specification

series NWINC = FAMINC - WW*WHRS
series lww = log(WW)
list X = const WE AX EXP2
list Z = X NWINC WA KL6 K618

heckit(lww,X,LFP,Z)
\end{scode}
\end{script}

% \section{Count data}
% \label{sec:poisson}

% also include example script for negative binomial (done in Verbeek
% example files).

%%% Local Variables: 
%%% mode: latex
%%% TeX-master: "gretl-guide"
%%% End: 