\chapter{Calendar dates}
\label{chap:calendar}

\section{Introduction}
\label{cal-intro}

Any software that aims to handle time-series data must have a good
built-in calendar. This is fairly straightforward in the current era,
with the Gregorian calendar now used universally for the dating of
socioeconomic observations. It is not so straightforward, however, if
when dealing with historical data that date back to before the
adoption of the Gregorian calendar in place of the Julian (an event
which first occurred in the principal Catholic countries in 1582 but
which took place at different dates in different countries over a span
of several centuries).

Gretl, like most data-oriented software, uses the Gregorian calendar
by default for all dates, thereby ensuring that dates are all
consecutive (the latter being a requirement of the ISO 8601 standard
for dates and times).\footnote{Gretl was not consistent in this regard
  prior to version 2017a: leap years were taken to be as defined by
  the Julian calendar prior to the adoption of the Gregorian calendar
  by Britain and its colonies in 1752.}

As you probably know, the Julian calendar adds a leap day (February
29) on each year that is divisible by 4 with no remainder. But this
over-compensates for the fact that a 365-day year is too short to keep
the calendar synchronized with the seasons. The Gregorian calendar
introduced a more complex rule which maintains better synchronization,
namely, each year divisible by 4 with no remainder is a leap year
\textit{unless} it's a centurial year (e.g.\ 1900) in which case it's
a leap year only if it is divisible by 400 with no remainder.  So the
years 1600 and 2000 were leap years on both calendars, but 1700, 1800,
and 1900 were leap years only on the Julian calendar.

The fact that the Julian calendar inserts leap days more frequently
means that the Julian date progressively (although very slowly) falls
behind the Gregorian date. For example, February 18 2017 (Gregorian)
is February 5 2017 on the Julian calendar. This further means that on
adoption of the Gregorian calendar it was necessary to skip several
days. In England, where the transition occurred in 1752, Wednesday
September 2 was directly followed by Thursday September 14.

In comparing calendars one wants to refer to a given day in terms that
are not specific to either calendar---but how to define a ``given
day''? This is accomplished by a count of days following some definite
event. Astronomers use the ``Julian day,'' whose count starts with a
particular confluence of astronomical events in the year known to the
Gregorian calendar (if one extrapolates it backward in time) as 4714
BC. Gretl uses a similar construction as a fulcrum, but the count of
what we call the ``epoch day'' starts at 1 on January 1, AD 1.  (This
is also the convention used by the \textsf{GLib} library, on which
gretl depends for most of its calendrical calculation.)

\section{Calendrical functions}
\label{cal-functions}

\section{Working with pre-Gregorian dates}
\label{cal-conversion}

Suppose we're working with historical data from eighteenth century
Britain (the same considerations would apply to sixteenth century
Spain or Italy). And we'd like to set up an historical calendar so
that we're sure we're putting archival data in the correct place.
Listing~\ref{ex:britain-1700s} shows how this can be done.

The first step is to find the epoch day corresponding to the Julian
date 1752-01-01 (which turns out to be 639551). Then we can build a
consistent series of epoch days, from which we can get both Julian and
Gregorian dates for 366 days starting on epoch day 639551. Finally, we
can construct a series, \texttt{hcal}, which switches calendar at the
right historical point.


\begin{script}[htbp]
  \caption{Make historical calendar for Britain in the 1700s}
  \label{ex:britain-1700s}
\begin{scode}
# 1752 was a leap year
nulldata 366
# give negative year to indicate Julian date
ed0 = epochday(-1752,1,1)
# consistent series of epoch day values
series ed = ed0 + index - 1
# Julian dates as ISO 8601 basic
series jdate = juldate(ed)
# Gregorian dates as ISO 8601 basic
series gdate = isodate(ed)
# Historical: switchover in September
series hcal = ed > epochday(-1752,9,2) ? gdate : jdate
# And let's take a look
print ed jdate gdate hcal -o
\end{scode}
\end{script}

Partial output from this script is shown below.

\begin{code}
              ed        jdate        gdate         hcal

  1       639551     17520101     17520112     17520101
  2       639552     17520102     17520113     17520102
  3       639553     17520103     17520114     17520103
  ...
244       639794     17520831     17520911     17520831
245       639795     17520901     17520912     17520901
246       639796     17520902     17520913     17520902
247       639797     17520903     17520914     17520914
248       639798     17520904     17520915     17520915
249       639799     17520905     17520916     17520916
\end{code}