\chapter{Calendar dates}
\label{chap:calendar}

\section{Introduction}
\label{cal-intro}

Any software that aims to handle time-series data must have a good
built-in calendar. This is fairly straightforward in the current era,
with the Gregorian calendar now used universally for the dating of
socioeconomic observations. It is not so straightforward, however,
when dealing with historical data recorded prior to the adoption of
the Gregorian calendar in place of the Julian, an event which first
occurred in the principal Catholic countries in 1582 but which took
place at different dates in different countries over a span of several
centuries.

Gretl, like most data-oriented software, uses the Gregorian calendar
by default for all dates, thereby ensuring that dates are all
consecutive (the latter being a requirement of the ISO 8601 standard
for dates and times).\footnote{Gretl was not consistent in this regard
  prior to version 2017a: leap years were taken to be as defined by
  the Julian calendar prior to the adoption of the Gregorian calendar
  by Britain and its colonies in 1752.}

As readers probably know, the Julian calendar adds a leap day
(February 29) on each year that is divisible by 4 with no
remainder. But this over-compensates for the fact that a 365-day year
is too short to keep the calendar synchronized with the seasons. The
Gregorian calendar introduced a more complex rule which maintains
better synchronization, namely, each year divisible by 4 with no
remainder is a leap year \textit{unless} it's a centurial year (e.g.\
1900) in which case it's a leap year only if it is divisible by 400
with no remainder.  So the years 1600 and 2000 were leap years on both
calendars, but 1700, 1800, and 1900 were leap years only on the Julian
calendar. While the average length of a Julian year is 365.25 days,
the Gregorian average is 365.2425 days. 

The fact that the Julian calendar inserts leap days more frequently
means that the Julian date progressively (although very slowly) falls
behind the Gregorian date. For example, February 18 2017 (Gregorian)
is February 5 2017 on the Julian calendar. On adoption of the
Gregorian calendar it was therefore necessary to skip several days. In
England, where the transition occurred in 1752, Wednesday September 2
was directly followed by Thursday September 14.

In comparing calendars one wants to refer to a given day in terms that
are not specific to either calendar---but how to define a ``given
day''? This is accomplished by a count of days following some definite
event. Astronomers use the ``Julian Day,'' whose count starts with a
particular coincidence of astronomical cycles in the year known to the
Gregorian calendar (if one extrapolates it backwards in time) as 4714
BC. Gretl uses a similar construction as a fulcrum, but the count of
what we call the ``epoch day'' starts at 1 on January 1, AD 1 (that
is, the first day of the Common Era), on the proleptic Gregorian
calendar.\footnote{The term ``proleptic,'' as applied to a calendar,
  indicates that it is extrapolated backwards or forwards relative to
  its period of actual historical use.} This is also the convention
used by the \textsf{GLib} library, on which gretl depends for most of
its calendrical calculation. Since \textsf{GLib} represents epoch days
as unsigned (32-bit) integers, this means that gretl does not support
dates prior to the Common---or, if you prefer, Christian---Era.

\section{Calendrical functions}
\label{cal-functions}

Gretl's calendrical functions are documented in the \GCR{}. In this
section we say a bit more about how these functions relate to each
other, and how they may be used to carry out some specific tasks.

A first point to note is that a daily date has three possible
representations within gretl. Two are associated with the ISO 8601
standard, namely the ``extended'' representation, \texttt{YYYY-MM-DD},
as for example \texttt{2017-02-19}, and the ``basic'' representation,
\texttt{YYYYMMDD} (for example \texttt{20170219}). As mentioned above,
such dates are by default taken to be relative to the (possibly
proleptic) Gregorian calendar. The third representation is as an
``epoch day'' (see above), for example \texttt{736379}, which is
calendar-independent and can therefore be used in converting from one
calendar to another.

\subsection{Decomposing a series of ``basic'' dates}

To generate from a series of dates in ISO 8601 basic format distinct
series holding year, month and day, the function \texttt{isoconv} can
be used. This function should be passed the original series followed
by ``pointers to'' the series to be filled out. For example, if we
have a series named \texttt{dates} in the prescribed format we might
do
%
\begin{code}
series y, m, d
isoconv(dates, &y, &m, &d)
\end{code}

This is mostly just a convenience function: provided the
\texttt{dates} input is valid on the (possibly proleptic) Gregorian
calendar it is equivalent to:
%
\begin{code}
series y = floor(dates/10000)
series m = floor((dates-10000*y)/100)
series d = dates - 10000*y - 100*m
\end{code}

However, there is some ``value added'': \texttt{isoconv} checks the
validity of the \texttt{dates} values. If the implied year, month and
day values for any \texttt{dates} observation do not correspond to a
valid Gregorian date in the Common Era, then all the derived series
will have value \texttt{NA} at that observation. (For example, the
implied month is not in the range 1--12, or the implied day is not in
the range of 1 to the number of days in the month, taking account of
Gregorian leap years.)

The \texttt{isoconv} function can also handle Julian dates, should
anyone have need for that facility. The convention is that if the
\textit{negative} of an ISO 8601 basic date is given as an argument,
the date is taken to be on the (possibly proleptic) Julian calendar.
Since dates BCE are not supported by gretl, hopefully this should not
give rise to ambiguity. The only difference from the standard usage of
\texttt{isoconv} is that Julian-only dates, such as \texttt{17000229},
are recognized as valid. So, for example,
%
\begin{code}
series y, m, d
isoconv(-dates, &y, &m, &d)
\end{code}
%
will accept a \texttt{dates} value of \texttt{17000229}, giving
(\texttt{y} = 1700, \texttt{m} = 2, \texttt{d} = 29), while this would
give \texttt{NA} values for year, month and day on the default
Gregorian calendar.

\subsection{Obtaining and using epoch days}

To convert from a Gregorian or Julian date to the corresponding epoch
day, you can use the \texttt{epochday} function, which takes as
arguments year, month and day on the given calendar. Similarly to
\texttt{isoconv}, the convention is that Gregorian dates are assumed
unless the year is given in the negative to flag a Julian date.
The following code fragment,
%
\begin{code}
edg = epochday(1700,1,1)
edj = epochday(-1700,1,1)
\end{code}
%
produces \texttt{edg} = 620548 and \texttt{edj} = 620558, indicating
that the two calendars differed by 10 days at the point in time
known as January 1, 1700, on the proleptic Gregorian calendar.

Taken together with the \texttt{isodate} and \texttt{juldate}
functions (which each take an epoch day argument and return an ISO
8601 basic date on, respectively, the Gregorian and Julian calendars),
\texttt{epochday} can be used to convert between the two calendars.
For example, what was the date in England (still on the Julian
calendar) on the day known to Italians as June 26, 1740 (Italy having
been on the Gregorian calendar since October 1582)?
%
\begin{code}
ed = epochday(1740,6,26)
english_date = juldate(ed)
printf "%.0f\n", english_date
\end{code}
%
We find that the English date was \texttt{17400615}, the 15th of June.
Working in the other direction, what Italian date corresponded to the
5th of November, 1740, in England?
%
\begin{code}
ed = epochday(-1740,11,5)
italian_date = isodate(ed)
printf "%.0f\n", italian_date
\end{code}
%
Answer: \texttt{17401116}; Guy Fawkes night in 1740 occurred on 
November 16 from the Italian point of view.

A further---and perhaps more practical---use of epoch days is checking
whether daily data is complete. Suppose we have what purports to be
7-day daily data on the Gregorian calendar with a starting date of
2015-01-01 and an ending date of 2016-12-31. How many observations
should there be? We can use the \texttt{dayspan} function, providing
as arguments the epoch-day values for the first and last dates and the
number of days per week.
\begin{code}
ed1 = epochday(2015,1,1)
ed2 = epochday(2016,12,31)
n = dayspan(ed1, ed2, 7)
\end{code}
%
We are told there should be \texttt{n} = 731 observations; if there
are fewer, there's something missing. If the days per week is less
than 7 (6 means skipping Sundays, 5 means skipping both Saturdays and
Sundays), \texttt{dayspan} returns the number of relevant days.


\section{Working with pre-Gregorian dates}
\label{cal-conversion}

Suppose we're working with historical data from eighteenth century
Britain (the same considerations would apply to sixteenth century
Spain or Italy). And we'd like to set up an historical calendar so
that we're sure we're putting archival data in the correct place.
For the purposes of this exercise we'll home in on the year 1752,
or more precisely the January to December period beginning on January
1, 1752 according to the Julian calendar that was in force at that
date. Listing~\ref{ex:britain-1752} shows how this can be done.

The first step is to find the epoch day corresponding to the Julian
date 1752-01-01 (which turns out to be 639551). Then we can build a
consistent series of epoch days, from which we can get both Julian and
Gregorian dates for 355 days starting on epoch day 639551. Note, 355
days because this was a short year: it was a leap year, but 11 days
were skipped in September in making the transition to the Gregorian
calendar. We can then construct a series, \texttt{hcal}, which
switches calendar at the right historical point.


\begin{script}[htbp]
  \caption{Historical calendar for Britain in 1752}
  \label{ex:britain-1752}
\begin{scode}
# 1752 was a short year on the British calendar!
nulldata 355
# give a negative year to indicate Julian date
ed0 = epochday(-1752,1,1)
# consistent series of epoch day values
series ed = ed0 + index - 1
# Julian dates as YYYYMMDD basic
series jdate = juldate(ed)
# Gregorian dates as YYYYMMDD
series gdate = isodate(ed)
# Historical: cut-over in September
series hcal = ed > epochday(-1752,9,2) ? gdate : jdate
# And let's take a look
print ed jdate gdate hcal -o
\end{scode}
\end{script}

Partial output from this script is shown below.
%
\begin{code}
              ed        jdate        gdate         hcal

  1       639551     17520101     17520112     17520101
  2       639552     17520102     17520113     17520102
...
245       639795     17520901     17520912     17520901
246       639796     17520902     17520913     17520902
247       639797     17520903     17520914     17520914
248       639798     17520904     17520915     17520915
...
355       639905     17521220     17521231     17521231
\end{code}

Notice that the observation labels (in the first column) are still
just index numbers. Perhaps we would prefer to have historical dates
in that role. To achieve this we can decompose the \texttt{hcal}
series into year, month and day, then use the special \texttt{genr
  markers} apparatus (see chapter~\ref{chap:datafiles}). Suitable
code is as follows:
\begin{code}
series y m d
isoconv(hcal, &y, &m, &d)
genr markers = "%04d-%02d-%02d", y, m, d
print ed jdate gdate hcal -o
\end{code}

After running this, we see (again, partial output)
%
\begin{code}
                     ed        jdate        gdate         hcal

1752-01-01       639551     17520101     17520112     17520101
1752-01-02       639552     17520102     17520113     17520102
...
1752-09-01       639795     17520901     17520912     17520901
1752-09-02       639796     17520902     17520913     17520902
1752-09-14       639797     17520903     17520914     17520914
1752-09-15       639798     17520904     17520915     17520915
...
1752-12-31       639905     17521220     17521231     17521231
\end{code}
