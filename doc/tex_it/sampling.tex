\chapter{Creare dei sotto-campioni}
\label{sampling}

\section{Introduzione}
\label{sample-intro}

Questo capitolo affronta alcune questioni correlate alla creazione di
sotto-campioni in un dataset.� possibile definire un sotto-campione
per un dataset in due modi diversi, che chiameremo rispettivamente
``impostazione'' del campione e ``restrizione'' del campione.
    
\section{Impostazione del campione}
\label{sample-set}

Per ``impostazione'' del campione, si intende la definizione di un
campione ottenuta indicando il punto iniziale e/o quello finale
dell'intervallo attuale del campione. Questa modalit� � usata
tipicamente con serie storiche; ad esempio se si hanno dati
trimestrali per l'intervallo da 1960:1 a 2003:4 e si vuole stimare una
regressione usando solo i dati degli anni '70, un comando adeguato �

\begin{code}
        smpl 1970:1 1979:4
\end{code}
      
Oppure se si vuole riservare la parte finale delle osservazioni
disponibili per eseguire una previsione "fuori dal campione", si pu�
usare il comando

\begin{code}
        smpl ; 2000:4
\end{code}
      
dove il punto e virgola significa ``mantenere inalterata
l'osservazione iniziale'' (e potrebbe essere usato in modo analogo al
posto del secondo parametro, indicando di mantenere inalterata
l'osservazione finale). Per ``inalterata'' in questo caso si intende
inalterata relativamente all'ultima impostazione eseguita con
\verb+smpl+, o relativamente all'intero dataset, se non � ancora stato
definito alcun sotto-campione in precedenza.  Ad esempio, dopo

\begin{code}
	smpl 1970:1 2003:4
	smpl ; 2000:4
\end{code}
      
l'intervallo del campione sar� da 1970:1 a 2000:4.� possibile anche
impostare l'intervallo del campione in modo incrementale o relativo:
in questo caso occorre indicare per il punto iniziale e finale uno
spostamento relativo, sotto forma di numero preceduto dal segno pi� o
dal segno meno (o da un punto e virgola per indicare nessuna
variazione). Ad esempio

\begin{code}
	smpl +1 ;
\end{code}
      
sposter� in avanti di un'osservazione l'inizio del campione,
mantenendo inalterata la fine del campione, mentre

\begin{code}
	smpl +2 -1
\end{code}

sposter� l'inizio del campione in avanti di due osservazioni e la fine
del campione indietro di una.Una caratteristica importante
dell'operazione di ``impostazione del campione'' descritta fin qui �
che il sotto-campione creato risulta sempre composto da un insieme di
osservazioni contigue. La struttura del dataset rimane quindi
inalterata: se si lavora su una serie trimestrale, dopo aver impostato
il campione la serie rimarr� trimestrale.
    

\section{Restrizione del campione}
\label{sample-restrict}

Per ``restrizione'' del campione si intende la definizione di un
campione ottenuta selezionando le osservazioni in base a un criterio
Booleano (logico), o usando un generatore di numeri casuali. Questa
modalit� � usata tipicamente con dati di tipo cross-section o panel.Si
supponga di avere dei dati di tipo cross-section che descrivono il
genere, il reddito e altre caratteristiche di un gruppo di individui e
si vogliano analizzare solo le donne presenti nel campione. Se si
dispone di una variabile dummy \verb+genere+, che vale 1 per gli
uomini e 0 per le donne, si potrebbe ottenere questo risultato con

\begin{code}
	smpl genere=0 --restrict
\end{code}
    
Oppure si supponga di voler limitare il campione di lavoro ai soli
individui con un reddito superiore ai 50.000 euro. Si potrebbe usare
\begin{code}
	smpl reddito>50000 --restrict
\end{code}

Qui sorge un problema: eseguendo in sequenza i due comandi visti
sopra, cosa conterr� il sotto-campione? Tutti gli individui con
reddito superiore a 50.000 euro o solo le donne con reddito superiore
a 50.000 euro? La risposta corretta � la seconda: la seconda
restrizione si aggiunge alla prima, ossia la restrizione finale � il
prodotto logico della nuova restrizione e di tutte le restrizioni
precedenti.  Se si vuole applicare una nuova restrizione
indipendentemente da quelle applicate in precedenza, occorre prima
re-impostare il campione alla sua lunghezza originaria, usando
    
\begin{code}
      smpl full
\end{code}

In alternativa, � possibile aggiungere l'opzione \verb+replace+ al
comando \verb+smpl+:

\begin{code}
      smpl income>50000 --restrict --replace
\end{code}

Questa opzione ha l'effetto di re-impostare automaticamente il
campione completo prima di applicare la nuova restrizione.A differenza
della semplice ``impostazione'' del campione, la ``restrizione'' del
campione pu� produrre un insieme di osservazioni non contigue nel
dataset originale e pu� anche modificare la struttura del
dataset.Questo fenomeno pu� essere osservato nel caso dei dati panel:
si supponga di avere un panel di cinque imprese (indicizzate dalla
variabile \verb+impresa+) osservate in ognuno degli anni identificati
dalla variabile \verb+anno+.  La restrizione

\begin{code}
	smpl anno=1995 --restrict
\end{code}

produce un dataset che non � pi� di tipo panel, ma cross-section per
l'anno 1995.  In modo simile
      
\begin{code}
	smpl impresa=3 --restrict
\end{code}

produce un dataset di serie storiche per l'impresa numero 3.Per questi
motivi (possibile non-contiguit� nelle osservazioni, possibile
cambiamento nella struttura dei dati) gretl si comporta in modo
diverso a seconda che si operi una ``restrizione'' del campione o una
semplice ``impostazione'' di esso. Nel caso dell'impostazione, il
programma memorizza semplicemente le osservazioni iniziali e finali e
le usa come parametri per i vari comandi di stima dei modelli, di
calcolo delle statistiche ecc. Nel caso della restrizione, il
programma crea una copia ridotta del dataset e la tratta come un
semplice dataset di tipo cross-section non datato.  \footnote{Con una
  eccezione: se si parte da un dataset panel bilanciato e la
  restrizione � tale da preservare la struttura di panel bilanciato
  (ad esempio perch� implica la cancellazione di tutte le osservazioni
  per una unit� cross-section), allora il dataset ridotto � ancora
  trattato come panel.}.  Se si vuole re-imporre un'interpretazione di
tipo "serie storiche" o "panel" al dataset ridotto, occorre usare il
comando \cmd{setobs}, o il comando dal men� ``Campione, Struttura
dataset'', se appropriato).  Il fatto che una ``restrizione'' del
campione comporti la creazione di una copia ridotta del dataset
originale pu� creare problemi quando il dataset � molto grande
(nell'ordine delle migliaia di osservazioni). Se si usano simili
dataset, la creazione della copia pu� causare l'esaurimento della
memoria del sistema durante il calcolo dei risultati delle
regressioni. � possibile aggirare il problema in questo modo:

\begin{enumerate}
\item Aprire il dataset completo e imporre la restrizione sul
  campione.
\item Salvare una copia del dataset ridotto su disco.
\item Chiudere il dataset completo e aprire quello ridotto.
\item Procedere con l'analisi.
\end{enumerate}

\section{Campionamento casuale}
\label{sample-random}

Se si usano dataset molto grandi (o se si intende studiare le
propriet� di uno stimatore), pu� essere utile estrarre un campione
casuale dal dataset completo. � possibile farlo ad esempio con

\begin{code}
	smpl 100 --random
\end{code}

che seleziona 100 osservazioni.  Se occorre che il campione sia
riproducibile, occorre per prima cosa impostare il seme del generatore
di numeri casuali, usando il comando \cmd{set}.Questo tipo di
campionamento � un esempio di ``restrizione'' del campione: viene
infatti generata una copia ridotta del dataset.

\section{I comandi del men� Campione}
\label{sample-menu}

Gli esempi visti finora hanno mostrato il comando testuale \cmd{set},
ma � possibile creare un sotto-campione usando i comandi del men�
\textsf{Campione} nell'interfaccia grafica del programma.I comandi del
men� permettono di ottenere gli stessi risultati delle varianti del
comando testuale \verb+smpl+, con la seguente eccezione: se si usa il
comando ``Campione, Imposta in base a condizione...'' e sul dataset �
gi� stato impostato un sotto-campione, viene data la possibilit� di
preservare la restrizione gi� attiva o di sostituirla (in modo analogo
a quanto avviene invocando l'opzione \verb+replace+ descritta
nelSection \ref{sample-restrict}.
    
%%% Local Variables: 
%%% mode: latex
%%% TeX-master: "gretl-guide-it"
%%% End: 

