\chapter{Introduzione}
\label{intro}

\section{Caratteristiche principali}
\label{features}

\app{Gretl} � un pacchetto econometrico che comprende una libreria
condivisa, un programma client a riga di comando e un'interfaccia
grafica.
    
\begin{description}
\item[Amichevole] \app{Gretl} offre un'interfaccia utente intuitiva,
  che permette di entrare subito nel vivo dell'analisi econometrica.
  Grazie all'integrazione con i libri di testo di Ramu Ramanathan, di
  Jeffrey Wooldridge, di James Stock e Mark Watson, il pacchetto offre
  molti file di dati e script di comandi, commentati e pronti all'uso.
  Gli utenti di gretl hanno inoltre a disposizione la documentazione
  sul programma e la mailing list \href{http://gretl.sourceforge.net/lists.html}{gretl-users}.
\item[Flessibile] � possibile scegliere il proprio metodo di lavoro
  preferito: dal punta-e-clicca interattivo alla modalit� batch,
  oppure una combinazione dei due approcci.
\item[Multi-piattaforma] La piattaforma di sviluppo di \app{Gretl} �
  Linux, ma il programma � disponibile anche per MS Windows e
  Mac OS X, e dovrebbe funzionare su qualsiasi sistema operativo simile a UNIX
  che comprenda le librerie di base richieste (si veda
  l'appendice~\ref{app-technote}).
\item[Open source] L'intero codice sorgente di \app{Gretl} �
  disponibile per chiunque voglia criticarlo, correggerlo o
  estenderlo. Si veda l'appendice~\ref{app-technote}.
\item[Sofisticato] \app{Gretl} offre un'ampia variet�
  di stimatori basati sui minimi quadrati, compresi i minimi quadrati
  a due stadi e i minimi quadrati non lineari.  Offre anche alcuni stimatori
  di massima verosimiglianza (ad es.\ logit, probit, tobit), oltre alla stima
  generica di massima verosimiglianza.
  Il programma supporta la stima di sistemi di equazioni simultanee, GARCH,
  ARMA, autoregressioni vettoriali e modelli vettoriali a correzione di errore.
\item[Accurato] \app{Gretl} � stato testato a fondo con il dataset di
  riferimento NIST. Si veda l'appendice~\ref{app-accuracy}.
\item[Pronto per internet] \app{Gretl} pu� scaricare i database da un
  server alla Wake Forest University; inoltre, comprende una funzionalit�
  di aggiornamento che controlla se � disponibile una nuova versione del programma
  e, nella versione per MS Windows, permette di aggiornarlo automaticamente.
\item[Internazionale] \app{Gretl} supporta le lingue inglese,
  francese, italiana, spagnola, polacca o tedesca, a seconda della lingua
  impostata sul computer.
\end{description}

\section{Ringraziamenti}
\label{ack}

La base di codice di \app{Gretl} deriva da quella del programma \app{ESL}
(``Econometrics Software Library''), scritto dal Professor
Ramu Ramanathan della University of California, San Diego. Siamo molto
grati al Professor Ramanathan per aver reso disponibile questo codice con
licenza GNU General Public License e per aver aiutato nello sviluppo di
\app{Gretl}.
 
Siamo anche grati agli autori di molti testi di econometria che hanno
permesso la distribuzione delle versioni \app{Gretl} dei dataset contenuti
nei loro libri. Questa lista al momento comprende William Greene,
autore di \emph{Econometric Analysis}, Jeffrey Wooldridge
(\emph{Introductory Econometrics: A Modern Approach}); James Stock e
Mark Watson (\emph{Introduction to Econometrics}); Damodar Gujarati
(\emph{Basic Econometrics}); Russell Davidson e James MacKinnon
(\emph{Econometric Theory and Methods}).

La stima GARCH in \app{Gretl} si basa sul codice pubblicato sul \emph{Journal of
Applied Econometrics} dai Prof. Fiorentini, Calzolari e Panattoni, mentre il
codice per generare i \emph{p}-value per i test Dickey Fuller � di James
MacKinnon. In ognuno dei casi sono grato agli autori per avermi permesso di
usare il loro lavoro.  

Per quanto riguarda l'internazionalizzazione di \app{Gretl}, vorrei ringraziare
Ignacio D�az-Emparanza (spagnolo), Michel Robitaille e Florent Bresson (francese),
Cristian Rigamonti (italiano), Tadeusz Kufel e Pawel Kufel (polacco), e
Markus Hahn e Sven Schreiber (tedesco).

\app{Gretl} ha beneficiato largamente del lavoro di molti sviluppatori di software
libero e open-source: per i dettagli si veda l'appendice~\ref{app-technote}.
Devo ringraziare Richard Stallman della Free Software Foundation per
il suo supporto al software libero in generale, ma in particolare per
aver accettato di ``adottare'' \app{Gretl} come programma GNU.

Molti utenti di \app{Gretl} hanno fornito utili suggerimenti e
segnalazioni di errori. Un ringraziamento particolare a Ignacio
D�az-Emparanza, Tadeusz Kufel, Pawel Kufel, Alan Isaac, Cristian
Rigamonti e Dirk Eddelbuettel, che cura il pacchetto \app{Gretl} per
Debian GNU/Linux.


\section{Installazione del programma}
\label{install}

\subsection{Linux}
\label{linux-install}

Sulla piattaforma Linux\footnote{In questo manuale
  verr� usata l'abbreviazione ``Linux'' per riferirsi al sistema
  operativo GNU/Linux. Ci� che viene detto a proposito di Linux vale
  anche per altri sistemi simili a UNIX, anche se potrebbero essere
  necessari alcuni adattamenti.}, � possibile compilare da s� il
codice di \app{Gretl}, oppure usare un pacchetto pre-compilato.
Pacchetti gi� pronti sono disponibili in formato \app{rpm}
(appropriato per sistemi Red Hat Linux e simili) e anche in formato
\app{deb} (per Debian GNU/Linux). Se si preferisce compilare da s� (o si usa un
sistema UNIX per cui non sono disponibili pacchetti pre-compilati), ecco come
procedere:
      
\begin{enumerate}
\item Scaricare il pi� recente pacchetto dei sorgenti di \app{Gretl}
  da \href{http://gretl.sourceforge.net/}{gretl.sourceforge.net}.
        
\item Decomprimere il pacchetto. Se si dispone delle utilit� GNU,
  usare il comando \cmd{tar xvfz gretl-N.tar.gz} (sostituire \cmd{N}
  con il numero di versione specifico del file scaricato).
\item Spostarsi nella directory del codice sorgente di Gretl appena
  creata (ad es. \verb+gretl-1.1.5+).
          
\item La sequenza basilare di comandi da eseguire �:
          
\begin{code}
    ./configure 
    make 
    make check
    make install
\end{code}

Tuttavia, potrebbe essere utile leggere per prima cosa il file
  \verb+INSTALL+ e/o eseguire
\begin{code}
    ./configure --help
\end{code}
per capire quali opzioni di configurazione sono disponibili.
Un'opzione che pu� essere utile utilizzare � \cmd{--prefix}; il
comportamento predefinito � quello di installare il programma in
\verb+/usr/local+, ma � possibile modificarlo.  Ad esempio
\begin{code}
    ./configure --prefix=/usr
\end{code}
installer� il tutto sotto la directory \verb+/usr+.  Se la procedura
di configurazione si interrompe perch� una delle librerie richieste
non � disponibile sul sistema, si veda l'appendice~\ref{app-technote}.
Sulla maggior parte dei sistemi, il comando \texttt{make install} richiede
di avere i privilegi di amministratore. Quindi, occorre assumere l'identit�
di \texttt{root} prima di eseguire  \texttt{make install}, oppure si pu�
usare il comando \texttt{sudo}.
\end{enumerate}

\app{Gretl} supporta il desktop \app{gnome}. Per abilitare il supporto, occorre
compilare il programma da s� (come descritto sopra), mentre per disabilitare le
funzionalit� specifiche di \app{gnome}, occorre usare l'opzione
\verb+--without-gnome+ con il comando \cmd{configure}.


\subsection{MS Windows}
\label{windows-install}

La versione MS Windows � disponibile sotto forma di file eseguibile
auto-estraente. Per installarlo, occorre scaricare
\verb+gretl_install.exe+ ed eseguire questo programma. Verr� chiesta
una posizione in cui installare il pacchetto (quella predefinita �
\verb+c:\userdata\gretl+).

\subsection{Aggiornamento}
\label{updating}


Se si ha un computer connesso a internet, all'avvio \app{Gretl} pu�
collegarsi al proprio sito web alla Wake Forest University per vedere
se sono disponibili aggiornamenti al programma. In caso positivo,
comparir� una finestra informativa. Per attivare questa funzionalit�,
occorre abilitare la casella ``Avvisa in caso di aggiornamenti di
gretl'' nel men� ``Strumenti, Preferenze, Generali...'' di \app{Gretl}.
      
La versione MS Windows di \app{Gretl} fa un passo in pi�: d� anche la
possibilit� di aggiornare automaticamente il programma. � sufficiente
seguire le indicazioni nella finestra pop-up: chiudere \app{Gretl} ed
eseguire il programma di aggiornamento ``gretl updater'' (che di
solito si trova vicino alla voce \app{Gretl} nel gruppo Programmi del
men� Avvio di Windows). Quando il programma di aggiornamento ha
concluso il suo funzionamento, � possibile avviare di nuovo
\app{Gretl}.
      
%%% Local Variables: 
%%% mode: latex
%%% TeX-master: "gretl-guide-it"
%%% End: 

