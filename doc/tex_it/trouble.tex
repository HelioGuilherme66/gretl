\chapter{Risoluzione dei problemi}
\label{trouble}



\section{Segnalazione dei bug}
\label{trouble-bugs}


Le segnalazioni dei bug sono benvenute. � difficile trovare errori di
calcolo in \app{gretl} (ma questa affermazione non costituisce alcuna
sorta di garanzia), ma � possibile imbattersi in bug o stranezze nel
comportamento dell'interfaccia grafica. Si tenga presente che
l'utilit� delle segnalazioni aumenta quanto pi� si � precisi nella
descrizione: cosa \emph{esattamente} non funziona, in che condizioni,
con quale sistema operativo? Se si ricevono messaggi di errore, cosa
dicono esattamente?

\section{Programmi ausiliari}
\label{trouble-programs}

Come detto in precedenza, \app{gretl} richiama alcuni altri programmi
per eseguire alcune operazioni (gnuplot per i grafici, {\LaTeX} per la
stampa ad alta qualit� dei risultati delle regressioni, GNU R).  Se
succede qualche problema durante questi collegamenti esterni, non �
sempre facile fornire un messaggio di errore abbastanza informativo.
Se il problema si verifica durante l'uso di \app{gretl} con
l'interfaccia grafica, pu� essere utile avviare \app{gretl} dal prompt
dei comandi (ad es. da una finestra di xterm nel sistema X window, o
da un ``Prompt di MS-DOS'' in MS Windows, nel qual caso occorre
digitare \cmd{gretlw32.exe}), invece che da un men� o da un'icona: la
finestra del terminale potr� quindi contenere messaggi di errore
aggiuntivi.  Si tenga anche presente che nella maggior parte dei casi
\app{gretl} assume che i programmi in questione siano disponibili nel
``percorso di esecuzione'' (path) dell'utente, ossia che possano
essere invocati semplicemente con il nome del programma, senza
indicare il percorso completo \footnote{L'eccezione a questa regola �
  costituita dall'invocazione di gnuplot in MS Windows, dove occorre
  indicare il percorso completo del programma.}. Quindi se un certo
programma non si avvia, conviene provare ad eseguirlo da un prompt dei
comandi, come descritto qui sotto.
      
\begin{center}
  \begin{tabular}{llll}
    Sistema & Grafica & Stampa & GNU R\\
    Sistema X window & gnuplot & latex, xdvi & R\\
    MS Windows & wgnuplot.exe & latex, windvi & RGui.exe\\
  \end{tabular}
\end{center}

Se il programma non si avvia dal prompt, non � un problema di
\app{gretl}: probabilmente la directory principale del programma non �
contenuta nel path dell'utente, oppure il programma non � stato
installato correttamente.  Per i dettagli su come modificare il path
dell'utente, si veda la documentazione o l'aiuto online per il proprio
sistema operativo.
    
%%% Local Variables: 
%%% mode: latex
%%% TeX-master: "gretl-guide-it"
%%% End: 

