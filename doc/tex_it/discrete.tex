\chapter{Variabili discrete}
\label{chap-discrete}

Quando una variabile pu� assumere solo un numero finito, tipicamente basso, di
valori, essa si pu� chiamare \emph{discreta}. Alcuni comandi di
\app{gretl} si comportano in modo leggermente diverso quando sono usati su
variabili discrete; in pi�, \app{gretl} fornisce alcuni comandi che si applicano
solo alle variabili discrete.


\section{Dichiarazione delle variabili discrete}
\label{discr-declare}

Quando si crea un file di dati da zero, nessuna variabile viene considerata
discreta. Per marcare una variabile come discreta, � possibile agire in due
modi:
\begin{enumerate}
\item Dall'interfaccia grafica, selezionare ``Variabile, Modifica attributi''
  dal men�. Apparir� una finestra di dialogo in cui appare la casella
  ``Tratta questa variabile come discreta''. La stessa finestra di dialogo pu�
  essere richiamata dal men� contestuale (facendo clic col tasto destro su una
  variabile) o premendo il tasto F2;
\item Dall'interfaccia a riga di comando, usando il comando \texttt{discrete}.
  Il comando accetta uno o pi� argomenti, che possono essere variabili o liste
  di variabili. Ad esempio:
\begin{code}
  list xlist = x1 x2 x3
  discrete z1 xlist z2
\end{code}
In questo modo � possibile dichiarare pi� variabili discrete con un solo
comando, cosa che al momento non � possibile fare usando l'interfaccia grafica.
L'opzione \texttt{--reverse} inverte la dichiarazione, ossia rende continua una
variabile discreta.
Ad esempio:
\begin{code}
  discrete pippo
  # ora pippo � discreta
  discrete pippo --reverse
  # ora piipo � continua
\end{code}
\end{enumerate}

Si noti che marcare una variabile come discreta non ne modifica il contenuto. �
quindi responsabilit� dell'utente usare correttamente questa funzione. Per
ricodificare una variabile continua in classi, � possibile usare il comando
\texttt{genr} e le sue funzioni aritmetiche come nell'esempio seguente:
\begin{code}
  nulldata 100
  # genera una variabile con media 2 e varianza 1
  genr x = normal() + 2
  # suddivide in 4 classi
  genr z = (x>0) + (x>2) + (x>4)
  # ora dichiara z come discreta
  discrete z
\end{code}

Quando si marca una variabile come discreta, questa impostazione viene ricordata
dopo il salvataggio del file.

\section{Comandi per le variabili discrete}
\label{discr-commands}

\subsection{Il comando \texttt{dummify}}
\label{discr-dummify}

Il comando \texttt{dummify} prende come argomento una serie $x$ e crea delle
variabili dummy per ognuno dei valori distinti presenti in $x$, che deve essere
stata dichiarata discreta in precedenza. Ad esempio:
\begin{code}
  open greene22_2
  discrete Z5 # marca Z5 come discreta
  dummify Z5
\end{code}

L'effetto di questi comandi � quello di generare 5 nuove variabili dummy, i cui
nomi vanno da \texttt{DZ5\_1} fino a \texttt{DZ5\_5}, che corrispondono ai
diversi valori presenti in \texttt{Z5}. Ossia, la variabile
\texttt{DZ5\_4} vale 1 dove \texttt{Z5} vale 4, e 0 altrove. Questa funzionalit�
� disponibile anche nell'interfaccia grafica, con il comando del men�
``Aggiungi, dummy per le variabili discrete selezionate''.

Il comando \texttt{dummify} pu� essere usato anche con la sintassi seguente:
\begin{code}
  list dlist = dummify(x)
\end{code}
che crea non solo le variabili dummy, ma anche una lista (si veda il paragrafo~\ref{named-lists})
che pu� essere usata in seguito. L'esempio seguente calcola le statistiche
descrittive per la variabile \texttt{Y} in corrispondenza di ogni valore di
\texttt{Z5}:
\begin{code}
  open greene22_2
  discrete Z5 # marca Z5 come discreta
  list foo = dummify(Z5)
  loop foreach i foo
    smpl $i --restrict --replace
    summary Y
  end loop
  smpl full
\end{code}
% $

Poich� \texttt{dummify} genera una lista, pu� anche essere usato direttamente in
comandi come \texttt{ols}, come in questo esempio:
\begin{code}
  open greene22_2
  discrete Z5 # marca Z5 come discreta
  ols Y 0 dummify(Z5)
\end{code}

\subsection{Il comando \texttt{freq}}
\label{discr-freq}

Il comando \texttt{freq} mostra le frequenze assolute e relative per una
variabile. Il modo in cui le frequenze vengono calcolate dipende dal carattere
discreto o continuo della variabile. Questo comando � disponibile anche
nell'interfaccia grafica, usando il comando del men� ``Variabile, Distribuzione
di frequenza''.

Per variabili discrete, le frequenze sono contate per ogni diverso valore
assunto dalla variabile. Per le variabili continue, i valori sono raggruppati in
``classi'' e quindi le frequenze sono calcolate per ogni classe.
Il numero di classi � calcolato in funzione del numero di osservazioni valide
nel campione selezionato al momento, nel modo seguente:

\begin{table}[htbp]
  \centering
  \begin{tabular}{cc}
\hline
  Osservazioni & Classi \\
\hline
  $8 \le n < 16$ & 5 \\
  $16 \le n < 50 $ & 7 \\
  $50 \le n \le 850 $ & $\lceil \sqrt{n} \rceil$  \\
  $n > 850 $ & 29 \\
\hline
\end{tabular}
\caption{Numero di classi per varie ampiezze campionarie}
\label{tab:bins}
\end{table}

Ad esempio, il codice seguente
\begin{code}
  open greene19_1
  freq TUCE
  discrete TUCE # marca TUCE come discreta
  freq TUCE
\end{code}
produce questo risultato
\begin{code}
Lettura del file dati /usr/local/share/gretl/data/greene/greene19_1.gdt
Periodicit�: 1, oss. max.: 32,
Intervallo delle osservazioni: 1-32

5 variabili elencate:
  0) const    1) GPA      2) TUCE     3) PSI      4) GRADE  

? freq TUCE

Distribuzione di frequenza per TUCE, oss. 1-32
Numero di intervalli = 7, media = 21,9375, deviazione standard = 3,90151

       Intervallo        P.med.  Frequenza    Rel.     Cum.

          <  13,417     12,000        1      3,12%    3,12% *
    13,417 - 16,250     14,833        1      3,12%    6,25% *
    16,250 - 19,083     17,667        6     18,75%   25,00% ******
    19,083 - 21,917     20,500        6     18,75%   43,75% ******
    21,917 - 24,750     23,333        9     28,12%   71,88% **********
    24,750 - 27,583     26,167        7     21,88%   93,75% *******
          >= 27,583     29,000        2      6,25%  100,00% **

Test per l'ipotesi nulla di distribuzione normale:
Chi-quadro(2) = 1,872 con p-value 0,39211

? discrete TUCE # marca TUCE come discreta

? freq TUCE
Distribuzione di frequenza per TUCE, oss. 1-32

          Frequenza    Rel.     Cum.

  12           1      3,12%    3,12% *
  14           1      3,12%    6,25% *
  17           3      9,38%   15,62% ***
  19           3      9,38%   25,00% ***
  20           2      6,25%   31,25% **
  21           4     12,50%   43,75% ****
  22           2      6,25%   50,00% **
  23           4     12,50%   62,50% ****
  24           3      9,38%   71,88% ***
  25           4     12,50%   84,38% ****
  26           2      6,25%   90,62% **
  27           1      3,12%   93,75% *
  28           1      3,12%   96,88% *
  29           1      3,12%  100,00% *

Test per l'ipotesi nulla di distribuzione normale:
Chi-quadro(2) = 1,872 con p-value 0,39211
\end{code}
Come si pu� vedere dall'esempio, viene calcolato automaticamente un test
Jarque--Bera per la normalit�.

Questo comando accetta due opzioni: \texttt{--quiet}, per evitare la stampa
dell'istogramma, e \texttt{--gamma}, per sostituire il test di normalit� con
il test non parametrico di Locke, la cui ipotesi nulla � che i dati seguano una
distribuzione Gamma.


\subsection{Il comando \texttt{xtab}}
\label{discr-xtab}

Il comando \texttt{xtab} ha la sintassi seguente
\begin{code}
  xtab lista-y ; lista-x
\end{code}
dove \texttt{lista-y} e \texttt{lista-x} sono liste di variabili discrete;
esso produce tabulazioni incrociate per ognuna delle variabili nella
\texttt{lista-y} (per riga) rispetto a ognuna delle variabili nella
\texttt{lista-x} (per colonna). Al momento, questa funzionalit� non �
accessibile dall'interfaccia grafica.

Ad esempio:
\begin{code}
  open greene22_2
  discrete Z* # Marca Z1-Z8 come discrete
  xtab Z1 Z4 ; Z5 Z6
\end{code}
produce questo risultato
\begin{code}
Tabulazione incrociata di Z1 (righe) rispetto a Z5 (colonne)

       [   1][   2][   3][   4][   5]  TOT.
  
[   0]    20    91    75    93    36    315
[   1]    28    73    54    97    34    286

TOTALE    48   164   129   190    70    601

Test chi-quadro di Pearson = 5,48233 (4 df, p-value = 0,241287)

Tabulazione incrociata di Z1 (righe) rispetto a Z6 (colonne)

       [   9][  12][  14][  16][  17][  18][  20]  TOT.
  
[   0]     4    36   106    70    52    45     2    315
[   1]     3     8    48    45    37    67    78    286

TOTALE     7    44   154   115    89   112    80    601

Test chi-quadro di Pearson = 123,177 (6 df, p-value = 3,50375e-24)

Tabulazione incrociata di Z4 (righe) rispetto a Z5 (colonne)

       [   1][   2][   3][   4][   5]  TOT.
  
[   0]    17    60    35    45    14    171
[   1]    31   104    94   145    56    430

TOTALE    48   164   129   190    70    601

Test chi-quadro di Pearson = 11,1615 (4 df, p-value = 0,0248074)

Tabulazione incrociata di Z4 (righe) rispetto a Z6 (colonne)

       [   9][  12][  14][  16][  17][  18][  20]  TOT.
  
[   0]     1     8    39    47    30    32    14    171
[   1]     6    36   115    68    59    80    66    430

TOTALE     7    44   154   115    89   112    80    601

Test chi-quadro di Pearson = 18,3426 (6 df, p-value = 0,0054306)
\end{code}

Il test chi-quadro di Pearson per l'indipendenza viene mostrato automaticamente
se la frequenza attesa nell'ipotesi di indipendenza supera 5 per almeno l'80 per
cento delle celle. L'opzione \texttt{--chi-square} fa in modo che il test venga
sempre mostrato.

Inoltre, le opzioni \texttt{--row} o \texttt{--column} fanno in modo che vengano
mostrate le percentuali di riga o di colonna.

Se si vuole incollare il risultato di \texttt{xtab} in qualche altra
applicazione, ad esempio un foglio di calcolo, � utile usare l'opzione
\texttt{--zeros}, che scrive il numero zero nelle celle con frequenza pari a
zero, invece di lasciarle vuote.

%%% Local Variables: 
%%% mode: latex
%%% TeX-master: "gretl-guide"
%%% End: 
