\chapter{L'interfaccia a riga di comando}
\label{cli}

\section{Gretl sul terminale}
\label{cli-console}

Il pacchetto \app{gretl} include il programma a riga di comando
\app{gretlcli}. Su Linux � possibile eseguirlo in una finestra di terminale
(xterm, rxvt, o simili), o in un terminale testuale. Su MS Windows pu� essere
eseguito in una finestra di terminale (quella che di solito viene chiamata
``prompt di MS-DOS'').  \app{gretlcli} ha un file di aiuto, a cui si
accede inserendo il comando ``help'' al prompt. Inoltre, pu� essere
usato in modalit� batch, inviando i risultati direttamente a un file
(si veda la \GCR).
    
Se \app{gretlcli} � linkato alla libreria \app{readline} (ci� avviene
automaticamente nella versione MS Windows, si veda anche
l'appendice~\ref{app-build}), � possibile richiamare e modificare i
comandi gi� eseguiti, inoltre � disponibile il completamento
automatico dei comandi. Per muoversi nella lista dei comandi gi�
eseguiti � possibile usare i tasti Freccia Su/Gi�, mentre per muoversi
sulla riga di comando � possibile usare i tasti Freccia a
destra/sinistra e le scorciatoie in stile Emacs\footnote{Questi sono i valori
  predefiniti per le scorciatoie da tastiera, che possono essere modificate
  seguendo le istruzioni contenute nel
  \href{http://cnswww.cns.cwru.edu/~chet/readline/readline.html}{manuale
    di readline}.}. Le scorciatoie pi� usate sono:
%    
\begin{center}
  \begin{tabular}{ll}
    \textit{Scorciatoia} & \textit{Effetto}\\
    \verb+Ctrl-a+ & Va all'inizio della riga\\
    \verb+Ctrl-e+ & Va alla fine della riga\\
    \verb+Ctrl-d+ & Cancella il carattere a destra\\
  \end{tabular}
\end{center}
%
dove ``\verb+Ctrl-a+'' significa premere il tasto ``\verb+a+'' mentre
si tiene premuto il tasto ``\verb+Ctrl+''.  Quindi se si vuole
correggere qualcosa all'inizio della riga di comando, \emph{non
  occorre} usare il tasto di cancellazione all'indietro su tutta la
riga, basta saltare all'inizio della riga e aggiungere o cancellare i
caratteri desiderati.  Inserendo le prime lettere del nome di un
comando e premendo il tasto Tab, readline cercher� di completare
automaticamente il nome del comando. Se esiste un solo modo di
completare le lettere inserite, verr� inserito il comando
corrispondente, altrimenti premendo Tab una seconda volta verr�
visualizzata una lista delle alternative possibili.
    
\section{La modalit� non interattiva}
\label{cli-syntax}

Probabilmente, il modo pi� utile per svolgere grosse quantit� di lavoro con
\app{gretlcli} consiste nell'utilizzare la modalit� ``batch'' (non interattiva),
in cui il programma legge ed esegue uno script di comandi, scrivendo il
risultato su un file. Ad esempio:

\begin{code}
gretlcli -b filescript > fileoutput
\end{code}

Il \textsl{filescript} fornito come argomento al programma deve specificare al
proprio interno il file di dati su cui operare, usando il comando
\cmd{open filedati}. Se non si utilizza l'opzione \texttt{-b} (che sta per ``batch''),
il programma rester� in attesa di input al termine dell'esecuzione dello script.
    
%%% Local Variables: 
%%% mode: latex
%%% TeX-master: "gretl-guide-it"
%%% End: 

