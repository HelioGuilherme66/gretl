\chapter{L'interfaccia a riga di comando}
\label{cli}



\section{Gretl sul terminale}
\label{cli-console}


Il pacchetto \app{gretl} include il programma a riga di comando
\app{gretlcli}. Su Linux � possibile esegurlo in un terminale testuale
o in un xterm (o simili), mentre su MS Windows pu� essere eseguito in
una finestra di terminale (quella che di solito viene chiamata
``prompt di MS-DOS'').  \app{gretlcli} ha un file di aiuto, a cui si
accede inserendo il comando ``help'' al prompt. Inoltre, pu� essere
usato in modalit� batch, inviando i risultati direttamente a un file
(si veda XXX).
    
Se \app{gretlcli} � linkato alla libreria \app{readline} (ci� avviene
automaticamente nella versione MS Windows, si veda anche l'[?]), �
possibile richiamare e modificare i comandi gi� eseguiti, inoltre �
disponibile il completamento automatico dei comandi. Per muoversi
nella lista dei comandi gi� eseguiti � possibile usare i tasti Freccia
Su/Gi�, mentre per muoversi sulla riga di comando � possibile usare i
tasti Freccia a destra/sinistra e le scorciatoie in stile
Emacs\footnote{In realt� quelli mostrati sono i valori predefiniti, ma
  possono essere modificati: si veda il
  \href{http://cnswww.cns.cwru.edu/~chet/readline/readline.html}{manuale
    di readline}.}. Le scorciatoie pi� usate sono:
    

\begin{center}
  \begin{tabular}{ll}
    Scorciatoia & Effetto\\
    \verb+Ctrl-a+ & Va all'inizio della riga\\
    \verb+Ctrl-e+ & Va alla fine della riga\\
    \verb+Ctrl-d+ & Cancella il carattere a destra\\
  \end{tabular}
\end{center}


dove ``\verb+Ctrl-a+'' significa premere il tasto ``\verb+a+'' mentre
si tiene premuto il tasto ``\verb+Ctrl+''.  Quindi se si vuole
correggere qualcosa all'inizio della riga di comando, \emph{non
  occorre} usare il tasto di cancellazione all'indietro su tutta la
riga, basta saltare all'inizio della riga e aggiungere o cancellare i
caratteri desiderati.  Inserendo le prime lettere del nome di un
comando e premendo il tasto Tab, readline cercher� di completare
automaticamente il nome del comando. Se esiste un solo modo di
completare le lettere inserite, verr� inserito il comando
corrispondente, altrimenti premendo Tab una seconda volta verr�
visualizzata una lista delle alternative possibili.
    

\section{Differenze con ESL di Ramanathan}
\label{cli-syntax}

\app{gretlcli} ha ereditato la sintassi di base dei comandi dal
programma \app{ESL} di Ramu Ramanathan e gli script di comandi
sviluppati per \app{ESL} dovrebbero essere utilizzabili con con pochi
cambiamenti: le uniche cose a cui stare attenti sono i comandi
multilinea e il comando \cmd{freq}.
    
\begin{itemize}
\item In \app{ESL} viene usato un punto e virgola come terminatore per
  molti comandi.  In \app{gretlcli} ho deciso di rimuovere questa
  caratteristica: il carattere punto e virgola � semplicemente
  ignorato, a parte in alcuni casi speciali in cui ha un significato
  particolare: come separatore per due liste nei comandi \cmd{ar} e
  \cmd{tsls}, e come segnaposto per l'osservazione iniziale o finale
  immutata nel comando \cmd{smpl}. In \app{ESL} il punto e virgola d�
  la possibilit� di spezzare le righe di comando su pi� righe dello
  schermo; in \app{gretlcli} si ottiene questo risultato inserendo una
  barra rovesciata \verb+\+ alla fine della riga da spezzare.
	
\item Per quanto riguarda \cmd{freq}, al momento non � possibile
  specificare intervalli personalizzati, come avviene in \app{ESL}.
  Inoltre, ai risultati del comando � stato aggiunto un test
  chi-quadro di normalit�.
	
\end{itemize}


Si noti anche che i comandi per usare la modalit� batch sono stati
semplificati. In \app{ESL} si usava, ad esempio:
      
\begin{code}
        esl -b filedati < fileinput > fileoutput
\end{code}

mentre in \app{gretlcli} si usa:
      
\begin{code}
	gretlcli -b fileinput > fileoutput
\end{code}


Il file di input viene trattato come un argomento del programma,
mentre il file di dati su cui operare va indicato all'interno del file
di input, usando il comando \cmd{open datafile} o il commento speciale
\verb+(* !+ filedati \verb+*)+.
    
%%% Local Variables: 
%%% mode: latex
%%% TeX-master: "gretl-guide-it"
%%% End: 

