\chapter{Variabili dipendenti discrete e censurate}
\label{discr-models}

\section{Modelli logit e probit}
\label{sec:logit-probit}

Capita spesso che si voglia specificare e stimare un modello in cui la variabile
dipendente non � continua  ma discreta. Un esempio tipico � quello di un modello
in cui la variabile dipendente � lo stato lavorativo di un individuo (1 =
occupato, 0 = disoccupato). Un modo comodo per formalizzare questa situazione
consiste nel considerare la variabile $y_i$ come una variabile aleatoria di
Bernoulli e analizzarne la distribuzione condizionata alle variabili esplicative
$x_i$, ossia
\begin{equation}
  \label{eq:qr-Bernoulli}
  y_i \left\{ 
    \begin{array}{ll} 1 & P_i \\ 0 & 1 - P_i \end{array}
    \right. 
\end{equation}
dove $P_i$ � una funzione delle variabili esplicative $x_i$.

Nella maggior parte dei casi, la funzione $P_i$ � una funzione di ripartizione,
applicata a una combinazione lineare delle $x_i$. Nel modello probit, si usa la
funzione di ripartizione normale, mentre il modello logit usa la funzione
logistica $\Lambda()$. Quindi si ha
\begin{eqnarray}
  \label{eq:qr-link}
  \textrm{probit} & \qquad & P_i = \Phi(z_i)  \\
  \textrm{logit}  & \qquad & P_i = \Lambda(z_i) = \frac{1}{1 + e^{-z_i}} \\
  & &z_i = \sum_{j=1}^k x_{ij} \beta_j
\end{eqnarray}
dove $z_i$ � chiamata la funzione \emph{indicatrice}. Si noti che in questo
caso, i coefficienti $\beta_j$ non possono essere interpretati come derivate
parziali di $E(y_i | x_i)$ rispetto a $x_{ij}$. Un modo equivalente di formulare
questo modello consiste nell'ipotizzare l'esistenza di una variabile non
osservata $y^*_i$ che pu� essere descritta dal modello
\begin{equation}
  \label{eq:qr-latent}
  y^*_i = \sum_{j=1}^k x_{ij} \beta_j + \varepsilon_i = z_i  +
  \varepsilon_i ;
\end{equation}
si osserva $y_i = 1$ quando $y^*_i > 0$ e $y_i = 0$ altrimenti. Se si assume
$\varepsilon_i$ come normale, si ottiene il modello probit, mentre il modello
logit assume che la funzione di densit� di $\varepsilon_i$ sia
\[
  \lambda(\varepsilon_i) =
  \pder{\Lambda(\varepsilon_i)}{\varepsilon_i} =
  \frac{e^{-\varepsilon_i}}{(1 + e^{-\varepsilon_i})^2} .
\]

\app{gretl} stima sia il modello probit che quello logit col metodo della
massima verosimiglianza, usando procedure di ottimizzazione numerica. La maggior
parte delle volte queste richiedono poche iterazioni per raggiungere la
convergenza, ma � possibile visualizzare i dettagli dell'algoritmo di
massimizzazione usando l'opzione \texttt{--verbose}.

Example: \ldots

\subsection{Ordered models}
\label{sec:ordered}

\section{The Tobit model}
\label{sec:tobit}

\subsection{Generalized Tobit model}
\label{sec:heckit}

include Heckman example script

\section{Count data}
\label{sec:poisson}

also include example script for negative binomial (done in Vebeek
example files).



%%% Local Variables: 
%%% mode: latex
%%% TeX-master: t
%%% End: 
