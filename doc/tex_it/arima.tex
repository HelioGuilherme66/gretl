\chapter{Modelli ARMA}
\label{arma-estimation}

\section{Rappresentazione e sintassi}
\label{arma-repr}

Il comando \cmd{arma} effettua la stima di modelli autoregressivi a media mobile
(ARMA); la rappresentazione pi� generale di un modello stimabile con \app{gretl}
� la seguente:
\begin{equation}
  \label{eq:general-arma}
  A(L) B(L^s) y_t = x_t \beta + C(L) D(L^s) \epsilon_t ,
\end{equation}
dove $L$ � l'operatore di ritardo ($L^n x_t = x_{t-n}$), $s$ � il numero di
sotto-periodi per le serie stagionali (ad esempio, 12 per serie mensili),
$x_t$ � un vettore di variabili esogene, e $\epsilon_t$ � un processo a
rumore bianco.

Il modello ARMA ``basilare'' si ottiene quando vale $x_t = 1$ e non ci sono
operatori stagionali. In questo caso, $B(L^s) = D(L^s) = 1$ e il modello diventa
\begin{equation}
  \label{eq:plain-arma}
  A(L) y_t = \mu + C(L) \epsilon_t ,
\end{equation}
dove, nella solita notazione, il vettore $\beta$ si riduce all'intercetta $\mu$.
� possibile scrivere l'equazione precedente in modo pi� esplicito come
\[
  y_t = \mu + \phi_1 y_{t-1} + \ldots + \phi_p y_{t-p} + 
  \epsilon_t + \theta_1 \epsilon_{t-1} + \ldots + \theta_q
  \epsilon_{t-q} ;
\]
la sintassi corrispondente in \app{gretl} � semplicemente
\begin{code}
  gretl p q ; y
\end{code}
dove \verb|p| e \verb|q| sono gli ordini di ritardo desiderati; questi possono
essere sia numeri, sia scalari pre-definiti. Il parametro $\mu$ pu� essere
omesso se necessario, aggiungendo l'opzione \cmd{--nc} al comando.

Se si vuole stimare un modello con variabili esplicative, la sintassi vista
sopra pu� essere estesa in questo modo
\begin{code}
  gretl p q ; y const x1 x2
\end{code}
Questo comando stimerebbe il seguente modello:
\[
  y_t = \beta_0 + x_{1,t} \beta_1 + x_{2,t} \beta_2 + 
  \phi_1 y_{t-1} + \ldots + \phi_p y_{t-p} + 
  \epsilon_t + \theta_1 \epsilon_{t-1} + \ldots + \theta_q \epsilon_{t-q} .
\]
Quello che \app{gretl} stima in questo caso � un modello ARMAX (ARMA +
variabili esogene), che � diverso da ci� che alcuni altri programmi chiamano
``modello di regressione con errori ARMA''. La differenza � evidente
considerando il modello stimato da \app{gretl}
\begin{equation}
  \label{eq:armax}
  A(L) y_t = x_t \beta + C(L) \epsilon_t ,
\end{equation}
e un modello di regressione con errori ARMA, ossia
\begin{eqnarray}
  \label{eq:reg-arma}
  y_t & = & x_t \beta + u_t \\
  A(L) u_t & = & C(L) \epsilon_t ;
\end{eqnarray}
l'ultimo si tradurrebbe nell'espressione seguente
\[
  A(L) y_t = A(L) \left(x_t \beta \right) + C(L) \epsilon_t ;
\]
la formulazione ARMAX ha il vantaggio che il coefficiente $\beta$
pu� essere interpretato immediatamente come l'effetto marginale delle
variabili $x_t$ sulla media condizionale di $y_t$.

La struttura del comando \cmd{arma} non permette di specificare modelli con
buchi nella struttura dei ritardi\footnote{In realt� questa limitazione pu�
essere aggirata per quanto riguarda la parte autoregressiva, includendo ritardi
della variabile dipendente nella lista delle variabili esogene. In questo modo
per�, anche se si ottengono stime corrette, rende inutilizzabile la funzionalit�
di previsione attualmente disponibile.}. Una struttura di ritardi pi� flessibile
pu� essere necessaria quando si analizzano serie che mostrano un marcato
andamento stagionale. In questo caso � possibile usare il modello completo
(\ref{eq:general-arma}). Ad esempio, la sintassi
\begin{code}
  arma 1 1 ; 1 1 ; y
\end{code}
permette di stimare un modello con quattro parametri:
\[
  ( 1 - \phi L )  ( 1 - \Phi L^s ) y_t = \mu + ( 1 + \theta L ) ( 1 + \Theta L^s ) \epsilon_t;
\]
assumendo che $y_t$ � una serie trimestrale (e quindi $s=4$), l'equazione
precedente pu� essere scritta in modo pi� esplicito come
\[
  y_t = \mu + \phi y_{t-1} + \Phi y_{t-4} - (\phi \cdot \Phi) y_{t-5} + 
  \epsilon_t + \theta \epsilon_{t-1} + \Theta \epsilon_{t-4} +
  (\theta \cdot \Theta) \epsilon_{t-5} .
\]

\section{Stima}
\label{arma-est}

L'algoritmo usato da \app{gretl} per stimare i parametri di un modello ARMA
� quello della massima verosimiglianza condizionale, noto anche come ``somma
dei quadrati condizionale'' (si veda Hamilton, 1994, pagina 132).
Questo metodo � esemplificato nello script \ref{jack-arma}, quindi ne verr�
data solo una breve descrizione qui.

\ldots

\section{Previsione}
\label{arma-fcast}


%%% Local Variables: 
%%% mode: latex
%%% TeX-master: "gretl-guide"
%%% End: 

