\chapter{Dati panel}
\label{chap-panel}

\section{Stima di modelli panel}

\subsection{Minimi quadrati ordinari}
\label{pooled-est}

Lo stimatore pi� semplice per i dati panel � quello ``pooled OLS'' (ossia i
minimi quadrati ordinari utilizzando allo stesso modo tutte le osservazioni del
campione). In genere questo stimatore non fornisce risultati ottimali, ma
rappresenta un metro di paragone per stimatori pi� complessi.

Quando si stima un modello usando i minimi quadrati ordinari (OLS) su dati
panel, � disponibile un tipo di test aggiuntivo: nel men� \textsf{Test} della
finestra del modello � il comando ``Diagnosi panel'', mentre nella versione a
riga di comando del programma � il comando \cmd{hausman}.

Per eseguire questo test, occorre specificare un modello senza alcuna variabile
dummy relativa alle unit� cross-section. Il test confronta il semplice modello
OLS con le due principali alternative: il modello a effetti fissi e quello a
effetti casuali.

Il modello a \emph{effetti fissi} sopprime la costante e aggiunge una variabile
dummy per tutte le unit� cross section, permettendo cos� all'intercetta della
regressione di variare per ogni unit�.  Viene eseguito un test \emph{F} per
testare l'ipotesi nulla che l'intercetta sia uguale per tutte le
unit�: un basso p-value per questo test depone contro l'ipotesi nulla e a favore
dell'adeguatezza del modello a effetti fissi.

Il modello a \emph{effetti casuali}, d'altra parte, scompone la varianza dei
residui in due parti: una specifica all'unit� cross section, o
``gruppo'', e una specifica all'osservazione particolare (la stima pu�
essere eseguita solo se il panel � abbastanza ``largo'', ossia se il
numero delle unit� cross section nel dataset � maggiore del numero dei
parametri da stimare).  La statistica LM di Breusch--Pagan testa
l'ipotesi nulla che il modello pooled OLS sia adeguato, rispetto all'alternativo
modello a effetti casuali.

Pu� accadere che il modello pooled OLS sia rifiutato nei confronti di
entrambe le alternative, a effetti fissi o casuali. Come giudicare
quindi l'adeguatezza relativa dei due metodi alternativi di stima? Il
test di Hausman (i cui risultati sono mostrati a patto che il modello
a effetti casuali possa essere stimato) cerca di risolvere questo
problema.  A patto che gli errori specifici di unit� o di gruppo siano
non correlati con le variabili indipendenti, lo stimatore a effetti
casuali � pi� efficiente dello stimatore a effetti fissi; nel caso
contrario lo stimatore a effetti casuali non � consistente e deve
essergli preferito lo stimatore a effetti fissi.  L'ipotesi nulla per
il test di Hausman � che l'errore specifico di gruppo non sia
correlato con le variabili indipendenti (e quindi che il modello a
effetti casuali sia preferibile). Un basso p-value per questo test
suggerisce di rifiutare il modello a effetti casuali in favore del
modello a effetti fissi.  

Per una discussione rigorosa di questo argomento, si veda il capitolo
14 di Greene (2000).
 
\section{I modelli a effetti fissi e casuali}
\label{panel-est}

La specificazione ``pooled OLS'' si pu� scrivere come
\[
y_{it} = X_{it}\beta + u_{it}
\]
dove $y_{it}$ � l'osservazione della variabile dipendente per l'unit� cross
section $i$ nel periodo $t$, $X_{it}$ � un vettore $1\times k$ di variabili
indipendenti osservate per l'unit� $i$ nel periodo $t$,
$\beta$ � un vettore $k\times 1$ di parametri, e $u_{it}$ � un termine di errore
o di disturbo specifico all'unit� $i$ nel periodo $t$.

I modelli a effetti fissi e casuali rappresentano due modi di scomporre l'errore
unitario $u_{it}$.  Per il modello a effetti fissi possiamo scrivere
\[
u_{it} = \alpha_i + \varepsilon_{it}
\]
Ossia, scomponiamo $u_{it}$ in una componente specifica all'unit� e invariante
nel tempo, $\alpha_i$, e un errore specifico all'osservazione,
$\varepsilon_{it}$.  Gli $\alpha_i$ sono quindi trattati come parametri fissi
(intercette $y$ specifiche per le unit�) da stimare. Questo pu� essere fatto
includendo una variabile dummy per ogni unit� cross section (e sopprimendo la
costante comune), oppure � possibile procedere sottraendo le medie di gruppo da
ognuna delle variabili e stimando un modello senza costante. Nell'ultimo caso la
variabile dipendente si pu� scrivere come
\[
\tilde{y}_{it} = y_{it} - \frac{1}{T_i} \sum_{t=1}^{T_i} y_{it}
\]
dove $T_i$ � il numero di osservazioni per l'unit� $i$; in modo simile si
possono riscrivere le variabili indipendenti. � possibile ottenere stime degli
$\alpha_i$ usando
\[
\hat{\alpha}_i = \frac{1}{T_i} \sum_{t=1}^{T_i} 
   \left(y_{it} - X_{it}\hat{\beta}\right)
\]

Questi due metodi sono numericamente equivalenti, e \app{gretl} sceglie tra i
due a seconda del numero di unit� cross section e di variabili indipendenti
(cercando di minimizzare l'uso delle memoria).

Per il modello a effetti casuali possiamo scrivere
\[
u_{it} = v_i + \varepsilon_{it}
\]
Invece di trattare gli $v_i$ come parametri fissi, li trattiamo come estrazioni
casuali da una certa distribuzione di probabilit�. Per stimare questo modello
occorre per prima cosa fare un inferenza a proposito delle rispettive varianze
di $v_i$ e $\varepsilon_{it}$. La varianza di $v_i$, $\sigma^2_v$, � di solito
chiamata la varianza ``between'' (esterna, visto che si riferisce alla
variazione tra le unit� cross section), mentre $\sigma^2_{\varepsilon}$ �
chiamata varianza ``within'' (interna).

\section{Esempio: la Penn World Table}
\label{PWT}

La Penn World Table (homepage a
\href{http://pwt.econ.upenn.edu/}{pwt.econ.upenn.edu}) � un ricco
dataset panel macroeconomico, che comprende 152 paesi sull'arco
temporale 1950--1992. I dati sono disponibili in formato \app{gretl}:
si veda la
\href{http://gretl.sourceforge.net/gretl_data_it.html}{pagina dei
  dati} di \app{gretl} (i dati sono liberamente scaricabili, anche se
non sono distribuiti nel pacchetto principale di
\app{gretl}).

L'esempio \ref{examp-pwt} apre il file
\verb+pwt56_60_89.gdt+, un sottoinsieme della Penn World Table che
contiene dati per 120 paesi negli anni 1960--89, su 20 variabili,
senza osservazioni mancanti (il dataset completo, anch'esso compreso
nel pacchetto pwt per \app{gretl}, contiene molte osservazioni
mancanti). Viene calcolata la crescita del PIL reale sul periodo
1960--89 per ogni paese, e viene regredita sul livello del PIL reale
dell'anno 1960, per analizzare l'ipotesi della ``convergenza'' (ossia
di una crescita pi� veloce da parte dei paesi che partono da una
situazione peggiore).

\begin{script}[htbp]
  \caption{Uso della Penn World Table}
  \label{examp-pwt}

\begin{code}
          open pwt56_60_89.gdt 
          # Per l'anno 1989 (l'ultima oss.) il ritardo 29 d� 1960, la prima oss. 
          genr gdp60 = RGDPL(-29) 
          # Calcola la crescita totale del PIL reale sui 30 anni
          genr gdpgro = (RGDPL - gdp60)/gdp60
          # Restringi il campione a una cross-section del 1989
          smpl --restrict YEAR=1989 
          # convergenza: i paesi che partono pi� indietro crescono di pi�?
          ols gdpgro const gdp60 
          # risultato: no! Proviamo una relazione inversa?
          genr gdp60inv = 1/gdp60 
          ols gdpgro const gdp60inv 
          # Ancora no. Proviamo a trattare l'Africa in modo speciale? 
          genr afdum = (CCODE = 1)
          genr afslope = afdum * gdp60 
          ols gdpgro const afdum gdp60 afslope 
\end{code}

\end{script}


%%% Local Variables: 
%%% mode: latex
%%% TeX-master: "gretl-guide-it"
%%% End: 

