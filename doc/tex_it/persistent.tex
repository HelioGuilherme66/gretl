\chapter{Oggetti persistenti}
\label{chap-persist}

%%% Work in progress, not really ready for translation yet.

%FIXME This chapter needs to deal with saving models too.

\section{Liste definite dall'utente}
\label{named-lists}

Molti comandi di \app{gretl} richiedono come argomenti una o pi� liste di
variabili. Per semplificare la loro gestione negli script di comandi, in
particolare nelle funzioni definite dall'utente, \app{gretl} offre la
possibilit� di creare \textit{liste definite dall'utente}.  

\subsection{Creazione e modifica di una lista}

Per creare una lista occorre usare il comando \texttt{list}, seguito dal nome
della lista, un segno di uguale, e, in alternativa, \texttt{null} (per creare
una lista vuota) o il nome di una o pi� variabili da inserire nella lista. Ad
esempio:
%
\begin{code}
list xlist = 1 2 3 4
list reglist = income price 
list empty_list = null
\end{code}

Il nome della lista deve iniziare con una lettera e deve essere composto
interamente da lettere, numeri o il carattere trattino basso (non da spazi). La
lunghezza massima del nome � di 15 caratteri. Quando si aggiungono variabili a
una lista � possibile riferirsi a essa per nome o per numero identificativo.

Una volta creata, una lista viene ``ricordata'' per l'intera durata della
sessione di \app{gretl} e pu� essere usata nel contesto di qualsiasi comando
\app{gretl} che accetta liste di variabili. Un semplice esempio � la
specificazione di una lista di regressori:
%
\begin{code}
list xlist = x1 x2 x3 x4
ols y 0 xlist
\end{code}

Le liste possono essere modificate in due modi. Per \textit{ridefinire} una
lista si usa la stessa sintassi usata per crearla. Ad esempio
%
\begin{code}
list xlist = 1 2 3
list xlist = 4 5 6
\end{code}

Dopo il secondo comando, \texttt{xlist} contiene solo le variabili 4,
5 e 6.

Per \textit{accodare} o \textit{premettere} variabili a una lista esistente,
� possibile usare il nome di una lista all'interno del comando di definizione
della lista stessa. Ad esempio, � possibile scrivere
%
\begin{code}
list xlist = xlist 5 6 7
list xlist = 9 10 xlist 11 12
\end{code}

\subsection{Interrogazione delle liste}

� possibile determinare se una variabile sconosciuta rappresenta una lista
usando la funzione \texttt{islist()}.
%
\begin{code}
series xl1 = log(x1)
series xl2 = log(x2)
list xlogs = xl1 xl2
genr is1 = islist(xlogs)
genr is2 = islist(xl1)
\end{code}

Il primo comando \texttt{genr} assegner� il valore 1 a \texttt{is1} visto che
\texttt{xlogs} in effetti � una lista. Il secondo comando genr assegner� il
valore 0 a \texttt{is2}, visto che \texttt{xl1} � una serie di dati, non una
lista.

� anche possibile determinare il numero di elementi che compongono una lista
usando la funzione \texttt{nelem()}.
%
\begin{code}
list xlist = 1 2 3
genr nl = nelem(xlist)
\end{code}

Lo scalare \texttt{nl} avr� un valore di 3, visto che
\texttt{xlist} contiene 3 membri.

� possibile mostrare i membri di una lista come illustrato in questa sessione
interattiva:
%
\begin{code}
? list xlist = x1 x2 x3
 # list xlist = x1 x2 x3
Added list 'xlist'
? list xlist print
 # list xlist = x1 x2 x3
\end{code}
%
Si noti che \texttt{print xlist} produrr� un effetto diverso, ossia mostrer� i
valori di tutte le variabili contenute in \texttt{xlist} (come ci si pu�
aspettare).

\subsection{Liste di variabili trasformate}

Data una lista di variabili, � possibile generare liste che contengono
trasformazioni di queste variabili usando una forma speciale dei comandi
\texttt{logs}, \texttt{lags}, \texttt{diff}, \texttt{ldiff},
\texttt{sdiff} o \texttt{square}. In pratica basta far seguire questi comandi
dal nome della lista tra parentesi. Ad esempio:
%
\begin{code}
list xlist = x1 x2 x3
list lxlist = logs(xlist)
list difflist = diff(xlist)
\end{code}

Quando si genera una lista di \textit{ritardi} in questo modo, � possibile
specificare il massimo ordine di ritardi, inserendolo per primo tra parentesi e
separandolo dal nome della lista con una virgola. Ad esempio
%
\begin{code}
list xlist = x1 x2 x3
list laglist = lags(2, xlist)
\end{code}
%
oppure
%
\begin{code}
scalar order = 4
list laglist = lags(order, xlist)
\end{code}

Questi comandi riempiranno \texttt{laglist} col numero specificato di ritardi
delle variabili presenti in \texttt{xlist}. Come avviene per il normale comando
\texttt{lags} � possibile omettere l'ordine, che sar� cos� determinato
automaticamente a seconda della frequenza dei dati. Un altro modo speciale di
utilizzare questi comandi consiste nell'indicare tra parentesi il nome di una
singola variabile al posto di quello di una lista, come in
%
\begin{code}
series lx = log(x)
list laglist = lags(4, lx)
\end{code}

Si noti che la sintassi normale per questi comandi, ad esempio \texttt{logs},
� semplicemente
%
\begin{code}
logs x1 x2 x3
\end{code}
%
Se \texttt{xlist} � una lista, � anche possibile scrivere
%
\begin{code}
logs xlist
\end{code}
%
ma questo comando non salver� i logaritmi in una lista; per avere questo
risultato occorre usare la sintassi
%
\begin{code}
list loglist = logs(xlist)
\end{code}


\subsection{Controllo dei valori mancanti}

\app{Gretl} offre varie funzioni per riconoscere e gestire i valori mancanti (si
veda la \GCR{} per i dettagli). In questa sede � utile ricordare che la funzione
\texttt{ok()} pu� essere usata con un argomento lista. Ad esempio:
%
\begin{code}
list xlist = x1 x2 x3
series xok = ok(xlist)
\end{code}
%
Dopo questi comandi, la serie \texttt{xok} avr� valore 1 per le osservazioni in
cui nessuna variabile tra \texttt{x1}, \texttt{x2}, e
\texttt{x3} ha un valore mancante, e valore 0 per le osservazioni che non
soddisfano questa condizione.


\section{Stringhe definite dall'utente}
\label{named-strings}

In alcuni casi pu� essere utile salvare una stringa (ossia una sequenza di
caratteri) sotto forma di variabile che possa essere riutilizzata.
\app{Gretl} offre questa funzionalit� a partire dalla versione 1.6.0.

Per \textit{definire} una variabile stringa � possibile usare uno dei comandi
seguenti: \texttt{string} o \texttt{sprintf}.  Il comando \texttt{string}
� il pi� semplice: basta scrivere, ad esempio
%
\begin{code}
string pippo = "Qualcosa da ricordare"
\end{code}
%
Il primo campo dopo \texttt{string} � il nome della variabile che verr� creata,
quindi segue il segno di uguale, infine la stringa da salvare, racchiusa tra
virgolette doppie. Quest'ultima pu� anche essere indicata come sequenza di
sotto-stringhe, come in questo esempio:
%
\begin{code}
string pezzi = "Primo " "e" " secondo"
\end{code}
%
Il comando \texttt{sprintf} � pi� flessibile. Funziona esattamente come
\texttt{printf}, tranne per il fatto che la stringa ``di formato'' deve
essere preceduta dal nome di una variabile stringa. Ad esempio,
%
\begin{code}
scalar x = 8
sprintf pippo "var%d", x
\end{code}
%
Per \textit{recuperare il valore} di una variabile stringa, basta usare il nome
della variabile preceduto dal carattere ``at'', \verb|@|.  In un comando
\app{gretl}, una sequenza di caratteri che segue il simbolo \verb|@| viene
trattata come il nome di una variabile stringa e il valore della variabile viene
sostituito sulla riga di comando, come avviene in questo esempio di sessione
interattiva:
%
\begin{code}
? scalar x = 8
 scalar x = 8
Generato lo scalare x (ID 2) = 8
? sprintf pippo "var%d", x
Salvata la stringa 'pippo'
? print "@foo"
var8
\end{code}

\subsection{Stringhe predefinite}

Oltre alle stringhe definite dall'utente, \app{gretl} contiene alcune variabili
stringa predefinite che possono essere utili per chi scrive funzioni che
includono comandi shell. Le stringhe predefinite sono mostrate nella
Tabella~\ref{tab-strings}.

\begin{table}[htbp]
\centering
\begin{tabular}{ll}
  \verb|@gretldir| & La directory di installazione di \app{gretl} \\
  \verb|@userdir| & La directory utente di \app{gretl} \\
  \verb|@gnuplot| & Percorso, o nome, dell'eseguibile \app{gnuplot} \\
  \verb|@tramo| & Percorso, o nome, dell'eseguibile \app{tramo} \\
  \verb|@x12a| & Percorso, o nome, dell'eseguibile \app{x-12-arima} \\
  \verb|@tramodir| & Directory dei dati di \app{tramo} \\
  \verb|@x12adir| & Directory dei dati di \app{x-12-arima} \\
\end{tabular}
\caption{Variabili stringa predefinite}
\label{tab-strings}
\end{table}

%%% Local Variables: 
%%% mode: latex
%%% TeX-master: "gretl-guide"
%%% End: 

