\chapter{Esempi svolti}
\label{chap:cheatsheet}

Questo capitolo spiega come eseguire alcuni compiti comuni (e altri meno comuni)
usando il linguaggio di scripting di \app{gretl}. Alcune delle tecniche mostrate,
ma non tutte, sono utilizzabili anche attraverso l'interfaccia grafica del
programma; sebbene questa possa sembrare pi� intuitiva e facile da utilizzare a
prima vista, incoraggiamo gli utenti a sfruttare le potenzialit� del linguaggio
di scripting di \app{gretl}, dopo aver preso confidenza col programma.

\section{Gestione dei dataset}

\subsection{Periodicit� ``strane''}

\emph{Problema:} si hanno dati rilevati ogni 3 minuti a partire dalle ore 9,
ossia ogni ora � suddivisa in 20 periodi.

\emph{Soluzione:}
\begin{code}
setobs 20 9:1 --special
\end{code}

\emph{Commento:} ora le funzioni come \texttt{sdiff()} (differenza ``stagionale'')
o i metodi di stima come l'ARIMA stagionale funzioneranno come ci si aspetta.

\subsection{Eliminare osservazioni mancanti in modo selettivo}

\emph{Problema:} si ha un dataset con molte variabili e si vuole restringere il
campione a quelle osservazioni per cui non ci sono osservazioni mancanti per
nessuna delle variabili \texttt{x1}, \texttt{x2} e \texttt{x3}.

\emph{Soluzione:}
\begin{code}
list X = x1 x2 x3
genr sel = ok(X)
smpl sel --restrict
\end{code}

\emph{Commento:} ora � possibile salvare il file con il comando \texttt{store} per
preservare una versione ristretta del dataset.

\subsection{Operazioni diverse a seconda dei valori di una variabile}

\emph{Problema:} si ha una variabile discreta \texttt{d} e si vuole eseguire
alcuni comandi (ad esempio, stimare un modello) suddividendo il campione a
seconda dei valori di \texttt{d}.

\emph{Soluzione:}
\begin{code}
matrix vd = values(d)
m = rows(vd)
loop for i=1..m
  scalar sel = vd[i]
  smpl (d=sel) --restrict --replace
  ols y const x
end loop
smpl full
\end{code}

\emph{Commento:} L'ingrediente principale � un loop, all'interno del quale �
possibile eseguire tutte le istruzioni volute per ogni valore di
\texttt{d}, a patto che siano istruzioni il cui uso � consentito all'interno di
un loop.

\section{Creazione e modifica di variabili}

\subsection{Generazione di un ARMA(1,1)}

\emph{Problema:} Generare $y_t = 0.9 y_{t-1} + \varepsilon_t - 0.5
\varepsilon_{t-1}$, con $\varepsilon_t \sim NIID(0,1)$.

\emph{Soluzione:}
\begin{code}
alpha = 0.9
theta = -0.5
series e = normal()
series y = 0
series y = alpha * y(-1) + e + theta * e(-1)
\end{code}

\emph{Commento:} L'istruzione \texttt{series y = 0} � necessaria perch�
l'istruzione successiva valuta \texttt{y} ricorsivamente, quindi occorre
impostare \texttt{y[1]}. Si noti che occorre usare la parola chiave
\texttt{series}, invece di scrivere \texttt{genr y = 0} o semplicemente
\texttt{y = 0}, per assicurarsi che \texttt{y} sia una serie e non uno scalare.

\subsection{Assegnazione condizionale}

\emph{Problema:} Generare $y_t$ secondo la regola seguente:
\[
  y_t = \left\{ 
    \begin{array}{ll} 
      x_t & \mathrm{for} \quad d_t = 1 \\ 
      z_t & \mathrm{for} \quad d_t = 0 
    \end{array}
    \right. 
\]

\emph{Soluzione:}
\begin{code}
series y = d ? x : z
\end{code}

\emph{Commento:} ci sono varie alternative a quella presentata. La prima �
quella di forza bruta usando i loop. Un'altra, pi� efficiente ma ancora
subottimale, � quella di usare \verb|y = d*x + (1-d)*z|. L'operatore di
assegnazione condizionale ternario non solo rappresenta la soluzione
numericamente pi� efficiente, ma � anche quella di pi� semplice lettura, una volta
che si � abituati alla sua sintassi, che per alcuni lettori pu� ricordare quella
della funzione \texttt{=IF()} nei fogli di calcolo.

\section{Trucchi utili}
\label{sec:cheat-neat}

\subsection{Dummy di interazione}

\emph{Problema:} si vuole stimare il modello $y_i = \mathbf{x}_i
\beta_1 + \mathbf{z}_i \beta_2 + d_i \beta_3 + (d_i \cdot \mathbf{z}_i)
\beta_4 + \varepsilon_t$, dove $d_i$ � una variabile dummy, mentre
$\mathbf{x}_i$ e $\mathbf{z}_i$ sono vettori di variabili esplicative.

\emph{Soluzione:}
\begin{code}
list X = x1 x2 x3
list Z = z1 z2
list dZ = null
loop foreach i Z
  series d$i = d * $i
  list dZ = dZ d$i
end loop 

ols y X Z d dZ
\end{code} 
%$

\emph{Commento:} incredibile cosa si pu� fare con la sostituzione delle
stringhe, vero?

\subsection{Volatilit� percepita}

\emph{Problema:} avendo dati raccolti ogni minuto, si vuole calcolare la
``realized volatility'' per ogni ora come $RV_t = \frac{1}{60}
\sum_{\tau=1}^{60} y_{t:\tau}^2$. Il campione parte da 1:1.

\emph{Soluzione:}
\begin{code}
smpl full
genr time
genr minute = int(time/60) + 1
genr second = time % 60
setobs minute second --panel
genr rv = psd(y)^2
setobs 1 1
smpl second=1 --restrict
store foo rv
\end{code}

\emph{Commento:} qui facciamo credere a \app{gretl} che il dataset sia di tipo
panel, dove i minuti sono le ``unit�'' e i secondi sono il ``tempo''; in questo
modo possiamo utilizzare la funzione speciale \texttt{psd()} (panel standard deviation).
Quindi eliminiamo semplicemente tutte le osservazioni tranne una per minuto e
salviamo i dati risultanti (\texttt{store foo rv} significa ``salva nel file di dati
\texttt{foo.gdt} la serie \texttt{rv}'').


%%% Local Variables: 
%%% mode: latex
%%% TeX-master: "gretl-guide"
%%% End: 
