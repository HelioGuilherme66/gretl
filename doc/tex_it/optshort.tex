As can be seen from section~\ref{cmd-cmd}, the behavior of many gretl
commands can be modified via the use of option flags. These take the
form of two dashes followed by a string which is somewhat descriptive
of the effect of the option.

Some options require a parameter, which must be joined to the option
``flag'' with an equals sign. Among the options that do \emph{not}
require a parameter, certain common ones have a short form---a single
dash followed by a single letter---and it is considered idiomatic to
use the short forms in hansl scripts. The table below shows the
relevant mapping: for any command which supports the long-form option
in the first column, the short form in the second column is also
supported.

\begin{center}
\begin{tabular}{lc}
long form & short form \\ [4pt]
\option{verbose} & \texttt{-v} \\
\option{quiet}   & \texttt{-q} \\
\option{robust}  & \texttt{-r} \\
\option{hessian} & \texttt{-h} \\
\option{window}  & \texttt{-w} 
\end{tabular}
\end{center}