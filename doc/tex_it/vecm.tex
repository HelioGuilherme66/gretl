\chapter{Cointegrazione e modelli vettoriali a correzione d'errore}
\label{chap:vecm}

\section{Introduzione}
\label{sec:VECM-intro}

I concetti correlati di cointegrazione e correzione d'errore sono stati al
centro della ricerca in macroeconometria negli ultimi anni. L'aspetto
interessante del Modello Vettoriale a Correzione di Errore (VECM) consiste nel
fatto che permette al ricercatore di inserire una rappresentazione di relazioni
di equilibrio economico in una specificazione abbastanza ricca basata sulle
serie storiche. Questo approccio supera l'antica dicotomia tra i modelli
strutturali, che rappresentavano fedelmente la teoria macroeconomica ma non si
adattavano ai dati, e l'analisi delle serie storiche, che era più precisa nel
riprodurre l'andamento dei dati, ma di difficile, se non impossibile,
interpretazione in termini di teoria economica.

L'idea basilare della cointegrazione è strettamente collegata al concetto di
radici unitarie (si veda la sezione~\ref{sec:uroot}).  Si supponga di avere un
insieme di variabili macroeconomiche di interesse, e di non poter rifiutare
l'ipotesi che alcune di queste variabili, considerate individualmente, siano
non-stazionarie. In particolare, si supponga che un sottoinsieme di queste
variabili siano individualmente integrate di ordine 1, o I(1), ossia che non
siano stazionarie, ma che la loro differenza prima sia stazionaria. Dati i
problemi di tipo statistico che sono associati all'analisi dei dati non
stazionari (ad esempio il problema della regressione spuria), l'approccio
tradizionale in questo caso consiste nel prendere la differenza prima delle
variabili prima di procedere con l'analisi statistica.

In questo modo però, si perde informazione importante. Può darsi che mentre le
variabili sono I(1) prese singolarmente, esista una loro combinazione lineare
che sia invece stazionaria, ossia I(0) (potrebbe esserci anche più di una
combinazione lineare). In altri termini, mentre l'insieme delle variabili è
libero di muoversi nel tempo, esistono comunque delle relazioni che legano fra
di loro le variabili; è possibile interpretare queste relazioni, o
\emph{vettori di cointegrazione} come condizioni di equilibrio.

Ad esempio, ipotizziamo di scoprire che la quantità di moneta, $M$, il livello
dei prezzi, $P$, il tasso di interesse nominale, $R$, e l'output, $Y$, siano
tutti I(1). Secondo la teoria standard della domanda di moneta, dovremmo
comunque aspettarci una relazione di equilibrio tra la quantità di moneta reale,
il tasso d'interesse e l'output; ad esempio
\[
m - p = \gamma_0 + \gamma_1 y + \gamma_2 r \qquad \gamma_1 > 0,
\gamma_2 < 0
\]
dove le variabili in minuscolo indicano i logaritmi. In equilibrio si ha quindi
\[
m - p - \gamma_1 y - \gamma_2 r = \gamma_0
\]
Nella realtà non ci si aspetta che questa condizione sia soddisfatta in ogni
periodo, ma occorre ammettere la possibilità di disequilibri di breve periodo.
Ma se il sistema ritorna all'equilibrio dopo un disturbo, ne consegue che
il vettore $x = (m, p, y, r)'$ è limitato da un vettore di cointegrazione
$\beta' = (\beta_1, \beta_2, \beta_3, \beta_4)$, tale che $\beta'x$ è
stazionario (con una media pari a $\gamma_0$). Inoltre, se l'equilibrio è
caratterizzato correttamente dal semplice modello visto sopra, si ha $\beta_2 =
-\beta_1$, $\beta_3 < 0$ e $\beta_4 > 0$. Queste proprietà sono testabili
attraverso l'analisi di cointegrazione.

Questa analisi consiste tipicamente in tre passi:
\begin{enumerate}
\item Test per verificare il numero di vettori di cointegrazione, ossia il 
  \emph{rango di cointegrazione} del sistema.
\item Stima di un VECM di rango appropriato, non soggetto ad altre restrizioni.
\item Test dell'interpretazione dei vettori di cointegrazione come condizioni di
  equilibrio, usando le restrizioni sugli elementi di questi vettori.
\end{enumerate}

Le sezioni seguenti approfondiscono ognuno dei passi, aggiungendo altre
considerazioni econometriche e spiegando come implementare l'analisi usando
\app{gretl}.

\section{Modelli vettoriali a correzione di errore (VECM) come rappresentazione di
un sistema cointegrato}
\label{sec:VECM-rep}

Si consideri un VAR di ordine $p$ con una parte deterministica data da $\mu_t$
(tipicamente un polinomio nel tempo). È possibile scrivere il processo
$n$-variato $y_t$ come
\begin{equation}
  \label{eq:VECM-VAR}
  y_t = \mu_t + A_1 y_{t-1} + A_2 y_{t-2} + \cdots + A_p y_{t-p} +
  \epsilon_t 
\end{equation}
Ma poiché $y_t \equiv y_{t-1} - \Delta y_t$ e $y_{t-i} \equiv
y_{t-1} - (\Delta y_{t-1} + \Delta y_{t-2} + \cdots + \Delta
y_{t-i+1})$, è possibile riscrivere l'equazione nel modo seguente:
\begin{equation}
  \label{eq:VECM}
  \Delta y_t = \mu_t + \Pi y_{t-1} + \sum_{i=1}^{p-1} \Gamma_i \Delta
  y_{t-i} + \epsilon_t ,
\end{equation}
dove $\Pi = \sum_{i=1}^p A_i$ e $\Gamma_k = -\sum_{i=k}^p A_i$.
Questa è la rappresentazione VECM della (\ref{eq:VECM-VAR}).

L'interpretazione della (\ref{eq:VECM}) dipende in modo cruciale da $r$, il
rango della matrice $\Pi$.
\begin{itemize}
\item Se $r = 0$, i processi sono tutti I(1).
\item Se $r = n$, $\Pi$ è invertibile e i processi sono tutti I(0).
\item La cointegrazione accade in tutti i casi intermedi, quando
  $0 < r < n$ e $\Pi$ può essere scritta come $\alpha \beta'$. In 
  questo caso, $y_t$ è I(1), ma la combinazione $z_t = \beta'y_t$ è I(0).
  Se, ad esempio, $r=1$ e il primo elemento di $\beta$ fosse $-1$, si
  potrebbe scrivere $z_t = -y_{1,t} + \beta_2 y_{2,t} + \cdots + \beta_n y_{n,t}$,
  che è equivalente a dire che
  \[
    y_{1_t} = \beta_2 y_{2,t} + \cdots + \beta_n y_{n,t} - z_t
  \]
  è una relazione di equilibrio di lungo periodo: le deviazioni
  $z_t$ possono non essere pari a zero, ma sono stazionarie. In questo caso,
  la (\ref{eq:VECM}) può essere scritta come
  \begin{equation}
    \label{eq:VECMab}
    \Delta y_t = \mu_t + \alpha \beta' y_{t-1} + \sum_{i=1}^{p-1} \Gamma_i 
    \Delta y_{t-i} + \epsilon_t ,
  \end{equation}
  e la stima viene effettuata simultaneamente su tutti i parametri.
  D'altra parte è utile notare che se $\beta$ fosse noto, $z_t$ sarebbe
  osservabile e tutti i restanti parametri potrebbero essere stimati con OLS. In
  pratica, la procedura stima per prima cosa $\beta$ e tutto il resto poi.
\end{itemize}

Il rango di $\Pi$ viene analizzato calcolando gli autovalori di una matrice ad
essa strettamente legata che ha rango pari a quello di $\Pi$, ma che per
costruzione è simmetrica e semidefinita positiva. Di conseguenza, tutti i suoi
autovalori sono reali e non negativi e i test sul rango di $\Pi$ possono quindi
essere condotti verificando quanti autovalori sono pari a 0.

Se tutti gli autovalori sono significativamente diversi da 0, tutti i processi
sono stazionari. Se, al contrario, c'è almeno un autovalore pari a 0, allora
il processo $y_t$ è integrato, anche se qualche combinazione lineare $\beta'y_t$
potrebbe essere stazionaria. All'estremo opposto, se non ci sono autovalori
significativamente diversi da 0, non solo il processo $y_t$ è non-stazionario,
ma vale lo stesso per qualsiasi combinazione lineare $\beta'y_t$; in altre
parole non c'è alcuna cointegrazione.

La stima procede tipicamente in due passi: per prima cosa si esegue una serie di
test per determinare $r$, il rango di cointegrazione. Quindi, per un certo rango
vengono stimati i parametri dell'equazione (\ref{eq:VECMab}). I due comandi
offerti da \app{gretl} per compiere queste operazioni sono rispettivamente
\texttt{coint2} e \texttt{vecm}.

La sintassi di \texttt{coint2} è
\begin{code}
  coint2 p ylist [ ; xlist ]
\end{code}
dove \texttt{p} è il numero di ritardi nella (\ref{eq:VECM-VAR}),
\texttt{ylist} è una lista che contiene le variabili $y_t$, e
\texttt{xlist} è una lista opzionale di variabili esogene.

La sintassi di \texttt{vecm} è
\begin{code}
  vecm p r ylist [ ; xlist [ ; zlist ] ]
\end{code}
dove \texttt{p} è il numero di ritardi nella (\ref{eq:VECM-VAR}),
\texttt{r} è il rango di cointegrazione, \texttt{ylist} è una lista
che contiene le variabili $y_t$, \texttt{xlist} è una lista opzionale di
variabili esogene, e \texttt{zlist} è un'altra lista opzionale di variabili
esogene che si ipotizza facciano parte delle relazioni di cointegrazione.

Entrambi i comandi supportano opzioni specifiche per trattare il kernel
deterministico $\mu_t$; queste sono illustrate nella sezione seguente.

\section{Interpretation of the deterministic kernel}
\label{sec:coint-5cases}

Several important inferential aspect of cointegrated systems depend on
what hypotheses one is willing to make on the deterministic terms,
which leads to the famous ``five cases.''

In equation (\ref{eq:VECM}), the term $\mu_t$ is usually understood to
take the form
\[
  \mu_t = \mu_0 + \mu_1 \cdot t .
\]
In order to have the model mimic as closely as possible the features
of the observed data, there is one preliminary question to settle. Do
the data appear to follow a deterministic trend? And if so, is it
linear or quadratic?

Once this is established, one should impose restrictions on $\mu_0$
and $\mu_1$ that are consistent with this choice. For example, suppose
that the data do not exhibit a discernible trend. This means
that $\Delta y_t$ is on average zero, so it is reasonable to assume
that its expected value is also zero. Write equation (\ref{eq:VECM})
as
\begin{equation}
  \label{eq:VECM-poly}
  \Gamma(L) \Delta y_t = \mu_0 + \mu_1 \cdot t + \alpha z_{t-1} +
  \epsilon_t ,
\end{equation}
where $z_{t} = \beta' y_{t}$ is assumed to be stationary and therefore
to possess finite moments. Taking unconditional expectations, we get
\[ 
  0 = \mu_0 + \mu_1 \cdot t + \alpha m_z .
\]
Since the left-hand side does not depend on $t$, the restriction
$\mu_1 = 0$ is a safe bet. As for $\mu_0$, there are just two ways to
make the above expression true: either $\mu_0 = 0$ or, less
restrictively, $\mu_0$ happens to be exactly equal to $-\alpha m_z$.
This possibility is less restrictive in that the vector $\mu_0$ may be
non-zero, but it is constrained to be a linear combination of the
columns of $\alpha$. But if this is the case, $\mu_0$ can be written
as $\alpha \cdot c$, and one may write (\ref{eq:VECM-poly}) as
\[
  \Gamma(L) \Delta y_t = \alpha \left[ \beta' \quad c \right] 
  \left[ \begin{array}{c} y_{t-1} \\ 1 \end{array} \right]  
  + \epsilon_t ,
\]
and the long-run relationship contains an intercept. This type of
restriction is usually written
\[
  \alpha'_{\perp} \mu_0 = 0 ,
\]
where $\alpha_{\perp}$ is the left null space of the matrix $\alpha$.

An intuitive understanding of the issue can be gained by means of a
simple example. Consider a series $x_t$ which behaves as follows
%      
\[ x_t = m + x_{t-1} + \varepsilon_t \] 
%
where $m$ is a real number and $\varepsilon_t$ is a white noise
process. As is easy to show, $x_t$ is a random walk which fluctuates
around a deterministic trend with slope $m$. In the special case $m$ =
0, the deterministic trend disappears and $x_t$ is a pure random walk.
    
Consider now another process $y_t$, defined by
%      
\[ y_t = k + x_t + u_t \] 
%
where, again, $k$ is a real number and $u_t$ is a white noise process.
Since $u_t$ is stationary by definition, $x_t$ and $y_t$ cointegrate:
that is, their difference
%      
\[ z_t = y_t - x_t = k + u_t \]
%	
is a stationary process. For $k$ = 0, $z_t$ is simple zero-mean white
noise, whereas for $k$ $\ne$ 0 the process $z_t$ is white noise with a
non-zero mean.
  
After some simple substitutions, the two equations above can be
represented jointly as a VAR(1) system
%      
\[ \left[ \begin{array}{c} y_t \\ x_t \end{array} \right] = \left[
  \begin{array}{c} k + m \\ m \end{array} \right] + \left[
  \begin{array}{rr} 0 & 1 \\ 0 & 1 \end{array} \right] \left[
  \begin{array}{c} y_{t-1} \\ x_{t-1} \end{array} \right] + \left[
  \begin{array}{c} u_t + \varepsilon_t \\ \varepsilon_t \end{array}
\right] \]
%	
or in VECM form
%      
\begin{eqnarray*}
  \left[  \begin{array}{c} \Delta y_t \\ \Delta x_t \end{array} \right]  & = & 
  \left[  \begin{array}{c} k + m \\ m \end{array} \right] +
  \left[  \begin{array}{rr} -1 & 1 \\ 0 & 0 \end{array} \right] 
  \left[  \begin{array}{c} y_{t-1} \\ x_{t-1} \end{array} \right] + 
  \left[  \begin{array}{c} u_t + \varepsilon_t \\ \varepsilon_t \end{array} \right] = \\
  & = & 
  \left[  \begin{array}{c} k + m \\ m \end{array} \right] +
  \left[  \begin{array}{r} -1 \\ 0 \end{array} \right]
  \left[  \begin{array}{rr} 1 & -1 \end{array} \right] 
  \left[  \begin{array}{c} y_{t-1} \\ x_{t-1} \end{array} \right] + 
  \left[  \begin{array}{c} u_t + \varepsilon_t \\ \varepsilon_t \end{array} \right] = \\
  & = & 
  \mu_0 + \alpha \beta^{\prime} \left[  \begin{array}{c} y_{t-1} \\ x_{t-1} \end{array} \right] + \eta_t = 
  \mu_0 + \alpha z_{t-1} + \eta_t ,
\end{eqnarray*}
%	
where $\beta$ is the cointegrazione vector and $\alpha$ is the
``loadings'' or ``adjustments'' vector.
     
We are now ready to consider three possible cases:
    
\begin{enumerate}
\item $m$ $\ne$ 0: In this case $x_t$ is trended, as we just saw; it
  follows that $y_t$ also follows a linear trend because on average it
  keeps at a fixed distance $k$ from $x_t$. The vector $\mu_0$ is
  unrestricted.
	
\item $m$ = 0 and $k$ $\ne$ 0: In this case, $x_t$ is not trended and
  as a consequence neither is $y_t$. However, the mean distance
  between $y_t$ and $x_t$ is non-zero. The vector
  $\mu_0$ is given by
%	  
  \[
  \mu_0 = \left[ \begin{array}{c} k \\ 0 \end{array} \right]
  \]
%	    
  which is not null and therefore the VECM shown above does have a
  constant term. The constant, however, is subject to the restriction
  that its second element must be 0. More generally,
  $\mu_0$ is a multiple of the vector $\alpha$. Note
  that the VECM could also be written as
%	  
  \[
  \left[ \begin{array}{c} \Delta y_t \\ \Delta x_t \end{array} \right]
  = \left[ \begin{array}{r} -1 \\ 0 \end{array} \right] \left[
    \begin{array}{rrr} 1 & -1 & -k \end{array} \right] \left[
    \begin{array}{c} y_{t-1} \\ x_{t-1} \\ 1 \end{array} \right] +
  \left[ \begin{array}{c} u_t + \varepsilon_t \\ \varepsilon_t
    \end{array} \right]
  \]
%	   
  which incorporates the intercept into the cointegrazione vector. This
  is known as the ``restricted constant'' case.
	
\item $m$ = 0 and $k$ = 0: This case is the most restrictive: clearly,
  neither $x_t$ nor $y_t$ are trended, and the mean distance between
  them is zero. The vector $\mu_0$ is also 0, which explains why this
  case is referred to as ``no constant.''
	
\end{enumerate}

In most cases, the choice between the three possibilities is based on
a mix of empirical observation and economic reasoning. If the
variables under consideration seem to follow a linear trend then we
should not place any restriction on the intercept. Otherwise, the
question arises of whether it makes sense to specify a cointegrazione
relationship which includes a non-zero intercept. One example where
this is appropriate is the relationship between two interest rates:
generally these are not trended, but the VAR might still have an
intercept because the difference between the two (the ``interest rate
spread'') might be stationary around a non-zero mean (for example,
because of a risk or liquidity premium).
    
The previous example can be generalized in three directions:
    
\begin{enumerate}
\item If a VAR of order greater than 1 is considered, the algebra gets
  more convoluted but the conclusions are identical.
\item If the VAR includes more than two endogenous variables the
  rango di cointegrazione $r$ can be greater than 1. In this case, $\alpha$
  is a matrix with $r$ columns, and the case with restricted constant
  entails the restriction that $\mu_0$ should be some linear
  combination of the columns of $\alpha$.
\item If a linear trend is included in the model, the deterministic
  part of the VAR becomes $\mu_0 + \mu_1 t$. The reasoning is
  practically the same as above except that the focus now centers on
  $\mu_1$ rather than $\mu_0$.  The counterpart to the ``restricted
  constant'' case discussed above is a ``restricted trend'' case, such
  that the cointegrazione relationships include a trend but the first
  differences of the variables in question do not.  In the case of an
  unrestricted trend, the trend appears in both the cointegrazione
  relationships and the first differences, which corresponds to the
  presence of a quadratic trend in the variables themselves (in
  levels).
\end{enumerate}

In order to accommodate the five cases, \app{gretl} provides the
following options to the \texttt{coint2} and \texttt{vecm} commands:
\begin{center}
  \begin{tabular}{cc}
    \hline
    $\mu_t$ & command option \\
    \hline
    0 & \verb|--nc| \\
    $\mu_0, \alpha_{\perp}'\mu_0 = 0 $ &  \verb|--rc| \\
    $\mu_0$ &  default \\
    $\mu_0 + \mu_1 t , \alpha_{\perp}'\mu_1 = 0$ &  \verb|--crt| \\
    $\mu_0 + \mu_1 t$ &  \verb|--ct| \\
    \hline
  \end{tabular}
\end{center}
Note that for this command the above options are mutually exclusive.
In addition, you have the option of using the \verb|--seasonal|
options, for augmenting $\mu_t$ with centered seasonal dummies.  In
each case, p-values are computed via the approximations by Doornik
(1998).

\section{I test di cointegrazione di Johansen}
\label{sec:johansen-test}

I due test di cointegrazione di Johansen vengono usati per stabilire il
rango di $\beta$, in altre parole il numero di vettori di cointegrazione
del sistema. Essi sono il test ``$\lambda$-max'', per le ipotesi sui singoli
autovalori, e il test ``traccia'', per le ipotesi congiunte.
Si supponga che gli autovalori $\lambda_i$ siano ordinati dal maggiore al
minore. L'ipotesi nulla per il test  ``$\lambda$-max'' sul
$i$-esimo autovalore è che sia $\lambda_i = 0$. Invece, il test traccia
corrispondente considera l'ipotesi che sia $\lambda_j = 0$ per ogni
$j \ge i$.

Il comando \cmd{coint2} di \app{gretl} esegue questi due test; la voce
corrispondente nel menù dell'interfaccia grafica è ``Modello, Serie Storiche,
COINT - Test di cointegrazione, Johansen''.

Come nel test ADF, la distribuzione asintotica dei test varia a seconda
del kernel deterministico $\mu_t$ incluso nel VAR (si veda la sezione
\ref{sec:coint-5cases}). Il codice seguente usa il file di dati
\cmd{denmark}, fornito insieme a \app{gretl}, per replicare l'esempio proposto
da Johansen nel suo libro del 1995.
%
\begin{code}
open denmark
coint2 2 LRM LRY IBO IDE --rc --seasonal
\end{code}
%
In questo caso, il vettore $y_t$ nell'equazione (\ref{eq:VECM}) comprende le
quattro variabili \cmd{LRM}, \cmd{LRY}, \cmd{IBO}, \cmd{IDE}. Il numero dei
ritardi equivale a $p$ nella (\ref{eq:VECM}) (ossia, il numero dei ritardi del
modello scritto in forma VAR). Di seguito è riportata parte
dell'output:

\begin{center}
\begin{code}
Test di Johansen:
Numero di equazioni = 4
Ordine dei ritardi = 2
Periodo di stima: 1974:3 - 1987:3 (T = 53)

Caso 2: costante vincolata
Rango Autovalore Test traccia p-value   Test Lmax  p-value
   0    0,43317     49,144 [0,1284]     30,087 [0,0286]
   1    0,17758     19,057 [0,7833]     10,362 [0,8017]
   2    0,11279     8,6950 [0,7645]     6,3427 [0,7483]
   3   0,043411     2,3522 [0,7088]     2,3522 [0,7076]
\end{code}
\end{center}

Sia il test traccia, sia quello $\lambda$-max portano ad accettare l'ipotesi nulla
che il più piccolo autovalore valga 0 (ultima riga della tabella), quindi possiamo
concludere che le serie non sono stazionarie. Tuttavia, qualche loro combinazione
lineare potrebbe essere I(0), visto che il test $\lambda$-max rifiuta l'ipotesi che
il rango di $\Pi$ sia 0 (anche se il test traccia dà un'indicazione meno netta
in questo senso, con un p-value pari a $0.1284$).

\section{Identification of the cointegrazione vectors}
\label{sec:johansen-ident}

The core problem in the estimation of equation (\ref{eq:VECM}) is to
find an estimate of $\Pi$ that has by construction rank $r$, so it can
be written as $\Pi = \alpha \beta'$, where $\beta$ is the matrix
containing the cointegrazione vectors and $\alpha$ contains the
``adjustment'' or ``loading'' coefficients whereby the endogenous
variables respond to deviation from equilibrium in the previous period.

Without further specification, the problem has multiple solutions (in
fact, infinitely many). The parameters $\alpha$ and $\beta$ are
under-identified: if all columns of $\beta$ are cointegrazione vectors,
then any arbitrary linear combinations of those columns is a
cointegrazione vector too.  To put it differently, if $\Pi = \alpha_0
\beta_0'$ for specific matrices $\alpha_0$ and $\beta_0$, then $\Pi$
also equals $(\alpha_0 \Sigma)(\Sigma^{-1} \beta_0')$ for any conformable
non-singular matrix $\Sigma$.  In order to find a unique solution, it
is therefore necessary to impose some restrictions on $\alpha$ and/or
$\beta$. It can be shown that the minimum number of restrictions that
is necessary to guarantee identification is $r^2$. Normalizing one
coefficient per column to 1 (or -1, according to taste) is a trivial first
step, which also helps in that the remaining coefficients can be
interpreted as the parameters in the equilibrium relations, but this
only suffices when $r=1$.

The method that \app{gretl} uses by default is known as the ``Phillips
normalization'', or ``triangular representation''.\footnote{For
  comparison with other studies, you may wish to normalize $\beta$
  differently.  Using the \texttt{set} command you can do 
  \verb|set vecm_norm diag| to select a normalization that simply scales the
  columns of the original $\beta$ such that $\beta_{ij} = 1$ for $i=j$
  and $i \leq r$, as used in the empirical section of Boswijk and
  Doornik (2004).  Another alternative is \verb+set vecm_norm first+,
  which scales $\beta$ such that the elements on the first row equal
  1.  (To return to the default: \texttt{set vecm\_norm phillips}.)
} The starting point is writing $\beta$ in partitioned form as in
\[
  \beta = \left[
    \begin{array}{c} \beta_1 \\ \beta_2  \end{array}
    \right] ,
\]
where $\beta_1$ is an $r \times r$ matrix and  $\beta_2$ is $(n-r)
\times r$. Assuming that $\beta_1$ has full rank, $\beta$ can be
post-multiplied by $\beta_1^{-1}$, giving
\[
  \hat{\beta} = \left[
    \begin{array}{c} I \\ \beta_2 \beta_1^{-1}  \end{array}
    \right] =
    \left[
    \begin{array}{c} I \\ -B \end{array}
  \right]  ,
\]

The coefficients that \app{gretl} produces are $\hat{\beta}$, with
$B$ known as the matrix of unrestricted coefficients. In
terms of the underlying equilibrium relationship, the Phillips
normalization expresses the system of $r$ equilibrium relations as
  \begin{eqnarray}
    y_{1,t} & = & b_{1,r+1} y_{r+1,t} + \ldots + b_{1,n} y_{n,t} \\
    y_{2,t} & = & b_{2,r+1} y_{r+1,t} + \ldots + b_{2,n} y_{n,t} \\
    & \vdots & \\
    y_{r,t} & = & b_{r,r+1} y_{r+1,t} + \ldots + b_{r,n} y_{r,t} 
  \end{eqnarray}
where the first $r$ variables are expressed as functions of the
remaining $n-r$.

Although the triangular representation ensures that the statistical
problem of estimating $\beta$ is solved, the resulting equilibrium
relationships may be difficult to interpret. In this case, the user
may want to achieve identification by specifying manually the system
of $r^2$ constraints that \app{gretl} will use to produce an estimate
of $\beta$.

As an example, consider the money demand system presented in section
9.6 of Verbeek (2004).  The variables used are \texttt{m} (the log of
real money stock M1), \texttt{infl} (inflation), \texttt{cpr} (the
commercial paper rate), \texttt{y} (log of real GDP) and \texttt{tbr}
(the Treasury bill rate).\footnote{This data set is available in the
  \texttt{verbeek} data package; see
  \url{http://gretl.sourceforge.net/gretl_data.html}.}

Estimation of $\beta$ can be performed via the commands
\begin{code}
  open money.gdt 
  smpl 1954:1 1994:4 
  vecm 6 2 m infl cpr y tbr --rc
\end{code}
and the relevant portion of the output reads
\begin{code}
Maximum likelihood estimates, observations 1954:1-1994:4 (T = 164)
Cointegration rank = 2
Case 2: Restricted constant

beta (vettori di cointegrazione, standard errors in parentheses)

m           1.0000       0.0000 
           (0.0000)     (0.0000) 
infl        0.0000       1.0000 
           (0.0000)     (0.0000) 
cpr        0.56108      -24.367 
          (0.10638)     (4.2113) 
y         -0.40446     -0.91166 
          (0.10277)     (4.0683) 
tbr       -0.54293       24.786 
          (0.10962)     (4.3394) 
const      -3.7483       16.751 
          (0.78082)     (30.909) 
\end{code}
Interpretation of the coefficients of the cointegrazione matrix $\beta$
would be easier if a meaning could be attached to each of its
columns. This is possible by hypothesizing the existence of two
long-run relationships: a money demand equation
\begin{equation}
  \label{eq:verbeek-mondem}
  \mbox{\tt m} = c_1 + \beta_1 \mbox{\tt infl} + \beta_2 \mbox{\tt
    y} + \beta_3 \mbox{\tt tbr}
\end{equation}
and a risk premium equation
\begin{equation}
  \label{eq:verbeek-premium}
 \mbox{\tt cpr} = c_2 + \beta_4 \mbox{\tt infl} +
   \beta_5 \mbox{\tt y} + \beta_6 \mbox{\tt tbr}
\end{equation}
which imply that the cointegrazione matrix can be normalized as
\[
  \beta = \left[
    \begin{array}{rr}
      -1 & 0 \\ \beta_1 & \beta_4 \\ 0 & -1 \\ \beta_2 & \beta_5
      \\ \beta_3 & \beta_6 \\ c_1 & c_2
    \end{array}
    \right]
\]

This renormalization can be accomplished by means of the
\texttt{restrict} command, to be given after the \texttt{vecm} command
or, in the graphical interface, by selecting the ``Test, Linear
Restrictions'' menu entry. The syntax for entering the restrictions
should be fairly obvious:\footnote{Note that in this context we are
  bending the usual matrix indexation convention, using the leading
  index to refer to the \textit{column} of $\beta$ (the particular
  vettore di cointegrazione).  This is standard practice in the literature,
  and defensible insofar as it is the columns of $\beta$ (the
  cointegrating relations or equilibrium errors) that are of primary
  interest.}
\begin{code}
restrict
  b[1,1] = -1
  b[1,3] = 0
  b[2,1] = 0
  b[2,3] = -1
end restrict
\end{code}
which produces

\begin{code}
Cointegrating vectors (standard errors in parentheses)

m          -1.0000       0.0000 
           (0.0000)     (0.0000) 
infl     -0.023026     0.041039 
        (0.0054666)   (0.027790) 
cpr         0.0000      -1.0000 
           (0.0000)     (0.0000) 
y          0.42545    -0.037414 
         (0.033718)    (0.17140) 
tbr      -0.027790       1.0172 
        (0.0045445)   (0.023102) 
const       3.3625      0.68744 
          (0.25318)     (1.2870) 
\end{code}

\section{Over-identifying restrictions}
\label{sec:johansen-overid}

Although $r^2$ restrictions must be imposed on a VECM system to
ensure identification of $\beta$, extra restrictions can be specified
in order to check via a likelihood-ratio statistic whether they hold
in the data. These extra restrictions are usually imposed to test
constraints stemming from the economic theory underlying the
equilibrium relationships.

\app{Gretl} is capable of testing general linear restrictions of the
form 
\begin{equation}
\label{eq:Rb}
R_b \vec{\beta} = q
\end{equation}
and/or
\begin{equation}
\label{eq:Ra}
R_a \vec{\alpha} = 0
\end{equation}
%
Note that the $\beta$ restriction may be non-homogeneous ($q \neq 0$)
but the $\alpha$ restriction must be homogeneous.  Nonlinear
restrictions are not supported, and neither are restrictions that
cross between $\beta$ and $\alpha$.  In the case where $r > 1$ such
restrictions may be in common across all the columns of $\beta$ (or
$\alpha$) or may be specific to certain columns of these matrices.
This is the case discussed in Boswijk (1995) and Boswijk and Doornik
(2004, section 4.4).

The restrictions (\ref{eq:Rb}) and (\ref{eq:Ra}) may be written in
explicit form as
\begin{equation}
\label{eq:vecbeta}
\vec{\beta} = H\phi + h_0
\end{equation}
and
\begin{equation}
\label{eq:vecalpha}
\vec{\alpha'} = G\psi
\end{equation}
respectively, where $\phi$ and $\psi$ are the free parameter vectors
associated with $\beta$ and $\alpha$ respectively.  We may refer
to the free parameters collectively as $\theta$ (the column vector
formed by concatenating $\phi$ and $\psi$).  \app{Gretl} uses this
representation internally when testing the restrictions.

If the list of restrictions that is passed to the \texttt{restrict}
command contains more constraints than necessary to achieve
identification, then an LR test is performed; moreover, the
\texttt{restrict} command can be given the \verb|--full| switch, in
which case full estimates for the restricted system are printed
(including the $\Gamma_i$ terms), and the system thus restricted
becomes the ``current model'' for the purposes of further tests.  Thus
you are able to carry out cumulative tests, as in Chapter 7 of
Johansen (1995).

\subsection{Syntax}
\label{sec:vecm-restr-syntax}

The full syntax for specifying the restriction is an extension of the
one exemplified in the previous section. Inside a
\texttt{restrict}\ldots\texttt{end restrict} block, valid statements
are of the form
\begin{center}
  \texttt{\emph{parameter linear combination}} = \emph{\texttt{scalar}}
\end{center}
where a parameter linear combination involves a weighted sum of
individual elements of $\beta$ or $\alpha$ (but not both in the same
combination); the scalar on the right-hand side must be 0 for
combinations involving $\alpha$, but can be any real number for
combinations involving $\beta$. Below, we give a few examples of valid
restrictions:
\begin{code}
  b[1,1] = 1.618
  b[1,4] + 2*b[2,5] = 0
  a[1,3] = 0
  a[1,1] - a[1,2] = 0
\end{code}

A special syntax is reserved for the case when a certain constraint
should be applied to all columns of $\beta$: in this case, one index is
given for each \texttt{b} term, and the square brackets are dropped.
Hence, the following syntax
\begin{code}
restrict
  b1 + b2 = 0
end restrict
\end{code}
corresponds to
\[
\beta = \left[
\begin{array}{rr}
\beta_{11} & \beta_{21} \\
-\beta_{11} & -\beta_{21} \\
\beta_{13} & \beta_{23} \\
\beta_{14} & \beta_{24}
\end{array}
\right]
\]
The same convention is used for $\alpha$: when only one index is given for
each \texttt{a} term, the restriction is presumed to apply to all $r$
rows of $\alpha$, or in other words the given variables are weakly
exogenous. For instance, the formulation
%
\begin{code}
restrict
  a3 = 0
  a4 = 0
end restrict
\end{code}
%
specifies that variables 3 and 4 do not respond to the deviation from
equilibrium in the previous period.  

Finally, a short-cut is available for setting up complex restrictions (but
currently only in relation to $\beta$): you can specify $R_b$ and $q$,
as in $R_b \vec{\beta} = q$, by giving the names of previously
defined matrices.  For example,
%
\begin{code}
matrix I4 = I(4)
matrix vR = I4**(I4~zeros(4,1))
matrix vq = mshape(I4,16,1)
restrict
  R = vR
  q = vq
end restrict
\end{code}
%
which manually imposes Phillips normalization on the $\beta$ estimates
for a system with rango di cointegrazione 4.
 
\subsection{An example}
\label{sec:vecm-overid-ex}

Brand and Cassola (2004) propose a money demand system for the Euro
area, in which they postulate three long-run equilibrium
relationships:
%
\begin{center}
\begin{tabular}{ll}
  money demand & $m = \beta_l l + \beta_y y$ \\
  Fisher equation & $\pi = \phi l$ \\
  Expectation theory of & $l = s$ \\ [-4pt]
  interest rates
\end{tabular}
\end{center}
%
where $m$ is real money demand, $l$ and $s$ are long- and short-term
interest rates, $y$ is output and $\pi$ is inflation.\footnote{A
  traditional formulation of the Fisher equation would reverse the
  roles of the variables in the second equation, but this detail is
  immaterial in the present context; moreover, the expectation theory
  of interest rates implies that the third equilibrium relationship
  should include a constant for the liquidity premium. However, since
  in this example the system is estimated with the constant term
  unrestricted, the liquidity premium gets merged in the system
  intercept and disappears from $z_t$.}  (The names for these
variables in the \app{gretl} data file are \verb|m_p|, \texttt{rl},
\texttt{rs}, \texttt{y} and \texttt{infl}, respectively.)

The rango di cointegrazione assumed by the authors is 3 and there are 5
variables, giving 15 elements in the $\beta$ matrix.  $3 \times 3 = 9$
restrictions are required for identification, and a just-identified
system would have $15 - 9 = 6$ unrestricted parameters.  However, the
three postulated long-run relationships feature only three
unrestricted parameters, so the over-identification rank is 3.

\begin{script}[htbp]
  \caption{Estimation of a money demand system with constraints on $\beta$}
  \label{brand-cassola-script}
Input:
\begin{scodebit}
open brand_cassola.gdt

# perform a few transformations
m_p = m_p*100
y = y*100
infl = infl/4
rs = rs/4
rl = rl/4

# replicate table 4, page 824
vecm 2 3 m_p infl rl rs y -q
genr ll0 = $lnl

restrict --full
  b[1,1] = 1
  b[1,2] = 0
  b[1,4] = 0
  b[2,1] = 0
  b[2,2] = 1
  b[2,4] = 0
  b[2,5] = 0
  b[3,1] = 0
  b[3,2] = 0
  b[3,3] = 1
  b[3,4] = -1
  b[3,5] = 0
end restrict
genr ll1 = $rlnl
\end{scodebit}
Partial output:
\begin{scodebit}
Unrestricted loglikelihood (lu) = 116.60268
Restricted loglikelihood (lr) = 115.86451
2 * (lu - lr) = 1.47635
P(Chi-Square(3) > 1.47635) = 0.68774

beta (vettori di cointegrazione, standard errors in parentheses)

m_p        1.0000       0.0000       0.0000 
          (0.0000)     (0.0000)     (0.0000) 
infl       0.0000       1.0000       0.0000 
          (0.0000)     (0.0000)     (0.0000) 
rl         1.6108     -0.67100       1.0000 
         (0.62752)   (0.049482)     (0.0000) 
rs         0.0000       0.0000      -1.0000 
          (0.0000)     (0.0000)     (0.0000) 
y         -1.3304       0.0000       0.0000 
        (0.030533)     (0.0000)     (0.0000) 
\end{scodebit}
%$
\end{script}

Example \ref{brand-cassola-script} replicates Table 4 on page 824 of
the Brand and Cassola article.\footnote{Modulo what appear to be a few
  typos in the article.} Note that we use the \verb|$lnl| accessor
after the \texttt{vecm} command to store the unrestricted
log-likelihood and the \verb|$rlnl| accessor after \texttt{restrict}
for its restricted counterpart. 

The example continues in script~\ref{brand-cassola-tab5}, where we
perform further testing to check whether (a) the income elasticity in
the money demand equation is 1 ($\beta_y = 1$) and (b) the Fisher
relation is homogeneous ($\phi = 1$). Since the \verb|--full| switch
was given to the initial \texttt{restrict} command, additional
restrictions can be applied without having to repeat the previous
ones.  (The second script contains a few \texttt{printf} commands,
which are not strictly necessary, to format the output nicely.)  It
turns out that both of the additional hypotheses are rejected by the
data, with p-values of $0.002$ and $0.004$.

\begin{script}[htbp]
  \caption{Further testing of money demand system}
  \label{brand-cassola-tab5}
Input:
\begin{scodebit}
restrict
  b[1,5] = -1
end restrict
genr ll_uie = $rlnl

restrict
  b[2,3] = -1
end restrict
genr ll_hfh = $rlnl

# replicate table 5, page 824
printf "Testing zero restrictions in cointegrazione space:\n"
printf "  LR-test, rank = 3: chi^2(3) = %6.4f [%6.4f]\n", 2*(ll0-ll1), \
	pvalue(X, 3, 2*(ll0-ll1))

printf "Unit income elasticity: LR-test, rank = 3:\n"
printf "  chi^2(4) = %g [%6.4f]\n", 2*(ll0-ll_uie), \
	pvalue(X, 4, 2*(ll0-ll_uie))

printf "Homogeneity in the Fisher hypothesis:\n"
printf "  LR-test, rank = 3: chi^2(4) = %6.3f [%6.4f]\n", 2*(ll0-ll_hfh), \
	pvalue(X, 4, 2*(ll0-ll_hfh))
\end{scodebit}
Output:
\begin{scodebit}
Testing zero restrictions in cointegrazione space:
  LR-test, rank = 3: chi^2(3) = 1.4763 [0.6877]
Unit income elasticity: LR-test, rank = 3:
  chi^2(4) = 17.2071 [0.0018]
Homogeneity in the Fisher hypothesis:
  LR-test, rank = 3: chi^2(4) = 15.547 [0.0037]  
\end{scodebit}
\end{script}

Another type of test that is commonly performed is the ``weak
exogeneity'' test. In this context, a variable is said to be weakly
exogenous if all coefficients on the corresponding row in the $\alpha$
matrix are zero. If this is the case, that variable does not adjust to
deviations from any of the long-run equilibria and can be considered
an autonomous driving force of the whole system.

The code in Example~\ref{brand-cassola-exog} performs this test for
each variable in turn, thus replicating the first column of Table 6 on
page 825 of Brand and Cassola (2004).  The results show that weak
exogeneity might perhaps be accepted for the long-term interest rate
and real GDP (p-values $0.07$ and $0.08$ respectively).

\begin{script}[htbp]
  \caption{Testing for weak exogeneity}
  \label{brand-cassola-exog}
Input:
\begin{scodebit}
restrict
  a1 = 0
end restrict
ts_m = 2*(ll0 - $rlnl)

restrict
  a2 = 0
end restrict
ts_p = 2*(ll0 - $rlnl)

restrict
  a3 = 0
end restrict
ts_l = 2*(ll0 - $rlnl)

restrict
  a4 = 0
end restrict
ts_s = 2*(ll0 - $rlnl)

restrict
  a5 = 0
end restrict
ts_y = 2*(ll0 - $rlnl)

loop foreach i m p l s y --quiet
  printf "\Delta $i\t%6.3f [%6.4f]\n", ts_$i, pvalue(X, 6, ts_$i)
end loop
\end{scodebit}
Output (variable, LR test, p-value):
\begin{scodebit}
\Delta m	18.111 [0.0060]
\Delta p	21.067 [0.0018]
\Delta l	11.819 [0.0661]
\Delta s	16.000 [0.0138]
\Delta y	11.335 [0.0786]
\end{scodebit}
%$
\end{script}

\subsection{Numerical solution methods}
\label{sec:vecm-opt}

In general, the ML estimator for the restricted VECM problem has no
closed form solution, hence the maximum must be found via numerical
methods.\footnote{The exception is restrictions that are homogeneous,
  common to all $\beta$ or all $\alpha$ (in case $r>1$), and involve
  either $\beta$ only or $\alpha$ only.  Such restrictions are handled
  via the modified eigenvalues method set out by Johansen (1995).  We
  solve directly for the ML estimator, without any need for iterative
  methods.}  In some cases convergence may be difficult, and
\app{gretl} provides several choices to solve the problem.

Two maximization methods are available in \app{gretl}. The default is
the switching algorithm set out in Boswijk and Doornik (2004).  The
alternative is a limited-memory variant of the BFGS algorithm (LBFGS),
using analytical derivatives.  This is invoked using the
\verb+--lbfgs+ flag with the \texttt{restrict} command.

The switching algorithm works by explicitly maximizing the likelihood
at each iteration, with respect to $\hat{\phi}$, $\hat{\psi}$ and
$\hat{\Omega}$ (the covariance matrix of the residuals) in turn.  This
method shares a feature with the basic Johansen eigenvalues procedure,
namely, it can handle a set of restrictions that does not fully
identify the parameters.

LBFGS, on the other hand, requires that the model be fully identified.
When using LBFGS, therefore, you may have to supplement the
restrictions of interest with normalizations that serve to identify
the parameters.  For example, one might use all or part of the
Phillips normalization (see section \ref{sec:johansen-ident}).

Neither the switching algorithm nor LBFGS is guaranteed to find the
global ML solution.\footnote{In developing \app{gretl}'s VECM-testing
  facilities we have considered a fair number of ``tricky cases'' from
  various sources. We'd like to thank Luca Fanelli of the University
  of Bologna and Sven Schreiber of Johann Wolfgang Goethe-Universit\"at,
  for their help in devising torture-tests for \app{gretl}'s VECM
  code.} The optimizer may end up at a local maximum (or, in the case
of the switching algorithm, at a saddle point).

The solution (or lack thereof) may be sensitive to the initial value
selected for $\theta$.  By default, \app{gretl} selects a starting
point using a deterministic method based on Boswijk (1995), but two
further options are available: the initialization may be adjusted
using simulated annealing (with the \verb+--jitter+ flag), or the user
may supply an explicit initial value for $\theta$.

The default initialization method is:
%
\begin{enumerate}
\item Calculate the unrestricted ML $\hat{\beta}$ using the
  Johansen procedure.
\item If the restriction on $\beta$ is non-homogeneous, use the
  method proposed by Boswijk (1995):
\begin{equation}
\phi_0 = -[(I_r \otimes \hat{\beta}_{\perp})'H]^+ 
  (I_r \otimes \hat{\beta}_{\perp})' h_0
\end{equation}
where $\hat{\beta}'_{\perp} \hat{\beta} = 0$ and $A^+$ denotes
the Moore--Penrose inverse of $A$.  Otherwise
\begin{equation}
\phi_0 = (H'H)^{-1} H' \vec{\hat{\beta}}
\end{equation}
\item $\vec{\beta_0} = H\phi_0 + h_0$.
\item Calculate the unrestricted ML $\hat{\alpha}$ conditional on
  $\beta_0$, as per Johansen:
\begin{equation}
\label{eq:Jalpha}
\hat{\alpha} = S_{01} \beta_0 (\beta'_0S_{11}\beta_0)^{-1}
\end{equation}
\item If $\alpha$ is restricted by $\vec{\alpha'} = G\psi$, then
  $\psi_0 = (G'G)^{-1}G'\,{\rm vec}(\hat{\alpha}')$ and
  $\vec{\alpha'_0} = G\psi_0$.
\end{enumerate}

However, as mentioned above, \app{gretl} offers the option of adjusting the
initialization using simulated annealing.  

The basic idea is this: we start at a certain point in the parameter
space, and for each of $n$ iterations (currently $n=4096$) we randomly
select a new point within a certain radius of the
previous one, and determine the likelihood at the new point.  If the
likelihood is higher, we jump to the new point; otherwise, we jump
with probability $P$ (and remain at the previous point with
probability $1-P$).  As the iterations proceed, the system gradually
``cools''---that is, the radius of the random perturbation is
reduced, as is the probability of making a jump when the likelihood
fails to increase.

In the course of this procedure many points in the parameter space are
evaluated, starting with the point arrived at by the deterministic
method, which we'll call $\theta_0$.  One of these points will be
``best'' in the sense of yielding the highest likelihood: call it
$\theta^*$.  This point may or may not have a greater likelihood than
$\theta_0$.  And the procedure has an end point, $\theta_n$, which may
or may not be ``best''.

The rule followed by \app{gretl} in selecting an initial value for $\theta$
based on simulated annealing is this: use $\theta^*$ if $\theta^* >
\theta_0$, otherwise use $\theta_n$.  That is, if we get an
improvement in the likelihood via annealing, we make full use of this;
on the other hand, if we fail to get an improvement we nonetheless
allow the annealing to randomize the starting point.  Experiments
indicated that the latter effect can be helpful.

If the default deterministic initialization fails, the alternative to
relying on randomization is for the user to specify initial values.
This is done by passing a predefined vector to the \texttt{set}
command with parameter \texttt{initvals}, as in
%
\begin{verbatim}
set initvals myvec
\end{verbatim}

The details depend on whether the switching algorithm or LBFGS is
used.  For the switching algorithm, there are two options for
specifying the initial values.  The more user-friendly one (for most
people, we suppose) is to specify a matrix that contains $\vec{\beta}$
followed by $\vec{\alpha}$. For example:
\begin{code}
open denmark.gdt
vecm 2 1 LRM LRY IBO IDE --rc --seasonals

matrix BA = {1, -1, 6, -6, -6, -0.2, 0.1, 0.02, 0.03}
set initvals BA
restrict
  b[1] = 1
  b[1] + b[2] = 0
  b[3] + b[4] = 0
end restrict
\end{code}

The other option, which is compulsory when using LBFGS, is to specify
the initial values in terms of the unrestricted parameters, $\phi$ and
$\psi$.  Getting this right is somewhat less obvious.  As mentioned
above, the implicit-form restriction $R\vec{\beta} = q$ has explicit
form $\vec{\beta} = H\phi + h_0$, where $H = R_{\perp}$, the right
nullspace of $R$.  The vector $\phi$ is shorter, by the number of
restrictions, than $\vec{\beta}$.  The savvy user will then see what
needs to be done.  The other point to take into account is that if
$\alpha$ is unrestricted, the \textit{effective} length of $\psi$ is
0, since it is then optimal to compute $\alpha$ using Johansen's
formula, conditional on $\beta$ (equation \ref{eq:Jalpha} above). An
example follows:
\begin{code}
open denmark.gdt
vecm 2 1 LRM LRY IBO IDE --rc --seasonals

matrix phi = {-8, -6}
set initvals phi
restrict --lbfgs
  b[1] = 1
  b[1] + b[2] = 0
  b[3] + b[4] = 0
end restrict
\end{code}

%%% Local Variables: 
%%% mode: latex
%%% TeX-master: "gretl-guide"
%%% End: 
