\chapter{Методы оценивания}
\label{chap:estimation}

Конечно, вы можете оценивать эконометрические модели через Hansl, не
имея при себе набора данных (в смысле, в котором мы здесь используем
этот термин) --- так же, как вы могли бы это сделать, например, в
\textsf{Matlab}. Вам понадобятся данные, но их можно загрузить в
матричной форме (см. функцию mread в Справочнике по командам Gretl)
или сгенерировать искусственно с помощью таких функций, как mnormal
или muniform. Вы можете создать свой собственный оценщик данных,
используя простые вещи из линейной алгебры в Hansl, а также у вас есть
доступ к более специализированным функциям, таким как mols (см. раздел
\ref{sec:mat-op}) и mrls (метод наименьших квадратов с ограничениями),
если они вам нужны.

Однако, если вам не нужно использовать метод оценки, который в
настоящее время не поддерживается Gretl, или есть сильное желание
изобретать велосипед, вы, вероятно, захотите воспользоваться
встроенными командами оценки, которые доступны в Hansl. Эти команды
ориентированы на серию и поэтому требуют набора данных. Они делятся на
две основные категории: стандартные процедуры и общие инструменты,
которые можно использовать для оценки широкого разнообразия моделей,
основанных на общих принципах.

\section{Стандартные процедуры оценки}
\label{sec:canned}

Слово «canned» обозначает «законсервированный», поэтому, возможно,
звучит не очень аппетитно, но на самом деле это широко используемый
термин. Под стандартными процедурами обычно подразумевают две вещи,
каждая из которых на самом деле очень привлекательна.
\begin{itemize}
\item Пользователю предоставляется довольно простой
  интерфейс. Необходимо указать несколько входов и, возможно, выбрать
  несколько опций, затем основная работа выполняется в библиотеке
  Gretl. Выводятся подробные результаты (оценки параметров плюс
  многочисленные вспомогательные статистические данные).
\item TАлгоритм написан на языке C опытными
  кодировщиками. Следовательно, он работает быстрее (возможно, намного
  быстрее) чем его реализация на интерпретируемом языке, таком как
  Hansl.
\end{itemize}
Большинство таких процедур имеют синтаксис 
\begin{flushleft}
\quad \textsl{commandname parameters options}
\end{flushleft}
где параметры обычно идут в форме списка серий: т.е. зависимая
переменная, за которой следуют регрессоры.

Последовательность процедур можно грубо разделить на следующие
категории:

\begin{center}
\begin{tabular}{l>{\raggedright\arraybackslash}p{.6\textwidth}}
Линейн., одно уравнение: & \cmd{ols}, \cmd{tsls}, \cmd{ar1},
\cmd{mpols} \\
Линейн., неск. уравнений: & \cmd{system}, \cmd{var}, \cmd{vecm} \\
Нелинейн., одно уравнение: & 
 \cmd{logit}, \cmd{probit}, \cmd{poisson}, \cmd{negbin}, \cmd{tobit},
 \cmd{intreg}, \cmd{logistic}, \cmd{duration} \\
Панель: & \cmd{panel}, \cmd{dpanel} \\
Разнообразн.: & \cmd{arima}, \cmd{garch}, \cmd{heckit},
  \cmd{quantreg}, \cmd{lad}, \cmd{biprobit}
\end{tabular}
\end{center}
Не позволяйте названиям вводить вас в заблуждение: например, команда
\cmd{probit} может оценивать упорядоченные модели, пробит модели
панели случайных эффектов, и т.д. «Домашний стиль» Hansl заключается в
том, чтобы придерживаться относительно небольшого количества командных
слов и различать методы внутри класса методов оценивания, таких как
Пробит, с помощью различных опций или характера предоставленных
данных.  Одновременные системы (SUR, FIML и т. д.) составляют главное
исключение из синтаксической сводки; для них требуется системный блок
- см. главу о многомерных моделях в руководстве пользователя Gretl.

\section{Общие инструменты оценки}
\label{sec:est-blocks}

Hansl предлагает три основных набора инструментов для определения
средств оценки помимо стандартного выбора. Ниже прилагаем краткий
обзор:

\begin{center}
\begin{tabular}{lll}
Команда & Метод оценки & Руководство пользователя \\[4pt]
\cmd{nls} & нелинейный метод наименьших квадратов  & глава 20 \\
\cmd{mle} & оценка максимального правдоподобия  & глава 21\\
\cmd{gmm} & обобщенный метод моментов  & глава 22
\end{tabular}
\end{center}

Каждая из этих команд принимает форму блока операторов (например,
\texttt{nls} \dots{} \texttt{end nls}). Пользователь должен
предоставить функцию для вычисления подобранной зависимой переменной
(nls), логарифма правдоподобия (mle) или GMM остатки (gmm). С
\texttt{nls} и \texttt{mle} аналитические производные рассматриваемой
функции относительно ее параметров могут быть предоставлены по
желанию.

Вероятно, наиболее широко используемым из этих инструментов является
\texttt{mle}. Hansl предлагает несколько стандартных методов
оценивания машинного обучения, но если вы натолкнулись на модель,
которую хотите оценить с помощью метода максимального правдоподобия, и
она не поддерживается, то все, что вам нужно сделать, это записать
логарифмическую вероятность в Hansl и пропустить ее через \texttt{mle}
аппарат.

\section{Аксессоры постоценки}
\label{sec:postest-accessors}

Все упомянутые выше методы являются командами, а не функциями; поэтому
они не возвращают значения. Однако после оценки модели --- с помощью
стандартной процедуры или одного из наборов инструментов --- с помощью
аксессоров вы можете получить большинство чисел, с которыми вы,
возможно, захотите работать при дальнейшем анализе.

Некоторые из таких средств доступа являются общими и доступны после
использования практически любого метода оценивания. Примеры включают
\dollar{coeff} и \dollar{stderr} (для получения векторов коэффициентов
и стандартных ошибок соответственно), \dollar{uhat} и \dollar{yhat}
(остатки и подобранные значения) и \dollar{vcv} (ковариационная
матрица коэффициентов). Некоторые же, напротив, специфичны для
определенных методов оценивания. Примеры здесь включают \dollar{jbeta}
(матрицу коинтеграции, следующую за оценкой VECM), \dollar{h}
(оцененный ряд условной дисперсии после оценки GARCH) и
\dollar{mnlprobs} (матрицу вероятностей исходов после полиномиальной
логит-оценки).

Полный список и описание аксессоров можно найти в Справочнике по
командам Gretl.

\section{Форматирование результатов оценки}
\label{sec:model-format}

Команды, упомянутые в этой главе, по умолчанию выводят довольно
подробный (и, надеемся, хорошо отформатированный) результат. Однако в
некоторых случаях вы можете использовать встроенные команды в качестве
вспомогательных шагов в реализации метода оценивания, который сам по
себе не является встроенным. В этом контексте стандартный вывод данных
может быть неуместным, и вы можете сами взять на себя инициативу по
представлению результатов.

Это можно сделать довольно легко. Во-первых, вы можете «подавить»
обычный вывод, используя опцию \option{quiet} со встроенными командами
оценки.\footnote{Для некоторых команд --\option{quiet} уменьшает, но
  не отменяет обычный вывод gretl. В этих случаях вы можете
  использовать \option{silent}. Обратитесь к Справочнику по командам
  Gretl, чтобы определить, какие команды разрешают эту опцию.}
Во-вторых, вы можете использовать команду \cmd{modprint} для создания
желаемого результата. Как обычно, подробности см. в справочнике по
командам Gretl.

\section{Именованные модели}

Выше мы говорили, что команды оценки в Hansl ничего не возвращают. Это
должно быть оценено по достоиснтву в следующем случае: можно
использовать специальный синтаксис, чтобы поместить модель в группу
именованных моделей. Для этой цели используется форма ``\verb|<-|'', а
не обычный символ присваивания ``\verb|=|''.  В основном, она
предназначена для использования в графическом интерфейсе gretl, но ее
также можно использовать в сценариях Hansl.

Как только модель сохранена таким образом, упомянутые выше аксессоры
могут использоваться особым образом, они соединяются точкой перед
названием целевой модели. Вот небольшой пример. (Обратите внимание,
что \dollar{ess} получает доступ к сумме ошибок квадратов или сумме
квадратов остатков в моделях, оцениваемых методом наименьших
квадратов.)

\begin{code}
diff y x
ADL <- ols y const y(-1) x(0 to -1)
ECM <- ols d_y const d_x y(-1) x(-1)
# the following two values should be equal
ssr_a = ADL.$ess
ssr_e = ECM.$ess
\end{code}

%% \label{LastPage}

%%% Local Variables: 
%%% mode: latex
%%% TeX-master: "hansl-primer"
%%% End: 
