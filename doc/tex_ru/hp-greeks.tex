\chapter{Использование букв греческого алфавита}
\label{chap:greeks}

Как упоминалось в главе ~\ref{chap:hello}, существует требование о
том, что идентификаторы Hansl должны быть в формате
ASCII. Единственное исключение из данного правила состоит в том, что
они могут принимать форму одной греческой буквы. Соответствующие
условия следующие:

\begin{itemize}
\item TЭто исключение применяется только к именам всех переменных
  кроме серий; названия серий всегда должны быть в формате ASCII.
\item Поддерживаемые греческие символы --- это 24 буквы (без ударения)
  основного греческого алфавита, за исключением omicron, который
  неотличим от латинского ‘o’.
\item Эти буквы могут использоваться в нижнем или верхнем регистре и
  составлять различные идентификаторы, как обычно в Hansl, за
  исключением нескольких заглавных букв, которые неразличимы в
  латинском и греческом алфавитах (`A', `B', `E', `K', `M', `N',
  \dots).
\item Греческие буквы должны быть закодированы в UTF-8.
\end{itemize}

В графическом интерфейсе Gretl (редактор сценариев и консоль) можно
ввести допустимые греческие буквы, если набрать латинскую букву при
нажатой клавише \texttt{Alt}. Ниже указаны сопоставления двух языков:
сначала нижний, затем верхний регистр.

\begin{center}
  \begin{tabular}{ccl@{\hskip 4em}ccl}
    Лат & Греч & & Лат & Греч \\[4pt]
    a & $\alpha$ & альфа & n & $\nu$ & ню \\
    b & $\beta$ & бета & p & $\pi$ & пи \\
    c & $\chi$ & хи & q & $\theta$ & тета \\
    d & $\delta$ & дельта & r & $\rho$ & ро \\
    e & $\epsilon$ & эпсилон & s & $\sigma$ & сигма \\
    f & $\phi$ & фи & t & $\tau$ & тау \\
    g & $\gamma$ & гамма & u & $\upsilon$ & ипсилон \\
    h & $\eta$ & эта & v & $\nu$ & ню \\
    i & $\iota$ & йота & w & $\omega$ & омега \\
    j & $\psi$ & пси & x & $\xi$ & хи \\
    k & $\kappa$ & каппа & y & $\upsilon$ & ипсилон \\
    l & $\lambda$ & лямбда & z & $\zeta$ & зета \\
    m & $\mu$ & мю \\
  \end{tabular}
\end{center}

\begin{center}
  \begin{tabular}{ccl}
    Latin & Greek \\[4pt]
    D & $\Delta$ & Дельта \\
    F & $\Phi$ & Фи \\
    G & $\Gamma$ & Гамма \\
    J & $\Psi$ & Пси \\
    L & $\Lambda$ & Лямбда \\
    P & $\Pi$ & Пи \\
    Q & $\Theta$ & Тета \\
    S & $\Sigma$ & Сигма \\
    U & $\Upsilon$ & Ипсилон \\
    W & $\Omega$ & Омега \\
    X & $\Xi$  & Хи \\
    Y & $\Upsilon$ & Ипсилон \\
  \end{tabular}
\end{center}

Небольшой совет: вероятно, вам будет неудобно использовать
идентификаторы из греческих букв в сценариях Hansl, которыми вы
намерены делиться с другими пользователями через Интернет, поскольку
нельзя быть уверенным в том, что кодировки текста сохранятся без
изменений. Это предупреждение может быть особенно полезным, если вы
или любой из предполагаемых получателей ваших скриптов работает на MS
Windows, поскольку Windows изначально не поддерживает UTF-8,
обязательную кодировку таких идентификаторов для использования с
Gretl.
