\documentclass{article}
\usepackage{array,url,gretl}
\usepackage[body={6.0in,8.5in},top=1in,left=1.25in,nohead]{geometry}

\begin{document}

\begin{center}
  {\large \textbf{Assessing gretl's accuracy: the NIST datasets}}
\end{center}

The U.S. National Institute of Standards and Technology (NIST)
publishes a set of statistical reference datasets.  The object
of this project is to ``improve the accuracy of statistical software
by providing reference datasets with certified computational results
that enable the objective evaluation of statistical software''.

As of May 2000 the website for the project can be found at:

\url{http://www.nist.gov/itl/div898/strd/general/main.html}

\noindent while the datasets are at

\url{http://www.nist.gov/itl/div898/strd/general/dataarchive.html}

For testing \textsc{gretl} I have made use of the datasets pertaining
to Linear Regression and Univariate Summary Statistics (the others
deal with ANOVA and nonlinear regression).

I quote from the NIST text ``Certification Method \& Definitions''
regarding their certified computational results (emphasis added):

\begin{quote}
  For all datasets, multiple precision calculations (accurate to 500
  digits) were made using the preprocessor and FORTRAN subroutine
  package of Bailey (1995, available from NETLIB). Data were read in
  exactly as multiple precision numbers and all calculations were made
  with this very high precision. The results were output in multiple
  precision, and only then rounded to fifteen significant digits.
  \textit{These multiple precision results are an idealization. They
    represent what would be achieved if calculations were made without
    roundoff or other errors.} Any typical numerical algorithm (i.e.\ 
  not implemented in multiple precision) will introduce computational
  inaccuracies, and will produce results which differ slightly from
  these certified values.
\end{quote}

It is not to be expected that results obtained from ordinary
statistical packages will agree exactly with NIST's multiple precision
benchmark figures.  But the benchmark provides a very useful test for
egregious errors and imprecision.  

In Table ~\ref{tab:linreg} below, ``OK'' means that \textsc{gretl}'s
output agrees---to the precision given by the program, which is less
than the 15 significant digits given by NIST---with the certified
values for all the NIST statistics, which include regression
coefficients and standard errors, sum of squared residuals or error
sum of squares (ESS), standard error of residuals, $F$ statistic and
$R^2$.

\begin{table}[htbp]
\begin{center}
{\setlength{\extrarowheight}{6pt}
  \begin{tabular}{lp{2in}p{3in}}

\textit{Dataset} & \textit{Model} & \textsc{gretl} \textit{performance} \\
 Norris     &  Simple linear regression &  OK \\
 Pontius    &  Quadratic &  OK \\
 NoInt1     &  Simple regression, no intercept &  OK (but see text)  \\
 NoInt2     &  Simple regression, no intercept & OK (but see text) \\
 Filip      &  10th degree polynomial &  Complains of 
               excessive multicollinearity, no estimates produced \\
 Longley    &  Multiple regression, six independent variables &  OK \\
 Wampler1   &  5th degree polynomial &  OK \\
 Wampler2   &  5th degree polynomial &  OK \\
 Wampler3   &  5th degree polynomial &  OK \\
 Wampler4   &  5th degree polynomial &  OK \\
 Wampler5   &  5th degree polynomial &  OK \\
  \end{tabular}}
\caption{NIST linear regression tests}
\label{tab:linreg}
\end{center}
\end{table}

As can be seen from the table, \textsc{gretl} does a good job of
tracking the certified results. (Total run time for the tests was
0.195 seconds on a 333MHz i686 machine running GNU/Linux.)  With the
\texttt{Filip} data set, where the model is
$$y_t = \beta_0 + \beta_1 x_t + \beta_2 x^2_t + \beta_3 x^3_t + \cdots
+ \beta_{10} x^{10}_t + \epsilon$$
\textsc{gretl} refuses to produce
estimates due to a high degree of multicollinearity (the popular
commercial econometrics program \textit{Eviews 3.1} also baulks at
this regression).  Other than that, the program produces accurate
coefficient estimates in all cases.

In the \texttt{NoInt1} and \texttt{NoInt2} datasets there is a
methodological disagreement over the calculation of the coefficient of
determination, $R^2$, where the regression does not have an intercept.
\textsc{gretl} reports the square of the correlation coefficient
between the fitted and actual values of the dependent variable in this
case, while the NIST figure is
$$R^2 = 1 - \frac{\mathrm{ESS}}{\sum y^2}$$
There is no universal
agreement among statisticians on the ``correct'' formula (see for
instance the discussion in Ramanathan, \textit{Introductory
  Econometrics}, 4th edition, Dryden, 1998, pp.\ 163--4).
\textit{Eviews 3.1} produces a different figure again (which has a
negative value for the \texttt{NoInt} test files).  The figure chosen
by NIST was obtained for these regressions using the command

\begin{center}
\texttt{genr r2alt = 1 - (\$ess/sum(y * y))}  
\end{center}

\noindent and the numbers thus
obtained were in agreement with the certified values, up to
\textsc{gretl}'s precision.

As for the univariate summary statistics, the certified values given
by NIST are for the sample mean, sample standard deviation and sample
lag-1 autocorrelation coefficient.  NIST note that the latter
statistic ``may have several definitions''.  The certified value is
computed as 
$$r_1 = \frac{\sum^T_2 (y_t - \bar{y})(y_{t-1} - \bar{y})}%
{\sum^T_1 (y_t - \bar{y})^2}$$
while \textsc{gretl} gives the
correlation coefficient between $y_t$ and $y_{t-1}$.  For the purposes
of comparison, the NIST figure was computed within \textsc{gretl} as
follows:

\begin{verbatim}
  genr y1 = y(-1)
  genr ybar = mean(y)
  genr devy = y - ybar
  genr devy1 = y1 - ybar
  genr ssy = sum(devy * devy)
  smpl 2 ;
  genr ssyy1 = sum(devy * devy1)
  genr rnist = ssyy1 / ssy
\end{verbatim}

\noindent The figure \texttt{rnist} was then compared with the certified value.

With this modification, all the summary statistics were in agreement
(to the precision given by \textsc{gretl}) for all datasets
(\texttt{PiDigits}, \texttt{ Lottery}, \texttt{Lew}, \texttt{Mavro},
\texttt{Michelso}, \texttt{NumAcc1}, \texttt{NumAcc2},
\texttt{NumAcc3} and \texttt{NumAcc4}).


\end{document}
