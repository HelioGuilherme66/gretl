\chapter{Opciones, argumentos y búsqueda de rutas}
\label{optarg}

\section{Llamada a gretl}
\label{optarg1}

\cmd{gretl} (en MS Windows, \cmd{gretl.exe})\footnote{En Linux,
  se instala un guion de ``envoltura'' denominado \texttt{gretl}. Este
  guion comprueba si la variable de entorno \texttt{DISPLAY} está determinada;
  en ese caso, lanza el programa GUI, \verb@gretl_x11@, y si no lo está,
  lanza el programa de líneas de instrucciones, \texttt{gretlcli}}.

--- Abre el programa y espera por las entradas del usuario.
      
\cmd{gretl} \textsl{archivodatos}
      
--- Inicia el programa con el archivo de datos indicado en su espacio de trabajo.
El archivo de datos puede estar en cualquiera de entre varios formatos (consulta
la \emph{Guía de Usuario de GRETL}); el programa va a intentar detectar el
formato del archivo y tratarlo adecuadamente. Consulta también más abajo la
Sección \ref{path-search} para ver el comportamiento en la búsqueda de rutas.
      
\cmd{gretl --help} (o \cmd{gretl -h})
      
--- Presenta un breve resumen de uso y finaliza el programa.
      
\cmd{gretl --version} (o \cmd{gretl -v})
      
--- Presenta la identificación de la versión del programa y finaliza el programa.
      
\cmd{gretl --english} (o \cmd{gretl -e})
      
--- Fuerza la utilización del idioma inglés en lugar de la traducción.
      
\cmd{gretl --run} \textsl{archivoguion} (o \cmd{gretl -r}
\textsl{archivoguion})
      
--- Inicia el programa y abre una ventana que muestra el archivo de guion
indicado, preparado para ejecutarse. Consulta más abajo la Sección
\ref{path-search} para ver el comportamiento en la búsqueda de rutas.
      
\cmd{gretl --db} \textsl{basedatos} (o \cmd{gretl -d}
\textsl{basedatos})

--- Inicia el programa y abre una ventana que muestra la base de datos indicada.
Si los archivos de base de datos (el archivo \texttt{.bin} y su archivo
adjunto \texttt{.idx}) no están en el directorio de base de datos predeterminado
del sistema, debes indicar la ruta completa. Consulta también la
\emph{Guía de Usuario de GRETL} para obtener detalles sobre las bases de datos.
      
\cmd{gretl --dump} (o \cmd{gretl -c})
      
--- Vierte la información de configuración del programa en un archivo de texto plano
(el nombre del archivo se presenta en una salida de resultados normal). Puede ser de
utilidad para solucionar problemas.
      
\cmd{gretl --debug} (o \cmd{gretl -g})

--- (Solo en MS Windows) Abre una ventana de consola para presentar cualquier
mensaje enviado a los flujos de ``salida de resultados típica'' o de
``fallo típico''. Esos mensajes habitualmente no son visibles en Windows;
y esto puede ser de utilidad para solucionar problemas.
      
\section{Diálogo de preferencias}
\label{guiprefs}

Se pueden configurar varias cosas en \app{GRETL}, en el menú de
``Herramientas, Preferencias''. Los elementos del menú por separado se dedican a la
elección de la fuente monoespaciada que se usa en la pantalla de resultados de \app{GRETL},
y, en algunas plataformas, la fuente utilizada en los menús y otros mensajes.
Las otras opciones se organizan en cinco etiquetas, del siguiente modo.
      
\textbf{General}: Aquí puedes configurar el directorio de base para los archivos
compartidos de \app{GRETL}. Además, hay varios cuadrados de verificación.
Si la configuración de tu idioma nativo no es la del inglés y el carácter de la
parte decimal no es el punto (``\texttt{.}''), no marcar ``Usar las preferencias
locales para el punto decimal'' va a hacer que \app{GRETL} utilice el punto de
todas formas. Marcar ``Permitir instrucciones de shell'' hace posible
invocar instrucciones del intérprete en los guiones y en la consola de \app{GRETL}
(esta facilidad no está dispoñible por defecto debido a razones de seguridad).
      
\textbf{Programas}: Puedes especificar los nombres o rutas de varios programas
de terceros que es posible invocar con \app{GRETL} bajo determinadas
condiciones.

\textbf{Editor}: Establece las preferencias referidas al editor de guiones de
\app{GRETL}.

\textbf{Red}: Establece el servidor en el que buscar bases de datos de
\app{GRETL}; y también si usar o no usar un proxy HTTP.
      
\textbf{HCCME}: Establece las preferencias relativas a la estimación de la matriz
robusta de covarianzas. Consulta la \emph{Guía de Usuario de GRETL} para más detalles.
      
\textbf{MPI}: Esto solo se muestra si GRETL está creado con apoyo para MPI
(Interfaz de Paso de Mensajes).
      
Las configuraciones elegidas mediante el diálogo de Preferencias se guardan
de una sesión de GRETL a la siguiente.
      
\section{Llamada a gretlcli}
\label{optarg2}

\cmd{gretlcli}

--- Abre el programa y espera por las entradas del usuario.
      
\cmd{gretlcli} \textsl{archivodatos}

--- Inicia el programa con el archivo de datos indicado en su espacio de trabajo.
El archivo de datos puede ser cualquiera de entre varios admitidos por
\app{GRETL} (consulta la \emph{Guía de Usuario de GRETL} para más detalles).
El programa intentará detectar el formato del archivo y tratarlo adecuadamente.
Consulta también la Sección \ref{path-search} para ver el comportamiento en la búsqueda de rutas.

\cmd{gretlcli --help} (o \cmd{gretlcli -h})

--- Presenta un breve resumen de uso.

\cmd{gretlcli --version} (o \cmd{gretlcli -v})

--- Presenta la identificación de la versión del programa.

\cmd{gretlcli --english} (o \cmd{gretlcli -e})

--- Fuerza la utilización del idioma inglés en lugar de la traducción.

\cmd{gretlcli --run} \textsl{archivoguion} (o \cmd{gretlcli -r}
\textsl{archivoguion}

--- Ejecuta las instrucciones de \textsl{archivoguion} y después pasa el control
de las entradas del usuario a la línea de instrucciones. Consulta la Sección
\ref{path-search} para ver el comportamiento en la búsqueda de rutas.

\cmd{gretlcli --batch} \textsl{archivoguion} (o \cmd{gretlcli -b}
\textsl{archivoguion})

--- Ejecuta las instrucciones de \textsl{archivoguion} y luego sale del programa.
Cuando utilices esta opción, probablemente vas a querer redirigir la salida de
resultados a un archivo. Consulta la Sección \ref{path-search} para ver el
comportamiento en la búsqueda de rutas.

Al usar las opciones \cmd{--run} y \cmd{--batch}, el archivo de guion en cuestión
debe invocar un archivo de datos para abrirlo. Esto puede hacerse utilizando
la instrucción \cmd{open} dentro del propio guion.
      
\section{Búsqueda de rutas}
\label{path-search}

Cuando se indica el nombre de un archivo de datos o de guion a \app{GRETL} o
a \app{gretlcli} en la línea de instrucciones, se busca el archivo del siguiente
modo:

\begin{enumerate}
\item ``Tal cual''.  Es decir, en el directorio de trabajo vigente o, cuando
  se indica la ruta completa, en la localización especificada.
\item En el directorio de GRETL del usuario (ver la Tabla \ref{tab-path} para los
  valores predeterminados; ten en cuenta que \texttt{PERSONAL} es un marcador
  de posición que Windows expande de un modo específico para el usuario y el
  idioma, típicamente implicando ``My Documents'' en los sistemas con idioma inglés).
\item En cualquier subdirectorio inmediato al del usuario en GRETL.
\item En el caso de un archivo de datos, la búsqueda continúa con el directorio principal
  de datos de \app{GRETL}. En el caso de un archivo de guion, la búsqueda continúa
  con el directorio de guiones del sistema. Consulta la Tabla \ref{tab-path} para ver
  las configuraciones por defecto. (\texttt{PREFIX} denota el directorio de base
  elegido en el momento en el que se instala \app{GRETL}.)
\item En el caso de archivos de datos, la búsqueda continúa entonces con todos los
  subdirectorios inmediatos al directorio principal de datos.
\end{enumerate}

\begin{table}[htbp]
  \caption{Configuraciones de las rutas predeterminadas}
  \label{tab-path}
  \begin{center}
    \begin{tabular}{lll}
       & \textit{Linux} & \textit{MS Windows} \\ [4pt]
      Directorio de usuario & \texttt{\$HOME/gretl} & 
        \verb@PERSONAL\gretl@ \\
      Directorio de datos del sistema & \texttt{PREFIX/share/gretl/data} & 
        \verb@PREFIX\gretl\data@ \\
      Directorio de guiones del sistema & \texttt{PREFIX/share/gretl/scripts} & 
        \verb@PREFIX\gretl\scripts@ \\
    \end{tabular}
  \end{center}
\end{table}

Por lo tanto no es necesario indicar la ruta completa para un archivo de datos
o de guion, excepto que quieras anular el mecanismo automático de búsqueda.
(Esto también se aplica con \app{gretlcli}, cuando indicas un nombre de archivo
como argumento para las instrucciones \cmd{open} o \cmd{run}.)

Cuando un guion de instrucciones contiene una consigna para abrir un archivo
de datos, el orden en el que se busca ese archivo es como el establecido más
arriba; excepto que también se busque en el directorio que contiene el guion, ya que
sería inmediatamente después de intentar encontrar el archivo de datos ``tal cual''.
      

\subsection{MS Windows}
\label{MS-behave}

En MS Windows, la información de la configuración de \app{GRETL} y \app{gretlcli}
se guarda en el registro de Windows. Se crea un conjunto apropiado de entradas
en el registro cuando se instala \app{GRETL} por primera vez, y las configuraciones
pueden cambiarse en el menú ``Herramientas, Preferencias'' de \app{GRETL}'. En el caso
de que alguien necesite hacer ajustes manuales en esta información, pueden encontrarse
las entradas (utilizando el programa típico de Windows \app{regedit.exe}) dentro
de \verb@Software\gretl@ en \verb@HKEY_LOCAL_MACHINE@ (el directorio principal
de \app{GRETL} y las rutas hasta diversos programas auxiliares) y en
\verb@HKEY_CURRENT_USER@ (todas las otras variables que se pueden configurar).
