\chapter{Opciones, argumentos y directorios}
\label{c-optarg}

\section{gretl}
\label{optarg1}

\cmd{\app{gretl}} (en MS Windows, \cmd{gretlw32.exe})\footnote{En
  Linux, se instala un ``lote de instrucciones de envoltura
  (wrapper)'' llamado \texttt{gretl}. Este lote de instrucciones
  comprueba si la variable de entorno \texttt{DISPLAY} está activada;
  si lo está, lanza el programa GUI, \verb@gretl_x11@, y si no, lanza
  el programa de línea de instrucciones, \texttt{gretlcli}.}
    
--- Abre el programa y espera las instruccionest del usuario.

\cmd{gretl} \textsl{archivo-de-datos}

--- Abre el programa con el archivo de datos indicado en su ventana de
trabajo. El archivo de datos puede estar en formato propio de
\app{gretl}, formato CSV, o formato BOX1 (véase el \emph{Gretl User's
  Guide}). El programa intenta detectar el formato del archivo y
procede en consecuencia. Véase también Sección \ref{path-search} abajo
para el comportamiento de búsqueda de directorios.

\cmd{gretl --help} (o \cmd{gretl -h})

--- Muestra un breve resumen del historial de la sesión y sale al
sistema operativo.

\cmd{gretl --version} (o \cmd{gretl -v})

--- Muestra la identificación de la versión del programa y sale al
sistema operativo.

\cmd{gretl --run} \textsl{archivo-de-instrucciones} (o \cmd{gretl -r}
\textsl{archivo-de-instrucciones})

--- Abre el programa con una ventana mostrando el archivo de
instrucciones indicado, listo para su ejecución. Véase Sección
\ref{path-search} abajo para las características de búsqueda de
directorios.

\cmd{gretl --db} \textsl{base-de-datos} (o \cmd{gretl -d}
\textsl{base-de-datos})

--- Abre el programa con una ventana en la que se muestra la base de
datos especificada. Si los archivos de la base de datos (el archivo
\texttt{.bin} y su acompañante \texttt{.idx} --- véase el \emph{Gretl
  User's Guide}) no están en el directorio de bases de datos por
defecto del sistema, es preciso especificar la ubicación completa.

Podemos configurar varios detalles de \app{gretl} con el menú
``Archivo, Preferencias''.


\begin{itemize}
\item El directorio base para los archivos compartidos de \app{gretl}.

\item El directorio base del usuario para archivos relacionados con
  \app{gretl}.

\item La instrucción para lanzar \app{gnuplot}.

\item La instrucción para lanzar GNU R.

\item La instrucción para visualizar archivos TeX DVI.

\item El directorio en donde empezar a buscar las bases de datos
  nativas de \app{gretl}.

\item El directorio en donde empezar a buscar las bases de datos de
  RATS 4.

\item El número IP del servidor de bases de datos de \app{gretl}.

\item El número IP y número de puerto del servidor proxy HTTP a
  utilizar para contactar con el servidor de bases de datos, en caso
  de que sea aplicable (si tenemos un cortafuegos).

\item Los programas de calculadora y edición a lanzar desde la barra
  de herramientas.

\item La fuente monoespacio a utilizar en los resultados de
  \app{gretl} en la pantalla.

\item La fuente a utilizar en los menús y otros mensajes. (Nota: esta
  opción no está presente cuando \app{gretl} está compilado para el
  escritorio \app{gnome}, ya que en este caso la elección de fuentes
  se controla directamente desde \app{gnome}.)

\end{itemize}

También hay algunas casillas para señalar. Señalando la casilla ``modo
experto'' se suprimen algunos mensajes de aviso, que se muestran en
caso contrario. Marcar la casilla ``Informar sobre las actualizaciones
de \app{gretl}'' hace que \app{gretl} intente conectar con el servidor
de actualizaciones al empezar el programa. Desmarcar ``Mostrar barra
de herramientas de \app{gretl}'' desactiva el icono del mismo. Si el
idioma del sistema no es el Inglés, y el carácter de punto decimal no
es el punto ``\texttt{.}'', desactivar la casilla ``Utilizar
configuración local para el punto decimal'' hará que \app{gretl}
utilice el punto a pesar de todo.

Finalmente, hay algunas elecciones alternativas: en la pestaña
``Abrir/Guardar camino'' puede establecerse el directorio por defecto
donde \app{gretl} tiene que buscar al abrir o guardar un archivo -
bien el directorio de usuario de \app{gretl} o bien el directorio de
trabajo actual.  En la pestaña ``Ficheros de datos'' puede
establecerse el sufijo a utilizar por defecto para los nombres de
archivos de datos. El sufijo normal es \texttt{.gdt} pero, si se
desea, se puede poner \texttt{.dat}, que era la norma en las versiones
anteriores del programa. Si se pone \texttt{.dat} por defecto,
entonces los archivos de datos serán guardados en el formato
``tradicional'' (véase el \emph{Gretl User's Guide}).  También es
posible seleccionar la acción a asociar con el icono de carpeta en la
barra de herramientas en esta misma pestaña de ``Ficheros de datos'':
se puede elegir entre abrir el listado de los archivos de datos
asociados con el libro de texto de Ramanathan, o con el libro de
Wooldridge.

Las opciones establecidas de esta manera se gestionan de modo
diferente dependiendo del contexto. En MS Windows se guardan en el
registro de Windows. En el escritorio de \app{gnome} se guardan en
\texttt{.gnome/gretl} en el directorio personal del usuario. En otros
casos, se guardan en un archivo llamado \texttt{.gretlrc} en el
directorio del usuario.
    

\section{gretlcli}
\label{optarg2}

\cmd{gretlcli}

--- Abre el programa y espera a que el usuario introduzca órdenes.

\cmd{gretlcli} \textsl{archivo-de-datos}

--- Abre el programa con un archivo de datos específico en su espacio
de trabajo. El archivo de datos puede estar en formato nativo de
\app{gretl}, formato CSV, o formato BOX1 (véase el \emph{Gretl User's
  Guide}). El programa intentará detectar el formato del archivo y
procederá en consecuencia. Véase también Sección \ref{path-search}
abajo, para más detalles sobre las características de búsqueda de
directorios.

\cmd{gretlcli --help} (o \cmd{gretlcli -h})

--- Muestra un breve resumen de uso (lista de instrucciones).

\cmd{gretlcli --version} (o \cmd{gretlcli -v})

--- Muestra la identificación de la versión del programa.

\cmd{gretlcli --pvalue} (o \cmd{gretlcli -p})

--- Abre el programa de modo que podemos especificar de manera
interactiva el valor p para varios estadísticos habituales.

\cmd{gretlcli --run} \textsl{archivo-de-guión} (o \cmd{gretlcli -r}
\textsl{archivo-de-guión})

--- Ejecuta las instrucciones que hay en \textsl{archivo-de-guión} y
después pasa el control a la línea de instrucciones. Véase Sección
\ref{path-search} para más detalles sobre las características de
búsqueda de directorios.

\cmd{gretlcli --batch} \textsl{archivo-de-guión} (o \cmd{gretlcli -b}
\textsl{archivo-de-guión})

--- Ejecuta las instrucciones que hay en \textsl{archivo-de-guión} y
después sale del programa.  Cuando utilice esta opción,
Vd. probablemente deseará redirigir la salida del programa a un
archivo. Véase Sección \ref{path-search} para más detalles sobre las
características de búsqueda de directorios.

Al usar las opciones \cmd{--run} y \cmd{--batch}, el guión de
instrucciones en cuestión tiene que abrir un archivo de datos. Es
posible hacer esto mediante la instrucción \cmd{open} dentro del lote.



\section{Directorios}
\label{path-search}

Cuando se indica el nombre de un archivo de datos o de un archivo de
lote de instrucciones a \app{gretl} o a \app{gretlcli} en la línea de
instrucciones, se busca el archivo de la siguiente manera:

\begin{enumerate}

\item ``Como tal''. Es decir, en el directorio de trabajo actual o, si
  se especifica la dirección completa, en el sitio especificado.

\item En el directorio del usuario de \app{gretl} (véase la Tabla
  \ref{tab-path} para valores dados por defecto).

\item En cualquier subdirectorio por debajo del directorio de usuario
  de \app{gretl}.

\item En el caso de archivos de datos, la búsqueda continúa con el
  directorio principal de \app{gretl}. En el caso de un lote de
  instrucciones, la búsqueda continúa en el directorio de lotes de
  instrucciones del sistema.  Véase la Tabla \ref{tab-path} para los
  valores dados por defecto.  (\texttt{PREFIX} denota el directorio
  base elegido en el momento en que fue instalado \app{gretl}.)

\item En el caso de archivos de datos, la búsqueda procede entonces
  hacia todos los subdirectorios por debajo del directorio principal
  de datos.

\end{enumerate}


\begin{table}[htbp]
  \caption{Directorios por defecto}
  \label{tab-path}
  \begin{center}
    \begin{tabular}{>{\raggedright\arraybackslash}p{120pt}
        >{\raggedright\arraybackslash}p{160pt}
        >{\raggedright\arraybackslash}p{160pt} }
      \textit{ } & \textit{Linux} & \textit{MS Windows} \\ [4pt]
      Directorio del usuario & \verb|$HOME/gretl| & \verb@PREFIX\gretl\user@ \\
      Directorio de datos del sistema & \texttt{PREFIX/share/gretl/data} & \verb@PREFIX\gretl\data@ \\
      Directorio de lotes de instrucciones del sistema & \texttt{PREFIX/share/gretl/scripts} & \verb@PREFIX\gretl\scripts@ \\
    \end{tabular}
  \end{center}
\end{table}

Por lo tanto no es necesario especificar el directorio completo de los
archivos de datos o de instrucciones a menos que se quiera prescindir
del mecanismo de búsqueda automática. (Esto también es válido para
\app{gretlcli} cuando se indica un nombre de archivo como argumento en
las instrucciones \cmd{open} o \cmd{run}.)

Cuando un lote de instrucciones contiene una instrucción para abrir un
archivo de datos, el orden de búsqueda del archivo de datos es como se
ha descrito arriba, con la excepción de que también se busca en el
directorio donde está el lote, inmediatamente después de intentar
encontrar el archivo de datos ``como tal''.


\subsection{MS Windows}
\label{MS-behave}


En MS Windows, la información sobre la configuración de \app{gretl} y
\app{gretlcli} se guarda en el registro de Windows. Cuando \app{gretl}
se instala por primera vez, se crea un conjunto de entradas de
registro que pueden cambiarse mediante el menú ``Archivo,
Preferencias''. En el caso de que haya que llevar a cabo algunos
ajustes de forma manual a esta información, las entradas se encuentran
(utilizando el programa estándar de Windows, \app{regedit.exe}) dentro
de \verb@Software\gretl@ en \verb@HKEY_LOCAL_MACHINE@ (el directorio
principal de \app{gretl} y la instrucción para invocar \app{gnuplot})
y \verb@HKEY_CURRENT_USER@ (las demás variables configurables).
      
