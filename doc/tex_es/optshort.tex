Como se puede observar en la sección~\ref{cmd-cmd}, el comportamiento de
muchas instrucciones de GRETL puede modificarse mediante el uso de
indicadores de opción. Estos toman la forma de dos caracteres de guion,
seguidos de una cadena de texto que describe un poco el efecto que tiene la opción.

Algunas opciones necesitan un parámetro, que debe unirse al ``indicador''
de opción por medio de un signo igual. Entre las opciones que \emph{no}
necesitan un parámetro, algunas habituales tienen una forma corta (un
único carácter de guion seguido de una única letra) y se considera
característico usar las formas cortas en los guiones de HANSL. La tabla
de abajo muestra las asignaciones pertinentes: para cualquier instrucción
de la primera columna que admita la opción de formato largo, también
se admite la forma corta de la segunda columna.

\begin{center}
\begin{tabular}{lc}
forma larga & forma corta \\ [4pt]
\option{verbose} & \texttt{-v} \\
\option{quiet}   & \texttt{-q} \\
\option{robust}  & \texttt{-r} \\
\option{hessian} & \texttt{-h} \\
\option{window}  & \texttt{-w} 
\end{tabular}
\end{center}
