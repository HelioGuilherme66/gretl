\chapter{Operadores}
\label{chap:operators}

\section{Precedencia}

La tabla~\ref{tab:ops} enumera los operadores disponibles en \app{GRETL} en
orden de precedencia decreciente: los operadores de la primera fila son los de
mayor preferencia, los de la segunda fila son los de segunda mayor preferencia,
y así sucesivamente. Los operadores que están en una misma fila tienen la misma
preferencia. Cuando operadores sucesivos tienen la misma preferencia, el orden
general de evaluación es de izquierda a derecha. Las excepciones son las
operaciones exponencial y multiplicación de una matriz traspuesta. La expresión
\verb|a^b^c| equivale a \verb|a^(b^c)|, no a \verb|(a^b)^c|, y de forma similar
\verb|A'B'C'| equivale a \verb|A'(B'(C'))|.

\begin{table}[htbp]
\caption{Precedencia de los operadores}
\label{tab:ops}
\begin{center}
\begin{tabular}{lllllllll}
\verb|()| & \verb|[]| & \texttt{.} & \verb|{}| \\
\texttt{!} & \texttt{++} & \verb|--| & \verb|^| & \verb|'| \\
\texttt{*} & \texttt{/} & \texttt{\%} & \verb+\+ & \texttt{**} \\
\texttt{+} & \texttt{-} & \verb|~| & \verb+|+ & \\
\verb|>| & \verb|<| & \verb|>=| & \verb|<=| & \texttt{..} \\
\texttt{==} & \texttt{!=} \\
\verb|&&| \\
\verb+||+ \\
\texttt{?:} \\
\end{tabular}
\end{center}
\end{table}

Además de las formas básicas mostradas en la Tabla, diversos operadores tienen
también una ``forma con punto'' (como en ``\texttt{.+}'' que se lee como
``punto más''). Estas son versiones al modo de los operadores básicos, pero
para ser utilizados exclusivamente con matrices; tienen la misma precedencia que
sus homólogos básicos. Los operadores con punto disponibles son los siguientes:

\begin{center}
\begin{tabular}{cccccccccc}
\verb|.^| & \texttt{.*} & \texttt{./} & \texttt{.+} &
 \texttt{.-} & \verb|.>| & \verb|.<| & \verb|.>=| &
 \verb|.<=| & \texttt{.=} \\
\end{tabular}
\end{center}

Cada operador básico se muestra una vez más en la siguiente lista junto con
una breve consideración sobre su significado. Aparte de los tres primeros
conjuntos de símbolos de agrupación, todos los operadores son binarios
excepto cuando se advierta de lo contrario.

\begin{longtable}{ll}
\verb|()| & Llamada a una función \\
\verb|[]|  & Indicación de subíndices \\
\texttt{.} & Pertenencia a un `bundle' (mira abajo) \\
\verb|{}|  & Definición de una matriz \\
\texttt{!} & NO lógico unario \\
\texttt{++} & Aumento (unario) \\
\verb|--| & Disminución (unaria) \\
\verb|^|  & Exponente \\
\verb|'|  & Transposición matricial (unaria) o multiplicación de traspuesta (binaria) \\
\texttt{*} & Multiplicación \\
\texttt{/} & División, ``división a la derecha'' matricial \\
\texttt{\%} & Módulo \\
\verb+\+    & ``División a la izquierda'' matricial \\
\texttt{**} & Producto de Kronecker \\
\texttt{+} & Suma \\
\texttt{-} & Resta \\
\verb|~| & Concatenación matricial horizontal \\
\verb+|+ & Concatenación matricial vertical \\
\verb|>| & Mayor que (booleano) \\
\verb|<| & Menor que (booleano) \\
\verb|>=| & Mayor o igual que \\
\verb|<=| & Menor o igual que \\
\texttt{..} & Rango desde--hasta (al elaborar listas) \\
\texttt{==} & Test de igualdad booleana \\
\texttt{!=} & Test de desigualdad booleana \\
\verb|&&| & Y lógico \\
\verb+||+ & O lógico \\
\texttt{?:} & Expresión condicional \\
\end{longtable}

Interpretar ``\texttt{.}'' como el operador de pertenencia  a un `bundle'
se limita al caso en que está inmediatamente precedido por el identificador
de un bundle, y seguido inmediatamente por un identificador válido (clave).

Puedes encontrar los detalles sobre la utilización de los operadores
relacionados con matrices (incluido el operador punto) en el capítulo sobre
matrices en la \textit{Guía de Usuario de GRETL}.

\section{Asignación}

Los operadores mencionados más arriba están todos ideados para ser utilizados
en el lado derecho de una expresión que asigna un valor a una variable (o que
simplemente calcula y presenta un valor ---consulta la instrucción \texttt{eval}).
Además, tienes el propio operador de asignación, ``\texttt{=}''. En efecto,
este tiene la precedencia más baja de todas: se evalúa todo el lado derecho
de la expresión antes de que tenga lugar la asignación.

Además del ``\texttt{=}'' plano, dispones de diversas versiones ``inflexionadas''
de asignación. Estas se pueden utilizar solo cuando la variable del lado izquierdo
ya está definida. La asignación inflexionada produce un valor que es una
función del valor previo en la izquierda y del valor calculado en la derecha.
Esos operadores se forman anteponiendo un símbolo de operador normal al signo igual.
Por ejemplo,
%
\begin{code}
y += x
\end{code}
%
El nuevo valor asignado a \texttt{y} por el enunciado de arriba es el valor
previo de \texttt{y} más \texttt{x}. Los otros operadores inflexionados
disponibles, que funcionan de un modo exactamente similar, son los
siguientes:

\begin{center}
\begin{tabular}{ccccccc}
\texttt{-=} & \texttt{*=} & \texttt{/=} & \verb|%=| & 
  \verb|^=| & \verb|~=| & \verb+|=+ \\
\end{tabular}
\end{center}

Además, para las matrices se proporciona una forma especial de asignación
inflexionada. Supongamos que la matriz \texttt{M} es $2 \times 2$. Si ejecutas
\texttt{M = 5}, esto tiene el efecto de sustituir \texttt{M} por una matriz
$1 \times 1$ con un único elemento 5. Pero si indicas \texttt{M .= 5}, esto
asigna el valor 5 a todos los elementos de \texttt{M} sin modificar la
dimensión de esta.

\section{Aumento y disminución}

Los operadores unarios \texttt{++} y \verb|--| siguen a su operando,
\footnote{El lenguaje C de programación también admite versiones de \texttt{++}
  y \verb|--| como prefijos, que aumentan o disminuyen su operando
  antes de proporcionar su valor. En \app{GRETL} únicamente se admite
  la forma como sufijo.}  que debe ser una variable de tipo escalar.
Su forma de utilización más simple es la de las expresiones aisladas, como
%
\begin{code}
j++  # abreviatura para j = j + 1
k--  # abreviatura para k = k - 1
\end{code}
%
Sin embargo, también se pueden integrar en expresiones más complejas,
en cuyo caso primero proporcionan el valor original de la variable en
cuestión, y después tienen el efecto secundario de aumentar o disminuir
el valor de la variable. Por ejemplo:
%
\begin{code}
scalar i = 3
k = i++
matrix M = zeros(10, 1)
M[i++] = 1
\end{code}
%
Después de la segunda linea, \texttt{k} tiene el valor 3 y \texttt{i}
tiene el valor 4. La última linea asigna el valor 1 al elemento 4 de la
matriz \texttt{M} y establece que \texttt{i} = 5.

\textit{Advertencia}: Como en el lenguaje C de programación, el operador
unario de incremento o disminución no se debe aplicar a una variable
junto con una referencia normal a la misma variable, en un único enunciado.
Esto es debido a que la orden de evaluación no está garantizada, dando
lugar a cierta ambigüedad. Considera lo siguiente:
%
\begin{code}
M[i++] = i # ¡No hagas esto!
\end{code}
%
Esto se supone que asigna el valor de \texttt{i} a \texttt{M[i]}, ahora
bien, ¿es el valor original o el aumentado? Actualmente, esto no está
definido.
