\chapter{El interfaz de l�nea de instrucciones}
\label{c-cli}



\section{Gretl en la consola}
\label{cli-console}

El paquete \app{gretl} incluye el programa
      de l�nea de instrucciones \app{gretlcli}. Este
      es, esencialmente, una versi�n actualizada del programa ESL de Ramu
      Ramanathan. En Linux, puede ejecutarse desde la consola, o en un xterm (o
      similar). En MS Windows, puede ejecutarse desde una ``ventana de
      MSDOS''. \app{Gretlcli} cuenta con su propio
      archivo de ayuda, al que puede accederse tecleando la palabra
      ``help'' desde el cursor. Es posible ejecutarlo en un
      proceso por lotes, y enviar los resultados directamente a un archivo (v�ase
      [?] arriba).

Si \app{gretlcli} est� conectado a la
      biblioteca \app{``readline''} (esto ocurre
      autom�ticamente en el caso de la versi�n MS Windows; v�ase tambi�n el [?]), la l�nea de instrucciones es recuperable y
      editable, y ofrece terminaci�n de las instrucciones. Podemos utilizar las
      flechas de Arriba y Abajo para pasar a las instrucciones que ya han sido
      tecleados previamente. En cualquier l�nea de instrucciones, podemos utilizar
      las flechas para movernos, junto con las combinaciones de teclas para
      editar de Emacs.\footnote{En realidad, las combinaciones dadas a
      continuaci�n son s�lo laos que hay por defecto; pueden ser
      personalizadas. V�ase el \href{http://cnswww.cns.cwru.edu/~chet/readline/readline.html}{manual de readline
      }.

} Las m�s comunes son :



\begin{tabular}{ll}
\textit{Combinaciones de teclas} & \textit{Efecto} \\ [4pt]
\verb+Ctrl-a+ & ir al inicio de la l�nea \\
\verb+Ctrl-e+ & ir al final de la l�nea \\
\verb+Ctrl-d+ & eliminar el car�cter de la derecha \\
\end{tabular}

donde ``\verb+Ctrl-a+'' significa pulsar la
      tecla ``\verb+a+'' a la vez que la tecla
      ``\verb+Ctrl+''. As�, si queremos cambiar algo en
      el inicio de una instrucci�n, \emph{no} es preciso ir hacia
      atr�s por toda la l�nea, borr�ndola con nuestro paso. Simplemente
      saltamos al inicio y a�adimos o eliminamos caracteres.

Al teclear las primeras letras de una instrucci�n, pulsando la tecla
      Tab el programa \app{readline} intentar� completar el
      nombre de la instrucci�n. Si es una combinaci�n �nica, saldr� la instrucci�n
      autom�ticamente. Si hay m�s de una posibilidad, pulsando Tab otra vez,
      se mostrar� una lista.



\section{Diferencias con ESL}
\label{cli-syntax}

Los lotes o guiones de instrucciones desarrollados para el programa \app{ESL}
      original de Ramanathan deber�an de ser utilizables en \app{gretlcli}
      con pocos o ning�n cambio: en lo �nico que hay que tener cuidado es en
      las instrucciones multilineales y la instrucci�n \cmd{freq}.


\begin{itemize}
\item \app{ESL} utiliza el punto y coma como
          terminaci�n de muchas de las instrucciones. Esto ha sido eliminado en
          \app{gretlcli}. El punto y coma simplemente se
          ignora, excepto en unos casos especiales donde tiene un significado
          determinante: como separador de dos listas en las instrucciones
          \cmd{ar} y \cmd{tsls}, o como marcador
          para una primera o �ltima observaci�n sin cambios en la instrucci�n
          \cmd{smpl}. En \app{ESL}, el punto y coma da la
          posibilidad de partir una instrucci�n larga de m�s de una linea;
          en \app{gretlcli} esto se hace con un
          \verb+\+ situado al final de la l�nea que va a
          continuarse.


\item Con \cmd{freq}, actualmente no se
          pueden especificar rangos definidos por el usuario como en
          \app{ESL}. Un contraste chi-cuadrado para
          normalidad ha sido a�adido a los resultados de esta instrucci�n.


\end{itemize}

N�tese que se ha simplificado la sintaxis de l�nea de instrucciones
      para un proceso por lotes. En \app{ESL} se tecleaba,
      por ejemplo
      
\begin{code}
        esl -b datafile < inputfile > outputfile
      
\end{code}

 mientras que en \app{gretlcli} hay
      que teclear: 
\begin{code}
    gretlcli -b inputfile > outputfile
      
\end{code}


      El archivo de entrada se trata como un
      argumento del programa; debe de especificar el archivo de datos que hay
      que usar de forma interna, utilizando la sintaxis \cmd{open fichero_de_datos}
      o el comentario especial \verb+(* !+
      \textsl{datafile} \verb+*)+

