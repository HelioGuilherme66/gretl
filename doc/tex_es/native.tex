\chapter{Data file details}
\label{app-datafile}



\section{Basic native format}
\label{native}

In \app{gretl}'s native data format,
	a data set is represented by two files.  One contains the
	actual data and the other information on how the data should
	be read.  To be more specific:


\begin{enumerate}
\item \emph{Actual data}: A
	    rectangular matrix of white-space separated numbers.  Each
	    column represents a variable, each row an observation on
	    each of the variables (spreadsheet style). Data columns
	    can be separated by spaces or tabs. The filename should
	    have the suffix \verb+.gdt+.  By default the
	    data file is ASCII (plain text).  Optionally it can be
	    gzip-compressed to save disk space. You can insert
	    comments into a data file: if a line begins with the hash
	    mark (\verb+#+) the entire line is
	    ignored. This is consistent with gnuplot and octave data
	    files.


\item \emph{Header}: The data file
	    must be accompanied by a header file which has the same
	    basename as the data file plus the suffix
	    \verb+.hdr+.  This file contains, in
	    order:


\begin{itemize}
\item (Optional) \emph{comments}
		on the data, set off by the opening string
		\verb+(*+ and the closing string
		\verb+*)+, each of these strings to occur
		on lines by themselves.


\item (Required) list of white-space separated
		\emph{names of the variables} in the
		data file. Names are limited to 8 characters, must
		start with a letter, and are limited to alphanumeric
		characters plus the underscore.  The list may continue
		over more than one line; it is terminated with a
		semicolon, \verb+;+.


\item (Required) \emph{observations} line of
		the form \verb+1 1 85+.  The first element
		gives the data frequency (1 for undated or annual
		data, 4 for quarterly, 12 for monthly).  The second
		and third elements give the starting and ending
		observations. Generally these will be 1 and the number
		of observations respectively, for undated data.  For
		time-series data one can use dates of the form
		\cmd{1959.1} (quarterly, one digit after
		the point) or \cmd{1967.03} (monthly, two
		digits after the point). See [?] below
		for special use of this line in the case of panel
		data.


\item The keyword
		\verb+BYOBS+.


\end{itemize}


\end{enumerate}

Here is an example of a well-formed data header file.

	
\begin{code} 
	  (* 
	  DATA9-6: 
	  Data on log(money), log(income) and interest rate from US. 
	  Source: Stock and Watson (1993) Econometrica 
	  (unsmoothed data) Period is 1900-1989 (annual data). 
	  Data compiled by Graham Elliott. 
	  *) 
	  lmoney lincome intrate ; 
	  1 1900 1989 BYOBS
\end{code}



	The corresponding data file contains three columns of data, each
	having 90 entries.



\section{Extensions to the basic data
	format}
\label{extensions}


	The options available in \app{gretl} data
	files are broader than the setup just described, in three
	ways:
      


\begin{enumerate}
\item If the \verb+BYOBS+ keyword is
	    replaced by \verb+BYVAR+, and followed by the
	    keyword \verb+BINARY+, this indicates that the
	    corresponding data file is in binary format.  Such data
	    files can be written from
	    \app{gretlcli} using the
	    \cmd{store} command with the
	    \cmd{-s} flag (single precision) or the
	    \cmd{-o} flag (double
	    precision).


\item If \verb+BYOBS+ is followed by the
	    keyword \verb+MARKERS+,
	    \app{gretl} expects a data file in
	    which the \emph{first column} contains strings (8
	    characters maximum) used to identify the observations.
	    This may be handy in the case of cross-sectional data
	    where the units of observation are identifiable:
	    countries, states, cities or whatever.  It can also be
	    useful for irregular time series data, such as daily stock
	    price data where some days are not trading days --- in
	    this case the observations can be marked with a date
	    string such as \cmd{10/01/98}.  (Remember the
	    8-character maximum.)  Note that \cmd{BINARY} and
	    \cmd{MARKERS} are mutually exclusive flags. Also note
	    that the ``markers'' are not considered to be a
	    variable: this column does not have a corresponding entry
	    in the list of variable names in the header
	    file.


\item If a file with the same base name as the data
	    file and header files, but with the suffix
	    \verb+.lbl+, is found, this is read to fill
	    out the descriptive labels for the data series. The format
	    of the (plain text) label file is simple: each line
	    contains the name of one variable (as found in the header
	    file), followed by one or more spaces, followed by the
	    descriptive label. Here is an example: 

	    \verb+price  New car price index, 1982 base year+

	    A label file of this sort is created automatically when
	    you save data from \app{gretl}, if
	    there is any descriptive information to be saved.  Such
	    information can be added under the ``Variable, Edit
	      label'' menu item.
	  


\end{enumerate}

