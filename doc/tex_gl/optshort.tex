Como pode verse na sección~\ref{cmd-cmd}, o comportamento de moitas
instrucións de GRETL pode modificarse mediante o uso de indicadores de
opción. Estes toman a forma de dous caracteres de guión, seguidos
dunha cadea de texto que describe un pouco o efecto que ten a opción.

Algunhas opcións requiren un parámetro, que debe unirse ao ``indicador''
de opción mediante un signo igual. Entre as opcións que \emph{non}
requiren un parámetro, algunhas habituais teñen unha forma curta (un
único carácter de guión seguido dunha única letra) e considérase
característico usar as formas curtas nos guións de HANSL. A táboa de
abaixo amosa as asignacións pertinentes: para calquera instrución da
primeira columna que admita a opción de formato longo, tamén se admite
a forma curta da segunda columna.

\begin{center}
\begin{tabular}{lc}
forma longa & forma curta \\ [4pt]
\option{verbose} & \texttt{-v} \\
\option{quiet}   & \texttt{-q} \\
\option{robust}  & \texttt{-r} \\
\option{hessian} & \texttt{-h} \\
\option{window}  & \texttt{-w} 
\end{tabular}
\end{center}
