\chapter{Opcións, argumentos e procura de rutas}
\label{optarg}

\section{Chamada a gretl}
\label{optarg1}

\cmd{gretl} (en MS Windows, \cmd{gretl.exe})\footnote{En Linux,
  instálase un guión de ``envoltura'' denominado \texttt{gretl}. Este
  guión comproba se a variable de contorna \texttt{DISPLAY} está determinada;
  en tal caso, lanza o programa GUI, \verb@gretl_x11@, e se non o está,
  lanza o programa de liñas de instrucións, \texttt{gretlcli}}.

--- Abre o programa e agarda polas achegas do usuario.
      
\cmd{gretl} \textsl{ficheirodatos}
      
--- Comeza o programa co ficheiro de datos indicado no seu espazo de traballo.
O ficheiro de datos pode estar en calquera de entre varios formatos (consulta
o \emph{Manual de Usuario de GRETL}); o programa vai intentar detectar o
formato do ficheiro e tratalo axeitadamente. Consulta tamén máis abaixo a
Sección \ref{path-search} para ver o comportamento na procura de rutas.
      
\cmd{gretl --help} (ou \cmd{gretl -h})
      
--- Presenta un breve resumo de uso e finaliza o programa.
      
\cmd{gretl --version} (ou \cmd{gretl -v})
      
--- Presenta a identificación da versión do programa e finaliza o programa.
      
\cmd{gretl --english} (ou \cmd{gretl -e})
      
--- Forza a utilización do idioma inglés en troques da tradución.
      
\cmd{gretl --run} \textsl{ficheiroguion} (ou \cmd{gretl -r}
\textsl{ficheiroguion})
      
--- Comeza o programa e abre unha xanela que amosa o ficheiro de guión
indicado, preparado para executarse. Consulta máis abaixo a Sección
\ref{path-search} para ver o comportamento na procura de rutas.
      
\cmd{gretl --db} \textsl{bancodatos} (ou \cmd{gretl -d}
\textsl{bancodatos})

--- Comeza o programa e abre unha xanela que amosa o banco de datos indicada.
Se os ficheiros de banco de datos (o ficheiro \texttt{.bin} e o seu ficheiro
adxunto \texttt{.idx}) non están no directorio de banco de datos predeterminado
do sistema, debes de indicar a ruta completa. Consulta tamén o
\emph{Manual de Usuario de GRETL} para obter detalles sobre os bancos de datos.
      
\cmd{gretl --dump} (ou \cmd{gretl -c})
      
--- Deita a información de configuración do programa nun ficheiro de texto plano
(o nome do ficheiro preséntase nunha saída de resultados normal). Pode ser de
utilidade para solucionar problemas.
      
\cmd{gretl --debug} (ou \cmd{gretl -g})

--- (Só en MS Windows) Abre unha xanela de consola para presentar calquera
mensaxe enviada a os fluxos de ``saída de resultados típica'' ou de
``fallo típico''. Esas mensaxes habitualmente non son visibles en Windows;
e isto pode ser de utilidade para solucionar problemas.
      
\section{Diálogo de preferencias}
\label{guiprefs}

Se poden configurar diversas cousas en \app{GRETL}, no menú de
``Ferramentas, Preferencias''. Os elementos do menú por separado dedícanse á
elección da fonte monoespazada que se usa na pantalla de resultados de \app{GRETL},
e, nalgunhas plataformas, a fonte utilizada nos menús e outras mensaxes.
As outras opcións organízanse en cinco lapelas, do seguinte xeito.
      
\textbf{Xeral}: Aquí podes configurar o directorio de base para os ficheiros
compartidos de \app{GRETL}. Ademais, hai varios cadriños de verificación.
Se a configuración do teu idioma nativo non é a do inglés e o carácter da
parte decimal non é o punto (``\texttt{.}''), non marcar ``Usar as preferencias
locais para o punto decimal'' vai facer que \app{GRETL} utilice de tódolos
xeitos o punto. Marcar ``Permitir instrucións do intérprete'' fai posible
chamar por instrucións do intérprete nos guións e na consola de \app{GRETL}
(esta facilidade non está dispoñible por defecto debido a razóns de seguridade).
      
\textbf{Programas}: Podes especificar os nomes ou rutas de diversos programas
de terceiros que é posible chamar con \app{GRETL} baixo determinadas
condicións.

\textbf{Editor}: Establece as preferencias referidas ao editor de guións de
\app{GRETL}.

\textbf{Rede}: Establece o servidor no que procurar bancos de datos de
\app{GRETL}; e tamén se usar ou non usar un proxy HTTP.
      
\textbf{HCCME}: Establece as preferencias relativas á estimación da matriz
robusta de covarianzas. Consulta o \emph{Manual de Usuario de GRETL} para máis detalles.
      
\textbf{MPI}: Isto só se amosa se GRETL está creado con apoio para MPI
(Interface de Paso de Mensaxes).
      
As configuracións escollidas mediante o diálogo de Preferencias gárdanse
dunha sesión de GRETL á seguinte.
      
\section{Chamada a gretlcli}
\label{optarg2}

\cmd{gretlcli}

--- Abre o programa e agarda polas achegas do usuario.
      
\cmd{gretlcli} \textsl{ficheirodatos}

--- Comeza o programa co ficheiro de datos indicado no seu espazo de traballo.
O ficheiro de datos pode ser calquera de entre varios admitidos por
\app{GRETL} (consulta o \emph{Manual de Usuario de GRETL} para máis detalles).
O programa vai intentar detectar o formato do ficheiro e tratalo axeitadamente.
Consulta tamén a Sección \ref{path-search} para ver o comportamento na procura de rutas.

\cmd{gretlcli --help} (ou \cmd{gretlcli -h})

--- Presenta un breve resumo de uso.

\cmd{gretlcli --version} (ou \cmd{gretlcli -v})

--- Presenta a identificación da versión do programa.

\cmd{gretlcli --english} (ou \cmd{gretlcli -e})

--- Forza a utilización do idioma inglés en troques da tradución.

\cmd{gretlcli --run} \textsl{ficheiroguion} (ou \cmd{gretlcli -r}
\textsl{ficheiroguion}

--- Executa as instrucións de \textsl{ficheiroguion} e logo pasa o control
das achegas do usuario á liña de instrucións. Consulta a Sección \ref{path-search}
para ver o comportamento na procura de rutas.

\cmd{gretlcli --batch} \textsl{ficheiroguion} (ou \cmd{gretlcli -b}
\textsl{ficheiroguion})

--- Executa as instrucións de \textsl{ficheiroguion} e logo sae do programa.
Cando utilices esta opción, probablemente vas querer redirixir a saída de
resultados a un ficheiro. Consulta a Sección \ref{path-search} para ver o
comportamento na procura de rutas.

Ao usar as opcións \cmd{--run} e \cmd{--batch}, o ficheiro de guión en cuestión
debe chamar por un ficheiro de datos para abrilo. Isto pode facerse utilizando
a instrución \cmd{open} dentro do propio guión.
      
\section{Procura de rutas}
\label{path-search}

Cando se indica o nome dun ficheiro de datos ou de guión a \app{GRETL} ou
a \app{gretlcli} na liña de instrucións, procúrase o ficheiro do seguinte
xeito:

\begin{enumerate}
\item ``Tal cal''.  Isto é, no directorio de traballo vixente ou, cando
  se indica a ruta completa, na localización especificada.
\item No directorio de GRETL do usuario (ver a Táboa \ref{tab-path} para os
  valores predeterminados; cae na conta de que \texttt{PERSONAL} é un marcador
  de posición que Windows expande dun xeito específico para o usuario e o
  idioma, tipicamente implicando ``My Documents'' nos sistemas con idioma inglés).
\item En calquera subdirectorio inmediato ao do usuario en GRETL.
\item No caso dun ficheiro de datos, a procura continúa co directorio principal
  de datos de \app{GRETL}. No caso dun ficheiro de guión, a procura continúa
  co directorio de guións do sistema. Consulta a Táboa \ref{tab-path} para ver
  as configuracións por defecto. (\texttt{PREFIX} denota o directorio de base
  escollido no momento no que se instala \app{GRETL}.)
\item No caso de ficheiros de datos, a procura continúa entón con tódolos
  subdirectorios inmediatos ao directorio principal de datos.
\end{enumerate}

\begin{table}[htbp]
  \caption{Configuracións das rutas predeterminadas}
  \label{tab-path}
  \begin{center}
    \begin{tabular}{lll}
       & \textit{Linux} & \textit{MS Windows} \\ [4pt]
      Directorio de usuario & \texttt{\$HOME/gretl} & 
        \verb@PERSONAL\gretl@ \\
      Directorio de datos do sistema & \texttt{PREFIX/share/gretl/data} & 
        \verb@PREFIX\gretl\data@ \\
      Directorio de guións do sistema & \texttt{PREFIX/share/gretl/scripts} & 
        \verb@PREFIX\gretl\scripts@ \\
    \end{tabular}
  \end{center}
\end{table}

Polo tanto non é necesario indicar a ruta completa para un ficheiro de datos
ou de guión, agás que queiras anular o mecanismo automático de procura.
(Isto tamén se aplica con \app{gretlcli}, cando indicas un nome de ficheiro
como argumento para as instrucións \cmd{open} ou \cmd{run}.)

Cando un guión de instrucións contén unha consigna para abrir un ficheiro
de datos, a orde na que se procura ese ficheiro é como a establecida máis
arriba; agás que tamén se procure no directorio que contén o guión, xa que
sería inmediatamente despois de intentar atopar o ficheiro de datos ``tal cal''.
      

\subsection{MS Windows}
\label{MS-behave}

En MS Windows, a información da configuración de \app{GRETL} e \app{gretlcli}
gárdase no rexistro de Windows. Se crea un conxunto axeitado de entradas
no rexistro cando se instala \app{GRETL} por primeira vez, e as configuracións
poden trocarse no menú ``Ferramentas, Preferencias'' de \app{GRETL}'. No caso
de que alguén necesite facer axustes manuais nesta información, poden atoparse
as entradas (utilizando o programa típico de Windows \app{regedit.exe}) dentro
de \verb@Software\gretl@ en \verb@HKEY_LOCAL_MACHINE@ (o directorio principal
de \app{GRETL} e as rutas ata diversos programas auxiliares) e en
\verb@HKEY_CURRENT_USER@ (tódalas outras variables que se poden configurar).
