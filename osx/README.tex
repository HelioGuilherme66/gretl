\documentclass[11pt]{article}
\usepackage[body={6.5in,9in},top=1in,left=1in,nohead]{geometry}
\usepackage[pdftex]{color}
\usepackage{hyperref}
\usepackage{myguide}
\usepackage[T1]{fontenc}
\usepackage{lucidabr}

\hypersetup{pdftitle={README for Gretl on OS X},
  pdfsubject={gretl},
  pdfauthor={Allin Cottrell},
  pdfkeywords={},
  pdfpagemode={None},
  colorlinks=true
}

\begin{document}

\begin{center}
{\color{gold} \titlefont gretl} \\[1ex]
version 1.9.7 for Mac OS X \\[2ex]

\textit{GNU Regression, Econometrics and Time-series Library\\
  This is free software under the GNU General Public License.}

\end{center}

\section{Installation}
\label{sec:install}

\textsf{Gretl} is distributed as a gzipped disk-image
(\texttt{.dmg.gz}).  Double-click to unzip it first.  Once you have a
\texttt{.dmg} file, double-click to mount the image, and open the
image in the Finder.  Then drag \textsf{Gretl} to somewhere suitable
on your system (preferably your \textsf{Applications} folder).  Once
that is done, you can eject the disk image and delete the
\texttt{.dmg} and \texttt{.dmg.gz} files.

If you already have an older version of \textsf{Gretl} installed, the
update via dragging may fail. In that case you should open the
\textsf{Applications} folder in the Finder, delete the \textsf{Gretl}
item, then try the drag again.

To run gretl, double-click on the gretl program icon.  You can drag
this icon onto the dock if you want a quick launcher.

\section{System requirements}
\label{sec:os}

This package was tested on Mac OS X 10.6.8 (``Snow Leopard'') and
is also known to work on OS X 10.7 (``Lion'').
Hopefully it will work on later releases of OS X but it will probably
not work on releases earlier than 10.4.

The package requires Apple's version of the X11 window system, which
is automatically included in a current OS X installation; in older
versions of OS X it is available as an optional install from the distribution DVD.
It also requires the GTK+ Framework, which can be found at the
following location:

\url{http://sourceforge.net/projects/gretl/files/osx/GTK+.framework.dmg/download}

\section{Trouble-shooting}

If double-clicking on the gretl icon doesn't work for you, try
starting X11, opening an xterm, navigating into the \textsf{bin}
folder inside \textsf{Gretl.app}, and executing 
\textsf{gretl} manually.  That is, do something like:

\begin{verbatim}
cd /Applications/Gretl.app/Contents/Resources/bin
./gretl
\end{verbatim}

Hopefully this will give you an informative error message.

If you are updating from a previous installation of gretl, it
is possible that your recorded user preferences may be out of
sync with current gretl.  The fix for that is to delete
your old preferences file.  In an xterm window, do:

\begin{verbatim}
rm -f ~/.gretl2rc
\end{verbatim}

\section{Documentation}
\label{sec:doc}

When you start gretl, please consult the program's Help menu for
documentation in various formats.

\vspace{.25in}

\raggedright
Allin Cottrell \\
Wake Forest University \\
\texttt{cottrell@wfu.edu} \\
October, 2011

\end{document}

%%% Local Variables: 
%%% mode: latex
%%% TeX-master: t
%%% End: 
