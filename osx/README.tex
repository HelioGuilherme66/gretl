\documentclass[11pt]{article}
\usepackage[body={6.5in,9in},top=1in,left=1in,nohead]{geometry}
\usepackage[pdftex]{color}
\usepackage{hyperref}
\usepackage{myguide}
\usepackage[T1]{fontenc}
\usepackage{lucidabr}

\hypersetup{pdftitle={README for Gretl on OS X},
  pdfsubject={gretl},
  pdfauthor={Allin Cottrell},
  pdfkeywords={},
  pdfpagemode={None},
  colorlinks=true
}

\begin{document}

\begin{center}
{\color{gold} \titlefont gretl} \\[1ex]
version 1.9.1 for Mac OS X \\[2ex]

\textit{GNU Regression, Econometrics and Time-series Library\\
  This is free software under the GNU General Public License.}

\end{center}

\section{Installation}
\label{sec:install}

\textsf{Gretl} is distributed either as a plain disk-image file
(with extension \texttt{.dmg}) or a gzipped disk-image
(\texttt{.dmg.gz}).  If it is gzipped, double-click to unzip it
first.  Once you have a \texttt{.dmg} file, double-click to
mount the image, and open the image in the Finder.
Then drag \textsf{Gretl} to somewhere suitable on your
system (preferably your \textsf{Applications} folder).  Once that
is done, you can eject the disk image and delete the \texttt{.dmg}
file (and the \texttt{.dmg.gz} file, if applicable).

To run gretl, double-click on the gretl program icon.  You can drag
this icon onto the dock if you want a quick launcher.

\section{System requirements}
\label{sec:os}

This gretl package was tested on Mac OS X v10.4.11 (``Tiger'').  
Hopefully it will work on later releases of OS X but it will
probably not work on earlier releases.

The package requires Apple's version of the X11 window system, which
is available as an optional install from the OS X distribution DVD.
It also requires the GTK+ Framework, which can be found at either
of the following locations:

\url{http://r.research.att.com/gtk2-framework.dmg} 

\url{http://ricardo.ecn.wfu.edu/pub/gretl/gtk2-framework.dmg}

\section{Trouble-shooting}

If double-clicking on the gretl icon doesn't work for you, try
starting X11, opening an xterm, navigating into the \textsf{bin}
folder inside \textsf{Gretl.app}, and executing 
\textsf{gretl} manually.  That is, do something like:

\begin{verbatim}
cd /Applications/Gretl.app/Contents/Resources/bin
./gretl
\end{verbatim}

Hopefully this will give you an informative error message.

If you are updating from a previous installation of gretl, it
is possible that your recorded user preferences may be out of
sync with current gretl.  The fix for that is to delete
your old preferences file.  In an xterm window, do:

\begin{verbatim}
rm -f ~/.gretl2rc
\end{verbatim}

\section{Documentation}
\label{sec:doc}

When you start gretl, please consult the program's Help menu for
documentation in various formats.

\vspace{.25in}

\raggedright
Allin Cottrell \\
Wake Forest University \\
\texttt{cottrell@wfu.edu} \\
November, 2009

\end{document}

%%% Local Variables: 
%%% mode: latex
%%% TeX-master: t
%%% End: 
