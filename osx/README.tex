\documentclass[11pt]{article}
\usepackage[body={6.5in,9in},top=1in,left=1in,nohead]{geometry}
\usepackage[pdftex]{color}
\usepackage{hyperref}
\usepackage{myguide}
\usepackage[T1]{fontenc}
\usepackage{lucidabr}

\hypersetup{pdftitle={README for Gretl on OS X},
  pdfsubject={gretl},
  pdfauthor={Allin Cottrell},
  pdfkeywords={},
  pdfpagemode={None},
  colorlinks=true
}

\begin{document}

\begin{center}
{\color{gold} \titlefont gretl} \\[1ex]
version 1.7.5 for Mac OS X \\[2ex]

\textit{GNU Regression, Econometrics and Time-series Library\\
  This is free software under the GNU General Public License.}

\end{center}

\section{Installation}
\label{sec:install}

After downloading the gretl disk image to your desktop, double-click
it to open. Then drag \textsf{Gretl} to somewhere suitable on your
system (such as your \textsf{Applications} folder or Desktop).

To run gretl, double-click on the gretl program icon.  You can drag
this icon onto the dock if you want a quick launcher.

\section{System requirements}
\label{sec:os}

This gretl package was built on Mac OS X v10.4.11 (``Tiger'').  I'm not
able to to test it on earlier versions of OS X; it may work on 10.3
but will almost certainly will not work on any Mac operating system
prior to 10.3.

The package requires Apple's version of the X11 window system, which
is available as an optional install from the Tiger distribution DVD.

The package does \textit{not} depend on third-party enhancements for
OS X such as \textsf{Fink}.

\section{Font problems?}
\label{sec:fonts}

If gretl will not start at all, or if it starts but fonts are missing
or font selection doesn't work (you should be able to select fonts
under the ``Preferences'' item on the File menu) try this fix:

\begin{enumerate}
\item Open a Terminal window.
\item Type the command \texttt{sudo fc-cache}
\item Hit the Enter key, then enter your password if prompted.
\item If you see something like \texttt{fc-cache: not found} then
   X11 is not installed (properly); otherwise the X11 font cache
   should be created/updated.
\item Restart gretl.
\end{enumerate}

\section{Other problems?}

If double-clicking on the gretl icon doesn't work for you, try
starting X11, opening an xterm, navigating into the \textsf{bin}
folder inside \textsf{Gretl.app}, and executing \textsf{gretl}
manually.  That is, do something like:

\begin{verbatim}
cd Gretl.app/Contents/Resources/bin
./gretl
\end{verbatim}

Hopefully this will give you an informative error message.

\section{Documentation}
\label{sec:doc}

When you start gretl, please consult the program's Help menu for
documentation in various formats.

\vspace{.25in}

\raggedright
Allin Cottrell \\
Wake Forest University \\
\texttt{cottrell@wfu.edu} \\
June, 2008

\end{document}

%%% Local Variables: 
%%% mode: latex
%%% TeX-master: t
%%% End: 
