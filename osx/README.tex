\documentclass[11pt]{article}
\usepackage{lucidabr}
\usepackage[pdftex]{color}
\usepackage{hyperref}
\usepackage{myguide}
\usepackage[body={6.5in,9in},top=1in,left=1in,nohead]{geometry}

\geometry{nohead}
\hypersetup{pdftitle={README for Gretl on OS X},
  pdfsubject={gretl},
  pdfauthor={Allin Cottrell},
  pdfkeywords={},
  pdfpagemode={None},
  colorlinks=true
}

\begin{document}

\begin{center}
{\color{gold} \titlefont gretl} \\[1ex]
version 1.2.8 for Mac OS X \\[2ex]

\textit{GNU Regression, Econometrics and Time-series Library\\
  This is free software under the GNU General Public License.}

\end{center}

\section{Installation}
\label{sec:install}

Just drag \textsf{Gretl\_Folder} to somewhere suitable on your system
(such as your Applications folder or Desktop).  Once it is in place
you can rename the folder if you wish (\textit{before} running the
program for the first time).

To run gretl, go into \textsf{Gretl\_Folder} and double-click on the
gretl program icon.  You can drag this icon onto the dock if you want
a quick launcher.  

Note: although you may rename the top-level gretl folder, do not move
or rename any components within that folder or the program will break.

\section{System requirements}
\label{sec:os}

This gretl package has been tested on Mac OS X v10.2.4 (``Jaguar'').
I gather it works on version 10.3 (``Panther''), although there may be
a problem with fonts (see below).  It almost certainly will
not work on any Mac operating system prior to 10.2.

The package requires Apple's X11 for Mac OS.  At one time this was
available as a ``beta'' for OS X v10.2.  It is currently (July, 2004)
available as an ``optional install from the third Mac OS X v10.3
Panther CD'' (quoting the Apple website).  If you're running v10.2 and
have not already installed X11 I think you're probably out of luck.

The package does \textit{not} depend on third-party enhancements for
OS X such as \textsf{Fink}.

\section{Fonts}
\label{sec:fonts}

When you start gretl you may find that the font used for the menus is
poorly chosen---or, worse, that the menus display lots of funny little
squares where letters should be.

If the menus are legible but the font is not good, you can try
selecting a different font under the ``File, Preferences, Menu font''
menu item.  If the menus are not legible at all, you might try this
repair:

\begin{enumerate}
\item Navigate to the \texttt{bin} folder inside
  \texttt{Gretl\_Folder}.
\item Open the file \texttt{gretl.sh} in a text editor.
\item Replace the line (near the top), \texttt{export
    GDK\_USE\_XFT=1}, with \texttt{export GDK\_USE\_XFT=0}, and save
  the file.
\item Restart gretl.
\end{enumerate}


\section{Documentation}
\label{sec:doc}

When you start gretl, please consult the program's Help menu for
documentation in various formats.

\section{Notes for OS X hackers}
\label{sec:hack}

You will notice that this is not a very slick packaging job.  I had
intended to produce a proper OS X ``bundle'' for gretl, but failed to
do so.  I'm not sure whether this is due to my lack of familiarity
with Mac concepts, to bugs in the tools I was using, or to problems
with the system on which I was trying to build the bundle (OS X
10.2.4, which had to be ``recovered'' after an abortive attempt to
update to 10.2.8).

Briefly, this was my experience:

\begin{itemize}
\item Tried \textsf{osacompile}, which I had read would compile a text
  representation of an AppleScript into a bundle skeleton using
  something like

\texttt{osacompile -o foo.app foo.txt}

But this produced a zero-byte output file.
\item Tried the Apple \textsf{Package Builder} app, having first arranged my
  files in the layout of a bundle (\texttt{Contents/MacOS},
  \texttt{Contents/Resources} and all that).  This produced a
  spiffy-looking package, but when I installed it nothing worked.  It
  didn't seem to act like a real bundle and the script wouldn't launch
  (with cryptic numeric error codes).
\item Tried the third-party \textsf{App Bundler}.  This looked
  promising, but when it came to actually building the bundle the
  program bombed out with an error message, ``Unable to create
  resource files.''
\end{itemize}

I'd appreciate any guidance on how to get this right.  In the
meantime, here's a description of how the package is currently set up.

\begin{itemize}
\item The gretl program icon represents a little AppleScript, which
  just activates X11, changes directory into the \textsf{bin} folder
  of the package, and launches \textsf{gretl.sh}.
\item The shell script \textsf{gretl.sh} then does most of the work,
  setting up the environment so that the \textsf{GTK} libraries and
  \textsf{gnuplot} will work in an arbitary location.  Once all the
  environment variables are set, this script launches the real
  executable, \textsf{gretl\_x11}.
\end{itemize}

If the AppleScript launcher doesn't work for you (Panther issues?) you
could try opening an xterm, going into the \textsf{bin} folder of the
gretl package, and executing \textsf{gretl.sh} manually. 

\vspace{.25in}

\raggedright
Allin Cottrell \\
Wake Forest University \\
\texttt{cottrell@wfu.edu} \\
July, 2004



\end{document}

%%% Local Variables: 
%%% mode: latex
%%% TeX-master: t
%%% End: 
