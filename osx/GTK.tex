\documentclass[11pt]{article}
\usepackage[body={6.5in,9in},top=1in,left=1in,nohead]{geometry}
\usepackage[pdftex]{color}
\usepackage{hyperref}
\usepackage{myguide,url}
\usepackage[T1]{fontenc}
\usepackage{lucidabr}

\hypersetup{pdftitle={README for GTK framework for Gretl on OS X},
  pdfsubject={gretl},
  pdfauthor={Allin Cottrell},
  pdfkeywords={},
  pdfpagemode={None},
  colorlinks=true
}

\begin{document}

\thispagestyle{empty}

\begin{center}
{\color{gold} \gtkfont GTK+}
\end{center}

Notes on the GTK package at

\url{http://sourceforge.net/projects/gretl/files/osx/GTK+.framework.dmg/download}

This is a distribution of the GTK libraries and friends, intended for
use with gretl on Mac OS X, version 10.6 (Snow Leopard) or higher.
It is a 32-bit i386 build (for intel Macs only).

\section{Installation}

Drag the \texttt{GTK+.framework} folder onto the \texttt{Frameworks}
folder. You may get an error message if another version of the GTK
framework is already installed. In that case you can do, at the
command prompt in a terminal window:

\verb|sudo rm -rf /Library/Frameworks/GTK+.framework|

or, if you want to preserve the existing version in case you need to
restore it later,
\begin{verbatim}
sudo mv /Library/Frameworks/GTK+.framework \
  /Library/Frameworks/GTK+.framework.old
\end{verbatim}

Then try the drag again.

\section{Source code}

If you wish to inspect or work with the source code that was used in
building this package, the following components can be found at
\url{http://ftp.acc.umu.se/pub/GNOME/sources/}.

\begin{tabular}{l}
glib-2.18.4 \\
atk-1.24.0 \\
pango-1.22.4 \\
gtk+-2.14.7 \\
\end{tabular}

Other source packages used are:

\begin{tabular}{ll}
pkg-config-0.23 & \url{pkgconfig.freedesktop.org} \\
gettext-0.18.1.1 & \url{www.gnu.org/s/gettext} \\
libpng-1.2.46 & \url{www.libpng.org} \\
fontconfig-2.8.0 & \url{www.freedesktop.org} \\
freetype-2.3.12 & \url{www.freetype.org} \\
pixman-0.22.2 & \url{cairographics.org} \\
cairo-1.10.2 & \url{cairographics.org} \\
\end{tabular}

\section{Developer files}

The framework includes various files needed for program development
using GTK: headers, pkgconfig files and so on. To find these, navigate
inside the framework folder to 

\verb|Versions/Current/Resources|

\vspace{.2in}

\raggedright
Allin Cottrell \\
Wake Forest University \\
October, 2011

\end{document}

%%% Local Variables: 
%%% mode: latex
%%% TeX-master: t
%%% End: 
