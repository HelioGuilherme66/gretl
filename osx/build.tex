\documentclass{article}
\usepackage[T1]{fontenc}
\usepackage[body={6.5in,9in},top=1in,left=1in,nohead]{geometry}
\usepackage{smallhds,lucidabr}
\usepackage{url}

\begin{document}

\setlength{\parskip}{1ex}
\setlength{\parindent}{0pt}

\begin{center}
  {\large \textbf{Building a gretl disk image for OS X}}\\[6pt]
Allin Cottrell, December 2007
\end{center}

\section{Objective}

To build a stand-alone disk image (dmg) of gretl, including a suitable
configured version of gnuplot, for Mac OS X.  The final user should be
able to download the dmg file, double-click to mount it, and drag the
Gretl.app folder (found ``inside'' the image) to an Applications
folder.  You'll use Fink in the build process but the final dmg should
not be dependent on Fink in any way; it will, however, be dependent on
Apple's X11.

\section{Overview}

First, here are the prerequisites:

\begin{itemize}
\item A fully functional installation of OS X.
\item Apple's X11 and the Xcode development package.  If these are
  not already installed, they should be found on the OS X installation
  DVDs.
\item A basic installation of Fink.
\item Source code for gretl and gnuplot.
\item A skeleton for Gretl.app plus some auxiliary scripts.
\end{itemize}

The method is as follows:

\begin{enumerate}
\item Install the Gretl.app skeleton.  This provides the ``space''
  into which you'll install gretl and gnuplot.
\item Under Fink, install the various required third-party packages
  (including the ``dev'' or developer components).  This includes
  GTK+ and friends (glib, atk, gdk, pango).
\item Configure and build gnuplot; install gnuplot into Gretl.app.
\item Configure and build gretl; install gretl into the right place
  inside the Gretl.app folder; delete some extraneous files.
\item Tar up various run-time files from your Fink installation and
  dump them into the appropriate place in Gretl.app (hence removing
  the dependency on Fink at run time).  This is the trickiest part.
\item Grab the latest gretl documentation and dump it into Gretl.app.
\item Create a compressed disk image containing Gretl.app.
\end{enumerate}

Steps 1, 2, 3 and 5 only need to be done once; thereafter you can
update Gretl.app with just steps 4, 6 and 7.

The following sections expand on each of the steps.

\section{The Gretl.app skeleton}

I'll make a gzipped tar file available.  This will contain a mostly
empty directory tree, but I'll include some ``generic'' files that
shouldn't depend on the particular OS X build platform.  This should
be unzipped in some suitable location; on the OS X system to which I
have access I've put it under \texttt{/Users/allin/dist}.

\url{http://ricardo.ecn.wfu.edu/~cottrell/gretl-osx/Gretl.app.tar.gz}

\section{Required Fink packages}

The exact line-up of these packages depends somewhat on the specific
OS X variant.  If a given package is available via OS X itself, then
you don't need to, and probably don't want to, install the
corresponding Fink package.  A case in point is libxml2, which is
supplied on recent OS X (but was not supplied in earlier variants).

The required packages will presumably include gtk+2, gtk+2-dev and
fftw3; recode may also be required; gnuplot is not required since
we'll be building that ourselves.  Libxml2 will hopefully be supplied
by OS X, and dlcompat doesn't seem to be needed any longer. It may be
helpful to install wget via Fink for build purposes.

\section{Building gnuplot}

Grab the patched source for gnuplot 4.2.2,
\url{http://ricardo.ecn.wfu.edu/~cottrell/gretl-osx/gnuplot-4.2.2-ac.tar.gz}.
Untar and configure.  FIXME need configure params, etc.


\section{Configuring and building gretl}

There's a file \texttt{myconf} in the osx subdirectory of the gretl
source.  You should use this, or a variant of it, to configure gretl.
Here's what it looks like:

\begin{verbatim}
export CFLAGS="-O2 -I/sw/include"
export LDFLAGS=-L/sw/lib
export CPPFLAGS=$CFLAGS
export PKG_CONFIG_PATH="/usr/lib/pkgconfig:/usr/X11R6/lib/pkgconfig:/sw/lib/pkgconfig"
export PATH=/Users/allin/dist/Gretl.app/Contents/Resources/bin:$PATH
./configure --prefix=/Users/allin/dist/Gretl.app/Contents/Resources \
  --disable-rpath --enable-build-doc
\end{verbatim}

After doing \texttt{make} and \texttt{make install} we run a script
named postinst to clear out unnecessary files and add a few extras.
This is also in the \texttt{osx} subdir of the source.

\begin{verbatim}
#!/bin/sh

# postinst: run this in the gretl build directory

# The directory above Gretl.app
TOPDIR=/Users/allin/dist
PREFIX=$TOPDIR/Gretl.app/Contents/Resources

rm -f $PREFIX/bin/gretl
rm -rf $PREFIX/include
rm -rf $PREFIX/share/aclocal
rm -rf $PREFIX/share/info
rm -rf $PREFIX/info
rm -rf $PREFIX/lib/pkgconfig
rm -f $PREFIX/lib/gretl-gtk2/*.la
rm -rf $PREFIX/share/emacs

install -m 644 osx/README.pdf $TOPDIR
install -m 755 osx/gretl.sh $PREFIX/bin/gretl
\end{verbatim}

\section{Copying Fink run-time files}

Needed: Explanation; sample script.

\section{Documentation files}

The canonical PDF documentation for gretl is available from
\url{ricardo.ecn.wfu.edu}.  You should do something like the following
(wget may be installed via Fink):

\begin{verbatim}
TOPDIR=/Users/allin/dist
TARG=$TOPDIR/Gretl.app/Contents/Resources/share/gretl/doc
rm -f gretl-guide.pdf
wget http://ricardo.ecn.wfu.edu/pub/gretl/manual/PDF/gretl-guide.pdf
cp gretl-guide.pdf $TARG
rm -f gretl-ref.pdf
wget http://ricardo.ecn.wfu.edu/pub/gretl/manual/PDF/gretl-ref.pdf
cp gretl-ref.pdf $TARG
\end{verbatim}

\section{Creation of dmg}

Below is a shell script to create the final compressed .dmg file.
There's a copy in the gretl source package, in the osx subdirectory
(called \texttt{dmg.sh}).  Obviously, you'll need to edit the line
that defines \texttt{TOPDIR}; hopefully the rest should be portable.

This script should be run from some ``neutral'' location outside of
the distribution tree; you don't want to get a recursive thing going,
whereby the dmg is included within itself.  I run \texttt{dmg.sh} from
\verb+~/bin+.

\begin{verbatim}
#!/bin/bash

# the directory above Gretl.app
TOPDIR=/Users/allin/dist

HERE=`pwd`
KB=`du -ks $TOPDIR | awk '{ print $1 }'`
KB=$((KB+640))
hdiutil create -size ${KB}k tmp.dmg -layout NONE
MYDEV=`hdid -nomount tmp.dmg`
sudo newfs_hfs -v gretl $MYDEV
hdiutil eject $MYDEV
hdid tmp.dmg
cd $TOPDIR && \
cp -a Gretl.app /Volumes/gretl && \
cp -a README.pdf /Volumes/gretl
cd $HERE
hdiutil eject $MYDEV
hdiutil convert -format UDZO tmp.dmg -o gretl.dmg && rm tmp.dmg

\end{verbatim}

\end{document}

%%% Local Variables: 
%%% mode: latex
%%% TeX-master: t
%%% End: 
