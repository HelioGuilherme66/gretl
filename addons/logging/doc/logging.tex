\documentclass{article}
\usepackage{doc,url,verbatim,fancyvrb}
\usepackage{pifont}
\usepackage{gretl}
\usepackage[letterpaper,body={6.3in,9.15in},top=.8in,left=1.1in]{geometry}

%\usepackage[a4paper,body={6.1in,9.7in},top=.8in,left=1.1in]{geometry}

\begin{document}

\setlength{\parindent}{0pt}
\setlength{\parskip}{1ex}

\newcommand{\argname}[1]{\textsl{#1}}

\title{logging version 1.0}
\author{Artur Tarassow}
\date{July 8, 2021}
\maketitle

THIS IS WORK IN PROGRESS

\section{Introduction}

This addon provides structured logging; that is, a means of recording
the history and progress of a computation as a log of events. There
are in principle two roles involved in logging---the \textit{coder}
(the writer of a hansl script or function package) and the
\textit{user} (the person running the script or package)---although
one person may play both roles.

The \textit{coder} gets to decide which events will be logged, and the
importance or ``level'' to be assigned to each event. This is done by
means of the five logging functions provided, namely \texttt{Debug},
\texttt{Info}, \texttt{Warn}, \texttt{Error} and \texttt{Critical} (in
increasing order of importance).  Each of these functions requires a
single string argument and offers no return value. For example, the
signature of \texttt{Info} is
\begin{code}
void Info (const string msg)
\end{code}

The \textit{user} chooses which log messages to have printed, by
selecting a threshold: print only messages of level $n$ or above.
This is done via the command
\begin{code}
set loglevel <level>
\end{code}
where \verb|<level>| can be given by number or name, as shown below.

\begin{center}
  \begin{tabular}{cl}
    number & name \\[4pt]
    0 & debug \\
    1 & info \\
    2 & warn \\
    3 & error \\
    4 & critical
  \end{tabular}
\end{center}

The default level is 2 or \texttt{warn}, so messages set via
the \texttt{Debug} and \texttt{Info} functions will not be printed
unless the user specifies a lower threshold.

\section{Remarks}

Using logging provides some advantages over using \texttt{print} or
\texttt{printf} statements:
\begin{enumerate}
\item It allows you to control the visibility and presentation of
  messages without editing the source code. For example, the code
\begin{code}
  Debug("This is a debugging message")
\end{code}
  will produce no output by default; such messages are printed only
  if the user selects a verbose level of logging.
\item It's cheap to leave debugging statements like this in the source
  code: the program evaluates the message only if it is currently
  called for.
\item Log messages can be written to a separate file which can be
  analysed afterwards. More on this below.
\end{enumerate}

Note that the message passed to a logging function does not have to a
fixed piece of text. You can incorporate current state information by
means of the \texttt{sprintf} function, as in this example
\begin{code}
  Warn(sprintf("The matrix X looks funny:\n%g\n", X))
\end{code}
which prints the elements of \texttt{X} following the message.

\section{Public functions}

This package provides five logging functions, namely \texttt{Debug},
\texttt{Info}, \texttt{Warn}, \texttt{Error} and \texttt{Critical}.
They all require a single string argument and offer no return
value. For example, the signature of Info is
\begin{code}
void Info (const string msg)
\end{code}

\section{Loglevel}

The importance of an event can be called its ``level''. Gretl
supports five log-levels, identified by either number or name, as follows:

\begin{center}
  \begin{tabular}{cl}
    number & name \\[4pt]
    0 & debug \\
    1 & info \\
    2 & warn \\
    3 & error \\
    4 & critical
  \end{tabular}
\end{center}

The user can select a log level via the command
\begin{code}
set loglevel <level>
\end{code}
And the current setting can be obtained via the accessor
\dollar{loglevel}.

The table below may be helpful in determining which level of logging
to use for which purpose.\footnote{It is borrowed from
  \url{https://docs.python.org/3/howto/logging.html}.}

\begin{center}
  \begin{tabular}{lp{0.7\textwidth}}
    Debug & Detailed information, typically of interest only when diagnosing
            problems.\\
    Info & Confirmation that things are working as expected.\\
    Warn & An indication that something unexpected happened, or indicative of some
           problem in the near future (e.g.\ ``disk space low''). The software is still working
           as expected.\\
    Error & Due to a more serious problem, the software has not been able to perform
            some function.\\
    Critical & A serious error, indicating that the program itself may be unable to
               continue running.
  \end{tabular}
\end{center}

Mumble, mumble
\begin{code}
include logging.gfn
Warn("Watch out!")      # will print a message to the console
Debug("I told you so")  # will not print anything
\end{code}
prints
\begin{code}
WARNING: Watch out!
\end{code}

\section{Timestamps}

Optionally, the user can arrange for each logging message to show a
timestamp. This is achieved via the command
\begin{code}
set logstamp on
\end{code}
where boolval should be on/1/true/TRUE. You can use off/0/false/FALSE
to turn off timestamps. (Mention \dollar{logstamp}?)

\section{Logging to file}

A common use of logging apparatus involves writing logged events to file.
\begin{code}
  set loglevel info
  string path_logfile = "mylog.txt" # target file where to append
  set logfile "@path_logfile"

  Info("This message will not be printed but appended to a file.")
  print "File content"
  eval readfile(path_logfile)
\end{code}
In this case the message goes to 'mylog.txt', and is not mixed in with
regular program output.

\section{Changelog}

v1.0 (July 2021)
- Initial version.

\end{document}
