\documentclass{article}
\usepackage[letterpaper,body={6.3in,9.15in},top=.8in,left=1.1in]{geometry}
\usepackage{fancyvrb,color,gretl}
\usepackage{amsmath}
\usepackage[authoryear]{natbib}
\usepackage[pdftex,hyperfootnotes=false]{hyperref}

\definecolor{steel}{rgb}{0.03,0.20,0.45}

\hypersetup{pdftitle={gretl + lasso},
            pdfauthor={Allin Cottrell},
            colorlinks=true,
            linkcolor=blue,
            urlcolor=red,
            citecolor=steel,
            bookmarksnumbered=true,
            plainpages=false
          }

\newcommand{\startappendices}{%
\newcounter{appcount}
\setcounter{appcount}{0}
\renewcommand{\thesection}{Appendix \Alph{appcount}}}

\newcommand{\myappendix}[1]{%
\addtocounter{appcount}{1}
\section{#1}}          

\setlength{\parskip}{1ex}
\setlength{\parindent}{0pt}

\newenvironment{funcdoc}
{\noindent\hrulefill\\[-12pt]}
{\medbreak}

\title{regls: regularized least squares in gretl}
\author{Allin Cottrell}

\begin{document}
\maketitle

\section{Introduction}
\label{sec:intro}

The \textsf{regls} addon is essentially a front-end for functionality
coded in C in the gretl \texttt{regls} plugin; to run the package you
will need gretl version 2020a or higher.  The plugin implements LASSO
\citep{tibshirani96}---by default via the Alternating Direction Method
of Multipliers (ADMM) algorithm as set out in \cite{boyd2010}; Ridge
regression, by default via Singular Value Decomposition; and the
``elastic net'' hybrid of LASSO and Ridge.

The best-known open-source implementation of regularized regression
(perhaps the best-known, period) is that provided by the
\textsf{glmnet} package for \textsf{R}. Since we make several
references to \textsf{glmnet} below we should state up front what
we're talking about. The authors of \textsf{glmnet} are Jerome
Friedman, Trevor Hastie, Rob Tibshirani, Balasubramanian Narasimhan
and Noah Simon. Current information on \textsf{glmnet} can be found at
\url{https://glmnet.stanford.edu/}; for further information on the
algorithms used in the package see \cite{glmnet10}.

This package supports LASSO, Ridge and elastic net via the
\texttt{regls()} function. We begin by discussing LASSO, which is the
default method.  Ridge is discussed in section~\ref{sec:ridge} and
elastic net in section~\ref{sec:elnet}.

We use the LASSO parameterization employed by Boyd \textit{et al}: the
objective is
\begin{equation}
  \label{eq:obj}
  \min_{\hat{\beta}} \quad \frac{1}{2}\,
  \sum_{i=1}^n (y_i - X_i\hat{\beta})^2 + \lambda \sum_{j=1}^k |\hat{\beta}_j|
\end{equation}
where $n$ is the number of observations, $k$ is the number of
candidate regressors (the number of columns of $X$) and
$\lambda \geq 0$ is the LASSO regularization hyperparameter. In this
context $\lambda = 0$ gives plain OLS, and at the other end of the
spectrum there exists a data-dependent value of $\lambda$, namely
\begin{equation}
  \label{eq:lmax}
  \lambda_{\max} = \|X'y\|_{\infty}
\end{equation}
which drives all elements of $\hat{\beta}$ to zero.  A key control
variable for our \texttt{lasso()} function is the scaled term
$s = \lambda/\lambda_{\max}$, such that $0 \leq s \leq 1$.

The \texttt{regls} function takes three arguments: a series (the
dependent variable), a list (the independent variables, not including
a constant) and a bundle to contain optional parameters; and it
returns a bundle, described below. Its signature is therefore
\begin{code}
function bundle regls (series y, list X, bundle parms)
\end{code}

The \texttt{parms} argument may be omitted, in which case all settings
assume their default values, described below.

One basic element in the \texttt{parms} bundle is a specification for
$\lambda$, which may take either of two forms, as follows:
\begin{enumerate}
\item under the key \texttt{lfrac} (``lambda fraction''), a scalar
  (single $s$ value) or vector (sequence of $s$ values); or
\item under the key \texttt{nlambda}, the number of $s$ values to be
  used ($\geq 4$), in which case the values will be assigned
  automatically.
\end{enumerate}
If \texttt{nlambda} is provided instead of \texttt{lfrac}, the
automatic $s$ vector is a logarithmically declining sequence starting
at 1 and finishing at 0.0001. For example, given \texttt{nlambda} = 5
the sequence will be $s$ = \{1, 0.1, 0.01, 0.001, 0.0001\}.

If neither \texttt{lfrac} nor \texttt{nlambda} is specified, the
default is as if \texttt{nlambda} were given as 25.

In case you wish to specify a sequence succinctly but with more
control, the package contains a utility function
\texttt{lambda\_sequence()}, which takes up to three arguments. The
first and second arguments (required) give the maximum $s$ and the
number of values, while the third (optional) argument can be used to
give the minimum $s$ (by default 0.0001). As with the \texttt{nlambda}
option the values are spaced logarithmically.  So if you were to do
\begin{code}
parms.lfrac = lambda_sequence(1, 20, 0.001)
\end{code}
the resulting sequence would be $s$ = \{1, 0.69519, 0.48329, \dots,
0.00144, 0.001\}.

A second basic member of the parameter bundle is \texttt{stdize}, a
boolean switch to toggle standardization of the data. The default is
to perform standardization (corresponding to a non-zero value of this
option), but if the data are already standardized on input
\texttt{stdize} may be set to 0.  The estimates include an intercept
(which is not subject to regularization) only if \texttt{stdize} is
on.

Another basic option is \texttt{verbosity}. This has a default value
of 1, meaning that \texttt{regls} prints out a certain amount of
information about its progress and/or results. Setting it to 0 makes
\texttt{regls} run (mostly) quietly; setting it to 2 produces more
output in some cases.

The further optional parameters, as well as the contents of the bundle
returned by \texttt{regls}, are best explained by reference to the
various modes of usage of the function, namely estimation with a
single value of $\lambda$; exploration of a range of $\lambda$ values
using a unified training sample; and (probably most relevant in
practice) search for optimal $\lambda$ via cross validation.

\section{Estimation with a single regularization}
\label{sec:single-lambda}

Suppose we have 1200 observations on some series \texttt{y} and list
\texttt{X} (with $k=100$ members) and we wish to train on the first
1000 observations, using $s = 0.2$, then predict for the remaining
200. And let's say the data are not pre-standardized. We might then
do:
\begin{code}
bundle parms = defbundle("lfrac", 0.2)
smpl 1 1000
bundle lb = regls(y, X, parms)
\end{code}
We'll then find the following in \texttt{lb}:
\begin{itemize}
\item \texttt{B}: The full vector of $k+1$ coefficients (including an
  intercept).
\item \texttt{nzb}: A vector holding only the non-zero coefficients.
\item \texttt{nzX}: A list identifying the regressors with non-zero
  coefficients.
\item \texttt{lmax}: The $\lambda_{\max}$ value for the standardized
  data.
\item \texttt{lambda}: The value of $\lambda = s\,\lambda_{\max}$, see
  (\ref{eq:lmax}) above.
\item \texttt{crit}: The minimized LASSO criterion, see (\ref{eq:obj})
  above.
\item \texttt{R2}: Coefficient of determination,
  $1 - \sum(y-\hat{y})^2/\sum(y-\bar{y})^2$.
\item \texttt{BIC}: The Bayesian Information Criterion for the
  estimated model.
\item \texttt{nobs}: The number of training observations used.
\item \texttt{lfrac}: The input value of $s$.
\item \texttt{stdize}: Whether \texttt{regls} did standardization or
  not.
\end{itemize}

To predict for the remainder of the observations we could then do:
\begin{code}
smpl 1001 1200
series pred = lincomb(lb.nzX, lb.nzb)
\end{code}

At some points below we refer to the coefficient of determination
(under the key \texttt{R2}) as ``$R^2$''. But note that with
regularized regression, unlike OLS, this figure is not equal to the
squared correlation between $y$ and $\hat{y}$.

\section{Exploring a range of regularizations}
\label{sec:simple-search}

Suppose we wish to compare results from several values of $\lambda$,
using all the training data. We might then revise the prior script as:
\begin{code}
matrix lamseq = lambda_sequence(1, 10)
bundle parms = defbundle("lfrac", lamseq)
smpl 1 1000
bundle lb = regls(y, X, parms)
\end{code}

In this case \texttt{lb.B} will be a matrix holding the full
coefficient vector for each $s$ (one column per $s$ value);
\texttt{lb.crit}, \texttt{R2} and \texttt{lb.BIC} will be column
vectors holding the LASSO criterion, $R^2$ and BIC value for each
$s$; and the bundle will contain these additional items:
\begin{itemize}
\item \texttt{lfmin}: The $s$ value which produces the smallest BIC
  value.
\item \texttt{idxmin}: The 1-based index value of \texttt{lfmin} in
  the vector passed as \texttt{lfrac}.
\end{itemize}

The BIC \citep{schwarz78} is calculated by \textsf{regls} as
$-2\ell(\hat{\beta}) + k^*(\lambda) \log n$, where $\ell(\hat{\beta})$
is the log-likelihood, based on the sum of squared residuals, and
$k^*(\lambda)$ the number of non-zero coefficients for the given
$\lambda$. This criterion provides a guide (though certainly not an
infallible one) to the likely out-of-sample performance of a model:
smaller values of BIC are better. Note that the LASSO criterion itself
does not offer such a guide (it is likely to decrease monotonically
along with $\lambda$), but it can be useful in comparing the
effectiveness of minimization algorithms (see section~\ref{sec:ccd}).

When multiple $\lambda$ values are specified, the vector \texttt{nzb}
and list \texttt{nzX} refer to the non-zero coefficients and
associated regressors obtained with $s$ = \texttt{lfmin} (the BIC
minimizer). Out-of-sample predictions using this $s$ can be obtained
via \texttt{lincomb(nzX, nzb)}.

\section{Optimizing via cross validation}
\label{sec:xvalid}

Searching for optimal $\lambda$ over the entire training sample we run
the risk of overfitting. The standard remedy is to divide the training
data into ``folds'' and do cross validation. The algorithm is then (in
pseudo-code):
\begin{code}
for each s value, s(j)
  MSE(j) = 0
end
for each fold, f(i)
  set the estimation sample to the complement of f(i)
  for each s value, s(j)
    perform regularized estimation using s(j) and predict for f(i)
    MSE(j) <- MSE(j) + MSE for f(i)
  end
end
\end{code}
We then perform regularized estimation on the full training data using the
$s$ value that yields the least total MSE on the above procedure (or
perhaps take an alternative approach---see below).

The basic options connected with cross validation (to be entered in
the parameter bundle passed to \texttt{regls}) are as follows:
\begin{itemize}
\item \texttt{xvalidate}: Boolean, trigger for doing cross validation (required).
\item \texttt{nfolds}: Integer, the number of folds (optional, default
  10).
\item \texttt{randfolds}: Boolean, whether the folds should be
  assigned randomly (optional, default 0).
\end{itemize}

At present the folds are either assigned at random or (by default)
they are sequences of consecutive observations. It may be worth adding
a facility to set the folds via a predefined series. A further point:
at present the folds are by construction all the same size---the
result of integer division of the number of training observations by
the number of folds, which means that any ``remainder'' training
observations are ignored. That could be generalized if it seems
worthwhile.

When cross validation is specified \texttt{regls} will print some
information on the performance of the values of $s$ used, a
snippet of which is shown below:
\begin{code}
          s        MSE         se
   1.000000   1.000000   0.063336
   0.615848   0.870633   0.057432
   0.379269   0.759641   0.048581
   0.233572   0.694237   0.043515
\end{code}
The \texttt{MSE} value is the mean across the folds, and \texttt{se}
is its standard error, computed as per \textsf{glmnet}.

While it would seem most natural to select for further prediction the
$s$ value that minimizes MSE on cross validation---call this
$s^*$---\textsf{glmnet} suggests an alternative policy: select the
largest $s$ that delivers an MSE within one standard error of the
minimum, which we'll call $s^{\dagger}$. It may be that $s^*$ and
$s^{\dagger}$ are the same value, but if not this policy gives the
benefit of the doubt to parsimony.

After cross validation, \texttt{regls} by default stores the full
coefficient matrix (one column per value of $s$, estimated on the full
training data) under the key \texttt{B}.  And the returned bundle also
holds the indices of both $s^*$ and $s^{\dagger}$, under the keys
\texttt{idxmin} and \texttt{idx1se} respectively. One can therefore
select the desired set of coefficients and obtain fitted values using
the full input list \texttt{X}---with \texttt{const} prepended unless
your data are pre-standardized.
\begin{code}
matrix optimal_b = lb.B[,idxmin] # or idx1se, or other column
list All = const X
series fitted = lincomb(All, optimal_b)
\end{code}
Note that \texttt{B} has as many rows as \texttt{X} has members, plus
one for the intercept.

But there's another method which may be more convenient: after cross
validation the \texttt{regls} return bundle holds a single
\texttt{nzb} vector and \texttt{nzX} list, such that fitted values can
be obtained thus:
\begin{code}
series fitted = lincomb(lb.nzX, lb.nzb)
\end{code}
By default it's the $s^*$ (\texttt{idxmin}) column that's selected for
forming \texttt{nzb}, but if you wish to use $s^{\dagger}$ you can
arrange for that by setting \texttt{use\_1se} to a non-zero value in
the parameter bundle passed to \texttt{regls}, as in
\begin{code}
parms.use_1se = 1
\end{code}

\subsection*{Further cross validation options}

Some additional cross validation options are supported.

\begin{itemize}
\item \texttt{seed}: Integer, you can supply this to control the
  randomization when \texttt{randfolds} is active, hence getting
  exactly repeatable results.
\item \texttt{single\_b}: Boolean. If non-zero it stops \texttt{regls}
  from estimating coefficients for all $s$ values after cross
  validation; only the selected ``best'' value ($s^*$ or
  $s^{\dagger}$) is used. The matrix \texttt{B} mentioned above is
  then holds just a single column. This may shave a little off the
  execution time.

\end{itemize}

\section{Execution speed}
\label{sec:speed}

According to the discussion in section 3.2.2 of \cite{boyd2010}: the
ADMM algorithm is reliable but is known \textit{not} to be fast (or
not if accurate results are wanted). However, we have been able to
accelerate ADMM to the point where execution time is unlikely to be an
issue, by two main means.
\begin{itemize}
\item We implemented the suggestion in section 3.4.1 of
  \cite{boyd2010}: letting the penalty factor $\rho$ vary across ADMM
  iterations to keep the magnitudes of the primary and dual residuals
  in rough balance. This turns out to be highly effective.
\item We implemented automatic ``farming out'' of cross validation to
  multiple \textsf{MPI} processes (when \textsf{MPI} is available on
  the host machine). It's possible to prevent this by adding
  \texttt{no\_mpi} to the parameter bundle with a non-zero value.
\end{itemize}
In one benchmark case we considered---with 1500 training observations,
101 covariates, 50 values of $\lambda$ and 10 randomized cross
validation folds---the execution time was about 13 seconds before
making the changes mentioned above, and about 1.5 seconds
thereafter.\footnote{On a desktop machine with 4 Intel i7 processors,
  running Linux.}

\section{Additional ADMM controls}
\label{sec:add-controls}

This section describes some additional controls over the ADMM
algorithm that can be passed to the \texttt{regls} function via the
\texttt{parms} bundle. Under the key \texttt{admmctrl} you can supply
a 3-vector whose elements are, in order:
\begin{itemize}
\item \texttt{rho}: a positive real number, the initial ADMM penalty
  parameter. It seems that $\rho = 8.0$ works well but higher or lower
  values might produce faster convergence in some cases.
\item \texttt{reltol}: the relative tolerance used in gauging whether
  the algorithm has converged sufficiently.
\item \texttt{abstol}: the absolute convergence tolerance (which will
  be scaled by the square root of the number of candidate regressors).
\end{itemize}
We have found that \texttt{reltol} and \texttt{abstol} values of
$10^{-4}$ and $10^{-6}$, respectively, produce reasonably accurate
results in a manageable number of iterations. Setting smaller values
will produce greater accuracy at the cost of more iterations.
Non-positive values of these terms are ignored, so one can, for
example, set a single element by passing a zero vector with just the
desired term set to a positive value.

\section{LASSO examples}
\label{sec:examples}

Besides the sample script supplied with the package, more examples can
be found in the directories \texttt{murder}, \texttt{wine} and
\texttt{fat} at \url{http://ricardo.ecn.wfu.edu/pub/gretl/lasso/}.
Some of these scripts incorporate comparison with \textsf{glmnet}.
The murder-rate and wine quality examples use real-world data; the
\texttt{fat} example is an artificial case with more regressors than
observations.

Note that it's necessary to run the scripts involving randomized cross
validation several times to get a good idea of what's going on: in
each case there seem to be a few ``favoured solutions'' of varying
probability. Sometimes one sees gretl finding the better one,
sometimes \textsf{glmnet}.

\section{Ridge regression}
\label{sec:ridge}

While LASSO involves $\ell_1$ regularization, Ridge uses $\ell_2$: the
penalty factor $\lambda$ applies to the sum of squared coefficients,
giving rise to the following objective:
\begin{equation}
  \label{eq:ridge-obj}
  \min_{\hat{\beta}} \quad
  \sum_{i=1}^n (y_i - X_i\hat{\beta})^2 + \lambda \sum_{j=1}^k \hat{\beta}_j^2
\end{equation}

In consequence, although a large value of $\lambda$ will shrink Ridge
estimates substantially relative to OLS it will not send any
coefficients to exactly zero as does LASSO. If the $X$ matrix exhibits
strong collinearity, LASSO will tend to eliminate most of the
collinear terms while Ridge will tend to distribute the predictive
weight across the terms, yielding several small coefficients instead
of one relatively substantial coefficient and a bunch of zeros.

To get the \texttt{regls} function to perform Ridge regression rather
than LASSO, set a value of 1 under the key \texttt{ridge} in the
\texttt{parms} bundle, as in
\begin{code}
parms.ridge = 1
\end{code}

Most of the points made above with respect to LASSO carry over to
Ridge. The same three modes of operation described in sections
\ref{sec:single-lambda} to \ref{sec:xvalid} (from estimation using a
single value of $\lambda$ to cross-validation with as many values as
you like) are available.

There is an important difference, however, in respect of the
calibration of $\lambda$. In the LASSO case there's an easily computed
$\lambda_{\max}$ (= $\|X'y\|_{\infty}$) which just suffices to force
all slope coefficients to zero and so, as explained above, the user is
asked to express the LASSO penalty as a fraction of this maximum. In
the case of Ridge there is generally no finite $\lambda$ that will
drive all coefficients to zero and so no ``natural'' maximum to serve
as a benchmark. We therefore offer the user three options for the
specification of ``\texttt{lfrac},'' controlled by the integer-valued
parameter \texttt{lambda\_scale}:
\begin{itemize}
\item \texttt{lambda\_scale} = 0: no scaling is performed. The
  ``\texttt{lfrac}'' values are taken as actual $\lambda$ values (and
  so do not have to be bounded by 1.0 above).
\item \texttt{lambda\_scale} = 1: we emulate \textsf{glmnet}. The
  largest value of $\lambda$ is set to $9.9 \times 10^{35}$, which
  will drive all coefficients to near-zero. The second-largest
  $\lambda$ (call it $\lambda_2$) is then set to 1000 times
  $\|X'y\|_{\infty}$, and subsequent values in the sequence are scaled
  in relation to $\lambda_2$. This is the default setting.
\item \texttt{lambda\_scale} = 2: we follow the suggestion of some
  practitioners, setting $\lambda_{\max}$ to the squared Frobenius
  norm of $X$, which will not drive all coefficients to near-zero but
  will impose substantial shrinkage in relation to OLS.
\end{itemize}

To be clear on the action of options 1 and 2 for
\texttt{lambda\_scale}, suppose our \texttt{lfrac} specification is
\begin{code}
lfrac = {1, 0.5, 0.25, 0.125}
\end{code}
Then if \texttt{lambda\_scale} = 1 this translates to
\begin{code}
lam2 = 1000 * infnorm(X'y)
effective_lambda = {9.9e35, lam2, 0.5*lam2, 0.25*lam2}
\end{code}
while if \texttt{lambda\_scale} = 2 it becomes
\begin{code}
lam1 = tr(X'X) # Frobenius norm squared
effective_lambda = {lam1, 0.5*lam1, 0.25*lam1, 0.125*lam1}
\end{code}
Note that the relevant matrix norms are computed after
standardization.

In addition to \texttt{BIC} and \texttt{R2} the return bundle from
Ridge regression contains \texttt{edf} (a scalar if a single $\lambda$
is specified, otherwise a column vector). This is the ``effective''
degrees of freedom, or number of free parameters, calculated via
the SVD of the matrix of regressors:
\begin{equation}
  \label{eq:edf}
\mbox{edf} = \sum_{i=1}^k\, \frac{\sigma_i^2}{\sigma_i^2 + \lambda}
\end{equation}
where the $\sigma_i$s are the singular values. As a measure of the
``size'' of a model this takes the place of the number of non-zero
coefficients in LASSO.

In the case of a single $\lambda$, when estimation is performed using
the default SVD method, further information is available: the return
bundle contains the covariance matrix of the parameter estimates
(other than the constant) under the key \texttt{vcv}. And if the
\texttt{verbosity} option is set to 2 you get a printout of the model,
showing standard errors, $z$ statistics and $P$-values.

\section{The CCD option}
\label{sec:ccd}

As stated above, the default algorithms used by \textsf{regls} for
LSSO and Ridge are ADMM and SVD, respectively. However, you have the
option---for both LASSO and Ridge---of using the CCD (Cyclical
Coordinate Descent) algorithm, as employed by \textsf{glmnet}. This is
governed by two additional keys in the \texttt{parms} bundle:
\begin{itemize}
\item \texttt{ccd}: boolean, default 0. Set this to 1 to use CCD.
\item \texttt{ccd\_toler}: a positive scalar setting the convergence
  tolerance for CCD. The default is $10^{-7}$; setting a smaller value
  will give greater accuracy at the expense of more iterations.
\end{itemize}

Using CCD will give results that are more directly comparable with
\textsf{glmnet}. Beyond that, practitioners are likely to ask, which
of the available algorithms is faster, and which more accurate? Those
questions are not so simple to answer.

First, speed and accuracy cannot be assessed independently. In the
case of LASSO, both ADMM and CCD are numerical algorithms that deliver
an approximation to a notional ``exact'' result, with the latitude of
the approximation governed by one or more tolerance parameters. Both
algorithms are effective but they are quite different; their tolerance
settings are not directly comparable and the default values are
somewhat arbitrary. If method A produces a more accurate result than
method B under default settings it is generally possible to reduce the
tolerance(s) for B such that it equals the accuracy of A (which may
reverse their ranking in terms of execution time).

Moreover, even the relative accuracy of two sets of results is not so
straightforward to assess. Both LASSO and Ridge are minimization
problems, so it's tempting to assume that---for a given dataset and
given $\lambda$---the method that produces the smaller criterion value
is more accurate. But the LASSO objective shown in equation
(\ref{eq:obj}) is the Lagrangian form of a problem which can also be
written as
\[
  \min_{\hat{\beta}} \quad
  \sum_{i=1}^n (y_i - X_i\hat{\beta})^2 \,\mbox{ subject to }\,
  \sum_{j=1}^k |\hat{\beta}_j| \leq t
\]
Part of the error of approximation may include failure to exactly meet
the condition on the right, in which case a seemingly better result
may actually be less accurate.

A similar point can be made about Ridge. Unlike LASSO the Ridge
problem has an analytic solution and by default \textsf{regls}
implements that solution via SVD, which is pretty much the gold
standard for accuracy in digital computation. In some cases, however,
CCD at its default tolerance ($\epsilon = 10^{-7}$) produces a
slightly smaller Ridge criterion value. Is it really doing better than
SVD? No, as can be shown by tightening the CCD tolerance. For example,
with $\epsilon = 10^{-12}$ or so CCD may reproduce the SVD result.

This sort of experiment is generally useful for assessing relative
accuracy. If the difference between the results delivered by methods A
and B shrinks as the tolerance for method B is tightened, we can be
reasonable confident that A's initial result was more accurate. On
that basis our impression is that, with tolerances at their respective
default values, both ADMM LASSO and SVD Ridge are generally somewhat
slower and somewhat more accurate than the CCD alternative.

\section{Elastic net}
\label{sec:elnet}

As mentioned above, elastic net is a hybrid of LASSO and Ridge. It
employs a combination of $\ell_1$ and $\ell_2$ penalties governed by a
hyperparameter $0 \leq \alpha \leq 1$. The objective is
\[
    \min_{\hat{\beta}} \quad \frac{1}{2}\,
    \sum_{i=1}^n (y_i - X_i\hat{\beta})^2 +
    \lambda \left(\frac{1-\alpha}{2} \sum_{j=1}^k \hat{\beta}_j^2
      + \alpha \sum_{j=1}^k |\hat{\beta}_j|\right)
  \]
Thus $\alpha = 1$ gives LASSO, $\alpha = 0$ gives Ridge, and anything
between gives a combination. It has been argued that better
out-of-sample prediction can be obtained in some cases by preserving
some highly collinear regressors \`a la Ridge, while sending some
coefficients to exact zero as in LASSO, and elastic net allows for
this.

In the \texttt{regls} function, elastic net is selected by specifying
a fractional value under the key \texttt{alpha} in the parameters
bundle. This automatically switches to the CCD algorithm
(section~\ref{sec:ccd}), so the \texttt{ccd\_toler} option becomes
applicable.\footnote{In principle the ADMM algorithm could handle
  elastic net, but to date we have not implemented such support.}

When elastic net is used on a sequence of $\lambda$s without cross
validation (see section~\ref{sec:simple-search}) \texttt{regls}
provides a BIC measure as a possible means of selecting the most
promising penalty factor. However, calculation of the effective number
of parameters in this case is a debatable matter. What \texttt{regls}
uses at present is a weighted average of the number of non-zero
coefficients as in LASSO, with weight $\alpha$, and the SVD-derived
``edf'' for Ridge as in equation (\ref{eq:edf}), with weight
$1-\alpha$.

Note that if cross validation is called for with elastic net, it is
only the $\lambda$ value that is optimized. To assess the efficacy of
various $\alpha$ values one would have to perform several cross
validation runs.

\section{Reference: public functions}
\label{sec:funcref}

\begin{funcdoc}
\begin{verbatim}
bundle regls (series y, list X, bundle parms)
\end{verbatim}
  Performs LASSO, Ridge or elastic net estimation given the dependent
  variable \texttt{y}, the regressors \texttt{X}, and options in
  \texttt{parms}. Returns a bundle containing the
  results. Table~\ref{tab:regls-parms} lists the parameters that can
  be passed via the \texttt{parms} argument.
\end{funcdoc}

\begin{table}[htbp]
  \centering
  \begin{tabular}{lll}
    \texttt{lfrac} & scalar or vector & $\lambda$-fraction(s) \\
    \texttt{nlambda} & integer & number of automatic $\lambda$s \\
    \texttt{stdize} & 0/1, default 1 & standardize the data \\
    \texttt{ridge} & 0/1, default 0 & do Ridge regression \\
    \texttt{verbosity} & 0, 1 or 2, default 1 & printing of output\\
    \texttt{xvalidate} & 0/1, default 0 & do cross validation\\
    \texttt{nfolds} & optional integer $>1$, default 10 & number of
                                                          folds \\
    \texttt{randfolds} & 0/1, default 0 & use random folds\\
    \texttt{use\_1se} & 0/1, default 0 & see section~\ref{sec:xvalid} \\
    \texttt{seed} & optional integer & controls random folds\\
    \texttt{single\_b} & 0/1, default 0 & see section~\ref{sec:xvalid}\\
    \texttt{no\_mpi} & 0/1, default 0 & see section~\ref{sec:speed}\\
    \texttt{admmctrl} & optional control vector & see
                                                  section~\ref{sec:add-controls} \\
    \texttt{ccd} & 0/1, default 0 & see section~\ref{sec:ccd} \\
    \texttt{ccd\_toler} & positive scalar, default $10^{-7}$ & see
                                                               section~\ref{sec:ccd}\\
    \texttt{alpha} & $0 \leq \alpha \leq 1$ (default 1) & see
                                                               section~\ref{sec:elnet}
  \end{tabular}
  \caption{Summary of parameters for the \texttt{regls} function}
  \label{tab:regls-parms}
\end{table}

\begin{funcdoc}
\begin{verbatim}
matrix lambda_sequence (scalar lmax, int K, scalar eps[0.0001])
\end{verbatim}
  Produces a column vector holding a logarithmic sequence of
  \texttt{K} values running from \texttt{lmax} to \texttt{eps}. It is
  required that $0 < \mbox{\texttt{lmax}} \le 1$ and
  $0 \le \mbox{\texttt{eps}} < \mbox{\texttt{lmax}}$. In context such
  values are interpreted as instances of $s = \lambda/\lambda_{\max}$.
\end{funcdoc}

%%\clearpage

\begin{funcdoc}
\begin{verbatim}
matrix regls_get_stats (const series y, const series yhat)
\end{verbatim}
  Returns a 2-vector holding MSE = $\sum(y - \hat{y})^2/n$
  and $R^2 = 1 - \sum(y - \hat{y})^2/\sum(y - \bar{y})^2$.
\end{funcdoc}

\begin{funcdoc}
\begin{verbatim}
scalar regls_pc_correct (const series y, const series yhat)
\end{verbatim}
  Returns the percentage of cases in which \texttt{yhat} rounded to
  the nearest integer equals \texttt{y}. Useful only when \texttt{y}
  is integer-valued.
\end{funcdoc}

\begin{funcdoc}
\begin{verbatim}
matrix regls_foldvec (int nobs, int nf)
\end{verbatim}
  Returns a column vector of length \texttt{nobs} in which \texttt{nf}
  successive blocks of length \texttt{nobs}/\texttt{nf} take on the
  values 1, 2,\dots, \texttt{nf}, respectively. Useful only for
  composing a \texttt{folds} vector than can be passed to
  \textsf{glmnet} for comparison with gretl when consecutive folds are
  used in cross validation.
\end{funcdoc}

\begin{funcdoc}
\begin{verbatim}
void regls_multiprint (const matrix B, const matrix lam,
                       const series y, list X,
                       const matrix crit[null])
\end{verbatim}
  Given a $k \times m$ (or $k+1 \times m$) coefficient matrix
  \texttt{B}, an $m$-vector of $\lambda$ values \texttt{lam}, a
  dependent variable \texttt{y} and a list \texttt{X} of $k$
  regressors, prints a summary table with $m$ rows showing $R^2$, the
  sum of absolute values of the coefficients, and df (the number of
  non-zero coefficients) associated with each $\lambda$. It is
  presumed that \texttt{B[,j]} holds LASSO estimates produced using
  \texttt{lam[j]}. If the optional \texttt{crit} matrix is provided it
  should be an $m$-vector holding the criterion value associated with
  each $\lambda$, which will be shown as a fifth column.

  As in the \texttt{regls} function, the \texttt{X} list should not
  contain a constant: if the \texttt{B} matrix has one more row than
  \texttt{X} has members, the constant is handled
  automatically. Listing~\ref{script:multiprint} shows how the
  information provided in the \texttt{regls} return bundle can be used
  to supply the arguments \texttt{B}, \texttt{lam} and \texttt{crit}.
\end{funcdoc}

\begin{funcdoc}
\begin{verbatim}
series glmnet_pred (matrix *Rb, list X)
\end{verbatim}
  Convenience function for handling results retrieved by gretl from
  \textsf{glmnet}. On entry \texttt{Rb} should hold the full
  coefficient vector (including any zeros) and \texttt{X} the full
  list of candidate regressors, and the return value is the result of
  \texttt{lincomb(X, Rb)}. On exit \texttt{Rb} holds only the non-zero
  coefficients, with row-names added based on the \texttt{X}
  list. This is then comparable with gretl's \texttt{nzb}.
\end{funcdoc}

\begin{script}
  \caption{Usage of \texttt{regls\_multiprint}}
  \label{script:multiprint}
\begin{scode}
set verbose off
include regls.gfn
open murder.gdt --quiet --frompkg=regls

# all available predictors w. no missing values
list X = population..LemasPctOfficDrugUn
smpl 1 800
bundle parms = defbundle("nlambda", 8, "verbosity", 0)
bundle lb = regls(murdPerPop, X, parms)
regls_multiprint(lb.B, lb.lfrac, murdPerPop, X, lb.crit)
\end{scode}
  \end{script}

\bibliographystyle{gretl}
\bibliography{gretl}

\clearpage
\startappendices

\myappendix{Comparison with glmnet}
\label{sec:comparison}

It's likely that potential users of this package will take
\textsf{R}'s \textsf{glmnet} as benchmark. So we'd be well advised to
ensure that our results are close to those from \textsf{glmnet} unless
we can demonstrate a good reason for divergence. We're at that point:
on some examples gretl's \texttt{regls} gets a smaller (better) LASSO
criterion value than \textsf{glmnet} for a given $\lambda$, and on
some examples \textit{vice versa}, within a quite small range of
variation. Similarly for the typical full \textsf{regls} workflow;
that is, randomized cross validation followed by out of sample
prediction. Sometimes we do a bit better than \textsf{glmnet},
sometimes not quite so well.

\subsection*{Different conventions}

It should be noted that \textsf{regls} and \textsf{glmnet} employ
different conventions in some respects. This does not affect the
comparison of reported coefficients or predicted values, but it can
make comparison of $\lambda$ values a little awkward. The LASSO
objective function and definition of $\lambda_{\max}$ used by
\textsf{regls} were stated in section~\ref{sec:intro}, but to be fully
explicit we should say that the $X$ and $y$ in equation
(\ref{eq:lmax}) for the maximum $\lambda$ are taken to be standardized
values.

In \textsf{glmnet} the objective (in the linear Gaussian case) is
\[
   \min_{\hat{\beta}} \quad \frac{1}{2n}\,
  \sum_{i=1}^n (y_i - X_i\hat{\beta})^2 + \lambda \sum_{j=1}^k |\hat{\beta}_j| 
\]
This differs from our equation (\ref{eq:obj}) in dividing the sum of
squared residuals (SSR) by $2n$ rather than 2. Since \textsf{glmnet}
is not actually using a different relative weighting of the SSR and
the sum of absolute coefficient values, it follows that their
``$\lambda$'' must be read as $n^{-1}$ times ours. Moreover, while we
take each $\lambda_i$ value to be $s_i$ times $\lambda_{\max}$ as
defined in equation (\ref{eq:lmax}), the $\lambda$ values printed by
\textsf{glmnet} are (in our notation)
\[
\lambda_i = s_i \cdot \lambda_{\max} \cdot \hat{\sigma}_y / n 
\]
where $\hat{\sigma}_y$ is the ML estimate of the standard deviation of
the dependent variable. To obtain the \textsf{glmnet} $\lambda$
corresponding to a given $s$ one can do:
\begin{code}
Rlam = s * b.lmax * sdc({y}) / b.nobs
\end{code}
where \texttt{b} is a bundle obtained via \texttt{regls} on the same
data, \texttt{y} is the dependent series and \texttt{sdc(\{y\})} gives
$\hat{\sigma}_y$. The current sample range must be the same as for
\texttt{b} to get $\hat{\sigma}_y$ right, but if \textsf{glmnet} was
told \textit{not} to standardize the data then this term should be
omitted as it is assumed to be 1.

\subsection*{Cross validation}

This needs to go in proper sequence somewhere\dots.

There's a significant difference between \textsf{regls} and
\textsf{glmnet} in respect of cross validation. This applies even if
the CCD algorithm is selected in \textsf{regls}, which results in
near-identical results for coefficients and predicted values when
simply testing a sequence of $\lambda$ values.

In \textsf{regls} cross validation, the entire training dataset is
standardized at the outset, then each fold gets its share of the
standardized data. The maximum $\lambda$ is also determined using the
full training set, so the same $\lambda$ sequence is used for each
fold. In \textsf{glmnet}, by contrast, both standardization and
calculation of the $\lambda$ sequence are done per fold. It's not
obvious \textit{a priori} which of these methods is going to be more
effective in selecting the $\lambda$ that will do best for
out-of-sample prediction, but our Monte Carlo analyses show that the
\textsf{regls} method is superior.

\subsection*{An extended test}

The proof of the pudding for LASSO is out-of-sample prediction
following cross validation. We therefore conducted an extended test of
this, with comparison to \textsf{glmnet}, using the murder-rate
dataset referenced by Ryan Tibshirani\footnote{See
  \url{https://www.stat.cmu.edu/~ryantibs/datamining/lectures/17-modr2.pdf}.}
and included with our \texttt{regls} package. We used the default ADMM
algorithm with \texttt{regls}.

For each of 800 trials we:
\begin{enumerate}
\item Randomized the order of the entire dataset.
\item Selected the first 1000 observations for training and the next
  200 for testing.
\item Ran both \texttt{regls} and \textsf{glmnet}, using 25 values
  of $s = \lambda/\lambda_{\max}$ and 10 consecutive folds.
\item Used the optimal $s$ from cross validation to obtain
  $\hat{\beta}$ on the full set of training data.
\item Obtained fitted values $X\hat{\beta}$ for the testing data and
  calculated the out of sample $R^2$.
\end{enumerate}

In fact we ran this procedure twice, first taking $s^*$ as the optimum
from cross validation and then using $s^{\dagger}$, following
\textsf{glmnet}'s recommendation (see section~\ref{sec:xvalid}). The
results are shown in Table~\ref{tab:bigtest}.

\begin{table}[htbp]
  \centering
  \begin{tabular}{lccccccc}
    \multicolumn{8}{c}{Case A: using $s^*$} \\
 & mean & s.d. & s.e.(mean) & 95\% c.i. & median & min & max \\
\textsf{gretl} & 0.4987 & 0.2223 & 0.0079 & 0.4833 - 0.5141 & 0.5451 & $-$2.4235 & 0.7081 \\
\textsf{glmnet} & 0.4825 & 0.2226 & 0.0079 & 0.4671 - 0.4979 & 0.5244 & $-$2.4184 & 0.7136 \\[8pt]
    \multicolumn{8}{c}{Case B: using $s^{\dagger}$} \\
 & mean & s.d. & s.e.(mean) & 95\% c.i. & median & min & max \\
\textsf{gretl} & 0.4817 & 0.1783 & 0.0063 & 0.4693 - 0.4940 & 0.5122 & $-$1.7316 & 0.6615 \\
\textsf{glmnet} & 0.4585 & 0.1762 & 0.0062 & 0.4463 - 0.4708 & 0.4843 & $-$1.7313 & 0.6571 \\
  \end{tabular}
  \caption{Out-of-sample $R^2$ statistics, 800 trials}
  \label{tab:bigtest}
\end{table}

In Case A, with the MSE-minimizing $s^*$ selected for prediction, the
performance of the two codes is very similar: the 95 percent
confidence intervals for mean $R^2$ are strongly overlapping.  Since
the algorithms employed are quite different, one might say that their
respective results are mutually supportive.

In Case B, following the ``within one standard error'' rule to give
the benefit of the doubt to parsimony, the results are not quite so
good for either code (although there's some overlap with the Case A
confidence intervals).

The negative minimum $R^2$ values call for comment. It seems that in 2
or 3 percent of the randomly reordered datasets the 200 testing
observations differed substantially from the 1000 training
observations. An $R^2 < 0$ means, of course, that the predictions of
the model are inferior to simply predicting the sample mean for all
observations in the given sample range, but since we're talking about
out-of-sample prediction the sample mean for the testing data should
be considered unknown when predictions are made.

The script that produced the Case A results above is shown on the
following page. For Case B it's necessary to add
\texttt{parms.use\_1se = 1} when forming the parameter bundle for
gretl's \texttt{regls}, and in the \textsf{R} section to change the
line that assigns to \texttt{Rb} to read
\begin{code}
Rb <- as.matrix(coef(m$glmnet.fit, s = m$lambda.1se))
\end{code}

\clearpage

\begin{script}
  \caption{Monte Carlo script for out-of-sample prediction}
  \label{script:mc}
\begin{scode}
set verbose off
include regls.gfn
open murder.gdt --quiet --frompkg=regls

# all available predictors without missing values
list X = population..LemasPctOfficDrugUn
list X0 = const X
list LL = murdPerPop X

# the lambdas we'll use
lamseq = lambda_sequence(1, 25)
bundle parms = defbundle("lfrac", lamseq, "stdize", 1, "verbosity", 0)
parms.xvalidate = 1
parms.nfolds = 10

matrix foldvec = regls_foldvec(1000, 10)
mwrite(foldvec, "folds.mat", 1) # for glmnet

K = 800
matrix OSR2 = zeros(K, 2)

loop i=1..K --quiet
   smpl full
   series sorter = normal()
   dataset sortby sorter

   smpl 1 1000 # training data
   bundle lb = regls(murdPerPop, X, parms)

   foreign language=R --send-data=LL
     if (! "glmnet" %in% (.packages())) {
        library(glmnet)
     }
     x <- as.matrix(gretldata[,2:ncol(gretldata)])
     y <- as.matrix(gretldata[,1])
     folds <- gretl.loadmat("folds.mat")
     m <- cv.glmnet(x, y, foldid = folds, family = "gaussian", alpha = 1,
      nlambda = 25, standardize = T, intercept = T)
     Rb <- as.matrix(coef(m$glmnet.fit, s = m$lambda.min))
     gretl.export(Rb, quiet=1)
   end foreign --quiet

   Rb = mread("Rb.mat", 1)

   smpl 1001 1200 # testing data
   series pred = lincomb(lb.nzX, lb.nzb)
   m = regls_get_stats(murdPerPop, pred)
   OSR2[i,1] = m[2]
   series Rpred = lincomb(X0, Rb)
   m = regls_get_stats(murdPerPop, Rpred)
   OSR2[i,2] = m[2]
endloop

mwrite(OSR2, "OSR2_800.mat")
\end{scode}
  \end{script}

\end{document}
