\documentclass{article}
\usepackage{lucidabr,amsmath}
\usepackage{verbatim,fancyvrb}
\usepackage{color,gretlcode}
\usepackage[authoryear]{natbib}
\usepackage{smallhds,url}

\usepackage[letterpaper,body={6.3in,9.15in},top=.8in,left=1.1in]{geometry}
%\usepackage[a4paper,body={6.1in,9.7in},top=.8in,left=1.1in]{geometry}

\begin{document}

\setlength{\parindent}{0pt}
\setlength{\parskip}{1ex}

\newcommand{\argname}[1]{\textsl{#1}}

\title{Ivpanel 0.5}
\author{Allin Cottrell}
\maketitle

\section{Introduction}

This package estimates three variants of panel models with
instrumental variables, namely fixed-effects, the ``between'' model,
and random effects (G2SLS). An extended discussion of these models can
be found in chapter 5 of \cite{baltagi05}. The sample script provided
with this package replicates the results obtained by Baltagi using
\textsf{Stata} for each of the three supported models.

The required arguments to the main public function \texttt{ivpanel()}
are as follows:

\begin{enumerate}
\item \argname{y}: a series, the dependent variable
\item \argname{X}: a list, the regressors, both exogenous and endogenous
\item \argname{Z}: a list, the instruments
\end{enumerate}

Note that exogenous variables in list \argname{X} should be repeated in
\argname{Z}, as in gretl's \texttt{tsls} command.

An optional fourth argument, \argname{case}, can be used to specify the
type of model. The argument is an integer switch that takes values 1,
2 or 3 (this may be extended in future).  A value of 1 means fixed
effects, and is implicit if no fourth argument is given; a value of 2
means to use the between estimator; and a value of 3 means G2SLS.

An optional final boolean argument, \argname{quiet}, controls the
printing of output: if \argname{quiet} is set to a non-zero value,
printing is suppressed.

\section{The ivpanel bundle}

By way of return value, \texttt{ivpanel()} offers a gretl bundle
containing the items shown in Table~\ref{tab:bun}.

\begin{table}[htbp]
\centering
\begin{tabular}{llp{.6\textwidth}}
  \textit{name}   & \textit{type} & \textit{description} \\[4pt]
  \texttt{case}   & scalar & the value of the \argname{case} argument on input \\
  \texttt{nobs}   & scalar & the total number of observations used \\
  \texttt{coeff}  & matrix & column vector of coefficients \\
  \texttt{stderr} & matrix & column vector of standard errors \\
  \texttt{vcv}    & matrix & the coefficient covariance matrix \\
  \texttt{uhat}   & matrix & the vector of residuals \\
  \texttt{yhat}   & matrix & the vector of fitted values \\
  \texttt{SSR}    & scalar & sum of squared residuals \\
  \texttt{sigma}  & scalar & standard error of the regression \\
  \texttt{df}     & scalar & degrees of the freedom for the regression \\
  \texttt{rsq}    & scalar & correlation-based $R^2$  \\
  \texttt{wald}   & scalar & Wald joint $\chi^2$ test for all regressors \\
  \texttt{modstr} & string & short description of the model \\
  \texttt{ystr}   & string & the name of the dependent variable \\
  \texttt{Estr}   & string & the names of the endogenous regressor(s) \\
  \texttt{Istr}   & string & the names of the (additional) instruments \\
  \texttt{dims}   & matrix & data dimensions (cases 1 and 3 only) \\
  \texttt{Fpool}  & scalar & $F$-test for joint significance of fixed
                             effects (case 1 only)
\end{tabular}
\caption{Items in ivpanel bundle}
\label{tab:bun}
\end{table}

Some comments on the bundle members follow.

\begin{itemize}

\item The $R^2$ value, \texttt{rsq}, is calculated as the square of the
  correlation between the dependent variable and the fitted values.

\item The \texttt{dims} vector, if present, contains three elements
  holding, respectively, the number of groups used and the minimum and
  maximum time-series spans.

\item The \texttt{Fpool} statistic for fixed effects is calculated as
  per \cite{wooldridge90}, adjusted for the panel case. The formula is
\[
 \frac{\mbox{SSRr} - \mbox{SSRu}}{\mbox{SSRf}} \times 
  \frac{\mbox{dfd}}{\mbox{dfn}}
\]

 where SSRf is the sum of squared residuals from the fixed-effects IV
 model and the other two SSR values are calculated thus:
\begin{enumerate}
\item Run fixed-effects first-stage regressions and save the fitted
values; use these to replace the endogenous regressors.

\item SSRr is then obtained via OLS, and SSRu from a fixed-effects 
regression. SSRu differs from SSRf in that it uses the ``raw''
residuals, without correction as one usually applies with two 
stage least squares (i.e.\ replacing the first-stage fitted 
values with the actual data for the endogenous regressors 
when computing the fitted values).
\end{enumerate}

\end{itemize}

\section{Auxiliary printing function}

The auxiliary public function \texttt{ivp\_print()} is provided to
pretty-print the results contained in the bundle provided by
\texttt{ivpanel()}; \texttt{ivp\_print()} takes that bundle as its
sole argument.

\bibliographystyle{gretl}
\bibliography{gretl}

\end{document}
