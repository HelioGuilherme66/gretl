\documentclass[11pt,english]{article}
\usepackage{mathpazo}
\usepackage[a4paper]{geometry}
\geometry{verbose,tmargin=3cm,bmargin=3cm,lmargin=3cm,rmargin=3cm}
\usepackage{array}
\usepackage{booktabs}
\usepackage{multirow}
\usepackage{fancyvrb}

\DefineVerbatimEnvironment%
{code}{Verbatim}
{fontsize=\small, xleftmargin=1em}

\newcommand{\noun}[1]{\textsc{#1}}
\newcommand{\dollar}[1]{\texttt{\$#1}}

%% Because html converters don't know tabularnewline
\providecommand{\tabularnewline}{\\}

\usepackage{babel}

\usepackage{fancyvrb,color}
\newenvironment{funcdoc}[1]
{\noindent\hrulefill\newline\texttt{#1}\par\noindent\hrulefill\par\medskip\par}
{\bigskip}

% ---------- Ripped from gretl.sty ----------------------------------------

\newcommand{\scriptname}{Example}

\newcommand{\app}[1]{\textsf{#1}}
\newcommand{\cmd}[1]{\texttt{#1}}
\newcommand{\varname}[1]{\texttt{#1}}
\newcommand{\option}[1]{\texttt{-{}-#1}}

\newcommand{\ttsl}[1]{\ttfamily{\textsl{#1}}\normalfont}

\newcommand{\GUG}{\emph{Gretl User's Guide}}

\newenvironment{textcode}{\par\small\ttfamily}
{\normalfont\normalsize\par}

\DefineVerbatimEnvironment%
{code}{Verbatim}
{fontsize=\small, xleftmargin=1em}

\DefineVerbatimEnvironment%
{scode}{Verbatim}
{frame=lines, framesep=2ex, fontsize=\small,
 formatcom=\color{myteal}, rulecolor=\color{mygray}}

\DefineVerbatimEnvironment%
{scodebit}{Verbatim}
{fontsize=\small, formatcom=\color{myteal}}

\DefineVerbatimEnvironment%
{scodebot}{Verbatim}
{frame=bottomline, framesep=2ex, fontsize=\small,
 formatcom=\color{myteal}, rulecolor=\color{mygray}}

\renewcommand{\arraystretch}{1.2}

\definecolor{mygray}{rgb}{0.85,0.85,0.85} 
\definecolor{myteal}{rgb}{0.0,0.25,0.15} 

\def\floatpagefraction{.8}

%% add script as float (Example)
\newcounter{script}[section]
\renewcommand \thescript
     {\ifnum \c@chapter>\z@ \thechapter.\fi \@arabic\c@script}
\def\fps@script{tbp}
\def\ftype@script{1}
\def\ext@script{los}
\def\fnum@script{\scriptname\nobreakspace\thescript}
\newenvironment{script}
               {\@float{script}}
               {\end@float}
\newenvironment{script*}
               {\@dblfloat{script}}
               {\end@dblfloat}
\newcommand\theHscript{\thechapter.\arabic{script}}

\newcommand{\tip}[1]{\par\vspace{4pt}
 \ding{43} {\small \sffamily #1}\par}

%% define a very simple bibliography environment

% \newcommand{\bibname}{References}
% \renewenvironment{thebibliography}
%      {\section*{\bibname}
%       \label{refs}
%       \addcontentsline{toc}{section}{\bibname}
%       \setlength\parindent{0pt}
%       \setlength\parskip{6pt}}
%       {}

% ---------- End rip ------------------------------------------------------


\newcommand{\ArgRet}[2]{%
  {\it Arguments}: {#1}%
  \ifx&#2&%
  \else
  \par\smallskip\noindent {\it Return type}: \texttt{#2}
  \fi%
  \par\medskip\par%
  }

\begin{document}

\title{The extra package\\
(a collection of various convenience functions for hansl programming) }

\date{January 2023, release 1.6}

\author{The \noun{gretl} team\thanks{Currently co-ordinated by Sven Schreiber.}}

\maketitle
\tableofcontents{}

\section{Usage}

This package is intended for hansl scripting, not for gretl's GUI.
(But of course other contributed function packages that make use of
functions in extra.gfn can provide GUI access for themselves.)

The usual one-time requirement is to do \cmd{pkg install extra.zip}
to get a copy on the local system (or install it via gretl's graphical
mechanism), and then in the respective hansl script have a line \cmd{include
extra.gfn}.

Note that functions that are exact lookalikes of \app{Matlab/Octave}
functions do not live here, but would go into the
\texttt{matlab\_utilities} package.

%%%%%%%%%%%%%%%%%%%%%%%%%%%%%%%

\section{Matrix-related functions}

\subsection{eliminate}

\ArgRet{\texttt{matrix vecA}}{matrix}

Each column of the input vecA is assumed to come from the operation vec(A)
on a square matrix, thus rows(vecA) must be a square number.
Returns vech(A), which is the result of pre-multiplying vec(A) with the
"elimination" matrix $L_m$.
If vecA has several columns, each column is
treated separately as described above (and the results stacked side-by-side).

\subsection{duplicate}

\ArgRet{\texttt{matrix vechA}}{matrix}

The input is a vector assumed to come from an operation like vech(A).
Returns vec(A), which is the result of pre-multiplying vech(A) with the
"duplication" matrix $D_m$. If vechA has several columns, each column is
treated separately as described above (and the results stacked side-by-side).

\subsection{nearPSD}

\ArgRet{\texttt{matrix} pointer \texttt{{*}m}, \texttt{scalar epsilon}
(optional)}{scalar}

Forces the matrix $m$ into the positive semi-definite region. Algorithm
ported from ``DomPazz'' in Stackoverflow,
apparently mimicking the nearPD() function in R. Because of re-scaling
(to correlation matrix), the \texttt{epsilon} criterion value should
implicitly apply to the correlation-based eigenvalues. The return
value 0 or 1 indicates whether \texttt{m} was altered or not.

\subsection{zeroifclose}

\ArgRet{\texttt{matrix} pointer \texttt{{*}m}, \texttt{scalar thresh}
(optional)}{scalar}

Sets elements of \texttt{m} to zero if they are really close. The
return value 0 or 1 indicates whether \texttt{m} was altered or not.

The default value for the threshold has been \texttt{1e-12} since extra
version 0.7. In some applications smaller (in absolute value) but mathematically
truly non-zero results may occur, in which case a smaller threshold can be chosen.

\subsection{drill}

\ArgRet{\texttt{matrices} array, \texttt{matrix rowspec} (optional),
  \texttt{matrix colspec} (optional)}{matrix}

This function "drills through" a matrix array and returns a matrix;
for example, \cmd{drill(x, 2, 3)} returns a vector with the
\texttt{[2,3]} elements of all matrices in the \texttt{x}
array. Omitting one of rowspec, colspec or entering "0" means to
select all rows or columns respectively.  (Of course at least one of
rowspec, colspec must be specified.)

\emph{Nota bene}: all matrices in the array must be of the same dimensions.

\subsection{combinations}

\ArgRet{\texttt{matrix x}, \texttt{int h}}{matrix} This function
returns a matrix whose rows are all the possible subsets of $x$
containing $h$ elements; for $h>2$, a recursive algorithm is used.

For example: \texttt{combinations(\{1,2,3\}, 2)} returns
\begin{code}
         1   2
         1   3
         2   3
\end{code}

The argument \texttt{x} must be a (row or column) vector. The returned
matrix will have $n \choose k$ rows if successful, otherwise it will be a 1x1
matrix with an NA value.

\emph{Nota bene}: The recursive algorithm used may be a little slow if
the number of elements of $x$ is large.

\subsection{qformvech}

\ArgRet{\texttt{matrix Xt}}{matrix}
This function relies on the relation
$vech(X'AX) = P (X \otimes X)' Q \; vech(A) = G \; vech(A)$,
where $P$ and $Q$ are certain interim results.
It takes the matrix $X'$ and returns the matrix $G$ such that the right-hand
side of the equalities becomes feasible, which should be
numerically more efficient than the direct application
of the left-hand side.


%%%%%%%%%%%%%%%%%%%%%%%%%%%%%%%%

\section{Other functions working without a dataset in place}

\subsection{scores2x2}

\ArgRet{\texttt{matrix in, bool verbose} (optional)}{matrix}

Computes some standard score measures for a $2\times 2$ contingency
table of the form:

\begin{center}
\begin{tabular}{cccc}
\toprule
 &  & \multicolumn{2}{c}{Observed}\tabularnewline
 &  & 1 & 0\tabularnewline
\midrule
\multirow{2}{*}{Predicted} & 1 & h(its) & f(alse)\tabularnewline
 & 0 & m(iss) & z(eros)\tabularnewline
\bottomrule
\end{tabular}
\end{center}

\noindent and $n=h+f+m+z$ (total observations). Returns a column vector with the
elements listed in Table \ref{tab:scores2x2}. The input is always
sanitized by taking the upper 2x2 part, using absolute values, and
integer-ization. Warnings are issued if \texttt{verbose} is 1.


\begin{table}[htbp]
\begin{tabular}{rlp{0.35\textwidth}c}
  \hline
  \textbf{Number} &\textbf{Acronym} &  \textbf{Description} &\textbf{Formula} \\
  \hline
  1 & POD & prob of detection & $\frac{h}{h+m}$ \\
2 & POFD & prob of false detection & $\frac{f}{f+z}$ \\
3 & HR & hit rate & $\frac{h+z}{n}$ \\
4 & FAR & false alarm rate & $\frac{f}{h+f}$ \\
5 & CSI & critical success index & $\frac{h}{h+f+m}$\\
6 & OR & odds ratio & $\frac{h \cdot z}{f \cdot m}$\\
7 & BIAS & bias score & $\frac{h+f}{h+m}$\\
8 & TSS & true skill stat ($POD-POFD$); also known as the
          Hanssen-Kuipers score & $\frac{h}{h+m} -\frac{f}{f+z}$. \\
9 & HSS & Heidke skill score & $2 \frac{h \cdot z - f \cdot m}{(h+m) \cdot (m+z)+(h+f) \cdot (f+z)}$ \\
10 & ETS & equitable threat score & $\frac{h \cdot z-f \cdot m}{(f+m) \cdot n+(h \cdot z-f \cdot m)}$ \\
11 & PRC & precision & $\frac{h}{h+f}$ \\
12 & FSC & $F$-Score & $2 \frac{PRC \cdot POD}{PRC+POD} = 2 \frac{h}{1+h+m}$.\\
  \hline
\end{tabular}
\caption{Elements returned by the \cmd{scores2x2} function}
\label{tab:scores2x2}
\end{table}


\subsection{truncnorm}

\ArgRet{\texttt{int n, scalar m, scalar sigma, scalar below, scalar
above}}{matrix}

Generates $n$ truncated normal random values. Specify mean \texttt{m}
and standard deviation \texttt{sigma}, and the left/right truncation values
\texttt{below} and \texttt{above}. (Pass NA for any one of them to
skip the respective truncation.) Returns a column vector of values.


\subsection{WSRcritical}

\ArgRet{\texttt{int n, scalar prob }(optional)\texttt{, bool forcenorm}
(optional)}{matrix}

Concerns the distribution of Wilcoxon's signed rank test statistic for
\texttt{n} trials (at least 4). Tries to find the critical values
(low/hi) where the two-sided area to the outside is as close as
possible to the given \texttt{prob} (default: 0.05). (Note that
``outside'' means including the critical values themselves in the
exact/discrete case.) If we end up in the interior region not covered
by the exact table (for \texttt{prob} far away from 0 and also from
1), we fall back to the normal approximation. The function returns a
column vector \verb|{lo; hi; epv}|, where \texttt{epv} is the actual
probability mass (close to \texttt{prob} but not equal in general for
small samples). \texttt{lo} and \texttt{hi} can be non-integers in the normal
approximation case. The normal approximation instead of the exact
table values can be enforced with the \texttt{forcenorm} argument
(default: zero, do not enforce).

See also the sister function \cmd{WSRpvalue}.

\subsection{WSRpvalue}

\ArgRet{\texttt{int n, scalar W, bool forcenorm} (optional)}{scalar}

Concerns the distribution of Wilcoxon's signed rank test statistic for
\texttt{n} trials (at least 4), returns $P(X\geq W)$. In the interior
region not covered by the exact table, the true value is $\geq$ 12.5\%
(and $\leq$87.5\%) according to the table used,\footnote{Source of the
  table: Wilfrid J Dixon and Frank J. Massey, Jr., Introduction to
  Statistical Analysis, 2nd ed. (New York: McGraw-Hill, 1957), pp.
  443-444.} so typically based on such values H0 would not be
rejected. We fall back to the normal approximation in this region. In
the extreme outer regions not explicitly covered by the table, the
deviation from 0 or 1 will be smaller than 0.5\% = 0.005. We return
values 0.001 or 0.999 as an approximation here. The test statistic
\texttt{W} should usually be an integer, but in case of bindings it
could be fractional as well; in this case we also fall back to the
normal approximation.

The normal approximation instead of the exact table values can be
enforced with the \texttt{forcenorm} argument (default: zero, do not
enforce).

See also the sister function \cmd{WSRcritical}.

\subsection{powerset}

\ArgRet{\texttt{strings S}}{}

Computes the powerset of the input S, i.e. all possible combinations
of the string elements in S. (Including the empty set /
empty string "".) Each combination yields one string in the output
array. Being a set, the ordering is not defined and arbitrary.


\subsection{onemode}

\ArgRet{\texttt{matrix v}}{}

Finds the mode of the empirical distribution of the input data.
If that is multi-modal, details of internal computer arithmetic can
influence which of the modes is actually found. Returns a 2-element column vector with the modal value and its absolute frequency. If v is an empty
matrix (comprises only nan values) a 1x1 matrix with nan is returned.


\subsection{splitfname}

\ArgRet{\texttt{string fn}}{strings (array)}

The idea is to take a file name or full path and extract three components:
\begin{enumerate}
\item The path prefix (may be empty; without the trailing / or \textbackslash)
\item The "base" component of the file name, without the extension and without the path prefix
\item The file extension (without the dot; may be empty)
\end{enumerate}
In principle should work with forward as well as backslashes and also with double forward slashes.

Example:

Input string: "/what/on/earth/isthisfile.gdt"

Output equivalent to:

\cmd{defarray("/what/on/earth", "isthisfile", "gdt")}

\subsection{multi\_instrings}

\ArgRet{\texttt{strings lookinhere}, \texttt{strings tofind}}{matrix}

Returns in a column vector the positions (indices) in `lookinhere' where any of the
strings from `tofind' occur. If there are duplicates in `tofind' then the output
may also contain duplicate indices. Use \cmd{uniq()} or \cmd{values()} afterwards if needed.

\subsection{mat2latex}

\ArgRet{\texttt{matrix X}, \texttt{bundle opts} (optional)}{string}

Produces a string containing the representation of matrix \texttt{X}
as a \LaTeX\ \texttt{tabular} environment. For example,
\begin{code}
eval mat2latex(mshape(seq(1,6), 2, 3))
\end{code}
produces
\begin{code}
\begin{tabular}{lccc}
\hline
 &	  &	 Col 1  &	 Col 2  &	 Col 3 \\ \hline
Row 1 & 1.000	 & 3.000	 & 5.000 \\
Row 2 & 2.000	 & 4.000	 & 6.000 \\
\hline
\end{tabular}
\end{code}
Note that, if a matrix possesses row or column names, they will be
automatically used as labels. Some features of the results can be
tweaked by setting appropriate keys in the \texttt{opts} bundle:
\begin{description}
\item[format]: a string, to be used for aligning columns. The default
  is to have the first column left-aligned, and the subsequent ones
  centered, as in \texttt{"lccc"}.
\item[decimals] The number of decimals to use (default=3).
\item[nacode] The string to use for missing entries (default: empty).
\item[rnames] A string array for row headings.
\item[cnames] A string array for column headings.
\end{description}

For example, the code
\begin{code}
  open credscore.gdt
  xtab OwnRent Selfempl --quiet
  s = mat2latex($result, _(decimals=0))
  print s
\end{code}
% $
produces
\begin{code}
\begin{tabular}{lccc}
\hline
 &	   0  &	    1  &	 TOTAL \\ \hline
   0 & 60	 & 4	 & 64 \\
   1 & 35	 & 1	 & 36 \\
TOTAL & 95	 & 5	 & 100 \\
\hline
\end{tabular}
\end{code}
which looks, when compiled, as
\begin{center}
\begin{tabular}{lccc}
\hline
 &	   0  &	    1  &	 TOTAL \\ \hline
   0 & 60	 & 4	 & 64 \\
   1 & 35	 & 1	 & 36 \\
TOTAL & 95	 & 5	 & 100 \\
\hline
\end{tabular}
\end{center}

%%%%%%%%%%%%%%%%%%%%%%%%%%%%%%%

\section{Functions requiring a dataset}

\subsection{correspondence}

\ArgRet{\texttt{series a, series b}}{scalar}

This function takes two series and establishes if there's a 1-to-1
relationship between them, in which case it returns 2.  If there's a
1-to-n relationship such that \texttt{a} could be interpreted as a
(mathematical) function of \texttt{b}, it returns 1. If there's no relationship
-- for example several different values of series \texttt{a} appear together
with some value of \texttt{b} -- it returns 0.

One of the possible use cases is to check whether two discrete series encode
the same variable. For example, the code:
\begin{code}
  open grunfeld.gdt
  c = correspondence($unit, firm)
\end{code}
% $
sets \texttt{c} to 2, indicating that the variable \texttt{firm} is in fact the
panel cross-sectional identifier.

\subsection{gap\_filler}

\ArgRet{\texttt{series x, int method }(optional)}{series}

Simple convenience function to crudely get rid of missing values
interspersed between valid observations. The function is meant to be
used with time series, or panel datasets with a time dimension. An
error is returned if the function is used with a cross-sectional
dataset.

Apart from the first argument (series), it accepts an integer
parameter as second argument, whose meaning is: 0: do nothing, leave
the gaps; 1: NAs are replaced with previous observations; 2: NAs are
replaced with a linear interpolation (this uses the internal function
\texttt{interpol()}). Returns the filled series.

The very existence of the "0" method for interpolation may look
bizarre at first sight, but it may make sense to have in the context
of batch processing, like in the following example (hopefully,
self-explanatory):
\begin{code}
k = 1
loop foreach i X
   series z_$i = gap_filler($i, action[k++])
endloop
\end{code}

Note that the function only replaces NAs between valid observations;
therefore, if the origin series has missing values at the beginning or
the end of the sample, they will be in the returned series too.

\subsection{winsor}

\ArgRet{\texttt{series x, scalar p} (optional), \texttt{scalar
phi} (optional)}{series}

Returns a trimmed (``winsorized'') version
of the series, where outliers are replaced with implicit threshold
values. Truncation quantiles are determined according to relative
tail frequencies \texttt{p} and \texttt{phi}. Default lower and upper
frequencies are 0.05, but re-settable with \texttt{p}. Pass \texttt{phi}
in addition to \texttt{p} for an asymmetric trimming, then \texttt{p}
determines only the lower frequency and \texttt{phi} the upper.

\subsection{fracorder}

\ArgRet{\texttt{series x, int order} (optional), \texttt{bool verbosity}
 (optional)}{matrix}

Meta function to invoke all the various ways in gretl to estimate the order of
fractional integration of the input series, namely the Local Whittle estimator,
the one by Geweke \& Porter-Hudak (GPH), and the Hurst exponent minus $0.5$.
The first two are executed through gretl's command \texttt{fractint}, the
latter via \texttt{hurst}.\footnote{Another estimation approach for the Hurst
exponent is provided in the user-contributed function package 'gen\_hurst'.}
% This function only works with gretl 2020c or higher.

Returns a matrix with three rows corresponding to the methods above; the four
columns contain (1) the point estimate, (2) its standard error, (3) the test
statistic for the null hypothesis of integration order zero, (4) the associated
p-value. For example, to obtain the standard error of the Local Whittle
estimator one picks the 1,2-element of the output matrix. The optional
'verbosity' switch is set to 0 (OFF) by default, otherwise the standard output
of the underlying commands is printed out.

The optional 'order' argument only applies to the Local Whittle and GPH
estimators and overrides gretl's default lag order of $\min(T/2, T^{0.6})$.

For the Hurst method a minimum of 128 observations is required, and test
results are never available. Also note that by construction this estimator can
only take values between $-0.5$ and $0.5$.

\subsection{put\_outofsmpl}

\ArgRet{\texttt{bundle modorsys}}{bundle}

This is a function intended for experts who need to grab out-of-sample data
inside their function (package), where that data is not usually available
because it is beyond a user function's scope. One use case is to access
pre-sample time-series data (initial values) that may be needed for
bootstrapping. Another application could be post-sample (future) data in the
context of a forecasting exercise. 
% related to earlier ticket https://sourceforge.net/p/gretl/todo/76/
% " Provide expert hansl function to grab out-of-sample data"

The function accepts a bundle behaving like gretl's \texttt{\$model} or 
\texttt{\$system} accessors, and returns a copy of that bundle with the 
following additional members (all of matrix type):

\begin{itemize}
\item \texttt{initY} -- initial values of the endogenous variables
  (empty 0x0 if no endogenous lags in model or system)
\item \texttt{futureY} -- future (post-sample) values of the endogenous
  variables if they exist, else empty 0x0
\item \texttt{futureX} -- future (post-sample) values of the exogenous
  variables if they exist, else empty 0x0
\end{itemize}

For the right-hand side terms there are different conventions between
single-equation (\texttt{\$model}) and generic simultaneous-equation
(\cmd{system} block) setups on the one hand, and specialized  \cmd{var} /
\cmd{vecm} commands on the other hand:
 
\begin{itemize}
\item Generic single-equation and generic \cmd{system}: 
  \texttt{futureX} will contain also deterministics and lagged terms 
  (internally taken directly from \texttt{\$xlist}).
 
\item \cmd{var} and \cmd{vecm}: 
  \texttt{futureX} will contain only the truly exogenous and 
   non-deterministic terms, for example not the lags of the 
   endogenous variables.
\end{itemize}

\section{Authors}
\begin{itemize}
\item gap\_filler, eliminate, duplicate, truncnorm, powerset,
  drill, correspondence, combinations, qformvech: Jack Lucchetti

\item nearPSD, zeroifclose, scores2x2, WSRcritical, WSRpvalue, onemode,
splitfname, multi\_instrings, fracorder, put\_outofsmpl:
Sven Schreiber

\item winsor: Sven Schreiber, original code JoshuaHe


\end{itemize}

\section{Changelog }
\begin{itemize}
\item December 2022: 1.6, introduce the put\_outofsmpl function; internal
  refactoring of the mat2latex() function
\item July 2022: 1.5, introduce the mat2latex function
\item June 2022: 1.4, retire the commute function (now in libgretl)
\item January 2022: 1.3, retire the mat2list function (now in
  libgretl), update gap\_filler
\item October 2021: 1.2, add qformvech
\item June 2021: 1.1, add combinations, and increase gretl version requirement to 2020c
\item November 2020: 1.0, add mat2list
\item September 2020: 0.8, add fracorder, remove bwritejson
\item July 2020: 0.7, add multi\_instrings and correspondence, add deprecation warning to
bwritejson, efficiency improvement for zeroifclose
\item January 2020: 0.6, add drill, bwritejson, onemode, splitfname;
  finally remove the retired sepstr2arr (use native strsplit instead);
  slightly revise gap\_filler; rearrange the documentation a little
\item October 2018: 0.5, fix small commute bug; retire sepstr2arr; add powerset,
  eliminate, duplicate
\item February 2018: 0.41, allow non-integer input in WSRpvalue
\item January 2018: 0.4, add WSRcritical, WSRpvalue
\item December 2017: 0.3, add scores2x2; switch to pdf help document
\item September 2017: 0.2, add winsor
\item July 2017: initial release
\end{itemize}

\end{document}

%%% Local Variables:
%%% mode: latex
%%% TeX-master: t
%%% End:
